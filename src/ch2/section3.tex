\section{Cyclic Groups and Cyclic Subgroups}

\Exercise1 Find all subgroups of $Z_{45} = \gen{x}$, giving a
generator for each. Describe the containments between these subgroups.
\begin{solution}
  The subgroups are generated by $x^d$ where $d$ divides $45$. And we
  have $\gen{x^a} \leq \gen{x^b}$ if $(b,45) \mid (a,45)$. This gives
  the following subgroup relationships:
  \begin{align*}
    Z_{45} = \gen{x} &> \gen{x^3}, \gen{x^5},
                       \gen{x^9}, \gen{x^{15}}, 1, \\
    \gen{x^3} &> \gen{x^9}, \gen{x^{15}}, 1, \\
    \gen{x^5} &> \gen{x^{15}}, 1, \\
    \gen{x^9} &> 1, \\
    \gen{x^{15}} &> 1, \\
    1 = \gen{x^0}\rlap. & \qedhere
  \end{align*}
\end{solution}

\Exercise2 If $x$ is an element of the finite group $G$ and
$\ord{x} = \ord{G}$, prove that $G = \gen{x}$. Give an explicit
example to show that this result need not be true if $G$ is an
infinite group.
\begin{proof}
  Let $x\in G$ where $\ord{x} = \ord{G} = n < \infty$. By
  Proposition~2, we know that $1, x, x^2, \dots, x^{n-1}$ are all
  distinct elements in $G$. But $G$ contains only $n$ elements, so
  this must be the entirety of $G$. Therefore $G = \gen{x}$.

  This is not always true if $\ord{x} = \ord{G} = \infty$. For
  example, in the additive group $\Z$, $\ord{2} = \ord{\Z} = \infty$
  but clearly $\Z$ is not generated by $2$.
\end{proof}

\Exercise3 Find all generators for $\Z/48\Z$.
\begin{solution}
  The generators are those residue classes whose representatives are
  relatively prime to $48$. Therefore the generators are $\bar1$,
  $\bar5$, $\bar7$, $\overline{11}$, $\overline{13}$, $\overline{17}$,
  $\overline{19}$, $\overline{23}$, $\overline{25}$, $\overline{29}$,
  $\overline{31}$, $\overline{35}$, $\overline{37}$, $\overline{41}$,
  $\overline{43}$, and $\overline{47}$.
\end{solution}

\Exercise4 Find all generators for $\Z/202\Z$.
\begin{solution}
  $202 = 2\cdot101$, so the generators are all residue classes having
  odd representatives excluding $\overline{101}$.
\end{solution}

\Exercise5 Find the number of generators for $\Z/49000\Z$.
\begin{solution}
  If $\varphi$ denotes the Euler $\varphi$-function, then the number
  of generators is given by
  \begin{align*}
    \varphi(49000)
    &= \varphi(2^3)\varphi(5^3)\varphi(7^2) \\
    &= 2^2(2 - 1)5^2(5 - 1)7(7 - 1) \\
    &= 4\cdot100\cdot42 \\
    &= 16800. \qedhere
  \end{align*}
\end{solution}

\Exercise6 In $\Z/48\Z$ write out all elements of $\gen{\bar{a}}$ for
every $\bar a$. Find all inclusions between subgroups in $\Z/48\Z$.
\begin{solution}
  The elements of each subgroup are
  \begin{align*}
    \Z/48\Z = \gen{\bar1}
    &= \{\bar0, \bar1, \bar2, \bar3, \dots, \overline{46},
      \overline{47}\}, \\
    \gen{\bar2} &= \{\bar0, \bar2, \bar4, \bar6, \dots,
                  \overline{44}, \overline{46}\}, \\
    \gen{\bar3} &= \{\bar0, \bar3, \bar6, \bar9, \dots,
                  \overline{42}, \overline{45}\}, \\
    \gen{\bar4} &= \{\bar0, \bar4, \bar8, \overline{12}, \dots,
                  \overline{40}, \overline{44}\}, \\
    \gen{\bar6} &= \{\bar0, \bar6, \overline{12}, \overline{18},
                  \overline{24}, \overline{30}, \overline{36},
                  \overline{42}\}, \\
    \gen{\bar8} &= \{\bar0, \bar8, \overline{16}, \overline{24},
                  \overline{32}, \overline{40}\}, \\
    \gen{\overline{12}} &= \{\bar0, \overline{12}, \overline{24},
                          \overline{36}\}, \\
    \gen{\overline{16}} &= \{\bar0, \overline{16}, \overline{32}\}, \\
    \gen{\overline{24}} &= \{\bar0, \overline{24}\}, \\
    \gen{\bar0} &= \{\bar0\}.
  \end{align*}
  And we have the following inclusions:
  \begin{align*}
    \gen{\bar0}, \gen{\bar1}, \gen{\bar2}, \gen{\bar3}, \gen{\bar4},
    \gen{\bar6}, \gen{\bar8}, \gen{\overline{12}}, \gen{\overline{16}},
    \gen{\overline{24}}
    &\leq \gen{\bar1}, \\
    \gen{\bar0}, \gen{\bar2}, \gen{\bar4}, \gen{\bar6}, \gen{\bar8},
    \gen{\overline{12}}, \gen{\overline{16}}, \gen{\overline{24}}
    &\leq \gen{\bar2}, \\
    \gen{\bar0}, \gen{\bar3}, \gen{\bar6}, \gen{\overline{12}},
    \gen{\overline{16}}, \gen{\overline{24}}
    &\leq \gen{\bar3}, \\
    \gen{\bar0}, \gen{\bar4}, \gen{\bar8}, \gen{\overline{12}},
    \gen{\overline{16}}, \gen{\overline{24}}
    &\leq \gen{\bar4}, \\
    \gen{\bar0}, \gen{\bar6}, \gen{\overline{12}}, \gen{\overline{24}}
    &\leq \gen{\bar6}, \\
    \gen{\bar0}, \gen{\bar8}, \gen{\overline{16}}, \gen{\overline{24}}
    &\leq \gen{\bar8}, \\
    \gen{\bar0}, \gen{\overline{12}}, \gen{\overline{24}}
    &\leq \gen{\overline{12}}, \\
    \gen{\bar0}, \gen{\overline{16}} &\leq \gen{\overline{16}}, \\
    \gen{\bar0}, \gen{\overline{24}} &\leq \gen{\overline{24}}, \\
    \gen{\bar0} &\leq \gen{\bar0}. \qedhere
  \end{align*}
\end{solution}

\Exercise7 Let $Z_{48} = \gen{x}$ and use the isomorphism
$\Z/48\Z\cong Z_{48}$ given by $\bar1\mapsto x$ to list all subgroups
of $Z_{48}$ as computed in the preceding exercise.
\begin{solution}
  The subgroups are $\gen{x}$, $\gen{x^2}$, $\gen{x^3}$, $\gen{x^4}$,
  $\gen{x^6}$, $\gen{x^8}$, $\gen{x^{12}}$, $\gen{x^{16}}$,
  $\gen{x^{24}}$, and $1$.
\end{solution}

\Exercise8 Let $Z_{48} = \gen{x}$. For which integers $a$ does the map
$\varphi_a$ defined by $\varphi_a\colon\bar1\mapsto x^a$ extend to an
{\em isomorphism} from $\Z/48\Z$ onto $Z_{48}$.
\begin{solution}
  Choose an $a$ with $(a,48) = d > 1$ and set $b = 48 / d$. If
  $\varphi_a$ is a homomorphism, then
  \begin{equation*}
    \varphi_a(\bar{b}) = \varphi_a(b\cdot\bar1)
    = \varphi_a(\bar1)^b = x^{ab} = (x^{48})^{a/d} = 1 = \varphi_a(\bar0).
  \end{equation*}
  Therefore, in this case, $\varphi_a$ is not an injection and thus
  not an isomorphism.

  This suggests that $\varphi_a$ extends to an isomorphism if and only
  if $(a,48) = 1$, which we will now prove. First we show that the
  function $\bar{b}\mapsto x^{ab}$ is well defined, i.e., that the
  value of the function is not affected by the choice of
  representative for $\bar{b}$. Suppose $\bar{b} = \bar{c}$. Then
  $48k = b - c$ for some integer $k$, and we have
  \begin{equation*}
    \varphi_a(\bar{b}) = x^{ab} = x^{a(48k + c)} = (x^{48})^{ak}x^{ac}
    = 1^{ak}x^{ac} = x^{ac} = \varphi_a(\bar{c}).
  \end{equation*}

  Now, $\varphi_a$ is certainly a homomorphism, since
  \begin{equation*}
    \varphi_a(\bar{b} + \bar{c}) = x^{a(b + c)} = x^{ab}x^{ac} =
    \varphi_a(\bar{b})\varphi_a(\bar{c}).
  \end{equation*}
  To show injectivity, suppose
  $\varphi_a(\bar{b}) = \varphi_a(\bar{c})$. Then $x^{ab} = x^{ac}$ or
  $x^{a(b - c)} = 1$. Hence $48\mid a(b - c)$ and since $(a,48) = 1$
  we have $48\mid(b - c)$. This shows that $\bar{b} = \bar{c}$.

  Finally, since $\ord{\Z/48\Z} = \ord{Z_{48}} < \infty$, we know that
  injectivity of $\varphi_a$ implies surjectivity, so that $\varphi_a$
  is an isomorphism.
\end{solution}

\Exercise9 Let $Z_{36} = \gen{x}$. For which integers $a$ does the map
$\psi_a$ defined by $\psi_a\colon\bar1\mapsto x^a$ extend to a {\em
  well defined homomorphism} from $\Z/48\Z$ into $Z_{36}$. Can
$\psi_a$ ever be a surjective homomorphism?
\begin{solution}
  Suppose $\bar{b} = \bar{c}$ for integers $b$ and $c$. If $\psi_a$ is
  well defined then $\psi_a(\bar{b}) = \psi_a(\bar{c})$, that is,
  $x^{ab} = x^{ac}$. Then $x^{a(b - c)} = 1$ so we must have
  $36\mid a(b-c)$. But $48\mid(b - c)$, so there is an integer $k$ for
  which $48k = b - c$ and we have that $36\mid48ak$. If we choose
  $\bar{b}$ and $\bar{c}$ so that $k = 1$, then we must have $3\mid a$
  as a necessary condition for $36\mid48ak$. It is also sufficient
  that $3\mid a$, since $36\mid144mk$.

  Since
  \begin{equation*}
    \psi_a(\bar{b} + \bar{c}) = x^{a(b + c)}
    = x^{ab}x^{ac} = \psi_a(\bar{b})\psi_a(\bar{c}),
  \end{equation*}
  we see that $\psi_a$ is a well defined homomorphism if and only if
  $3\mid a$.

  Lastly, suppose $\psi_a(\bar{b}) = x$. Since $a = 3k$ for some
  integer $k$, we have
  \begin{equation*}
    x = \psi_a(\bar{b}) = x^{ab} = x^{3kb}.
  \end{equation*}
  Therefore $x^{3kb - 1} = 1$ and we see that $36$ divides $3kb -
  1$. But this is impossible since if $36m = 3kb - 1$ then
  $1 = 3kb - 36m = 3(kb - 12m)$ and $3\mid1$, a contradiction. So the
  homomorphism $\psi_a$ can never be surjective.
\end{solution}

\Exercise{10} What is the order of $\overline{30}$ in $\Z/54\Z$? Write
out all the elements and their orders in $\gen{\overline{30}}$.
\begin{solution}
  Since $\ord{\bar1} = 54$, by Proposition~5~(2) we have
  \begin{equation*}
    \ord{\overline{30}} = \ord{30\cdot\bar1} = \frac{54}{(30,54)}
    = \frac{54}{6} = 9.
  \end{equation*}
  Then
  \begin{equation*}
    \gen{\overline{30}} = \{\bar0, \bar6, \overline{12},
    \overline{18}, \overline{24}, \overline{30}, \overline{36},
    \overline{42}, \overline{48}\}
  \end{equation*}
  where
  \begin{align*}
    \ord{\bar0} &= 1, \\
    \ord{\bar6} &= 9, \\
    \ord{\overline{12}} &= 9, \\
    \ord{\overline{18}} &= 3, \\
    \ord{\overline{24}} &= 9, \\
    \ord{\overline{30}} &= 9, \\
    \ord{\overline{36}} &= 3, \\
    \ord{\overline{42}} &= 9, \\
    \ord{\overline{48}} &= 9. \qedhere
  \end{align*}
\end{solution}

\Exercise{11} Find all cyclic subgroups of $D_8$. Find a proper
subgroup of $D_8$ which is not cyclic.
\begin{solution}
  The cyclic subgroups are
  \begin{align*}
    \gen{1} &= 1, \\
    \gen{r} = \gen{r^3} &= \{1, r, r^2, r^3\}, \\
    \gen{r^2} &= \{1, r^2\}, \\
    \gen{s} &= \{1, s\}, \\
    \gen{sr} &= \{1, sr\}, \\
    \gen{sr^2} &= \{1, sr^2\}, \\
    \gen{sr^3} &= \{1, sr^3\}.
  \end{align*}
  A proper subgroup that is not cyclic is
  $\gen{s, r^2} = \{1, r^2, s, sr^2\}$.
\end{solution}

\Exercise{12} Prove that the following groups are {\em not} cyclic:
\label{exercise-prove-groups-not-cyclic}
\begin{enumerate}
\item $Z_2\times Z_2$
  \begin{proof}
    Let $Z_2 = \gen{x}$. Checking each element of $Z_2\times Z_2$, we
    see that none generate the whole group:
    \begin{align*}
      \gen{(1,1)} &= \{(1,1)\}, \\
      \gen{(1,x)} &= \{(1,1), (1,x)\}, \\
      \gen{(x,1)} &= \{(1,1), (x,1)\}, \\
      \intertext{and}
      \gen{(x,x)} &= \{(1,1), (x,x)\}.
    \end{align*}
    Therefore $Z_2\times Z_2$ is not cyclic.
  \end{proof}
\item $Z_2\times\Z$
  \begin{proof}
    If $Z_2\times\Z$ is cyclic, then it must have a generator of the
    form $(1, n)$ or $(x, n)$ for some $n\in\Z$. But $(1,n)$ cannot be
    a generator since it only generates elements whose first component
    is $1$.

    So the generator must have the form $(x,n)$. Now $n$ can only be
    $1$ or $-1$, since otherwise we could not get all the integers in
    the second component. But neither of these is a generator since,
    for example, $(1,1)$ is not in either cyclic subgroup. Therefore
    $Z_2\times\Z$ is not cyclic.
  \end{proof}
\item $\Z\times\Z$
  \begin{proof}
    Any generator for $\Z\times\Z$ must have the form $(\pm1,\pm1)$
    since there is no other way to generate all of the integers in
    each component. But every element in a subgroup generated by
    $(\pm1,\pm1)$ must have components which differ only in sign. For
    example, none of these elements will generate $(1,2)$. Therefore
    $\Z\times\Z$ is not cyclic.
  \end{proof}
\end{enumerate}

\Exercise{13} Prove that the following pairs of groups are {\em not}
isomorphic:
\begin{enumerate}
\item $\Z\times Z_2$ and $\Z$
  \begin{proof}
    Isomorphisms preserve order of elements, so $\Z\times Z_2$ cannot
    be isomorphic to $\Z$ since the element $(0,x)$ in $\Z\times Z_2$
    has order $2$ but no element in $\Z$ has order $2$.
  \end{proof}
\item $\Q\times Z_2$ and $\Q$
  \begin{proof}
    Again, $\Q$ has no elements of order $2$, but $\ord{(0,x)} = 2$ in
    $\Q\times Z_2$. Hence the two groups are not isomorphic.
  \end{proof}
\end{enumerate}

\Exercise{14} Let $\sigma =
(1\;2\;3\;4\;5\;6\;7\;8\;9\;10\;11\;12)$. For each of the following
integers $a$ compute $\sigma^a$:
$a = 13, 65, 626, 1195, -6, -81, -570$ and $-1211$.
\begin{solution}
  Since $\ord{\sigma} = 12$, the powers of $\sigma$ consist of exactly
  $12$ distinct elements. We can use the Division Algorithm to reduce
  arbitrary powers to their least residues. For example,
  \begin{equation*}
    626 = 52(12) + 2,
  \end{equation*}
  so $\sigma^{626} = (\sigma^{12})^{52}\sigma^2 = \sigma^2$. Applying
  this process for each of the given values produces the following
  permutations:
  \begin{align*}
    \sigma^{13} = \sigma^{1(12) + 1} = \sigma
    &= (1\;2\;3\;4\;5\;6\;7\;8\;9\;10\;11\;12) \\
    \sigma^{65} = \sigma^{5(12) + 5} = \sigma^5
    &= (1\;6\;11\;4\;9\;2\;7\;12\;5\;10\;3\;8) \\
    \sigma^{626} = \sigma^{52(12)+2} = \sigma^2
    &= (1\;3\;5\;7\;9\;11)(2\;4\;6\;8\;10\;12) \\
    \sigma^{1195} = \sigma^{99(12)+7} = \sigma^7
    &= (1\;8\;3\;10\;5\;12\;7\;2\;9\;4\;11\;6) \\
    \sigma^{-6} = \sigma^{-1(12)+6} = \sigma^6
    &= (1\;7)(2\;8)(3\;9)(4\;10)(5\;11)(6\;12) \\
    \sigma^{-81} = \sigma^{-7(12)+3} = \sigma^3
    &= (1\;4\;7\;10)(2\;5\;8\;11)(3\;6\;9\;12) \\
    \sigma^{-570} = \sigma^{-48(12)+6} = \sigma^6
    &= (1\;7)(2\;8)(3\;9)(4\;10)(5\;11)(6\;12) \\
    \sigma^{-1211} = \sigma^{-101(12)+1} = \sigma
    &= (1\;2\;3\;4\;5\;6\;7\;8\;9\;10\;11\;12). \qedhere
  \end{align*}
\end{solution}

\Exercise{15} Prove that $\Q\times\Q$ is not cyclic.
\begin{proof}
  In Exercise~\ref{exercise-prove-groups-not-cyclic} we showed that
  $\Z\times\Z$ is not cyclic. But $\Z\times\Z$ is a subgroup of
  $\Q\times\Q$, and a cyclic group cannot have a non-cyclic
  subgroup. Therefore $\Q\times\Q$ is not cyclic.
\end{proof}

\Exercise{16} Assume $\ord{x} = n$ and $\ord{y} = m$. Suppose that $x$
and $y$ {\em commute}: $xy = yx$. Prove that $\ord{xy}$ divides the
least common multiple of $m$ and $n$. Need this be true if $x$ and $y$
do {\em not} commute? Give an example of commuting elements $x,y$ such
that the order of $xy$ is not equal to the least common multiple of
$\ord{x}$ and $\ord{y}$.
\begin{solution}
  Let $\ell$ be the least common multiple of $m$ and $n$. Then there
  are integers $a$ and $b$ such that $am = \ell$ and $bn = \ell$. So
  if $\ord{x} = m$ and $\ord{y} = n$ for commuting elements $x$ and
  $y$, then
  \begin{equation*}
    (xy)^\ell = x^\ell y^\ell = (x^m)^a(y^n)^b = 1.
  \end{equation*}
  Therefore $\ord{xy}$ must divide $\ell$ by Proposition~3, which
  completes the proof.

  We note that this need not be true if $x$ and $y$ do not
  commute. For example, in the symmetric group $S_3$,
  $\ord{(1\,2)} = \ord{(2\,3)} = 2$ but $(1\,2)(2\,3) = (1\,2\,3)$
  which has order $3$. Clearly $3\nmid2$.

  Finally, for an example where $x$ and $y$ commute but the order of
  $xy$ does not equal the least common multiple of $\ord{x}$ and
  $\ord{y}$, consider the cyclic group $Z_{10}$. This group is abelian
  so all elements commute, and we have $\ord{x^2} = 5$ and
  $\ord{x^3} = 10$, but $\ord{x^5} = 2 \neq 10$.
\end{solution}

\Exercise{17} Find a presentation for $Z_n$ with one generator.
\begin{solution}
  We know that $Z_n$ can be generated by a single element $x$ which
  satisfies the one relation $x^n = 1$. Moreover, any group generated
  by a single element and satisfying only this relation must be
  isomorphic to $Z_n$, since all cyclic groups of the same order are
  isomorphic. So one possible presentation is
  \begin{equation*}
    Z_n = \gen{x\mid x^n = 1}.\qedhere
  \end{equation*}
\end{solution}

\Exercise{18} Show that if $H$ is any group and $h$ is an element of
$H$ with $h^n = 1$, then there is a unique homomorphism from
$Z_n = \gen{x}$ to $H$ such that $x\mapsto h$.
\begin{proof}
  Define the function $\varphi\colon Z_n\to H$ by
  \begin{equation*}
    \varphi(x^n) = h^n.
  \end{equation*}
  First we need to show that $\varphi$ is well defined. Suppose
  $x^a = x^b$. Then $x^{a-b} = 1$ so $n\mid(a-b)$ (Proposition~3). So
  there is an integer $c$ such that $cn = a - b$. We may then write
  $a = cn + b$ so that
  \begin{equation*}
    \varphi(x^a) = h^a = h^{cn+b} = (h^n)^ch^b = h^b = \varphi(x^b)
  \end{equation*}
  as required.

  To show that $\varphi$ is a homomorphism, consider two arbitrary
  elements $y = x^k$ and $z = x^\ell$ in $Z_n$. By the exponent rules
  established in Exercise~\ref{exercise-exponent-rules}, we have
  \begin{equation*}
    \varphi(yz) = \varphi(x^kx^\ell) = \varphi(x^{k+\ell})
    = h^{k+\ell} = h^kh^\ell = \varphi(x^k)\varphi(x^\ell)
    = \varphi(y)\varphi(z),
  \end{equation*}
  so $\varphi$ is indeed a homomorphism.

  Lastly, to show uniqueness, suppose $\psi\colon Z_n\to H$ is any
  homomorphism such that $\psi(x) = h$. Then for any integer $k$, we
  wish to show that we must have
  \begin{equation*}
    \psi(x^k) = h^k.
  \end{equation*}
  Note that we only need to consider $0\leq k\leq n-1$ since any other
  power is equal to one of these. We now proceed by induction on
  $k$. $\psi(x) = h$ by assumption, so the base case is
  satisfied. Suppose $\psi(x^k) = h^k$ for some nonnegative integer
  $k$. Then by the definition of a homomorphism and by the inductive
  hypothesis,
  \begin{equation*}
    \psi(x^{k+1}) = \psi(x^kx)
    = \psi(x^k)\psi(x) = h^kh = h^{k+1},
  \end{equation*}
  which establishes that $\varphi = \psi$ and thus completes the
  proof.
\end{proof}

\Exercise{19} Show that if $H$ is any group and $h$ is an element of
$H$, then there is a unique homomorphism from $\Z$ to $H$ such that
$1\mapsto h$.
\begin{proof}
  Let $\varphi\colon\Z\to H$ be given by
  \begin{equation*}
    \varphi(n) = h^n.
  \end{equation*}
  Then $\varphi$ is a function which maps $1$ to $h$. It is also a
  homomorphism, since for any integers $m$ and $n$,
  \begin{equation*}
    \varphi(m + n) = h^{m+n} = h^mh^n = \varphi(m)\varphi(n).
  \end{equation*}
  Finally, this homomorphism is uniquely determined because any
  homomorphism $\psi\colon\Z\to H$ such that $\psi(1) = h$ must
  satisfy $\psi(n) = \psi(n1) = \psi(1)^n = h^n$.
\end{proof}

\Exercise{20} Let $p$ be a prime and let $n$ be a positive
integer. Show that if $x$ is an element of the group $G$ such that
$x^{p^n} = 1$ then $\ord{x} = p^m$ for some $m\leq n$.
\begin{proof}
  If $x^{p^n} = 1$, then by Proposition~3 we have $\ord{x}$ divides
  $p^n$. But the only integers that divide a prime power $p^n$ are
  smaller prime powers $p^m$ (including $p^0 = 1$). Therefore
  $\ord{x} = p^m$ for some nonnegative integer $m$ with $m\leq n$.
\end{proof}

% \Exercise{21} Let $p$ be an odd prime and let $n$ be a positive
% integer. Use the Binomial Theorem to show that
% $(1 + p)^{p^{n-1}}\equiv1\pmod{p^n}$ but
% $(1 + p)^{p^{n-2}}\not\equiv1\pmod{p^n}$. Deduce that $1 + p$ is an
% element of order $p^{n-1}$ in the multiplicative group
% $(\Z/p^n\Z)^\times$.
% \begin{proof}
%   Consider the binomial coefficient
%   \begin{equation*}
%     \binom{p^{n-1}}k = \frac{p^{n-1}(p^{n-1}-1)\cdots(p^{n-1}-k+1)}{k!}.
%   \end{equation*}
% \end{proof}

% \Exercise{22} Let $n$ be an integer $\geq3$. Use the Binomial Theorem
% to show that $(1 + 2^2)^{2^{n-2}}\equiv1\pmod{2^n}$ but
% $(1 + 2^2)^{2^{n-3}}\not\equiv1\pmod{2^n}$. Deduce that $5$ is an
% element of order $2^{n-2}$ in the multiplicative group
% $(\Z/2^n\Z)^\times$.

\Exercise{23} Show that $(\Z/2^n\Z)^\times$ is not cyclic for any
$n\geq3$.
\begin{proof}
  By Theorem~7 we know that if $(\Z/2^n\Z)^\times$ is cyclic, then it
  must have at most one subgroup with order $2$. Therefore the proof
  will be complete if we can show that $(\Z/2^n\Z)^\times$ has more
  than one distinct subgroup of order $2$. This is equivalent to
  showing that the group has more than one element with order $2$.

  For $n\geq3$, we have
  \begin{equation*}
    (2^n-1)^2 \equiv (-1)^2 \equiv 1 \pmod{2^n}
  \end{equation*}
  and
  \begin{align*}
    (2^{n-1}-1)^2
    &= 2^{2n-2} - 2^n + 1 \\
    &= 2^{n-2}2^n - 2^n + 1 \\
    &\equiv 1 \pmod{2^n}.
  \end{align*}
  As long as $n\geq3$, both of the elements $\overline{2^n-1}$ and
  $\overline{2^{n-1}-1}$ are distinct from $\bar1$ and hence have
  order $2$. And the two elements are distinct, since
  $2^{n-1}\not\equiv2^n\pmod{2^n}$. Hence $(\Z/2^n\Z)^\times$ cannot
  be cyclic.
\end{proof}

\Exercise{24} Let $G$ be a finite group and let $x\in G$.
\begin{enumerate}
\item Prove that if $g\in N_G(\gen{x})$ then $gxg^{-1} = x^a$ for some
  $a\in\Z$.
  \begin{proof}
    If $g\in N_G(\gen{x})$ then $g\gen{x}g^{-1} = \gen{x}$. Since
    $gxg^{-1}\in g\gen{x}g^{-1}$, we must have $gxg^{-1}\in\gen{x}$ or
    $gxg^{-1} = x^a$ for some integer $a$.
  \end{proof}
\item Prove conversely that if $gxg^{-1} = x^a$ for some $a\in\Z$ then
  $g\in N_G(\gen{x})$.
  \begin{proof}
    Suppose $gxg^{-1} = x^a$ for an integer $a$.

    First we will show that
    \begin{equation}
      \label{eq:conjugate-of-power-is-power-of-conjugate}
      gx^kg^{-1} = (gxg^{-1})^k
      \quad\text{for all $k\in\Z$}.
    \end{equation}
    This obviously holds for $k = 0$, so suppose $k\neq0$. Since
    $gx^{-1}g^{-1} = (gxg^{-1})^{-1}$, it is sufficient to show that
    \eqref{eq:conjugate-of-power-is-power-of-conjugate} is true for
    positive integers $k$.

    We proceed by induction on $k$. The base case is trivial. Suppose
    $gx^kg^{-1} = (gxg^{-1})^k$ for some positive integer $k$. Then
    \begin{align*}
      gx^{k+1}g^{-1}
      &= gx^kxg^{-1} = gx^k(g^{-1}g)xg^{-1} \\
      &= (gx^kg^{-1})(gxg^{-1}) = (gxg^{-1})^{k+1},
    \end{align*}
    so \eqref{eq:conjugate-of-power-is-power-of-conjugate} holds for
    all integers $k$.

    Now suppose $y\in g\gen{x}g^{-1}$. Then there is $k\in\Z$ such
    that $y = gx^kg^{-1}$. From the preceding paragraph, we then have
    $y = (gxg^{-1})^k = x^{ak}$. Therefore $y\in\gen{x}$ so that
    $g\gen{x}g^{-1}\leq\gen{x}$.

    But we know that $\ord{g\gen{x}g^{-1}} = \ord{\gen{x}}$ by
    Exercise~\ref{exercise-conjugation-is-automorphism}. Since $x$ has
    finite order ($G$ is finite), it follows that
    $g\gen{x}g^{-1} = \gen{x}$ and $g\in N_G(\gen{x})$.
  \end{proof}
\end{enumerate}

\Exercise{25} Let $G$ be a cyclic group of order $n$ and let $k$ be an
integer relatively prime to $n$. Prove that the map $x\mapsto x^k$ is
surjective. Use Lagrange's Theorem to prove the same is true for any
finite group of order $n$.
\begin{proof}
  We will prove the general result directly. Fix an integer $k$
  relatively prime to $n$ and let $\varphi$ denote the map
  $x\mapsto x^k$ for the group $G$, where $\ord{G} = n$. Since
  $(n,k) = 1$, we may find $a,b\in\Z$ such that
  \begin{equation*}
    ak + bn = 1.
  \end{equation*}

  Now let $g\in G$ be arbitrary and consider the image of $g^a$ under
  $\varphi$. We have
  \begin{equation*}
    \varphi(g^a) = g^{ak} = g^{1 - bn} = g(g^n)^{-b}.
  \end{equation*}
  We know by Langrange's Theorem that $\ord{g}$ divides $n$ (since
  $\gen{g}$ is a cyclic subgroup of order $\ord{g}$), so $g^n =
  1$. We then have
  \begin{equation*}
    \varphi(g^a) = g,
  \end{equation*}
  which completes the proof that $\varphi$ is surjective.
\end{proof}
