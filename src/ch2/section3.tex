\section{Cyclic Groups and Cyclic Subgroups}

\Exercise1 Find all subgroups of $Z_{45} = \gen{x}$, giving a
generator for each. Describe the containments between these subgroups.
\begin{solution}
  The subgroups are generated by $x^d$ where $d$ divides $45$. And we
  have $\gen{x^a} \leq \gen{x^b}$ if $(b,45) \mid (a,45)$. This gives
  the following subgroup relationships:
  \begin{align*}
    Z_{45} = \gen{x} &> \gen{x^3}, \gen{x^5},
                       \gen{x^9}, \gen{x^{15}}, 1, \\
    \gen{x^3} &> \gen{x^9}, \gen{x^{15}}, 1, \\
    \gen{x^5} &> \gen{x^{15}}, 1, \\
    \gen{x^9} &> 1, \\
    \gen{x^{15}} &> 1, \\
    1 = \gen{x^0}\rlap. & \qedhere
  \end{align*}
\end{solution}

\Exercise2 If $x$ is an element of the finite group $G$ and
$\ord{x} = \ord{G}$, prove that $G = \gen{x}$. Give an explicit
example to show that this result need not be true if $G$ is an
infinite group.
\begin{proof}
  Let $x\in G$ where $\ord{x} = \ord{G} = n < \infty$. By
  Proposition~2, we know that $1, x, x^2, \dots, x^{n-1}$ are all
  distinct elements in $G$. But $G$ contains only $n$ elements, so
  this must be the entirety of $G$. Therefore $G = \gen{x}$.

  This is not always true if $\ord{x} = \ord{G} = \infty$. For
  example, in the additive group $\Z$, $\ord{2} = \ord{\Z} = \infty$
  but clearly $\Z$ is not generated by $2$.
\end{proof}

\Exercise3 Find all generators for $\Z/48\Z$.
\begin{solution}
  The generators are those residue classes whose representatives are
  relatively prime to $48$. Therefore the generators are $\bar1$,
  $\bar5$, $\bar7$, $\overline{11}$, $\overline{13}$, $\overline{17}$,
  $\overline{19}$, $\overline{23}$, $\overline{25}$, $\overline{29}$,
  $\overline{31}$, $\overline{35}$, $\overline{37}$, $\overline{41}$,
  $\overline{43}$, and $\overline{47}$.
\end{solution}

\Exercise4 Find all generators for $\Z/202\Z$.
\begin{solution}
  $202 = 2\cdot101$, so the generators are all residue classes having
  odd representatives excluding $\overline{101}$.
\end{solution}

\Exercise5 Find the number of generators for $\Z/49000\Z$.
\begin{solution}
  If $\varphi$ denotes the Euler $\varphi$-function, then the number
  of generators is given by
  \begin{align*}
    \varphi(49000)
    &= \varphi(2^3)\varphi(5^3)\varphi(7^2) \\
    &= 2^2(2 - 1)5^2(5 - 1)7(7 - 1) \\
    &= 4\cdot100\cdot42 \\
    &= 16800. \qedhere
  \end{align*}
\end{solution}
