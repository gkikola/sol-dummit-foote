\section{Cyclic Groups and Cyclic Subgroups}

\Exercise1 Find all subgroups of $Z_{45} = \gen{x}$, giving a
generator for each. Describe the containments between these subgroups.
\begin{solution}
  The subgroups are generated by $x^d$ where $d$ divides $45$. And we
  have $\gen{x^a} \leq \gen{x^b}$ if $(b,45) \mid (a,45)$. This gives
  the following subgroup relationships:
  \begin{align*}
    Z_{45} = \gen{x} &> \gen{x^3}, \gen{x^5},
                       \gen{x^9}, \gen{x^{15}}, 1, \\
    \gen{x^3} &> \gen{x^9}, \gen{x^{15}}, 1, \\
    \gen{x^5} &> \gen{x^{15}}, 1, \\
    \gen{x^9} &> 1, \\
    \gen{x^{15}} &> 1, \\
    1 = \gen{x^0}\rlap. & \qedhere
  \end{align*}
\end{solution}

\Exercise2 If $x$ is an element of the finite group $G$ and
$\ord{x} = \ord{G}$, prove that $G = \gen{x}$. Give an explicit
example to show that this result need not be true if $G$ is an
infinite group.
\begin{proof}
  Let $x\in G$ where $\ord{x} = \ord{G} = n < \infty$. By
  Proposition~2, we know that $1, x, x^2, \dots, x^{n-1}$ are all
  distinct elements in $G$. But $G$ contains only $n$ elements, so
  this must be the entirety of $G$. Therefore $G = \gen{x}$.

  This is not always true if $\ord{x} = \ord{G} = \infty$. For
  example, in the additive group $\Z$, $\ord{2} = \ord{\Z} = \infty$
  but clearly $\Z$ is not generated by $2$.
\end{proof}

\Exercise3 Find all generators for $\Z/48\Z$.
\begin{solution}
  The generators are those residue classes whose representatives are
  relatively prime to $48$. Therefore the generators are $\bar1$,
  $\bar5$, $\bar7$, $\overline{11}$, $\overline{13}$, $\overline{17}$,
  $\overline{19}$, $\overline{23}$, $\overline{25}$, $\overline{29}$,
  $\overline{31}$, $\overline{35}$, $\overline{37}$, $\overline{41}$,
  $\overline{43}$, and $\overline{47}$.
\end{solution}

\Exercise4 Find all generators for $\Z/202\Z$.
\begin{solution}
  $202 = 2\cdot101$, so the generators are all residue classes having
  odd representatives excluding $\overline{101}$.
\end{solution}

\Exercise5 Find the number of generators for $\Z/49000\Z$.
\begin{solution}
  If $\varphi$ denotes the Euler $\varphi$-function, then the number
  of generators is given by
  \begin{align*}
    \varphi(49000)
    &= \varphi(2^3)\varphi(5^3)\varphi(7^2) \\
    &= 2^2(2 - 1)5^2(5 - 1)7(7 - 1) \\
    &= 4\cdot100\cdot42 \\
    &= 16800. \qedhere
  \end{align*}
\end{solution}

\Exercise6 In $\Z/48\Z$ write out all elements of $\gen{\bar{a}}$ for
every $\bar a$. Find all inclusions between subgroups in $\Z/48\Z$.
\begin{solution}
  The elements of each subgroup are
  \begin{align*}
    \Z/48\Z = \gen{\bar1}
    &= \{\bar0, \bar1, \bar2, \bar3, \dots, \overline{46},
      \overline{47}\}, \\
    \gen{\bar2} &= \{\bar0, \bar2, \bar4, \bar6, \dots,
                  \overline{44}, \overline{46}\}, \\
    \gen{\bar3} &= \{\bar0, \bar3, \bar6, \bar9, \dots,
                  \overline{42}, \overline{45}\}, \\
    \gen{\bar4} &= \{\bar0, \bar4, \bar8, \overline{12}, \dots,
                  \overline{40}, \overline{44}\}, \\
    \gen{\bar6} &= \{\bar0, \bar6, \overline{12}, \overline{18},
                  \overline{24}, \overline{30}, \overline{36},
                  \overline{42}\}, \\
    \gen{\bar8} &= \{\bar0, \bar8, \overline{16}, \overline{24},
                  \overline{32}, \overline{40}\}, \\
    \gen{\overline{12}} &= \{\bar0, \overline{12}, \overline{24},
                          \overline{36}\}, \\
    \gen{\overline{16}} &= \{\bar0, \overline{16}, \overline{32}\}, \\
    \gen{\overline{24}} &= \{\bar0, \overline{24}\}, \\
    \gen{\bar0} &= \{\bar0\}.
  \end{align*}
  And we have the following inclusions:
  \begin{align*}
    \gen{\bar0}, \gen{\bar1}, \gen{\bar2}, \gen{\bar3}, \gen{\bar4},
    \gen{\bar6}, \gen{\bar8}, \gen{\overline{12}}, \gen{\overline{16}},
    \gen{\overline{24}}
    &\leq \gen{\bar1}, \\
    \gen{\bar0}, \gen{\bar2}, \gen{\bar4}, \gen{\bar6}, \gen{\bar8},
    \gen{\overline{12}}, \gen{\overline{16}}, \gen{\overline{24}}
    &\leq \gen{\bar2}, \\
    \gen{\bar0}, \gen{\bar3}, \gen{\bar6}, \gen{\overline{12}},
    \gen{\overline{16}}, \gen{\overline{24}}
    &\leq \gen{\bar3}, \\
    \gen{\bar0}, \gen{\bar4}, \gen{\bar8}, \gen{\overline{12}},
    \gen{\overline{16}}, \gen{\overline{24}}
    &\leq \gen{\bar4}, \\
    \gen{\bar0}, \gen{\bar6}, \gen{\overline{12}}, \gen{\overline{24}}
    &\leq \gen{\bar6}, \\
    \gen{\bar0}, \gen{\bar8}, \gen{\overline{16}}, \gen{\overline{24}}
    &\leq \gen{\bar8}, \\
    \gen{\bar0}, \gen{\overline{12}}, \gen{\overline{24}}
    &\leq \gen{\overline{12}}, \\
    \gen{\bar0}, \gen{\overline{16}} &\leq \gen{\overline{16}}, \\
    \gen{\bar0}, \gen{\overline{24}} &\leq \gen{\overline{24}}, \\
    \gen{\bar0} &\leq \gen{\bar0}. \qedhere
  \end{align*}
\end{solution}

\Exercise7 Let $Z_{48} = \gen{x}$ and use the isomorphism
$\Z/48\Z\cong Z_{48}$ given by $\bar1\mapsto x$ to list all subgroups
of $Z_{48}$ as computed in the preceding exercise.
\begin{solution}
  The subgroups are $\gen{x}$, $\gen{x^2}$, $\gen{x^3}$, $\gen{x^4}$,
  $\gen{x^6}$, $\gen{x^8}$, $\gen{x^{12}}$, $\gen{x^{16}}$,
  $\gen{x^{24}}$, and $1$.
\end{solution}

\Exercise8 Let $Z_{48} = \gen{x}$. For which integers $a$ does the map
$\varphi_a$ defined by $\varphi_a\colon\bar1\mapsto x^a$ extend to an
{\em isomorphism} from $\Z/48\Z$ onto $Z_{48}$.
\begin{solution}
  Choose an $a$ with $(a,48) = d > 1$ and set $b = 48 / d$. If
  $\varphi_a$ is a homomorphism, then
  \begin{equation*}
    \varphi_a(\bar{b}) = \varphi_a(b\cdot\bar1)
    = \varphi_a(\bar1)^b = x^{ab} = (x^{48})^{a/d} = 1 = \varphi_a(\bar0).
  \end{equation*}
  Therefore, in this case, $\varphi_a$ is not an injection and thus
  not an isomorphism.

  This suggests that $\varphi_a$ extends to an isomorphism if and only
  if $(a,48) = 1$, which we will now prove. First we show that the
  function $\bar{b}\mapsto x^{ab}$ is well defined, i.e., that the
  value of the function is not affected by the choice of
  representative for $\bar{b}$. Suppose $\bar{b} = \bar{c}$. Then
  $48k = b - c$ for some integer $k$, and we have
  \begin{equation*}
    \varphi_a(\bar{b}) = x^{ab} = x^{a(48k + c)} = (x^{48})^{ak}x^{ac}
    = 1^{ak}x^{ac} = x^{ac} = \varphi_a(\bar{c}).
  \end{equation*}

  Now, $\varphi_a$ is certainly a homomorphism, since
  \begin{equation*}
    \varphi_a(\bar{b} + \bar{c}) = x^{a(b + c)} = x^{ab}x^{ac} =
    \varphi_a(\bar{b})\varphi_a(\bar{c}).
  \end{equation*}
  To show injectivity, suppose
  $\varphi_a(\bar{b}) = \varphi_a(\bar{c})$. Then $x^{ab} = x^{ac}$ or
  $x^{a(b - c)} = 1$. Hence $48\mid a(b - c)$ and since $(a,48) = 1$
  we have $48\mid(b - c)$. This shows that $\bar{b} = \bar{c}$.

  Finally, since $\ord{\Z/48\Z} = \ord{Z_{48}} < \infty$, we know that
  injectivity of $\varphi_a$ implies surjectivity, so that $\varphi_a$
  is an isomorphism.
\end{solution}

\Exercise9 Let $Z_{36} = \gen{x}$. For which integers $a$ does the map
$\psi_a$ defined by $\psi_a\colon\bar1\mapsto x^a$ extend to a {\em
  well defined homomorphism} from $\Z/48\Z$ into $Z_{36}$. Can
$\psi_a$ ever be a surjective homomorphism?
\begin{solution}
  Suppose $\bar{b} = \bar{c}$ for integers $b$ and $c$. If $\psi_a$ is
  well defined then $\psi_a(\bar{b}) = \psi_a(\bar{c})$, that is,
  $x^{ab} = x^{ac}$. Then $x^{a(b - c)} = 1$ so we must have
  $36\mid a(b-c)$. But $48\mid(b - c)$, so there is an integer $k$ for
  which $48k = b - c$ and we have that $36\mid48ak$. If we choose
  $\bar{b}$ and $\bar{c}$ so that $k = 1$, then we must have $3\mid a$
  as a necessary condition for $36\mid48ak$. It is also sufficient
  that $3\mid a$, since $36\mid144mk$.

  Since
  \begin{equation*}
    \psi_a(\bar{b} + \bar{c}) = x^{a(b + c)}
    = x^{ab}x^{ac} = \psi_a(\bar{b})\psi_a(\bar{c}),
  \end{equation*}
  we see that $\psi_a$ is a well defined homomorphism if and only if
  $3\mid a$.

  Lastly, suppose $\psi_a(\bar{b}) = x$. Since $a = 3k$ for some
  integer $k$, we have
  \begin{equation*}
    x = \psi_a(\bar{b}) = x^{ab} = x^{3kb}.
  \end{equation*}
  Therefore $x^{3kb - 1} = 1$ and we see that $36$ divides $3kb -
  1$. But this is impossible since if $36m = 3kb - 1$ then
  $1 = 3kb - 36m = 3(kb - 12m)$ and $3\mid1$, a contradiction. So the
  homomorphism $\psi_a$ can never be surjective.
\end{solution}

\Exercise{10} What is the order of $\overline{30}$ in $\Z/54\Z$? Write
out all the elements and their orders in $\gen{\overline{30}}$.
\begin{solution}
  Since $\ord{\bar1} = 54$, by Proposition~5~(2) we have
  \begin{equation*}
    \ord{\overline{30}} = \ord{30\cdot\bar1} = \frac{54}{(30,54)}
    = \frac{54}{6} = 9.
  \end{equation*}
  Then
  \begin{equation*}
    \gen{\overline{30}} = \{\bar0, \bar6, \overline{12},
    \overline{18}, \overline{24}, \overline{30}, \overline{36},
    \overline{42}, \overline{48}\}
  \end{equation*}
  where
  \begin{align*}
    \ord{\bar0} &= 1, \\
    \ord{\bar6} &= 9, \\
    \ord{\overline{12}} &= 9, \\
    \ord{\overline{18}} &= 3, \\
    \ord{\overline{24}} &= 9, \\
    \ord{\overline{30}} &= 9, \\
    \ord{\overline{36}} &= 3, \\
    \ord{\overline{42}} &= 9, \\
    \ord{\overline{48}} &= 9. \qedhere
  \end{align*}
\end{solution}
