\chapter{Subgroups}

\section{Definition and Examples}

Let $G$ be a group.

\Exercise1 In each of (a)--(e) prove that the specified subset is a
subgroup of the given group.
\begin{enumerate}
\item the set of complex numbers of the form $a + ai$, $a\in\R$ (under
  addition)
  \begin{proof}
    Call the set $G$. $G$ is obviously nonempty. For any $a + ai$ and
    $b + bi$ in $G$, we have
    \begin{equation*}
      (a + ai) - (b + bi) = (a - b) + (a - b)i\in G,
    \end{equation*}
    so by Proposition~1, $G\leq\C$.
  \end{proof}
\item the set of complex numbers of absolute value $1$, i.e., the unit
  circle in the complex plane (under multiplication)
  \begin{proof}
    Let $G$ denote the complex numbers of absolute value $1$, and let
    $\overline{z}$ denote the conjugate of $z$. Then $G$ is nonempty
    and for any $z,w\in G$, we have
    \begin{equation*}
      \abs{zw^{-1}} = \abs{z}\abs{w^{-1}}
      = \abs{z}\frac{\abs{\overline{w}}}{\abs{w}^2} = 1,
    \end{equation*}
    so $zw^{-1}\in G$. Therefore $G\leq\C^\times$.
  \end{proof}
\item for fixed $n\in\Z^+$ the set of rational numbers whose
  denominators divide $n$ (under addition)
  \begin{proof}
    Let $G$ denote the subset in question. $G$ is clearly not
    empty. Let $a,b\in G$ be arbitrary. Then there is $x,y\in\Z$ and
    $k,\ell\in\Z^+$ so that $a = x/(kn)$ and $b = y/(\ell n)$. Then we
    have
    \begin{equation*}
      a - b = \frac{x}{kn} - \frac{y}{\ell n}
      = \frac{\ell x - ky}{\ell kn} = \frac{z}{mn}
    \end{equation*}
    where $z\in\Z$ and $m\in\Z^+$. Therefore $a - b\in G$ so that
    $G\leq\Q$.
  \end{proof}
\item for fixed $n\in\Z^+$ the set of rational numbers whose
  denominators are relatively prime to $n$ (under addition)
  \begin{proof}
    Again, let $G$ denote the subset, which is clearly nonempty. Take
    $a/b$ and $c/d$ in $G$, so that $(b,n) = (d,n) = 1$. Then
    \begin{equation*}
      \frac{a}b - \frac{c}d = \frac{ad - bc}{bd}.
    \end{equation*}
    Let $k = (bd, n)$. If $k > 1$, then there is a prime number $m$
    which divides $k$. Then $m\mid bd$ which implies $m\mid b$ or
    $m\mid d$, which is impossible since $m\mid n$. Therefore $k = 1$
    and $a/b - c/d \in G$. So $G\leq\Q$.
  \end{proof}
\item the set of nonzero real numbers whose square is a rational
  number (under multiplication)
  \begin{proof}
    Let $G$ be the set in question, which is clearly nonempty. If
    $a,b\in G$, then $a^2,b^2\in\Q$. Then
    \begin{equation*}
      \left(\frac{a}b\right)^2 = \frac{a^2}{b^2} \in \Q,
    \end{equation*}
    so $a/b\in G$. Hence $G\leq\R^\times$.
  \end{proof}
\end{enumerate}

\Exercise2 In each of (a)--(e) prove that the specified subset is {\em
  not} a subgroup of the given group:
\begin{enumerate}
\item the set of $2$-cycles in $S_n$ for $n\geq3$
  \begin{proof}
    For any $n\geq3$, the $2$-cycles $(1\,2)$ and $(2\,3)$ are members
    of $S_n$, yet $(2\,3)(1\,2) = (1\,3\,2)$ which is not a
    $2$-cycle. So this set is not a subgroup.
  \end{proof}
\item the set of reflections in $D_{2n}$ for $n\geq3$
  \begin{proof}
    Since $s$ and $sr$ are reflections, but $s(sr) = r$ is not, this
    set is not closed under the group operation so it is not a
    subgroup.
  \end{proof}
\item for $n$ a composite integer $>1$ and $G$ a group containing an
  element of order $n$, the set $\{x\in G\mid\ord{x}=n\}\cup\{1\}$
  \begin{proof}
    Let $p\mid n$ for $p$ a prime. Then $(x^{n/p})^p = x^n = 1$, so
    $\ord{x^{n/p}}<n$ and the set is not closed under the group
    operation.
  \end{proof}
\item the set of (positive and negative) odd integers in $\Z$ together
  with $0$
  \begin{proof}
    Since $1 + 1 = 2$, this set is not closed under addition and is
    therefore not a subgroup.
  \end{proof}
\item the set of real numbers whose square is a rational number (under
  addition)
  \begin{proof}
    $\sqrt2$ and $\sqrt3$ are in this subset, but $\sqrt2 + \sqrt3$ is
    not, so this cannot be a subgroup.
  \end{proof}
\end{enumerate}
