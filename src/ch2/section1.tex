\chapter{Subgroups}

\section{Definition and Examples}

Let $G$ be a group.

\Exercise1 In each of (a)--(e) prove that the specified subset is a
subgroup of the given group.
\begin{enumerate}
\item the set of complex numbers of the form $a + ai$, $a\in\R$ (under
  addition)
  \begin{proof}
    Call the set $G$. $G$ is obviously nonempty. For any $a + ai$ and
    $b + bi$ in $G$, we have
    \begin{equation*}
      (a + ai) - (b + bi) = (a - b) + (a - b)i\in G,
    \end{equation*}
    so by Proposition~1, $G\leq\C$.
  \end{proof}
\item the set of complex numbers of absolute value $1$, i.e., the unit
  circle in the complex plane (under multiplication)
  \begin{proof}
    Let $G$ denote the complex numbers of absolute value $1$, and let
    $\overline{z}$ denote the conjugate of $z$. Then $G$ is nonempty
    and for any $z,w\in G$, we have
    \begin{equation*}
      \abs{zw^{-1}} = \abs{z}\abs{w^{-1}}
      = \abs{z}\frac{\abs{\overline{w}}}{\abs{w}^2} = 1,
    \end{equation*}
    so $zw^{-1}\in G$. Therefore $G\leq\C^\times$.
  \end{proof}
\item for fixed $n\in\Z^+$ the set of rational numbers whose
  denominators divide $n$ (under addition)
  \begin{proof}
    Let $G$ denote the subset in question. $G$ is clearly not
    empty. Let $a,b\in G$ be arbitrary. Then there is $x,y\in\Z$ and
    $k,\ell\in\Z^+$ so that $a = x/(kn)$ and $b = y/(\ell n)$. Then we
    have
    \begin{equation*}
      a - b = \frac{x}{kn} - \frac{y}{\ell n}
      = \frac{\ell x - ky}{\ell kn} = \frac{z}{mn}
    \end{equation*}
    where $z\in\Z$ and $m\in\Z^+$. Therefore $a - b\in G$ so that
    $G\leq\Q$.
  \end{proof}
\item for fixed $n\in\Z^+$ the set of rational numbers whose
  denominators are relatively prime to $n$ (under addition)
  \begin{proof}
    Again, let $G$ denote the subset, which is clearly nonempty. Take
    $a/b$ and $c/d$ in $G$, so that $(b,n) = (d,n) = 1$. Then
    \begin{equation*}
      \frac{a}b - \frac{c}d = \frac{ad - bc}{bd}.
    \end{equation*}
    Let $k = (bd, n)$. If $k > 1$, then there is a prime number $m$
    which divides $k$. Then $m\mid bd$ which implies $m\mid b$ or
    $m\mid d$, which is impossible since $m\mid n$. Therefore $k = 1$
    and $a/b - c/d \in G$. So $G\leq\Q$.
  \end{proof}
\item the set of nonzero real numbers whose square is a rational
  number (under multiplication)
  \begin{proof}
    Let $G$ be the set in question, which is clearly nonempty. If
    $a,b\in G$, then $a^2,b^2\in\Q$. Then
    \begin{equation*}
      \left(\frac{a}b\right)^2 = \frac{a^2}{b^2} \in \Q,
    \end{equation*}
    so $a/b\in G$. Hence $G\leq\R^\times$.
  \end{proof}
\end{enumerate}

\Exercise2 In each of (a)--(e) prove that the specified subset is {\em
  not} a subgroup of the given group:
\begin{enumerate}
\item the set of $2$-cycles in $S_n$ for $n\geq3$
  \begin{proof}
    For any $n\geq3$, the $2$-cycles $(1\,2)$ and $(2\,3)$ are members
    of $S_n$, yet $(2\,3)(1\,2) = (1\,3\,2)$ which is not a
    $2$-cycle. So this set is not a subgroup.
  \end{proof}
\item the set of reflections in $D_{2n}$ for $n\geq3$
  \begin{proof}
    Since $s$ and $sr$ are reflections, but $s(sr) = r$ is not, this
    set is not closed under the group operation so it is not a
    subgroup.
  \end{proof}
\item for $n$ a composite integer $>1$ and $G$ a group containing an
  element of order $n$, the set $\{x\in G\mid\ord{x}=n\}\cup\{1\}$
  \begin{proof}
    Let $p\mid n$ for $p$ a prime. Then $(x^{n/p})^p = x^n = 1$, so
    $\ord{x^{n/p}}<n$ and the set is not closed under the group
    operation.
  \end{proof}
\item the set of (positive and negative) odd integers in $\Z$ together
  with $0$
  \begin{proof}
    Since $1 + 1 = 2$, this set is not closed under addition and is
    therefore not a subgroup.
  \end{proof}
\item the set of real numbers whose square is a rational number (under
  addition)
  \begin{proof}
    $\sqrt2$ and $\sqrt3$ are in this subset, but $\sqrt2 + \sqrt3$ is
    not, so this cannot be a subgroup.
  \end{proof}
\end{enumerate}

\Exercise3 Show that the following subsets of the dihedral group $D_8$
are actually subgroups:
\begin{enumerate}
\item $\{1,r^2,s,sr^2\}$
  \begin{proof}
    This is a finite group so it suffices to show that it is closed
    under the group operation of composition. We have
    \begin{align*}
      r^2(r^2) &= 1, \\
      r^2(s) &= sr^2, \\
      r^2(sr^2) &= s, \\
      s(r^2) &= sr^2, \\
      s^2 &= 1, \\
      s(sr^2) &= r^2, \\
      sr^2(r^2) &= s, \\
      sr^2(s) &= r^2, \\
      sr^2(sr^2) &= 1.
    \end{align*}
    Therefore this subset is a subgroup.
  \end{proof}
\item $\{1,r^2,sr,sr^3\}$
  \begin{proof}
    Again, we can simply enumerate the possibilities. We find that
    \begin{equation*}
      r^2(r^2) = sr(sr) = sr^3(sr^3) = 1,
    \end{equation*}
    \begin{equation*}
      r^2(sr) = sr(r^2) = sr^3,
    \end{equation*}
    \begin{equation*}
      r^2(sr^3) = sr^3(r^2) = sr,
    \end{equation*}
    and
    \begin{equation*}
      sr(sr^3) = sr^3(sr) = r^2.
    \end{equation*}
    Therefore this is a subgroup.
  \end{proof}
\end{enumerate}

\Exercise4 Give an explicit example of a group $G$ and an infinite
subset $H$ of $G$ that is closed under the group operation but is not
a subgroup of $G$.
\begin{solution}
  Let $G = \R^\times$ with the operation of multiplication. Then if
  $H$ is the nonzero integers, $H$ is closed under multiplication but
  is not a subgroup since it is not closed under inverses (for
  example, $2$ has no inverse in $H$).
\end{solution}

\Exercise5 Prove that $G$ cannot have a subgroup $H$ with
$\ord{H} = n - 1$, where $n = \ord{G} > 2$.
\begin{proof}
  If such a subgroup $H$ does exist, then it must exclude exactly one
  element $g$ from $G$. Since $\ord{H}\geq2$, we can take a nonidentity
  element $h\in H$.

  Consider the element $gh$. If $gh\not\in H$, then $gh = g$ and
  cancellation implies that $h$ is the identity, which is a
  contradiction. On the other hand, if $gh\in H$, then
  $(gh)h^{-1} = g\in H$, a contradiction. So the subgroup $H$ does not
  exist.
\end{proof}

\Exercise6 Let $G$ be an abelian group. Prove that
$\{g\in G\mid\ord{g}<\infty\}$ is a subgroup of $G$ (called the {\em
  torsion subgroup} of $G$). Give an explicit example where this set
is not a subgroup when $G$ is non-abelian.
\begin{solution}
  Let $G$ be abelian and let $H$ be the elements of $G$ having finite
  order. $H$ is nonempty since $1\in H$. Suppose $a,b\in H$. Then
  $\ord{a} = m$ and $\ord{b} = n$ for some finite $m$ and $n$. Since
  $G$ is abelian we have
  \begin{equation*}
    (ab^{-1})^{mn} = a^{mn}(b^{mn})^{-1} = 1.
  \end{equation*}
  Therefore $ab^{-1}\in H$ and $H$ is a subgroup of $G$.

  Now, for a non-abelian counterexample, consider the group of
  invertible functions from $\R\to\R$ under function composition. Let
  $f\colon\R\to\R$ and $g\colon\R\to\R$ be given by
  \begin{equation*}
    f(x) = -x
    \quad\text{and}\quad
    g(x) = 1 - x.
  \end{equation*}
  Then $f$ and $g$ have order $2$ but $f\circ g$, given by
  $x\mapsto x - 1$, has infinite order.
\end{solution}

\Exercise7 Fix some $n\in\Z$ with $n > 1$. Find the torsion subgroup
of $\Z\times(\Z/n\Z)$. Show that the set of elements of infinite order
together with the identity is {\em not} a subgroup of this direct
product.
\begin{solution}
  Let $G = \Z\times(\Z/n\Z)$ (with componentwise addition) and let $H$
  be the torsion subgroup. Since every nonzero integer has infinite
  order, members of $H$ must have the form $(0, k)$ for
  $k\in\Z/n\Z$. But $\Z/n\Z$ is a finite group, so all of its members
  have finite order. Therefore $H = \{ (0, k) \mid k\in\Z/n\Z \}$. And
  we know that this is a subgroup by the previous exercise.

  Now let $K$ be the set of elements of $G$ having infinite order
  together with the identity. Then $(1, 1)\in K$ and $(-1, 0)\in K$,
  but $(1,1) + (-1,0) = (0, 1)\not\in K$. Therefore $K$ is not a
  subgroup of $G$.
\end{solution}

\Exercise8 Let $H$ and $K$ be subgroups of $G$. Prove that $H\cup K$
is a subgroup if and only if either $H\subseteq K$ or $K\subseteq H$.
\begin{proof}
  Suppose $H\cup K$ is a subgroup of $G$. If $H\subseteq K$ then we
  are done. Suppose $H\not\subseteq K$ so that there is $h\in H$ such
  that $h\not\in K$. Let $k$ be any element in $K$. Since
  $h,k\in H\cup K$, we must have $hk\in H\cup K$. But if $hk\in K$,
  then $hk(k^{-1}) = h\in K$, which contradicts the choice of $h$. So
  $hk\in H$. And $h^{-1}\in H$, so $h^{-1}(hk) = k\in H$. Hence every
  element of $K$ is in $H$ so that $K\subseteq H$.

  Conversely, suppose $H\subseteq K$. Then $H\cup K = K$ is a
  subgroup. Similarly, if $K\subseteq H$, then $H\cup K = H$ is a
  subgroup. This completes the proof.
\end{proof}

\Exercise9 Let $G = GL_n(F)$, where $F$ is any field. Define
\begin{equation*}
  SL_n(F) = \{A\in GL_n(F) \mid \det(A) = 1\}
\end{equation*}
(called the {\em special linear group}). Prove that
$SL_n(F)\leq GL_n(F)$.
\begin{proof}
  First, note that $SL_n(F)$ is nonempty since the identity matrix $I$
  has determinant $1$. Now let $A,B\in SL_n(F)$. We know from linear
  algebra that
  \begin{equation*}
    \det(AB^{-1}) = \det(A)\det(B^{-1}) = \det(A)\det(B)^{-1} = 1,
  \end{equation*}
  so $AB^{-1}\in SL_n(F)$, which shows that $SL_n(F)$ is a subgroup.
\end{proof}

\Exercise{10}
\begin{enumerate}
\item Prove that if $H$ and $K$ are subgroups of $G$ then so is their
  intersection $H\cap K$.
  \begin{proof}
    Since $1\in H$ and $1\in K$, $1\in H\cap K$ and the intersection
    is nonempty. For any $a,b\in H\cap K$, we must have $ab^{-1}\in H$
    since $a,b\in H$ and $H$ is a subgroup. Similarly we must have
    $ab^{-1}\in K$, so $ab^{-1}\in H\cap K$ and $H\cap K\leq G$.
  \end{proof}
\item Prove that the intersection of an arbitrary nonempty collection
  of subgroups of $G$ is again a subgroup of $G$ (do not assume the
  collection is countable).
  \begin{proof}
    Let $H_\alpha$ be a subgroup of $G$ for all $\alpha$ belonging to
    some set of indices $A$. Let
    \begin{equation*}
      H = \bigcap_{\alpha\in A} H_\alpha.
    \end{equation*}
    Then $1\in H$ so $H$ is nonempty. If $a,b\in H$, then for any
    $\alpha$ we have $a,b\in H_\alpha$, so $ab^{-1}\in H_\alpha$ and
    we see that $ab^{-1}\in H$ as well. Hence $H\leq G$.
  \end{proof}
\end{enumerate}

\Exercise{11} Let $A$ and $B$ be groups. Prove that the following sets
are subgroups of the direct product $A\times B$:
\begin{enumerate}
\item $\{(a, 1) \mid a\in A\}$
  \begin{proof}
    Call the set $H$. Then $(1,1)\in H$ so $H$ is nonempty. For any
    $a_1,a_2\in A$ we have $a_1a_2^{-1}\in A$, so
    $(a_1,1)(a_2,1)^{-1} = (a_1a_2^{-1}, 1)\in H$. Therefore
    $H\leq A\times B$.
  \end{proof}
\item $\{(1, b) \mid b\in B\}$
  \begin{proof}
    The proof is almost the same as in part (a): $H$ is nonempty, and
    for any $b_1,b_2\in B$ we have
    $(1,b_1)(1,b_2)^{-1} = (1,b_1b_2^{-1})\in H$, so
    $H\leq A\times B$.
  \end{proof}
\item $\{(a, a) \mid a\in A\}$, where here we assume $B = A$ (called
  the {\em diagonal subgroup})
  \begin{proof}
    Again, call the subset $H$. $(1,1)\in H$ so $H$ is nonempty. For
    any $a_1,a_2\in A$, we have $a_1a_2^{-1}\in A$ so
    $(a_1,a_1)(a_2,a_2)^{-1} = (a_1a_2^{-1}, a_1a_2^{-1})\in
    H$. Therefore $H$ is a subgroup of $A^2$.
  \end{proof}
\end{enumerate}

\Exercise{12} Let $A$ be an abelian group and fix some $n\in\Z$. Prove
that the following sets are subgroups of $A$:
\begin{enumerate}
\item $\{a^n \mid a\in A\}$
  \begin{proof}
    Call the subset $H$. Then $1^n = 1\in H$ so $H$ is nonempty. If
    $a^n,b^n\in H$ then, since $A$ is abelian,
    \begin{equation*}
      a^n(b^n)^{-1} = a^n(b^{-1})^n = (ab^{-1})^n.
    \end{equation*}
    Therefore $a^n(b^n)^{-1}\in H$ and $H\leq A$.
  \end{proof}
\item $\{a\in A\mid a^n = 1\}$
  \begin{proof}
    Again, call the set $H$. Then $1\in H$ so $H$ is nonempty. Suppose
    $a,b\in H$. Then $a^n = 1$ and
    $(b^{-1})^n = (b^n)^{-1} = 1^{-1} = 1$. Since $A$ is abelian, we
    have $(ab^{-1})^n = a^n(b^{-1})^n = 1$ so $ab^{-1}\in
    H$. Therefore $H\leq A$.
  \end{proof}
\end{enumerate}

\Exercise{13} Let $H$ be a subgroup of the additive group of rational
numbers with the property that $1/x\in H$ for every nonzero element
$x$ of $H$. Prove that $H = 0$ or $\Q$.
\begin{proof}
  Suppose $H$ is a subgroup of $\Q$ with the given property. Certainly
  $0\in H$. If $H = 0$ then there is nothing left to prove, so suppose
  $H\neq0$. Then $x\in H$ for some nonzero $x\in\Q$. And we may take
  $x$ to be positive, since if $x < 0$ then $-x > 0$ and $-x\in H$
  since $H$ is closed under additive inverses.

  Write $x = a/b$ for positive integers $a$ and $b$. Since $H$ is
  closed under addition, we have
  \begin{equation*}
    bx = \overbrace{\frac{a}b + \frac{a}b
      + \cdots + \frac{a}b}^{\text{$b$ terms}} = a\in H.
  \end{equation*}
  Also, $a$ is nonzero, so by hypothesis $1/a\in H$. By the same
  reasoning as above, we have $a(1/a) = 1\in H$. And since $H$ is
  closed under addition and inverses, this shows that $\Z\subseteq H$.

  Now, let $r\in\Q$ be arbitrary and write $r = p/q$ for integers $p$
  and $q$ (with $q$ nonzero). Since $q\in H$, we have $1/q\in H$ and
  so $p(1/q) = p/q\in H$. This shows that $\Q\subseteq H$. But
  $H\subseteq\Q$, so $H = \Q$ and the proof is complete.
\end{proof}

\Exercise{14} Show that $\{x\in D_{2n} \mid x^2 = 1\}$ is not a
subgroup of $D_{2n}$ (here $n\geq3$).
\begin{proof}
  In $D_{2n}$, $s^2 = 1$ and $(sr)^2 = srsr = s^2r^{-1}r = 1$, so
  these elements are in the subset. However, their product $s(sr) = r$
  has order $n>2$. So this set is not closed under the group operation
  and thus is not a subgroup.
\end{proof}
