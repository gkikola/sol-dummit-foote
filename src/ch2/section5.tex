\section{The Lattice of Subgroups of a Group}

\Exercise1 Let $H$ and $K$ be subgroups of $G$. Exhibit all possible
sublattices which show only $G$, $1$, $H$, $K$ and their joins and
intersections. What distinguishes the different drawings?
\begin{solution}
  In general, when $H$ and $K$ are distinct, with a nontrivial
  intersection and a join that is a proper subgroup, the sublattice
  might look something like the following:
  \begin{center}
    \begin{tikzpicture}
      \node at (0,3.5) {};
      \node (G) at (0,3) {$G$};
      \node (join) at (0,2) {$\gen{H,K}$};
      \node (H) at (-1,1) {$H$};
      \node (K) at (1,1) {$K$};
      \node (cap) at (0,0) {$H\cap K$};
      \node (1) at (0,-1) {$1$};
      \node at (0,-1.5) {};
      \draw (G) -- (join) -- (H) -- (cap) -- (K) -- (join);
      \draw (cap) -- (1);
    \end{tikzpicture}
  \end{center}
  In other cases, we may have $H = K$, or $H\geq K$:
  \begin{center}
    \begin{tikzpicture}
      \begin{scope}[xshift=-1.5cm]
        \node at (0,3.5) {};
        \node (G) at (0,3) {$G$};
        \node (HK) at (0,1.5){$H = K$};
        \node (1) at (0,0) {$1$};
        \node at (0,-0.5) {};
        \draw (G) -- (HK) -- (1);
      \end{scope}
      \begin{scope}[xshift=1.5cm]
        \node at (0,3.5) {};
        \node (G) at (0,3) {$G$};
        \node (H) at (0,2) {$H$};
        \node (K) at (0,1) {$K$};
        \node (1) at (0,0) {$1$};
        \node at (0,-0.5) {};
        \draw (G) -- (H) -- (K) -- (1);
      \end{scope}
    \end{tikzpicture}
  \end{center}
  There are several other possibilities, for example one of $H$ or $K$
  could be trivial, or we could have $\gen{H,K} = G$, and various
  other options. The drawings are distinguished by the relationships
  between the various subgroups.
\end{solution}

\Exercise2 In each of (a) to (d) list all subgroups of $D_{16}$ that
satisfy the given condition.
\begin{enumerate}
\item Subgroups that are contained in $\gen{sr^2,r^4}$
  \begin{solution}
    From the lattice given in the text, we see that the subgroups
    contained in $\gen{sr^2,r^4}$ are $\gen{sr^2,r^4}$, $\gen{sr^6}$,
    $\gen{sr^2}$, $\gen{r^4}$, and $1$.
  \end{solution}
\item Subgroups that are contained in $\gen{sr^7, r^4}$
  \begin{solution}
    $\gen{sr^7, r^4} = \gen{sr^3, r^4}$, so the subgroups contained in
    this subgroup are $\gen{sr^3, r^4}$, $\gen{r^4}$, $\gen{sr^3}$,
    $\gen{sr^7}$, and $1$.
  \end{solution}
\item Subgroups that contain $\gen{r^4}$
  \begin{solution}
    The subgroups containing $\gen{r^4}$ are $\gen{r^4}$,
    $\gen{sr^2,r^4}$, $\gen{s,r^4}$, $\gen{r^2}$, $\gen{sr^3,r^4}$,
    $\gen{sr^5,r^4}$, $\gen{s, r^2}$, $\gen{r}$, $\gen{sr, r^2}$, and
    $D_{16}$ itself.
  \end{solution}
\item Subgroups that contain $\gen{s}$
  \begin{solution}
    The subgroups containing $\gen{s}$ are $\gen{s}$, $\gen{s,r^4}$,
    $\gen{s,r^2}$, and $D_{16}$.
  \end{solution}
\end{enumerate}
