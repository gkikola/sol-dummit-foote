\section{The Lattice of Subgroups of a Group}

\Exercise1 Let $H$ and $K$ be subgroups of $G$. Exhibit all possible
sublattices which show only $G$, $1$, $H$, $K$ and their joins and
intersections. What distinguishes the different drawings?
\begin{solution}
  In general, when $H$ and $K$ are distinct, with a nontrivial
  intersection and a join that is a proper subgroup, the sublattice
  might look something like the following:
  \begin{center}
    \begin{tikzpicture}
      \node at (0,3.5) {};
      \node (G) at (0,3) {$G$};
      \node (join) at (0,2) {$\gen{H,K}$};
      \node (H) at (-1,1) {$H$};
      \node (K) at (1,1) {$K$};
      \node (cap) at (0,0) {$H\cap K$};
      \node (1) at (0,-1) {$1$};
      \node at (0,-1.5) {};
      \draw (G) -- (join) -- (H) -- (cap) -- (K) -- (join);
      \draw (cap) -- (1);
    \end{tikzpicture}
  \end{center}
  In other cases, we may have $H = K$, or $H\geq K$:
  \begin{center}
    \begin{tikzpicture}
      \begin{scope}[xshift=-1.5cm]
        \node at (0,3.5) {};
        \node (G) at (0,3) {$G$};
        \node (HK) at (0,1.5){$H = K$};
        \node (1) at (0,0) {$1$};
        \node at (0,-0.5) {};
        \draw (G) -- (HK) -- (1);
      \end{scope}
      \begin{scope}[xshift=1.5cm]
        \node at (0,3.5) {};
        \node (G) at (0,3) {$G$};
        \node (H) at (0,2) {$H$};
        \node (K) at (0,1) {$K$};
        \node (1) at (0,0) {$1$};
        \node at (0,-0.5) {};
        \draw (G) -- (H) -- (K) -- (1);
      \end{scope}
    \end{tikzpicture}
  \end{center}
  There are several other possibilities, for example one of $H$ or $K$
  could be trivial, or we could have $\gen{H,K} = G$, and various
  other options. The drawings are distinguished by the relationships
  between the various subgroups.
\end{solution}

\Exercise2 In each of (a) to (d) list all subgroups of $D_{16}$ that
satisfy the given condition.
\begin{enumerate}
\item Subgroups that are contained in $\gen{sr^2,r^4}$
  \begin{solution}
    From the lattice given in the text, we see that the subgroups
    contained in $\gen{sr^2,r^4}$ are $\gen{sr^2,r^4}$, $\gen{sr^6}$,
    $\gen{sr^2}$, $\gen{r^4}$, and $1$.
  \end{solution}
\item Subgroups that are contained in $\gen{sr^7, r^4}$
  \begin{solution}
    $\gen{sr^7, r^4} = \gen{sr^3, r^4}$, so the subgroups contained in
    this subgroup are $\gen{sr^3, r^4}$, $\gen{r^4}$, $\gen{sr^3}$,
    $\gen{sr^7}$, and $1$.
  \end{solution}
\item Subgroups that contain $\gen{r^4}$
  \begin{solution}
    The subgroups containing $\gen{r^4}$ are $\gen{r^4}$,
    $\gen{sr^2,r^4}$, $\gen{s,r^4}$, $\gen{r^2}$, $\gen{sr^3,r^4}$,
    $\gen{sr^5,r^4}$, $\gen{s, r^2}$, $\gen{r}$, $\gen{sr, r^2}$, and
    $D_{16}$ itself.
  \end{solution}
\item Subgroups that contain $\gen{s}$
  \begin{solution}
    The subgroups containing $\gen{s}$ are $\gen{s}$, $\gen{s,r^4}$,
    $\gen{s,r^2}$, and $D_{16}$.
  \end{solution}
\end{enumerate}

\Exercise3 Show that the subgroup $\gen{s, r^2}$ of $D_8$ is
isomorphic to $V_4$.
\begin{proof}
  The subgroup $\gen{s, r^2}$ consists of the elements
  $\{1, s, r^2, sr^2\}$, and $V_4 = \{1, a, b, c\}$. Note that both
  groups are abelian and of order $4$.

  Define the mapping $\varphi\colon\gen{s,r^2}\to V_4$ by
  \begin{equation*}
    \varphi(1) = 1,
    \quad
    \varphi(s) = a,
    \quad
    \varphi(r^2) = b,
    \quad\text{and}\quad
    \varphi(sr^2) = c.
  \end{equation*}
  We now directly verify that $\varphi$ is a homomorphism:
  \begin{align*}
    \varphi(s^2) &= \varphi(1) = 1 = a^2 = \varphi(s)^2, \\
    \varphi(sr^2) &= c = ab = \varphi(s)\varphi(r^2), \\
    \varphi(ssr^2) &= \varphi(r^2) = b = ac = \varphi(s)\varphi(sr^2), \\
    \varphi(r^4) &= \varphi(1) = 1 = b^2 = \varphi(r^2)^2, \\
    \varphi(r^2sr^2) &= \varphi(s) = a = bc = \varphi(r^2)\varphi(sr^2), \\
    \varphi((sr^2)^2) &= \varphi(1) = 1 = c^2 = \varphi(sr^2)^2.
  \end{align*}
  Since both groups are abelian, this is enough to show that $\varphi$
  is a homomorphism. But $\varphi$ is clearly also a bijection, so
  $\varphi$ is an isomorphism and $\gen{s,r^2}\cong V_4$.
\end{proof}

\Exercise4 Use the given lattice to find all pairs of elements that
generate $D_8$ (there are $12$ pairs).
\begin{solution}
  First, we know that $D_8 = \gen{s,r}$. Now, looking at the cyclic
  subgroups in the lattice, we see that the only subgroup containing
  both $\gen{s}$ and $\gen{rs}$ is $D_8$ itself. Hence
  $\gen{s, rs} = D_8$. Similarly, the only subgroup containing
  $\gen{s}$ and $\gen{r^3s}$ is $D_8$, so $\gen{s, r^3s} =
  D_8$. Continuing in this way, we can find all the pairs that
  generate $D_8$ (noting that $\gen{r} = \gen{r^3}$):
  \begin{multline*}
    \gen{s,r}, \gen{s,r^3}, \gen{s,rs}, \gen{s,r^3s},
    \gen{r^2s,r}, \gen{r^2s,r^3}, \\
    \gen{r^2s,rs}, \gen{r^2s,r^3s},
    \gen{r,rs}, \gen{r^3,rs}, \gen{r^3,r^3s}, \gen{r,r^3s}.
  \end{multline*}
  No other pairing can generate all of $D_8$.
\end{solution}

\Exercise5 Use the given lattice to find all elements $x\in D_{16}$
such that $D_{16} = \gen{x, s}$ (there are $8$ such elements $x$).
\begin{solution}
  Note that $\gen{r} = \gen{r^3} = \gen{r^5} = \gen{r^7}$. We now
  proceed as in the previous problem, pairing $\gen{s}$ with other
  cyclic subgroups such that all of $D_{16}$ is the smallest group
  containing both subgroups. We find the following generating pairs:
  \begin{equation*}
    \gen{s,r},\gen{s,r^3},\gen{s,r^5},\gen{s,r^7},
    \gen{s,sr^3}, \gen{s,sr^7},\gen{s,sr^5},\gen{s,sr}.
    \qedhere
  \end{equation*}
\end{solution}

\Exercise6 Use the given lattices to help find the centralizers of
every element in the following groups:
\begin{enumerate}
\item $D_8$
  \begin{solution}
    Since $s$ commutes with $r^2$, we see from the lattice that
    $C_{D_8}(s) = \gen{s, r^2}$ (this centralizer cannot be all of
    $D_8$ since $s$ does not commute with $r$). $r^2$ commutes with
    everything (it is in the center of $D_8$), so
    $C_{D_8}(r^2) = D_8$. By similar reasoning, we find the following
    centralizers:
    \begin{align*}
      C_{D_8}(1) &= D_8, \\
      C_{D_8}(r) &= \gen{r}, \\
      C_{D_8}(r^2) &= D_8, \\
      C_{D_8}(r^3) &= \gen{r}, \\
      C_{D_8}(s) &= \gen{s, r^2}, \\
      C_{D_8}(rs) &= \gen{rs,r^2}, \\
      C_{D_8}(r^2s) &= \gen{s, r^2}, \\
      C_{D_8}(r^3s) &= \gen{rs, r^2}. \qedhere
    \end{align*}
  \end{solution}
\item $Q_8$
  \begin{solution}
    We know that $-1$ commutes with every element, but $i$, $j$, and
    $k$ do not commute with each other. Therefore
    \begin{align*}
      C_{Q_8}(1) &= Q_8, \\
      C_{Q_8}(-1) &= Q_8, \\
      C_{Q_8}(i) = C_{Q_8}(-i) &= \gen{i}, \\
      C_{Q_8}(j) = C_{Q_8}(-j) &= \gen{j}, \\
      C_{Q_8}(k) = C_{Q_8}(-k) &= \gen{k}. \qedhere
    \end{align*}
  \end{solution}
\item $S_3$
  \begin{solution}
    From the lattice we see that every nontrivial subgroup is maximal,
    so the centralizer of each cycle is either the subgroup generated
    by that cycle, or else all of $S_3$. But none of $(1\,2)$,
    $(1\,3)$, $(2\,3)$, and $(1\,2\,3)$ commute with each other, so
    none of the centralizers can be all of $S_3$, aside from
    $C_{S_3}(1)$. This gives
    \begin{align*}
      C_{S_3}(1) &= S_3, \\
      C_{S_3}(1\,2) &= \gen{(1\,2)}, \\
      C_{S_3}(1\,3) &= \gen{(1\,3)}, \\
      C_{S_3}(2\,3) &= \gen{(2\,3)}, \\
      C_{S_3}(1\,2\,3) = C_{S_3}(1\,3\,2) &= \gen{(1\,2\,3)}. \qedhere
    \end{align*}
  \end{solution}
\item $D_{16}$
  \begin{solution}
    We use similar reasoning as we did for $D_8$.
    \begin{align*}
      C_{D_{16}}(1) &= D_{16}, \\
      C_{D_{16}}(r) = C_{D_{16}}(r^2) = C_{D_{16}}(r^3) &= \gen{r}, \\
      C_{D_{16}}(r^5) = C_{D_{16}}(r^6) = C_{D_{16}}(r^7) &= \gen{r}, \\
      C_{D_{16}}(r^4) &= D_{16}, \\
      C_{D_{16}}(s) = C_{D_{16}}(sr^4) &= \gen{s, r^4}, \\
      C_{D_{16}}(sr) = C_{D_{16}}(sr^5) &= \gen{sr^5, r^4}, \\
      C_{D_{16}}(sr^2) = C_{D_{16}}(sr^6) &= \gen{sr^2, r^4}, \\
      C_{D_{16}}(sr^3) = C_{D_{16}}(sr^7) &= \gen{sr^3, r^4}. \qedhere
    \end{align*}
  \end{solution}
\end{enumerate}
