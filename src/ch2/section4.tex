\section{Subgroups Generated by Subsets of a Group}

\Exercise1 Prove that if $H$ is a subgroup of $G$ then $\gen{H} = H$.
\begin{proof}
  Let $H$ be a subgroup of $G$. Certainly $H\leq\gen{H}$. Now suppose
  $h\in\gen{H}$. Then
  \begin{equation*}
    h\in\bigcap_{\substack{H\subseteq K \\ K\leq G}} K.
  \end{equation*}
  But $H$ itself is a subgroup of $G$ containing itself as a subset,
  so by definition $h\in H$. This shows that $\gen{H}\leq H$ so that
  $\gen{H} = H$ as required.
\end{proof}

\Exercise2 Prove that if $A$ is a subset of $B$ then
$\gen{A}\leq\gen{B}$. Give an example where $A\subseteq B$ with
$A\neq B$ but $\gen{A}=\gen{B}$.
\begin{proof}
  Let $a\in\gen{A}$. Since $A\subseteq B$ we have $a\in B$ so that
  $a\in\gen{B}$. This shows that $\gen{A}\leq\gen{B}$.

  For the requested example, simply consider $Z_4$ with $A = \{x\}$
  and $B = \{x, x^3\}$. Certainly $A\subset B$ but
  $\gen{A} = \gen{B}$.
\end{proof}

\Exercise3 Prove that if $H$ is an abelian subgroup of a group $G$
then $\gen{H, Z(G)}$ is abelian. Give an explicit example of an
abelian subgroup $H$ of a group $G$ such that $\gen{H, C_G(H)}$ is not
abelian.
\begin{proof}
  Let $H\leq G$ be an abelian subgroup and let $g,h\in\gen{H, Z(G)}$
  be arbitrary. By Proposition~9, $g$ and $h$ can be written as a
  finite product
  \begin{equation*}
    g = g_1^{\epsilon_1},\dots,g_m^{\epsilon_m}
    \quad\text{and}\quad
    h = h_1^{\delta_1},\dots,h_n^{\delta_n},
  \end{equation*}
  where each $g_i$ and $h_i$ (not necessarily distinct) is in
  $H\cup Z(G)$. Now, members of $H$ commute with each other and with
  members of $Z(G)$, and members of $Z(G)$ commute with each other and
  with members of $H$. Therefore all of the elements in $H\cup Z(G)$
  commute with one another, so we may write
  \begin{equation*}
    gh = g_1^{\epsilon_1},\dots,g_m^{\epsilon_m}
    h_1^{\delta_1},\dots,h_n^{\delta_n}
    = h_1^{\delta_1},\dots,h_n^{\delta_n}
    g_1^{\epsilon_1},\dots,g_m^{\epsilon_m} = hg.
  \end{equation*}
  Since $g,h\in\gen{H, Z(G)}$ were arbitrary, this shows that
  $\gen{H, Z(G)}$ is abelian.

  To show that $\gen{H, C_G(H)}$ is not necessarily abelian, consider
  the dihedral group $D_8$ with $H = \{1, r^2\}$. Since $1$ and $r^2$
  are in $Z(D_8)$, we have $C_G(H) = D_8$. Therefore
  $\gen{H, C_G(H)} = D_8$ is not abelian, even though $H$ is abelian.
\end{proof}

\Exercise4 Prove that if $H$ is a subgroup of $G$ then $H$ is
generated by the set $H - \{1\}$.
\begin{proof}
  If $H = \{1\}$ then $H - \{1\}$ is the empty set which indeed
  generates the trivial subgroup $H$. So suppose $\ord{H} > 1$ and
  pick a nonidentity element $h\in H$. Since
  $1 = hh^{-1}\in\gen{H - \{1\}}$ (Proposition~9), we see that
  $H\leq\gen{H - \{1\}}$. By minimality of $\gen{H - \{1\}}$, the
  reverse inclusion also holds so that $\gen{H - \{1\}} = H$.
\end{proof}
