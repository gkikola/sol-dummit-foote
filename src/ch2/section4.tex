\section{Subgroups Generated by Subsets of a Group}

\Exercise1 Prove that if $H$ is a subgroup of $G$ then $\gen{H} = H$.
\begin{proof}
  Let $H$ be a subgroup of $G$. Certainly $H\leq\gen{H}$. Now suppose
  $h\in\gen{H}$. Then
  \begin{equation*}
    h\in\bigcap_{\substack{H\subseteq K \\ K\leq G}} K.
  \end{equation*}
  But $H$ itself is a subgroup of $G$ containing itself as a subset,
  so by definition $h\in H$. This shows that $\gen{H}\leq H$ so that
  $\gen{H} = H$ as required.
\end{proof}

\Exercise2 Prove that if $A$ is a subset of $B$ then
$\gen{A}\leq\gen{B}$. Give an example where $A\subseteq B$ with
$A\neq B$ but $\gen{A}=\gen{B}$.
\begin{proof}
  Let $a\in\gen{A}$. Since $A\subseteq B$ we have $a\in B$ so that
  $a\in\gen{B}$. This shows that $\gen{A}\leq\gen{B}$.

  For the requested example, simply consider $Z_4$ with $A = \{x\}$
  and $B = \{x, x^3\}$. Certainly $A\subset B$ but
  $\gen{A} = \gen{B}$.
\end{proof}

\Exercise3 Prove that if $H$ is an abelian subgroup of a group $G$
then $\gen{H, Z(G)}$ is abelian. Give an explicit example of an
abelian subgroup $H$ of a group $G$ such that $\gen{H, C_G(H)}$ is not
abelian.
\begin{proof}
  Let $H\leq G$ be an abelian subgroup and let $g,h\in\gen{H, Z(G)}$
  be arbitrary. By Proposition~9, $g$ and $h$ can be written as a
  finite product
  \begin{equation*}
    g = g_1^{\epsilon_1},\dots,g_m^{\epsilon_m}
    \quad\text{and}\quad
    h = h_1^{\delta_1},\dots,h_n^{\delta_n},
  \end{equation*}
  where each $g_i$ and $h_i$ (not necessarily distinct) is in
  $H\cup Z(G)$. Now, members of $H$ commute with each other and with
  members of $Z(G)$, and members of $Z(G)$ commute with each other and
  with members of $H$. Therefore all of the elements in $H\cup Z(G)$
  commute with one another, so we may write
  \begin{equation*}
    gh = g_1^{\epsilon_1},\dots,g_m^{\epsilon_m}
    h_1^{\delta_1},\dots,h_n^{\delta_n}
    = h_1^{\delta_1},\dots,h_n^{\delta_n}
    g_1^{\epsilon_1},\dots,g_m^{\epsilon_m} = hg.
  \end{equation*}
  Since $g,h\in\gen{H, Z(G)}$ were arbitrary, this shows that
  $\gen{H, Z(G)}$ is abelian.

  To show that $\gen{H, C_G(H)}$ is not necessarily abelian, consider
  the dihedral group $D_8$ with $H = \{1, r^2\}$. Since $1$ and $r^2$
  are in $Z(D_8)$, we have $C_G(H) = D_8$. Therefore
  $\gen{H, C_G(H)} = D_8$ is not abelian, even though $H$ is abelian.
\end{proof}

\Exercise4 Prove that if $H$ is a subgroup of $G$ then $H$ is
generated by the set $H - \{1\}$.
\begin{proof}
  If $H = \{1\}$ then $H - \{1\}$ is the empty set which indeed
  generates the trivial subgroup $H$. So suppose $\ord{H} > 1$ and
  pick a nonidentity element $h\in H$. Since
  $1 = hh^{-1}\in\gen{H - \{1\}}$ (Proposition~9), we see that
  $H\leq\gen{H - \{1\}}$. By minimality of $\gen{H - \{1\}}$, the
  reverse inclusion also holds so that $\gen{H - \{1\}} = H$.
\end{proof}

\Exercise5 Prove that the subgroup generated by any two distinct
elements of order $2$ in $S_3$ is all of $S_3$.
\begin{proof}
  There are three elements of order $2$ in $S_3$, namely $(1\,2)$,
  $(1\,3)$, and $(2\,3)$. For $\gen{(1\,2), (1\,3)}$ we have
  \begin{equation*}
    (2\,3) = (1\,3)(1\,2)(1\,3),
    \quad
    (1\,2\,3) = (1\,3)(1\,2),
    \quad\text{and}\quad
    (1\,3\,2) = (1\,2)(1\,3),
  \end{equation*}
  so $\gen{(1\,2), (1\,3)} = S_3$. By symmetry, we also have
  \begin{equation*}
    \gen{(1\,2), (2\,3)} = \gen{(1\,3), (2\,3)} = S_3.
  \end{equation*}
  Therefore the desired result holds.
\end{proof}

\Exercise6 Prove that the subgroup of $S_4$ generated by $(1\,2)$ and
$(1\,2)(3\,4)$ is a noncyclic group of order $4$.
\begin{proof}
  Let $a = (1\,2)$ and $b = (1\,2)(3\,4)$. Note that $a^2 = b^2 = 1$,
  and $ab = ba = (3\,4)$. We see that the set
  $A = \{1, a, b, (3\,4)\}$ is closed under composition and inverses
  and hence is a subgroup of order $4$. Therefore $\gen{a, b} = A$ and
  we see that $A$ is noncyclic since in particular it has two distinct
  elements with order $2$.
\end{proof}

\Exercise7 Prove that the subgroup of $S_4$ generated by $(1\,2)$ and
$(1\,3)(2\,4)$ is isomorphic to the dihedral group of order $8$.
\begin{proof}
  Let $a = (1\,2)$, $b = (1\,3)(2\,4)$ and $c = (1\,3\,2\,4)$. It is
  easy to check that $ab = c$ so $c\in\gen{a,b}$.

  Clearly $a^2 = c^4 = 1$. Since $b$ is a product of disjoint
  $2$-cycles, it follows that $b^{-1} = b$ so that
  \begin{equation*}
    ca = aba = a(ab)^{-1} = ac^{-1}.
  \end{equation*}
  Since $a$ and $c$ satisfy all the same relations as do $s$ and $r$
  in $D_8$, it follows that there is a homomorphism
  $\varphi\colon D_8\to\gen{a,b}$ defined by
  \begin{equation*}
    \varphi(s^ir^j) = a^i(ab)^j,
    \quad\text{where $i\in\{0,1\}$ and $j\in\{0,1,2,3\}$}.
  \end{equation*}
  Since $c$, $c^2$, and $c^3$ are all distinct, it follows that
  $\varphi$ is injective. And $\varphi$ is surjective since every
  finite product of powers of $a$ and $b$ can be reduced to the form
  $a^i(ab)^j$ in the same way that elements of $D_8$ can be expressed
  in the form $s^ir^j$. $\varphi$ is bijective, so it is an
  isomorphism and we have $D_8\cong\gen{a,b}$.
\end{proof}

\Exercise8 Prove that $S_4 = \gen{(1\,2\,3\,4), (1\,2\,4\,3)}$.
\begin{proof}
  Let $A = \gen{(1\,2\,3\,4),(1\,2\,4\,3)}$. By inspection, we find that
  \begin{equation*}
    (1\,4\,2) = (1\,2\,4\,3)(1\,2\,3\,4).
  \end{equation*}
  Therefore $A$ contains an element of order $3$ as well as elements
  of order $4$. Thus $3$ and $4$ both divide $\ord{A}$. But $\ord{A}$
  also divides $24$, so the only possibilities for $\ord{A}$ are $12$
  and $24$.

  To eliminate $12$ as a possible order, note that
  \begin{equation*}
    (1\,2) = (1\,2\,3\,4)(1\,2\,4\,3)^3(1\,2\,3\,4)
  \end{equation*}
  and
  \begin{equation*}
    (1\,3\,2\,4) = (1\,3)(2\,4)
    = (1\,2\,3\,4)(1\,2\,4\,3)(1\,2\,3\,4)(1\,2\,4\,3)^2.
  \end{equation*}
  So by the previous exercise, we know that $A$ contains a subgroup
  isomorphic to $D_8$. Therefore $8$ divides $\ord{A}$ so we must have
  $A = S_4$.
\end{proof}

\Exercise9 Prove that $SL_2(\F_3)$ is the subgroup of $GL_2(\F_3)$
generated by $\begin{pmatrix} 1 & 1 \\ 0 & 1 \end{pmatrix}$ and
$\begin{pmatrix} 1 & 0 \\ 1 & 1 \end{pmatrix}$.
\begin{proof}
  Let
  \begin{equation*}
    A =
    \begin{pmatrix}
      1 & 1 \\
      0 & 1
    \end{pmatrix}
    \quad\text{and}\quad
    B =
    \begin{pmatrix}
      1 & 0 \\
      1 & 1
    \end{pmatrix}.
  \end{equation*}
  Note that $A,B\in SL_2(\F_3)$. We are told that we may assume that
  the subgroup $SL_2(\F_3)$ has order $24$, so we can show that
  $\gen{A,B} = SL_2(\F_3)$ if we can show that it has more than $12$
  elements (since the order of $\gen{A,B}$ must divide $24$).

  The matrices $I$, $A$, and $B$, make three elements, so we need to
  find ten more:
  \begin{align*}
    A^2 &= \begin{pmatrix} 1 & 2 \\ 0 & 1 \end{pmatrix}, &
    B^2 &= \begin{pmatrix} 1 & 0 \\ 2 & 1 \end{pmatrix}, \\
    AB &= \begin{pmatrix} 2 & 1 \\ 1 & 1 \end{pmatrix}, &
    (AB)^2 &= \begin{pmatrix} 2 & 0 \\ 0 & 2 \end{pmatrix}, \\
    (AB)^3 &= \begin{pmatrix} 1 & 2 \\ 2 & 2 \end{pmatrix}, &
    BA &= \begin{pmatrix} 1 & 1 \\ 1 & 2 \end{pmatrix}, \\
    A^2B^2 &= \begin{pmatrix} 2 & 2 \\ 2 & 1 \end{pmatrix}, &
    ABA &= \begin{pmatrix} 2 & 0 \\ 1 & 2 \end{pmatrix}, \\
    BAB &= \begin{pmatrix} 2 & 1 \\ 0 & 2 \end{pmatrix}, &
    A^2B &= \begin{pmatrix} 0 & 2 \\ 1 & 1 \end{pmatrix}.
  \end{align*}
  So $\ord{\gen{A,B}} = 24$ and $\gen{A,B} = SL_2(\F_3)$.
\end{proof}
