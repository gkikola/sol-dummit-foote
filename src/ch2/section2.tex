\section[Centralizers and Normalizers]
{Centralizers and Normalizers, Stabilizers and Kernels}

\Exercise1 Prove that
\begin{equation*}
  C_G(A) = \{g\in G \mid \text{$g^{-1}ag = a$ for all $a\in A$}\}.
\end{equation*}
\begin{proof}
  By multiplying on the left by $g$ and on the right by $g^{-1}$, we
  see that $g^{-1}ag = a$ if and only if $gag^{-1} = a$. Therefore the
  above set is a valid alternative way to define the centralizer of
  $A$.
\end{proof}

\Exercise2 Prove that $C_G(Z(G)) = G$ and deduce that $N_G(Z(G)) = G$.
\begin{proof}
  Let $g\in G$ be arbitrary. If $a\in Z(G)$, then in particular,
  $ga = ag$ which shows that $gag^{-1} = a$. Therefore
  $g\in C_G(Z(G))$ for any $g\in G$, so $G\leq C_G(Z(G))$. But we know
  $C_G(Z(G))\leq G$, so this establishes equality.

  Since $C_G(A)\leq N_G(A)$ for any $A\subseteq G$, we must have
  $N_G(Z(G)) = G$.
\end{proof}

\Exercise3 Prove that if $A$ and $B$ are subsets of $G$ with
$A\subseteq B$ then $C_G(B)$ is a subgroup of $C_G(A)$.
\begin{proof}
  Suppose $A$ and $B$ are as stated. Let $g\in C_G(B)$. Then
  $gbg^{-1} = b$ for any $b\in B$. But $A\subseteq B$, so
  $gag^{-1} = a$ for any $a\in A$ as well. Therefore $g\in
  C_G(A)$. This shows that $C_G(B)\subseteq C_G(A)$, and since both
  are subgroups of $G$, we have $C_G(B)\leq C_G(A)$.
\end{proof}

\Exercise4 For each of $S_3$, $D_8$, and $Q_8$ compute the
centralizers of each element and find the center of each group. Does
Lagrange's Theorem simplify your work?
\begin{solution}
  The centralizer of $1$ (for any group) is the entire group. The
  centralizers of the other elements can be computed directly. For
  example, $C_{S_3}((1\,2))$ must at minimum include $1$ and $(1\,2)$
  itself. We can test the other elements directly (note
  $(1\,2)^{-1} = (1\,2)$):
  \begin{align*}
    (1\,2)(1\,3)(1\,2) &= (2\,3), \\
    (1\,2)(2\,3)(1\,2) &= (1\,3), \\
    (1\,2)(1\,2\,3)(1\,2) &= (1\,3\,2), \\
    (1\,2)(1\,3\,2)(1\,2) &= (1\,2\,3).
  \end{align*}
  So, $C_{S_3}((1\,2)) = \{ 1, (1\,2) \}$.

  We can use Lagrange's Theorem to reduce some of the checking. For
  example, let $a = (1\,3)$. Then $C_{S_3}(a)$ must include the
  subgroup $\{1, (1\,3)\}$, so $2$ divides $\ord{C_{S_3}(a)}$. On the
  other hand, $\ord{C_{S_3}(a)}$ divides $\ord{S_3} = 6$. Therefore
  there are only two possibilities, either $\ord{C_{S_3}(a)} = 2$ or
  $6$. Since $(1\,3)$ does not commute with $(1\,2)$, we know that the
  order must be $2$. So $C_{S_3}(a) = \{ 1, (1\,3) \}$. Similarly, we
  find $C_{S_3}((2\,3)) = \{ 1, (2\,3) \}$.

  Now let $a = (1\,2\,3)$. We have $a^{-1} = (1\,3\,2) = a^2$ so
  $C_{S_3}(a)$ must contain the cyclic subgroup $\{1, a, a^2\}$ and we
  see that $3$ divides $\ord{C_{S_3}(a)}$. So the order is either $3$
  or $6$. But it must be $3$, since $(1\,2\,3)$ does not commute with
  $(1\,2)$, for example. So $C_{S_3}(a) = \{1, a, a^2\}$. Similarly,
  $C_{S_3}((1\,3\,2))$ is this same set.

  From the above results, we see that the center of $S_3$ is
  $Z(S_3) = \{1\}$, since no non-identity element commutes with every
  element of $S_3$.

  Similarly, we may find the centralizers of $D_8$:
  \begin{align*}
    C_{D_8}(r) &= \{1, r, r^2, r^3\}, \\
    C_{D_8}(r^2) &= D_8, \\
    C_{D_8}(r^3) &= \{1, r, r^2, r^3\}, \\
    C_{D_8}(s) &= \{1, r^2, s, sr^2\}, \\
    C_{D_8}(sr) &= \{1, r^2, sr, sr^3\}, \\
    C_{D_8}(sr^2) &= \{1, r^2, s, sr^2\}, \\
    C_{D_8}(sr^3) &= \{1, r^2, sr, sr^3\}.
  \end{align*}
  And we see that $Z(D_8) = \{1, r^2\}$.

  Finally, for $Q_8$, we have:
  \begin{align*}
    C_{Q_8}(-1) &= Q_8, \\
    C_{Q_8}(i) &= \{1, -1, i, -i\}, \\
    C_{Q_8}(-i) &= \{1, -1, i, -i\}, \\
    C_{Q_8}(j) &= \{1, -1, j, -j\}, \\
    C_{Q_8}(-j) &= \{1, -1, j, -j\}, \\
    C_{Q_8}(k) &= \{1, -1, k, -k\}, \\
    C_{Q_8}(-k) &= \{1, -1, k, -k\}.
  \end{align*}
  And $Z(Q_8) = \{1, -1\}$.
\end{solution}

\Exercise5 In each of parts (a) to (c) show that for the specified
group $G$ and subgroup $A$ of $G$, $C_G(A) = A$ and $N_G(A) = G$.
\begin{enumerate}
\item $G = S_3$ and $A = \{1, (1\,2\,3), (1\,3\,2)\}$.
  \begin{solution}
    $A$ is a cyclic subgroup generated by $(1\,2\,3)$, so
    $A\leq C_G(A)$. By Lagrange's Theorem, $3$ divides $\ord{C_G(A)}$
    which divides $\ord{S_3} = 6$, so either $\ord{C_G(A)} = 3$ or it
    is $6$. But it can't be $6$ since, for example, $(1\,2)$ does not
    commute with $(1\,2\,3)$. Therefore $\ord{C_G(A)} = 3$ and we see
    that $C_G(A) = A$.

    Since $C_G(A)\leq N_G(A)$, we again have either $\ord{N_G(A)} = 3$
    or $6$. However,
    \begin{equation*}
      (1\,2)A(1\,2) = \{1, (1\,3\,2), (1\,2\,3)\} = A,
    \end{equation*}
    so $\ord{N_G(A)} > 3$. Therefore $N_G(A) = G$.
  \end{solution}
\item $G = D_8$ and $A = \{1, s, r^2, sr^2\}$.
  \begin{solution}
    The elements of $A$ all commute with one another and in fact form
    a subgroup of $D_8$. By Lagrange, $\ord{C_G(A)} = 4$ or $8$. But
    $r$ does not commute with $s$, for example, so $\ord{C_G(A)} = 4$
    and we have $C_G(A) = A$.

    Since $C_G(A)\leq N_G(A)$, we must have either $N_G(A) = A$ or
    $N_G(A) = G$. Since
    \begin{equation*}
      rAr^{-1} = \{ 1, sr^2, r^2, s \} = A,
    \end{equation*}
    we must have $N_G(A) = G$.
  \end{solution}
\item $G = D_{10}$ and $A = \{1, r, r^2, r^3, r^4\}$.
  \begin{solution}
    $A$ is a cyclic subgroup. Again, by Lagrange, we must have
    $\ord{C_G(A)} = 5$ or $10$. But $s$ and $r$ do not commute, so it
    must be the former. Hence $C_G(A) = A$.

    For the normalizer, we simply note that
    \begin{equation*}
      sAs = \{ 1, r^4, r^3, r^2, r\} = A,
    \end{equation*}
    so $N_G(A) = G$.
  \end{solution}
\end{enumerate}
