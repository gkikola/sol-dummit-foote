\section{More on Cosets and Lagrange's Theorem}

Let $G$ be a group.

\Exercise1 Which of the following are permissible orders for subgroups
of a group of order $120$: $1$, $2$, $5$, $7$, $9$, $15$, $60$, $240$?
For each permissible order give the corresponding index.
\begin{solution}
  The permissible orders are $1$, $2$, $5$, $15$, and $60$. The other
  orders do not divide $120$ and so by Lagrange's Theorem are not
  possible.

  The subgroup of order $1$ would have index $120$, the subgroup of
  order $2$ would have index $60$, the subgroup of order $5$ would
  have index $24$, the subgroup with order $15$ would have index $8$,
  and the subgroup of order $60$ would have index $2$.
\end{solution}

\Exercise2 Prove that the lattice of subgroups of $S_3$ in Section~2.5
is correct (i.e., prove that it contains all subgroups of $S_3$ and
that their pairwise joins and intersections are correctly drawn).
\begin{proof}
  The subgroups contained in the lattice are $1$, $\gen{(1\,2)}$,
  $\gen{(1\,3)}$, $\gen{(2\,3)}$, $\gen{(1\,2\,3)}$, and $S_3$
  itself. Since $\ord{S_3} = 6$, any nontrivial subgroups must have
  order $2$ or $3$. Since $\gen{(1\,2\,3)} = \gen{(1\,3\,2)}$, all
  cyclic subgroups are accounted for.

  Now suppose $S_3$ has a non-cyclic proper subgroup $H$. Say $H$ is
  generated by $\sigma$ and $\tau$. Then $\ord{H} = 3$ and
  $H = \{1, \sigma, \tau\}$. But $\ord{\sigma}$ must divide $\ord{H}$,
  so $\ord{\sigma} = 3$. Then $\sigma$ and $\sigma^2$ are distinct,
  and we must have $\tau = \sigma^2$. Hence $H$ is not cyclic, a
  contradiction. This shows that all proper subgroups of $S_3$ are
  cyclic. Therefore all subgroups are present in the lattice.

  Note that the subgroups of order $2$ cannot themselves be subgroups
  of $\gen{(1\,2\,3)}$, since $2\nmid3$. Therefore every nontrivial
  subgroup is maximal, and the lattice is correct.
\end{proof}

\Exercise3 Prove that the lattice of subgroups of $Q_8$ in Section~2.5
is correct.
\begin{proof}
  By Lagrange's Theorem the possible subgroups of $Q_8$ have orders
  $1$, $2$, $4$, and $8$. So every nontrivial subgroup has order $2$
  or $4$. The only element of $Q_8$ having order $2$ is $-1$, so
  $\gen{-1}$ is the only possible subgroup with that order. Every
  other nonidentity element has order $4$. The only subgroups of order
  $4$ are $\gen{i}$, $\gen{j}$, and $\gen{k}$, since all elements of
  $Q_8$ belong to one of these subgroups.

  $\gen{-1}$ is contained in each of $\gen{i}$, $\gen{j}$, and
  $\gen{k}$, and the latter three are maximal by Lagrange. Therefore
  the lattice is correct.
\end{proof}

\Exercise4 Show that if $\ord{G} = pq$ for some primes $p$ and $q$
(not necessarily distinct) then either $G$ is abelian or $Z(G) = 1$.
\begin{proof}
  Let $G$ be as stated. If $G$ is abelian, there is nothing left to
  prove, so suppose $G$ is not abelian. Then $Z(G)$ is proper. By
  Lagrange, there are only three possibilities for the order of
  $Z(G)$: the order is either $1$, $p$, or $q$.

  Now assume that $Z(G)$ is not trivial. Then without loss of
  generality we may suppose that $\ord{Z(G)} = p$. Since the center of
  a group is always normal, we may consider the quotient group
  $G/Z(G)$. Again by Lagrange, we have
  \begin{equation*}
    \ord{G/Z(G)} = \frac{\ord{G}}{\ord{Z(G)}} = \frac{pq}p = q.
  \end{equation*}
  Now, $q$ is prime, so we may apply Corollary~10 to conclude that
  $G/Z(G)$ is cyclic. Then by
  Exercise~\ref{exercise:quotient-group:center-quotient-cyclic}, it
  follows that $G$ is abelian, which is a contradiction. Therefore
  $Z(G)$ is the trivial subgroup, and the proof is complete.
\end{proof}

\Exercise5 Let $H$ be a subgroup of $G$ and fix some element $g\in G$.
\begin{enumerate}
\item Prove that $gHg^{-1}$ is a subgroup of $G$ of the same order as
  $H$.
  \begin{proof}
    $1\in gHg^{-1}$, so $gHg^{-1}$ is nonempty. Let
    $x,y\in gHg^{-1}$. Then $x = gh_1g^{-1}$ and $y = gh_2g^{-1}$
    for some $h_1,h_2\in H$. We have
    \begin{align*}
      xy^{-1} &= (gh_1g^{-1})(gh_2g^{-1})^{-1} \\
              &= (gh_1g^{-1})(gh_2^{-1}g^{-1}) \\
              &= gh_1(g^{-1}g)h_2^{-1}g^{-1} \\
              &= gh_1h_2^{-1}g^{-1} \in gHg^{-1}.
    \end{align*}
    By the subgroup criterion, $gHg^{-1}\leq G$.

    Now define the map $\varphi\colon H\to gHg^{-1}$ by
    $\varphi(h) = ghg^{-1}$. Then $\varphi$ is clearly surjective. It
    is also injective, since $gh_1g^{-1} = gh_2g^{-1}$ implies that
    $h_1 = h_2$ by the left and right cancellation laws. This shows
    that $\ord{H} = \ord{gHg^{-1}}$ (in fact they are isomorphic).
  \end{proof}
\item Deduce that if $n\in\Z^+$ and $H$ is the unique subgroup of $G$
  of order $n$ then $H\trianglelefteq G$.
  \begin{proof}
    For any $g\in G$, we know from the above that $gHg^{-1}\leq G$ and
    that $\ord{gHg^{-1}} = \ord{H} = n$. Since $H$ is the only
    subgroup of $G$ with order $n$, it follows that $H =
    gHg^{-1}$. Since this is true for every $g\in G$,
    $H\trianglelefteq G$ by definition.
  \end{proof}
\end{enumerate}

\Exercise6 Let $H\leq G$ and let $g\in G$. Prove that if the right
coset $Hg$ equals {\em some} left coset of $H$ in $G$ then it equals
the left coset $gH$ and $g$ must be in $N_G(H)$.
\begin{proof}
  Suppose $Hg = aH$ for some $a\in G$. Since $g\in Hg$, we have
  $g\in aH$. Since the (right) cosets of $H$ form a partition of $G$
  (Proposition~4), this implies that $aH = gH$. Therefore $Hg = gH$
  and $g\in N_G(H)$.
\end{proof}

\Exercise7 Let $H\leq G$ and define a relation $\sim$ on $G$ by
\begin{equation*}
  a\sim b \quad\text{if and only if}\quad b^{-1}a\in H.
\end{equation*}
Prove that $\sim$ is an equivalence relation and describe the
equivalence class of each $a\in G$. Use this to prove Proposition~4.
\begin{proof}
  For any $a\in G$, we certainly have $a^{-1}a = 1\in H$, so $a\sim a$
  and $\sim$ is reflexive. If $a\sim b$ for $a,b\in G$ then, since $H$
  must be closed under inverses, we have
  $a^{-1}b = (b^{-1}a)^{-1} \in H$ so that $b\sim a$, and $\sim$ is
  symmetric. Lastly, if $a\sim b$ and $b\sim c$ for $a,b,c\in G$, then
  \begin{equation*}
    c^{-1}a
    = c^{-1}(bb^{-1})a
    = (c^{-1}b)(b^{-1}a) \in H,
  \end{equation*}
  so $a\sim c$ and we see that $\sim$ is transitive. This shows that
  $\sim$ is an equivalence relation.

  Note that the equivalence classes of $\sim$ form a partition of $G$
  (see Proposition~2 of Section~0.1). It is not difficult to see that
  $a\sim b$ if and only if $a$ and $b$ belong to the same left coset
  of $H$. Therefore the left cosets of $H$ form a partition of $G$,
  providing an alternative proof for Proposition~4.
\end{proof}

\Exercise8 Prove that if $H$ and $K$ are finite subgroups of $G$ whose
orders are relatively prime then $H\cap K = 1$.
\begin{proof}
  Let $H$ and $K$ be as stated, and let $n = \ord{H\cap K}$. We know
  that the intersection of two subgroups is a subgroup, so
  $H\cap K\leq H$. By Lagrange's Theorem, $n$ must divide
  $\ord{H}$. But $H\cap K\leq K$ also, so $n$ must divide
  $\ord{K}$. And since $\ord{H}$ and $\ord{K}$ have no common divisor
  other than $1$, we must have $n = 1$. Therefore $H\cap K$ is the
  trivial subgroup.
\end{proof}

\Exercise9 Let $G$ be a finite group and let $p$ be a prime dividing
$\ord{G}$. Let $\mathcal{S}$ denote the set of $p$-tuples of elements
of $G$ the product of whose coordinates is $1$:
\begin{equation*}
  \mathcal{S} = \{(x_1,x_2,\dots,x_p) \mid
  \text{$x_i\in G$ and $x_1x_2\cdots x_p = 1$}\}.
\end{equation*}
\begin{enumerate}
\item Show that $\mathcal{S}$ has $\ord{G}^{p-1}$ elements, hence has
  order divisible by $p$.
  \begin{proof}
    Let $\mathcal{T}$ be the set of all $(p-1)$-tuples of elements of
    $G$. Then $\mathcal{T}$ has $\ord{G}^{p-1}$ elements. We show that
    $\mathcal{S}$ and $\mathcal{T}$ share the same cardinality.

    Define $\varphi\colon\mathcal{T}\to\mathcal{S}$ by
    \begin{equation*}
      \varphi(x_1,x_2,\dots,x_{p-1})
      = (x_1,x_2,\dots,x_{p-1},(x_1x_2\cdots x_{p-1})^{-1}).
    \end{equation*}
    Given two elements $\alpha,\beta\in\mathcal{S}$, if
    $\alpha = \beta$ then clearly their first $p-1$ coordinates must
    be equal, so $\mathcal{T}$ is injective. Now suppose
    $\alpha\in\mathcal{S}$. Then the image of the first $p-1$
    coordinates of $\alpha$ under $\mathcal{T}$ must be $\alpha$,
    since the last coordinate is completely determined by the others
    (group inverses are unique). Therefore $\varphi$ is a bijection,
    so $\ord{\mathcal{S}} = \ord{\mathcal{T}} = \ord{G}^{p-1}$.
  \end{proof}
\end{enumerate}
Define the relation $\sim$ on $\mathcal{S}$ by letting
$\alpha\sim\beta$ if $\beta$ is a cyclic permutation of $\alpha$.
\begin{enumerate}
  \setcounter{enumi}1
\item Show that a cyclic permutation of an element of $\mathcal{S}$ is
  again an element of $\mathcal{S}$.
  \begin{proof}
    Let $\alpha = (x_1,x_2,\dots,x_p)\in\mathcal{S}$. Then
    $x_1x_2\cdots x_p = 1$. Multiplying on the left by $x_p$ and then
    on the right by $x_p^{-1}$ gives
    \begin{equation*}
      x_px_1x_2\cdots x_{p-1} = x_px_p^{-1} = 1.
    \end{equation*}
    Therefore $(x_p,x_1,x_2,\dots,x_{p-1})\in\mathcal{S}$. We may
    repeat this process $p-1$ times to see that all cyclic
    permutations of $\alpha$ are in $\mathcal{S}$.
  \end{proof}
\item Prove that $\sim$ is an equivalence relation on $\mathcal{S}$.
  \begin{proof}
    Let $\alpha,\beta,\gamma\in\mathcal{S}$ be arbitrary. $\alpha$ is
    a cyclic permutation of itself (namely the identity permutation),
    so $\alpha\sim\alpha$ and $\sim$ is reflexive.

    If $\alpha\sim\beta$ then $\beta$ is a cyclic permutation of
    $\alpha$, so by taking the inverse of this permutation we see that
    $\alpha$ is a cyclic permutation of $\beta$. Therefore
    $\beta\sim\alpha$, and $\sim$ is symmetric.

    Lastly, if $\alpha\sim\beta$ and $\beta\sim\gamma$ then $\gamma$
    is a cyclic permutation of $\beta$ and $\beta$ is a cyclic
    permutation of $\alpha$, and by taking the composition of these
    two permutations we see that $\gamma$ is a cyclic permutation of
    $\alpha$, so that $\alpha\sim\gamma$. Thus $\sim$ is transitive
    and the proof is complete.
  \end{proof}
\item Prove that an equivalence class contains a single element if and
  only if it is of the form $(x,x,\dots,x)$ with $x^p = 1$.
  \begin{proof}
    Take any $\alpha\in\mathcal{S}$ and let $[\alpha]$ denote the
    equivalence class containing $\alpha$.

    First, if $\alpha$ is the only element in $[\alpha]$, then all
    cyclic permutations of $\alpha$ must be the same. This is not
    possible unless all coordinates of $\alpha$ are the same. So
    $\alpha$ has the form $(x,x,\dots,x)$, where $x^p = 1$.

    Conversely, if $\alpha$ has the form $(x,x,\dots,x)$ with all
    coordinates the same, then every cyclic permutation of $\alpha$
    will leave $\alpha$ unchanged. Therefore $[\alpha]$ contains only
    the one element.
  \end{proof}
\item Prove that every equivalence class has order $1$ or $p$ (this
  uses the fact that $p$ is a {\em prime}). Deduce that
  $\ord{G}^{p-1} = k + pd$, where $k$ is the number of classes of size
  $1$ and $d$ is the number of classes of size $p$.
  \begin{proof}
    Again, let $\alpha\in\mathcal{S}$ with $[\alpha]$ denoting the
    corresponding equivalence class, and let
    \begin{equation*}
      \alpha = (x_1,x_2,\dots,x_p),
    \end{equation*}
    with the $x_i$'s not necessarily distinct. Suppose $[\alpha]$
    contains exactly $n$ members. Then $1\leq n\leq p$. For all
    $k\in\Z$, we must have
    \begin{equation}
      \label{eq:quotient-group:cauchy:cond-for-xi-eq-xj}
      x_i = x_j
      \quad\text{whenever}\quad
      i + kn \equiv j \pmod{p},
    \end{equation}
    since $\alpha$ can only be cycled $n$ times before arriving back
    at itself.

    Now there are two cases, either $n = p$ or $1\leq n<p$. In the
    first case there is nothing left to prove, so assume $1\leq
    n<p$. Then $(n,p) = 1$ since $p$ is prime. So by
    Exercise~\ref{exercise:prelim:z-mod-nz-cross}, we know that $n$
    has a multiplicative inverse, $n^{-1}$, modulo $p$. Then, taking
    $k = n^{-1}$, \eqref{eq:quotient-group:cauchy:cond-for-xi-eq-xj}
    tells us that $x_i = x_j$ whenever $i+1\equiv
    j\pmod{p}$. Therefore $x_{i+1} = x_i$ for all $i$ with
    $1\leq i<p$, and $x_1 = x_p$. But then every coordinate of
    $\alpha$ is the same, so $[\alpha]$ has only one member. Hence
    $n = 1$ in this case.

    We have shown that if $[\alpha]$ has exactly $n$ elements, then
    $n = 1$ or $n = p$. Since the equivalence classes partition
    $\mathcal{S}$, we see that
    \begin{equation*}
      \ord{G}^{p-1} = \ord{\mathcal{S}} = k + pd,
    \end{equation*}
    where $k$ is the number of classes of size $1$ and $d$ is the
    number of classes of size $p$.
  \end{proof}
\item Since $\{(1,1,\dots,1)\}$ is an equivalence class of size $1$,
  conclude from (e) that there must be a nonidentity element $x$ in
  $G$ with $x^p = 1$, i.e., $G$ contains an element of order $p$.
  \begin{proof}
    Since $p$ divides $\ord{G}$, it certainly divides
    $\ord{G}^{p-1}$. So $p\mid(k + pd)$. But this implies that
    $p\mid k$. Therefore $k > 1$, so there must be more than one
    equivalence class of $\sim$ having only one element. One of these
    must be of the form $(x,x,\dots,x)$ where $x$ is not the identity
    of $G$. Therefore this $x$ is such that $x^p = 1$, so $\ord{x}$
    divides $p$. We know $x$ is not the identity, so $\ord{x} = p$.
  \end{proof}
\end{enumerate}

\Exercise{10}
\label{exercise:quotient-group:index-subgroup-intersection}
Suppose $H$ and $K$ are subgroups of finite index in the (possibly
infinite) group $G$ with $\ord{G:H} = m$ and $\ord{G:K} = n$. Prove
that
\begin{equation*}
  \text{l.c.m.}(m,n)\leq\ord{G:H\cap K}\leq mn.
\end{equation*}
Deduce that if $m$ and $n$ are relatively prime then
$\ord{G:H\cap K} = \ord{G:H}\cdot\ord{G:K}$.
\begin{proof}
  For any $g\in G$, consider the cosets $gH$, $gK$, and $g(H\cap
  K)$. First, if $x\in g(H\cap K)$, then $x\in gH$ and $x\in gK$ so
  $g(H\cap K)\subseteq(gH\cap gK)$. On the other hand, if $x\in gH$
  and $x\in gK$, then $g^{-1}x\in H\cap K$, so $x\in g(H\cap K)$ and
  we have $(gH\cap gK)\subseteq g(H\cap K)$. Therefore
  \begin{equation*}
    gH\cap gK = g(H\cap K)
    \quad\text{for all $g\in G$}.
  \end{equation*}
  Now each coset of $H\cap K$ is the intersection of one coset of $H$
  and one coset of $K$. There are exactly $mn$ such intersections, so
  $\ord{G:H\cap K}$ is finite and is at most $mn$.

  As we will show in
  Exercise~\ref{exercise:quotient-group:subgroup-of-subgroup-index}
  below, we must have
  \begin{equation*}
    \ord{G:H\cap K} = \ord{G:H}\ord{H:H\cap K}.
  \end{equation*}
  So $\ord{G:H}$ divides $\ord{G:H\cap K}$. That is, if
  $\ord{G:H\cap K} = s$, then $m\mid s$. By the same argument, we know
  that $n\mid s$ also. Therefore $s\geq[m,n]$, where $[m,n]$ denotes
  the least common multiple of $m$ and $n$. This completes the proof
  of the inequality.

  Finally, if $(m,n) = 1$, then $[m,n] = mn$, and we see that
  $\ord{G:H\cap K}$ must equal $mn$.
\end{proof}

\Exercise{11}
\label{exercise:quotient-group:subgroup-of-subgroup-index}
Let $H\leq K\leq G$. Prove that $\ord{G:H} = \ord{G:K}\cdot\ord{K:H}$
(do not assume $G$ is finite).
\begin{proof}
  Since cosets of $H$ are contained within cosets of $K$, if $H$ has
  infinite index in $K$ or if $K$ has infinite index in $G$ then $H$
  has infinite index in $G$ also. So we will assume that $\ord{K:H}$
  and $\ord{G:K}$ are finite.

  Let $m,n\in\Z^+$ where
  \begin{equation*}
    m = \ord{G:K}
    \quad\text{and}\quad
    n = \ord{K:H}.
  \end{equation*}
  Let $g_1,g_2,\dots,g_m$ be representatives of the distinct cosets of
  $K$ in $G$, and let $k_1,k_2,\dots,k_n$ be representatives of the
  distinct cosets of $H$ in $K$. Take any element $a\in G$. Since the
  cosets of $K$ partition $G$, $a$ belongs to exactly one coset
  $g_iK$, so $a = g_ib$ for some $b\in K$. And since the cosets of $H$
  partition $K$, $b$ belongs to exactly one coset $k_jH$, so
  $b = k_jc$ for some $c\in H$. Then $a = g_ik_jc$, where $i$ and $j$
  are uniquely determined.

  Note that, within each coset $g_iK$, we cannot have
  $g_ik_{j_1}H = g_ik_{j_2}H$ with $j_1\neq j_2$. For, if this is
  possible, then let $x$ belong to this common coset. Then
  $x = g_ik_{j_1}h_1$ and $x = g_ik_{j_2}h_2$ for some $h_1,h_2\in
  H$. Multiplying on the left by $g_i^{-1}$ then gives
  $k_{j_1}h_1 = k_{j_2}h_2$, and we see that $k_{j_1}$ and $k_{j_2}$
  are representatives of the same coset of $H$ in $K$, which is a
  contradiction.

  Therefore the cosets of $H$ partition $G$ into $mn$ disjoint
  subsets, so
  \begin{equation*}
    \ord{G:H} = mn = \ord{G:K}\cdot\ord{K:H}. \qedhere
  \end{equation*}
\end{proof}

\Exercise{12} Let $H\leq G$. Prove that the map $x\mapsto x^{-1}$
sends each left coset of $H$ in $G$ onto a right coset of $H$ and
gives a bijection between the set of left cosets and the set of right
cosets of $H$ in $G$ (hence the number of left cosets of $H$ in $G$
equals the number of right cosets).
\begin{proof}
  Let $\varphi$ be a mapping between the set of left cosets of $H$ in
  $G$ to the set of right cosets, given by
  \begin{equation*}
    \varphi(xH) = Hx^{-1}.
  \end{equation*}
  First we show that $\varphi$ is well defined. Suppose $x$ and $y$
  are representatives of the same left coset $gH$. Then $x = gh_1$ and
  $y = gh_2$ for some $h_1,h_2\in H$. So
  $x^{-1} = h_1^{-1}g^{-1}\in Hg^{-1}$ and
  $y^{-1} = h_2^{-1}g^{-1}\in Hg^{-1}$, and we see that $\varphi$
  sends both $xH$ and $yH$ to the same right coset $Hg^{-1}$ and is
  therefore well defined.

  To show that $\varphi$ is a bijection, we simply exhibit a two-sided
  inverse function. Let $\psi$ send the set of right cosets of $H$ in
  $G$ onto the set of left cosets via the map $Hx \mapsto x^{-1}H$. By
  the same argument as before, $\psi$ is well
  defined. $\psi\circ\varphi$ and $\varphi\circ\psi$ are obviously the
  identity, so $\psi = \varphi^{-1}$ and $\varphi$ is a bijection. It
  follows that the number of left cosets of $H$ in $G$ is equal to the
  number of right cosets.
\end{proof}

\Exercise{13} Fix any labelling of the vertices of a square and use
this to identify $D_8$ as a subgroup of $S_4$. Prove that the elements
of $D_8$ and $\gen{(1\,2\,3)}$ do not commute in $S_4$.
\begin{proof}
  Let the vertices of a square be labelled $1,2,3,4$ in a clockwise
  fashion. Then every element in $D_8$ induces a distinct permutation
  of these vertices. It is easy to see that these permutations form a
  subgroup of $S_4$. $r$ is identified with $(1\,2\,3\,4)$ and $s$ is
  identified with $(2\,4)$.

  We have
  \begin{equation*}
    (1\,2\,3\,4)(1\,2\,3)
    = (1\,3\,2\,4)
    \neq (1\,3\,4\,2)
    = (1\,2\,3)(1\,2\,3\,4)
  \end{equation*}
  and
  \begin{equation*}
    (2\,4)(1\,2\,3)
    = (1\,4\,2\,3)
    \neq (1\,2\,4\,3)
    = (1\,2\,3)(2\,4).
  \end{equation*}
  Since the generators of $D_8$ and the generator of $\gen{(1\,2\,3)}$
  do not commute, we see that the elements in the two subgroups do not
  in general commute with one another.
\end{proof}

\Exercise{14} Prove that $S_4$ does not have a normal subgroup of
order $8$ or a normal subgroup of order $3$.
\begin{proof}
  Suppose $S_4$ has a normal subgroup $H$ of order $8$. Now, consider
  that the elements $(1\,2\,3\,4)$ and $(1\,2\,4\,3)$ cannot both be
  in $H$ since they generate all of $S_4$, as we showed in
  Exercise~\ref{exercise:S4-generated-by-two-4cycles}. But
  \begin{equation*}
    (1\,2\,3\,4) = (1\,4)(3\,2)(1\,3)
    \quad\text{and}\quad
    (1\,2\,4\,3) = (3\,4)(3\,2)(1\,3),
  \end{equation*}
  so $H$ cannot contain all of the $2$-cycles $(1\,4)$, $(3\,2)$,
  $(3\,4)$, and $(1\,3)$. We see then that $S_4$ contains an element
  $\sigma$ of order $2$ which does not belong to $H$. Therefore
  $H\cap\gen{\sigma} = 1$ and we have by Corollary~15 that
  $H\gen{\sigma}\leq S_4$. And by Proposition~13 we see that
  \begin{equation*}
    \ord{H\gen{\sigma}}
    = \frac{\ord{H}\ord{\gen{\sigma}}}{\ord{H\cap\gen{\sigma}}}
    = 8\cdot2 = 16.
  \end{equation*}
  Now $S_4$, a group of order $24$, has a subgroup of order $16$,
  which contradicts Lagrange's Theorem. Therefore $H$ does not exist:
  there is no normal subgroup of order $8$ in $S_4$.

  Next, suppose that $S_4$ has a normal subgroup $K$ of order $3$. We
  know by Corollary~10 that any subgroup of order $3$ is cyclic. Now,
  $S_4$ has more than subgroup of order $3$, for example
  \begin{equation*}
    \gen{(1\,2\,3)} = \{1, (1\,2\,3), (1\,3\,2)\}
    \quad\text{and}\quad
    \gen{(2\,3\,4)} = \{1, (2\,3\,4), (2\,4\,3)\}.
  \end{equation*}
  So $K$ has a trivial intersection with some subgroup $\gen{\tau}$ of
  order $3$. Then since $K$ is normal we know $K\gen{\tau}\leq S_4$
  having order $3\cdot3 = 9$, but this is impossible. Therefore $S_4$
  has no normal subgroup of order $3$.
\end{proof}

\Exercise{15} Let $G = S_n$ and for fixed $i\in\{1,2,\dots,n\}$ let
$G_i$ be the stabilizer of $i$. Prove that $G_i\cong S_{n-1}$.
\begin{proof}
  Let $G$ act on $\{1,2,\dots,n\}$ and fix some $i$ from this latter
  set. Now suppose $\sigma\in G_i$. We can always write $\sigma$ as a
  product of disjoint cycles using the Cycle Decomposition Algorithm
  presented in Section~1.3. Each of the cycles in the cycle
  decomposition of sigma must not contain $i$, since $i$ needs to be
  stabilized. Therefore $G_i$ consists of all permutations of the set
  $\{1,2,\dots,n\} - \{i\}$, that is it is the permutations of a set
  with $n-1$ elements. And it has been shown that, for finite sets $A$
  and $B$, $S_A\cong S_B$ when $\ord{A} = \ord{B}$. Therefore
  $G_i\cong S_{n-1}$.
\end{proof}

\Exercise{16} Use Lagrange's Theorem in the multiplicative group
$(\Z/p\Z)^\times$ to prove {\em Fermat's Little Theorem}: if $p$ is a
prime then $a^p\equiv a\pmod{p}$ for all $a\in\Z$.
\begin{proof}
  Let $p$ be a prime. Then
  \begin{equation*}
    (\Z/p\Z)^\times = \{\bar1,\bar2,\bar3,\dots,\overline{p-1}\}.
  \end{equation*}
  Now either $a$ is a multiple of $p$ or not. If not, then
  $\bar{a}\in(\Z/p\Z)^\times$ and by Lagrange's Theorem and
  Corollary~9, we know that $\ord{\bar{a}}$ must divide
  $p-1$. Therefore $\bar{a}^{p-1} = \bar1$, so
  \begin{equation*}
    a^{p-1}\equiv 1\pmod{p},
  \end{equation*}
  and multiplying both sides by $a$ gives the desired result.

  The other possibility is that $a$ and $p$ are not relatively
  prime. In this case, we have
  \begin{equation*}
    a^p\equiv0\equiv a\pmod{p},
  \end{equation*}
  and the result still holds.
\end{proof}

\Exercise{17} Let $p$ be a prime and let $n$ be a positive
integer. Find the order of $\bar{p}$ in
\begin{equation*}
  (\Z/(p^n-1)\Z)^\times
\end{equation*}
and deduce that $n\mid\varphi(p^n-1)$ (here $\varphi$ is Euler's
function).
\begin{solution}
  Note that $p^n\equiv1\pmod{p^n-1}$ so $\ord{\bar p}\leq n$. On the
  other hand, if $p^k\equiv1\pmod{p^n-1}$, then $(p^n-1)\mid(p^k-1)$
  and we see that $k\geq n$. Therefore $\ord{\bar p} = n$.

  Since $(\Z/(p^n-1)\Z)^\times$ has order $\varphi(p^n - 1)$, we know
  by Corollary~9 that $n\mid\varphi(p^n-1)$.
\end{solution}
