\chapter{Quotient Groups and Homomorphisms}

\section{Definitions and Examples}

Let $G$ and $H$ be groups.

\Exercise1 Let $\varphi\colon G\to H$ be a homomorphism and let $E$ be
a subgroup of $H$. Prove that $\varphi^{-1}(E)\leq G$ (i.e., the
preimage or pullback of a subgroup under a homomorphism is a
subgroup). If $E\trianglelefteq H$ prove that
$\varphi^{-1}(E)\trianglelefteq G$. Deduce that
$\ker\varphi\trianglelefteq G$.
\begin{proof}
  Note that $\varphi(1) = 1\in E$ so $\varphi^{-1}(E)$ is
  nonempty. Suppose $a, b \in \varphi^{-1}(E)$, so that
  $\varphi(a) = x$ and $\varphi(b) = y$ for some $x,y\in E$. Then,
  since $\varphi$ is a homomorphism, we have
  \begin{equation*}
    \varphi(ab^{-1}) = \varphi(a)\varphi(b)^{-1} = xy^{-1} \in E,
  \end{equation*}
  which shows that $ab^{-1}\in\varphi^{-1}(E)$. By the subgroup
  criterion, this shows that $\varphi^{-1}(E)\leq G$.

  Now suppose that $E$ is a normal subgroup of $H$. Let $g\in G$ and
  $n\in\varphi^{-1}(E)$. Then $\varphi(g) = h$ for some $h\in H$ and
  $\varphi(n) = x$ for some $x\in E$. We have
  \begin{align*}
    \varphi(gng^{-1})
    &= \varphi(g)\varphi(n)\varphi(g)^{-1} \\
    &= hxh^{-1}.
  \end{align*}
  But $hxh^{-1}\in E$ since $E\trianglelefteq H$, so
  $gng^{-1}\in\varphi^{-1}(E)$. The choice of $g$ and $n$ were
  arbitrary, so this shows that $\varphi^{-1}(E)\trianglelefteq G$.

  Lastly, if we let $E$ be the trivial subgroup of $H$, then the above
  shows that $\ker\varphi = \varphi^{-1}(E) \trianglelefteq G$ since
  the trivial subgroup is always normal.
\end{proof}

\Exercise2 Let $\varphi\colon G\to H$ be a homomorphism of groups with
kernel $K$ and let $a,b\in\varphi(G)$. Let $X\in G/K$ be the fiber
above $a$ and let $Y$ be the fiber above $b$, i.e.,
$X = \varphi^{-1}(a)$, $Y = \varphi^{-1}(b)$. Fix an element $u$ of
$X$ (so $\varphi(u) = a$). Prove that if $XY = Z$ in the quotient
group $G/K$ and $w$ is any member of $Z$, then there is some $v\in Y$
such that $uv = w$.
\begin{proof}
  Let $v = u^{-1}w$. We want to show that $v\in Y$, or
  $\varphi(v) = b$. Since $\varphi$ is a homomorphism and since
  $Z = \varphi^{-1}(ab)$, we have
  \begin{align*}
    \varphi(v) &= \varphi(u^{-1}w) \\
               &= \varphi(u)^{-1}\varphi(w) \\
               &= a^{-1}(ab) \\
               &= (a^{-1}a)b \\
               &= b.
  \end{align*}
  So $v\in Y$ as required.
\end{proof}

\Exercise3 Let $A$ be an abelian group and let $B$ be a subgroup of
$A$. Prove that $A/B$ is abelian. Give an example of a non-abelian
group $G$ containing a proper normal subgroup $N$ such that $G/N$ is
abelian.
\begin{solution}
  Let $a_1B, a_2B \in A/B$, where $a_1,a_2\in A$. Since $A$ is
  abelian, we have
  \begin{equation*}
    a_1Ba_2B = (a_1a_2)B = (a_2a_1)B = a_2Ba_1B.
  \end{equation*}
  Therefore $A/B$ is abelian.

  For the second part of the problem, let $G$ be the non-abelian
  dihedral group $D_8$ and let $N$ the proper normal subgroup
  $\gen{r^2}$. In the text it was shown that $G/N \cong V_4$, the
  Klein four-group, which is abelian. Therefore $G/N$ is abelian even
  though $G$ is not.
\end{solution}
