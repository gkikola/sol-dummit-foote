\chapter{Quotient Groups and Homomorphisms}

\section{Definitions and Examples}

Let $G$ and $H$ be groups.

\Exercise1 Let $\varphi\colon G\to H$ be a homomorphism and let $E$ be
a subgroup of $H$. Prove that $\varphi^{-1}(E)\leq G$ (i.e., the
preimage or pullback of a subgroup under a homomorphism is a
subgroup). If $E\trianglelefteq H$ prove that
$\varphi^{-1}(E)\trianglelefteq G$. Deduce that
$\ker\varphi\trianglelefteq G$.
\begin{proof}
  Note that $\varphi(1) = 1\in E$ so $\varphi^{-1}(E)$ is
  nonempty. Suppose $a, b \in \varphi^{-1}(E)$, so that
  $\varphi(a) = x$ and $\varphi(b) = y$ for some $x,y\in E$. Then,
  since $\varphi$ is a homomorphism, we have
  \begin{equation*}
    \varphi(ab^{-1}) = \varphi(a)\varphi(b)^{-1} = xy^{-1} \in E,
  \end{equation*}
  which shows that $ab^{-1}\in\varphi^{-1}(E)$. By the subgroup
  criterion, this shows that $\varphi^{-1}(E)\leq G$.

  Now suppose that $E$ is a normal subgroup of $H$. Let $g\in G$ and
  $n\in\varphi^{-1}(E)$. Then $\varphi(g) = h$ for some $h\in H$ and
  $\varphi(n) = x$ for some $x\in E$. We have
  \begin{align*}
    \varphi(gng^{-1})
    &= \varphi(g)\varphi(n)\varphi(g)^{-1} \\
    &= hxh^{-1}.
  \end{align*}
  But $hxh^{-1}\in E$ since $E\trianglelefteq H$, so
  $gng^{-1}\in\varphi^{-1}(E)$. The choice of $g$ and $n$ were
  arbitrary, so this shows that $\varphi^{-1}(E)\trianglelefteq G$.

  Lastly, if we let $E$ be the trivial subgroup of $H$, then the above
  shows that $\ker\varphi = \varphi^{-1}(E) \trianglelefteq G$ since
  the trivial subgroup is always normal.
\end{proof}

\Exercise2 Let $\varphi\colon G\to H$ be a homomorphism of groups with
kernel $K$ and let $a,b\in\varphi(G)$. Let $X\in G/K$ be the fiber
above $a$ and let $Y$ be the fiber above $b$, i.e.,
$X = \varphi^{-1}(a)$, $Y = \varphi^{-1}(b)$. Fix an element $u$ of
$X$ (so $\varphi(u) = a$). Prove that if $XY = Z$ in the quotient
group $G/K$ and $w$ is any member of $Z$, then there is some $v\in Y$
such that $uv = w$.
\begin{proof}
  Let $v = u^{-1}w$. We want to show that $v\in Y$, or
  $\varphi(v) = b$. Since $\varphi$ is a homomorphism and since
  $Z = \varphi^{-1}(ab)$, we have
  \begin{align*}
    \varphi(v) &= \varphi(u^{-1}w) \\
               &= \varphi(u)^{-1}\varphi(w) \\
               &= a^{-1}(ab) \\
               &= (a^{-1}a)b \\
               &= b.
  \end{align*}
  So $v\in Y$ as required.
\end{proof}

\Exercise3 Let $A$ be an abelian group and let $B$ be a subgroup of
$A$. Prove that $A/B$ is abelian. Give an example of a non-abelian
group $G$ containing a proper normal subgroup $N$ such that $G/N$ is
abelian.
\begin{solution}
  Let $a_1B, a_2B \in A/B$, where $a_1,a_2\in A$. Since $A$ is
  abelian, we have
  \begin{equation*}
    a_1Ba_2B = (a_1a_2)B = (a_2a_1)B = a_2Ba_1B.
  \end{equation*}
  Therefore $A/B$ is abelian.

  For the second part of the problem, let $G$ be the non-abelian
  dihedral group $D_8$ and let $N$ the proper normal subgroup
  $\gen{r^2}$. In the text it was shown that $G/N \cong V_4$, the
  Klein four-group, which is abelian. Therefore $G/N$ is abelian even
  though $G$ is not.
\end{solution}

\Exercise4 Prove that in the quotient group $G/N$,
$(gN)^\alpha = g^\alpha N$ for all $\alpha\in\Z$.
\begin{proof}
  First, if $\alpha = 0$, then $(gN)^0 = 1N = g^0N$, so the statement
  is true in this case. We also know by Proposition~5 that
  $(gN)^{-1} = g^{-1}N$. So it will suffice to prove the statement
  only for positive $\alpha$, which we will do by induction on
  $\alpha$.

  The case where $\alpha = 1$ is trivial. For the inductive step,
  suppose that the statement $(gN)^k = g^kN$ holds for some particular
  $k\geq1$. Then
  \begin{align*}
    (gN)^{k+1}
    &= (gN)^kgN \\
    &= g^kNgN \\
    &= (g^kg)N \\
    &= g^{k+1}N.
  \end{align*}
  By induction, we conclude that $(gN)^\alpha = g^\alpha N$ for all
  $\alpha\geq1$, so the proof is complete.
\end{proof}

\Exercise5 Use the preceding exercise to prove that the order of the
element $gN$ in $G/N$ is $n$, where $n$ is the smallest positive
integer such that $g^n\in N$ (and $gN$ has infinite order if no such
positive integer exists). Give an example to show that the order of
$gN$ in $G/N$ may be strictly smaller than the order of $g$ in $G$.
\begin{solution}
  Fix an element $gN$ in $G/N$. First, if possible, let $n$ be the
  smallest positive integer such that $g^n\in N$. Then $g^nN =
  1N$. So, by the previous exercise, we know that $(gN)^n = 1N$. This
  shows that $\ord{gN}\leq n$. On the other hand, if $m$ is any
  positive integer with $(gN)^m = 1N$ then, using the previous
  exercise again, $g^mN = 1N$ so that $g^m\in N$. Since $n$ is the
  smallest positive integer with $g^n\in N$, this shows that
  $\ord{gN}\geq n$, which completes the proof for the case of finite
  order.

  Next, suppose that there is no such $n$. Then for each positive
  integer $k$, $g^k\not\in N$. If $gN$ were to have finite order, say
  $(gN)^m = 1N$, then the previous exercise would show that
  $g^m\in N$, giving a contradiction. This shows that $gN$ has
  infinite order, which completes the proof.

  Lastly, for the example, consider $G = Z_4$, the cyclic group of
  order $4$. Let $x$ be a generator for $G$ and take
  $N = \gen{x^2} = \{1, x^2\}$. Now the element $x^2$ has order $2$ in
  $G$, but $x^2N = 1N$ has order $1$ in $G/N$.
\end{solution}

\Exercise6 Define $\varphi\colon\R^\times\to\{\pm1\}$ by letting
$\varphi(x)$ be $x$ divided by the absolute value of $x$. Describe the
fibers of $\varphi$ and prove that $\varphi$ is a homomorphism.
\begin{solution}
  The fiber above $1$ is the positive reals, and the fiber above $-1$
  is the negative reals.

  Let $x,y\in\R^\times$ be arbitrary. Then
  \begin{equation*}
    \varphi(xy)
    = \frac{xy}{\abs{xy}}
    = \frac{x}{\abs{x}}\cdot\frac{y}{\abs{y}}
    = \varphi(x)\varphi(y),
  \end{equation*}
  so $\varphi$ is a homomorphism.
\end{solution}

\Exercise7 Define $\pi\colon\R^2\to\R$ by $\pi((x,y)) = x + y$. Prove
that $\pi$ is a surjective homomorphism and describe the kernel and
fibers of $\pi$ geometrically.
\begin{solution}
  For any $(x_1,y_1), (x_2,y_2)\in\R^2$, we have
  \begin{align*}
    \pi((x_1,y_1) + (x_2,y_2))
    &= \pi((x_1 + x_2, y_1 + y_2)) \\
    &= (x_1 + x_2) + (y_1 + y_2) \\
    &= (x_1 + y_1) + (x_2 + y_2) \\
    &= \pi((x_1,y_1)) + \pi((x_2,y_2)),
  \end{align*}
  so $\pi$ is a homomorphism. And for any $x\in\R$, we have
  $\pi((x, 0)) = x + 0 = x$, so $\pi$ is also surjective.

  $\ker\pi$ is simply the diagonal line whose equation is $x + y =
  0$. And for $a\in\R$, the fiber over $a$ is the line with equation
  $x + y = a$, which is just a translate of the kernel.
\end{solution}

\Exercise8 Let $\varphi\colon\R^\times\to\R^\times$ be the map sending
$x$ to the absolute value of $x$. Prove that $\varphi$ is a
homomorphism and find the image of $\varphi$. Describe the kernel and
the fibers of $\varphi$.
\begin{solution}
  For any $x,y\in\R^\times$ we have
  \begin{equation*}
    \varphi(xy) = \abs{xy} = \abs{x}\abs{y} = \varphi(x)\varphi(y)
  \end{equation*}
  and $\varphi$ is a homomorphism. Its image is $\R^+$, the positive
  reals.

  The kernel of $\varphi$ is the set $\{-1,1\}$, since
  $\varphi(\pm1) = 1$ and no other real number has an absolute value
  of $1$. Likewise, the fiber over the real number $a$ is $\{-a, a\}$.
\end{solution}

\Exercise9 Define $\varphi\colon\C^\times\to\R^\times$ by
$\varphi(a + bi) = a^2 + b^2$. Prove that $\varphi$ is a homomorphism
and find the image of $\varphi$. Describe the kernel and the fibers of
$\varphi$ geometrically (as subsets of the plane).
\begin{solution}
  Let $a + bi$ and $c + di$ be any members of $\C^\times$. Then
  \begin{align*}
    \varphi((a + bi)(c + di))
    &= \varphi((ac - bd) + (ad + bc)i) \\
    &= (ac - bd)^2 + (ad + bc)^2 \\
    &= a^2c^2 - 2abcd + b^2d^2 + a^2d^2 + 2abcd + b^2c^2 \\
    &= a^2(c^2 + d^2) + b^2(c^2 + d^2) \\
    &= (a^2 + b^2)(c^2 + d^2) \\
    &= \varphi(a + bi)\varphi(c + di),
  \end{align*}
  and $\varphi$ is a homomorphism. Let $a+bi\in\C^\times$. Since $a$
  and $b$ cannot both be zero, $a^2 + b^2 > 0$. So
  $\im\varphi\subseteq\R^+$. But for any $a\in\R^+$, we have
  $\varphi(\sqrt{a} + 0i) = a$ and we see that $\im\varphi = \R^+$.

  The kernel of $\varphi$ is the set
  \begin{equation*}
    \ker\varphi = \{a + bi\in\C^\times \mid a^2 + b^2 = 1\}.
  \end{equation*}
  This is a circle of radius 1 centered at the origin in the complex
  plane. For $a\in\R^+$, the fiber over $a$ is the circle of radius
  $\sqrt{a}$.
\end{solution}
