\section{The Isomorphism Theorems}

Let $G$ be a group.

\Exercise1 Let $F$ be a finite field of order $q$ and let
$n\in\Z^+$. Prove that
\begin{equation*}
  \ord{GL_n(F):SL_n(F)} = q - 1.
\end{equation*}
\begin{proof}
  In Exercise~\ref{exercise:quotient-group:GL-mod-SL} we saw that
  $GL_n(F)/SL_n(F)\cong F^\times$. Therefore
  \begin{equation*}
    \ord{GL_n(F):SL_n(F)}
    = \ord{GL_n(F)/SL_n(F)}
    = \ord{F^\times}
    = q - 1,
  \end{equation*}
  since $F^\times$ consists of all members of $F$ excluding the $0$
  element.
\end{proof}

\Exercise2 Prove all parts of the Lattice Isomorphism Theorem.
\begin{proof}
  First we show that there is a bijection from the set $\mathcal{A}$
  of subgroups $A$ of $G$ containing $N$ onto the set
  $\overline{\mathcal{A}}$ of subgroups $\overline{A} = A/N$ of
  $G/N$. Let $\pi\colon G\to G/N$ be the natural projection of $G$
  onto $G/N$. Then define the map
  $\Phi\colon\mathcal{A}\to\overline{\mathcal{A}}$ by
  \begin{equation*}
    \Phi(A) = \pi(A) = \{aN \mid a\in A\}.
  \end{equation*}
  That $\Phi(A)\leq G/N$ for any $A\leq G$ is easy to check: $\Phi(A)$
  is nonempty since it includes $1N$, and if $a,b\in A$, then
  \begin{equation*}
    (aN)(bN)^{-1} = (ab^{-1})N\in\Phi(A).
  \end{equation*}

  To show $\Phi$ is injective, suppose $\Phi(A) = \Phi(B)$. Let
  $a\in A$. Then $\pi(a) = \pi(b)$ for some $b\in B$, so
  $b^{-1}a\in N$ and $a\in bN$. Since $N\leq B$, this shows that
  $a\in B$ so that $A\subseteq B$. A similar argument will show that
  $A\supseteq B$ and so $A = B$.

  To see that $\Phi$ is surjective, let $\overline{A} = A/N$ be a
  subgroup of $G/N$. We saw in
  Exercise~\ref{exercise:quotient-group:preimage-of-hom-is-subgroup}
  that the complete preimage of a subgroup in $G/N$ is a subgroup of
  $G$, so there is $A\in\mathcal{A}$ such that
  $\Phi(A) = \overline{A}$.

  We have shown that $\Phi$ is a bijection. Now suppose $A,B\leq G$
  with $N\leq A$ and $N\leq B$.
  \begin{enumerate}
  \item $A\leq B$ if and only if $\overline{A}\leq\overline{B}$.

    If $A\leq B$, then every coset of $N$ in $A$ is clearly also a
    coset of $N$ in $B$, so that $\overline{A}\leq\overline{B}$. On
    the other hand, if $\overline{A}\leq\overline{B}$ then for any
    $a\in A$, we have $aN\in\overline{B}$ so that $b^{-1}a\in N$ for
    some $b\in B$, which implies $a\in bN\subseteq B$, so $A\leq B$.
  \item If $A\leq B$, then
    $\ord{B:A} = \ord{\overline{B}:\overline{A}}$.

    Define the map $\psi\colon B/A\to\overline{B}/\overline{A}$ by
    \begin{equation*}
      \psi(bA) = (bN)\overline{A} = \bar{b}\overline{A}.
    \end{equation*}

    First we show that $\psi$ is well defined. Suppose $b_1A = b_2A$,
    so that $b_1$ and $b_2$ are representatives of the same coset of
    $A$ in $B$. Then $b_2^{-1}b_1\in A$ so
    $(b_2^{-1}b_1)N \in \overline{A}$. This implies that
    $(b_1N)\overline{A} = (b_2N)\overline{A}$, or
    $\overline{b_1}\overline{A} = \overline{b_2}\overline{A}$ as
    required.

    Next, we show injectivity. If $\psi(b_1A) = \psi(b_2A)$ then
    $(b_1N)\overline{A} = (b_2N)\overline{A}$ so that
    $(b_2^{-1}b_1)N\in\overline{A}$ or $b_2^{-1}b_1N = aN$ for some
    $a\in A$. Then $b_2^{-1}b_1\in A$ so $b_1A = b_2A$.

    Finally, surjectivity is clear, since any coset $(bN)\overline{A}$
    of $\overline{A}$ in $\overline{B}$ can be obtained from the image
    of the coset $bA$.

    We have shown that there is a bijection between the elements of
    $B/A$ and the elements of $\overline{B}/\overline{A}$, so
    $\ord{B:A} = \ord{\overline{B}:\overline{A}}$.
  \item $\overline{\gen{A, B}} = \gen{\overline{A}, \overline{B}}$.

    $xN\in\overline{\gen{A,B}}$ if and only if $x\in\gen{A,B}$, if and
    only if
    \begin{equation*}
      x = x_1x_2\cdots x_n,
      \quad\text{where $x_i\in A\cup B$ for each $i$}.
    \end{equation*}
    But this is true if and only if
    \begin{equation*}
      xN = (x_1N)(x_2N)\cdots(x_nN),
      \quad\text{$x_iN\in\overline{A}\cup\overline{B}$ for each $i$},
    \end{equation*}
    if and only if $xN\in\gen{\overline{A},\overline{B}}$. Therefore
    $\overline{\gen{A,B}} = \gen{\overline{A},\overline{B}}$.
  \item $\overline{A\cap B} = \overline{A}\cap\overline{B}$.

    $xN \in \overline{A\cap B}$ if and only if $x\in A\cap B$ if and
    only if $x\in A$ and $x\in B$, and this is true if and only if
    $xN\in\overline{A}$ and $xN\in\overline{B}$ or
    $xN\in\overline{A}\cap\overline{B}$.
  \item $A\trianglelefteq G$ if and only if
    $\overline{A}\trianglelefteq\overline{G}$.

    Suppose $A\trianglelefteq G$. Then if $g\in G$ and $a\in A$, we
    have $gag^{-1}\in A$, and
    \begin{equation*}
      (gN)(aN)(g^{-1}N)
      = (gag^{-1})N
      \in\overline{A}.
    \end{equation*}
    Therefore $\overline{A}\trianglelefteq\overline{G}$.

    Conversely, suppose
    $\overline{A}\trianglelefteq\overline{G}$. Then if
    $\bar{g}\in\overline{G}$ and $\bar{a}\in\overline{A}$, we have
    \begin{equation*}
      \bar{g}\bar{a}\bar{g}^{-1}
      = (gag^{-1})N \in \overline{A},
    \end{equation*}
    so $gag^{-1}\in A$. Hence $A\trianglelefteq G$. \qedhere
  \end{enumerate}
\end{proof}

\Exercise3 Prove that if $H$ is a normal subgroup of $G$ of prime
index $p$ then for all $K\leq G$ either
\begin{enumerate}
\item $K\leq H$ or
\item $G = HK$ and $\ord{K:K\cap H} = p$.
\end{enumerate}
\begin{proof}
  Let $H$ have prime index $p$ as stated. Since $K\leq N_G(H) = G$, we
  may apply the Second Isomorphism Theorem to see that $KH\leq G$ and
  $H\trianglelefteq KH$. And $KH = HK$ by Proposition~14. Now consider
  the index of $HK$ in $G$.

  We know by
  Exercise~\ref{exercise:quotient-group:subgroup-of-subgroup-index}
  that
  \begin{equation*}
    \ord{G:H} = \ord{G:HK}\cdot\ord{HK:H}.
  \end{equation*}
  But $\ord{G:H}$ is prime, so there are only two possibilities for
  $\ord{G:HK}$: Either $HK$ has index $1$, in which case $HK = G$, or
  $\ord{G:HK} = p$. In the latter case, $\ord{HK:H} = 1$ so $H = HK$
  which implies that $K\leq H$.

  So either $K\leq H$ or $G = HK$. And if $G = HK$, then the Second
  Isomorphism Theorem tells us that $K/(H\cap K)\cong HK/H$, so
  \begin{equation*}
    \ord{K:H\cap K} = \ord{HK:H} = \ord{G:H} = p. \qedhere
  \end{equation*}
\end{proof}
