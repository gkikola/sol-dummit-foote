\section{Transpositions and the Alternating Group}

\Exercise1 In Exercises~\ref{exercise:intro-group:perm1} and
\ref{exercise:intro-group:perm2} of Section~1.3 you were asked to find
the cycle decomposition of some permutations. Write each of these
permutations as a product of transpositions. Determine which of these
is an even permutation and which is an odd permutation.
\begin{solution}
  From Exercise~\ref{exercise:intro-group:perm1}, we have
  \begin{align*}
    \sigma &= (1\,3\,5)(2\,4) = (1\,5)(1\,3)(2\,4) \\
    \tau &= (1\,5)(2\,3) \\
    \sigma^2 &= (1\,5\,3) = (1\,3)(1\,5) \\
    \sigma\tau &= (2\,5\,3\,4) = (2\,4)(2\,3)(2\,5) \\
    \tau\sigma &= (1\,2\,4\,3) = (1\,3)(1\,4)(1\,2) \\
    \intertext{and}
    \tau^2\sigma &= (1\,3\,5)(2\,4) = (1\,5)(1\,3)(2\,4).
  \end{align*}
  Of these, we can see that $\tau$ and $\sigma^2$ are even, and the
  rest are odd.

  From Exercise~\ref{exercise:intro-group:perm2}, we get
  \begin{align*}
    \sigma &= (1\;13\;5\;10)(3\;15\;8)(4\;14\;11\;7\;12\;9) \\
           &= (1\;10)(1\;5)(1\;13)(3\;8)(3\;15)
             (4\;9)(4\;12)(4\;7)(4\;11)(4\;14) \\
    \tau &= (1\;14)(2\;9\;15\;13\;4)(3\;10)(5\;12\;7)(8\;11) \\
           &= (1\;14)(2\;4)(2\;13)(2\;15)(2\;9)(3\;10)
             (5\;7)(5\;12)(8\;11) \\
    \sigma^2 &= (1\;5)(3\;8\;15)(4\;11\;12)(7\;9\;14)(10\;13) \\
           &= (1\;5)(3\;15)(3\;8)(4\;12)(4\;11)(7\;14)(7\;9)(10\;13) \\
    \sigma\tau &= (1\;11\;3)(2\;4)(5\;9\;8\;7\;10\;15)(13\;14) \\
           &= (1\;3)(1\;11)(2\;4)(5\;15)(5\;10)(5\;7)(5\;8)(5\;9)(13\;14) \\
    \tau\sigma &= (1\;4)(2\;9)(3\;13\;12\;15\;11\;5)(8\;10\;14) \\
           &= (1\;4)(2\;9)(3\;5)(3\;11)(3\;15)(3\;12)(3\;13)(8\;14)(8\;10) \\
    \intertext{and}
    \tau^2\sigma &= (1\;2\;15\;8\;3\;4\;14\;11\;12\;13\;7\;5\;10) \\
           &= (1\;10)(1\;5)(1\;7)(1\;13)(1\;12)(1\;11)(1\;14)
             (1\;4)(1\;3)(1\;8)(1\;15)(1\;2).
  \end{align*}
  From these, we see that only $\sigma$, $\sigma^2$ and $\tau^2\sigma$
  are even. The rest are odd.
\end{solution}

\Exercise2 Prove that $\sigma^2$ is an even permutation for every
permutation $\sigma$.
\begin{proof}
  We know that $\sigma$ can be written as a product of
  transpositions. Let it be written as a product of $n$
  transpositions,
  \begin{equation*}
    \sigma = \sigma_1\sigma_2\cdots\sigma_n,
  \end{equation*}
  where each $\sigma_i$ is a transposition. Then we can write
  \begin{equation*}
    \sigma^2 = (\sigma_1\cdots\sigma_n)(\sigma_1\cdots\sigma_n).
  \end{equation*}
  Since $\sigma^2$ has been written as a product of $2n$
  transpositions (i.e., an even number of transpositions), $\sigma^2$
  is even.
\end{proof}

\Exercise3 \label{exercise:quotient-group:S-n-gen-adj}
Prove that $S_n$ is generated by
$\{(i\;\;i+1) \mid 1\leq i\leq n - 1\}$.
\begin{proof}
  We use induction on $n$. For $n = 1$, $S_n$ is trivial and is
  generated by the empty set. Now suppose the statement is true for
  $S_n$, where $n$ is some positive integer, and consider $S_{n+1}$.

  For each $n$, let
  \begin{equation*}
    T_n = \{(i\;\;i+1) \mid 1\leq i\leq n-1\}.
  \end{equation*}
  We want to show that every permutation in $S_{n+1}$ can be written
  as a product of members of $T_{n+1}$. Let $\sigma\in S_{n+1}$, and
  write $\sigma$ as a product of transpositions.

  For each transposition $(j\;k)$ with $j < k$, there are two
  cases. First, if $1\leq j < k \leq n$, then the transposition
  belongs to $S_n$ and by the induction hypothesis can be written as a
  product of members of $T_n$. In this case, we are done.

  The remaining case is where $k = n+1$. If $j = n$, then $(j\;k)$ is
  already in $T_{n+1}$ and we are done, so suppose $j < n$. Let
  \begin{equation*}
    \tau = (k - 1\;\;k)(j\;\;k-1)(k-1\;\;k).
  \end{equation*}
  Notice that $\tau(j) = k$ and $\tau(k) = j$, with all other values
  fixed. Then $(j\;\;k-1)$ can, by the inductive hypothesis, be
  written as a product of members of $T_n$. As $(k-1\;\;k)$ belongs to
  $T_{n+1}$, this completes the inductive step. By induction, the
  result holds for all positive integers $n$.
\end{proof}

\Exercise4
\label{exercise:quotient-group:S-n-gen-12-123n}
Show that $S_n = \gen{(1\,2), (1\,2\,3\,\dots\,n)}$ for all $n\geq2$.
\begin{proof}
  Note that, for $i$ with $1\leq i\leq n-2$,
  \begin{equation*}
    (1\,2\,3\,\dots\,n)(i\;\;i+1)(1\,2\,3\,\dots\,n)^{-1}
    = (i+1\;\;i+2).
  \end{equation*}
  It follows that
  \begin{equation*}
    \{(i\;\;i+1)\mid1\leq i\leq n-1\}
    \subseteq \gen{(1\,2), (1\,2\,3\,\dots\,n)},
  \end{equation*}
  so by Exercise~\ref{exercise:quotient-group:S-n-gen-adj}, we are
  done.
\end{proof}

\Exercise5 Show that if $p$ is prime, $S_p = \gen{\sigma,\tau}$ where
$\sigma$ is any transposition and $\tau$ is any $p$-cycle.
\begin{proof}
  Let $G = \gen{\sigma,\tau}$. Relabel the elements of $S_p$ so that
  \begin{equation*}
    \sigma = (0\;1\;2\;\dots\;p-1)
    \quad\text{and}\quad
    \tau = (0\;k).
  \end{equation*}
  Now observe that
  \begin{equation*}
    \sigma\tau\sigma^{-1} = (1\;\;k+1),
  \end{equation*}
  where $k+1$ is understood to be the least residue of $k+1$, modulo
  $p$. Similarly,
  \begin{equation*}
    \sigma(1\;\;k+1)\sigma^{-1} = (2\;\;k+2)
  \end{equation*}
  and, in general,
  \begin{equation*}
    \sigma(n-1\;\;k+n-1)\sigma^{-1} = (n\;\;k+n),
  \end{equation*}
  where again all terms are reduced modulo $p$. This shows that
  $(n\;\;k+n)\in G$ for all positive integers $n$.

  Taking $n = k$, we see that $(k\;2k)\in G$. With $n = 2k$, we have
  $(2k\;3k)\in G$, and in general, $((m-1)k\;\;mk)\in G$ for all
  positive integers $m$.

  But
  \begin{equation*}
    (0\;k)(k\;2k)(0\;k) = (0\;2k),
  \end{equation*}
  so $(0\;2k)\in G$. And
  \begin{equation*}
    (0\;2k)(2k\;3k)(0\;2k) = (0\;3k),
  \end{equation*}
  so $(0\;3k)\in G$. Continuing in this way, we see that
  $(0\;nk)\in G$ for all positive integers $n$.

  Since $p$ is prime, $k$ is relatively prime to $p$ and therefore has
  a multiplicative inverse mod $p$. Let $a$ be this inverse, so that
  $ak\equiv1\pmod{p}$. Then we see that $(0\;ak) = (0\;1)\in G$.

  Both $(0\,1)$ and $(0\;1\;2\;\dots\;p-1)$ belong to $G$, so by
  Exercise~\ref{exercise:quotient-group:S-n-gen-12-123n}, $G = S_p$.
\end{proof}

\Exercise6 Show that $\gen{(1\,3), (1\,2\,3\,4)}$ is a proper subgroup
of $S_4$. What is the isomorphism type of this subgroup?
\begin{solution}
  Let $\sigma = (1\,2\,3\,4)$ and $\tau = (1\,3)$. Note that
  $\sigma^4 = \tau^2 = 1$. Moreover,
  \begin{equation*}
    \sigma\tau = (1\,4)(2\,3)
  \end{equation*}
  and
  \begin{equation*}
    \tau\sigma^{-1} = (1\,3)(1\,4\,3\,2) = (1\,4)(2\,3),
  \end{equation*}
  so $\sigma\tau = \tau\sigma^{-1}$ and we see that $\sigma$ and
  $\tau$ satisfy exactly the same relations in $S_4$ as $r$ and $s$ do
  in $D_8$. We can therefore define a surjective homomorphism
  $\varphi\colon D_8\to\gen{\sigma,\tau}$ with
  \begin{equation*}
    \varphi(r) = \sigma
    \quad\text{and}\quad
    \varphi(s) = \tau.
  \end{equation*}
  This shows that $\gen{\sigma,\tau}$ has at most $8$ elements and is
  thus a proper subgroup of $S_4$. However, it is easy to directly
  verify that $\varphi$ maps distinct elements in $D_8$ to distinct
  elements in $\gen{\sigma,\tau}$, so $\varphi$ is an isomorphism and
  $\gen{\sigma,\tau}\cong D_8$.
\end{solution}

\Exercise7
\label{exercise:quotient-group:tetrahedron-rot-iso-A4}
Prove that the group of rigid motions of a tetrahedron is isomorphic
to $A_4$.
\begin{proof}
  In Exercise~\ref{exercise:intro-groups:tetrahedron-sym-iso-sub-S4},
  we showed that the group $G$ of rigid motions of a tetrahedron is
  isomorphic to a subgroup of $S_4$. We will show that this subgroup
  has $12$ elements and consists only of even permutations, so that it
  must be $A_4$.

  We have shown in
  Exercise~\ref{exercise:subgroups:rigid-motion-tetrahedron-12} that
  there are $12$ elements in $G$. One of these is the identity, eight
  more are rotations that fix one vertex and permute the remaining
  three in a $3$-cycle. A $3$-cycle is an even permutation by
  Proposition~25, so each of these elements in $G$ correspond to
  elements in $A_4$.

  Finally, the remaining three rotations in $G$ are $180^\circ$
  rotations leaving no fixed points, each transposing two pairs of
  vertices. A product of two transpositions is an even permutation, so
  these elements of $G$ also correspond to elements in
  $A_4$. Therefore $G\cong A_4$.
\end{proof}

\Exercise8
\label{exercise:quotient-group:A4-lattice}
Prove the lattice of subgroups of $A_4$ given in the text is correct.
\begin{proof}
  We know from the previous exercise,
  Exercise~\ref{exercise:quotient-group:tetrahedron-rot-iso-A4}, that
  $A_4$ is isomorphic to the group of rigid motions of a
  tetrahedron. By the discussion in Section~3.2 of the text, this
  group has no subgroup of order $6$.

  By Exercise~\ref{exercise:classify:groups-4}, any subgroup of order
  $4$ must be isomorphic to either $Z_4$ or $V_4$, where the latter is
  the Klein $4$-group. But we know from the previous exercise that the
  rotational symmetries of the tetrahedron each have order $1$, $2$,
  or $3$, so $A_4$ cannot have a cyclic subgroup of order $4$. On the
  other hand, it is not difficult to verify that
  $\gen{(1\,2)(3\,4), (1\,3)(2\,4)}\cong V_4$, and this is the only
  subgroup of order $4$ since $A_4$ only has three elements of order
  $2$.

  By Lagrange, the only remaining possibilities for nontrivial proper
  subgroups are orders $2$ and $3$. These must be cyclic subgroups,
  and since $A_4$ has three elements of order $2$ and eight elements
  of order $3$, we see that the cyclic subgroups shown in the lattice
  are the only possibilities.

  The indices shown on the lattice are also easy to verify using
  Lagrange's Theorem. Therefore the lattice in the text is correct.
\end{proof}

\Exercise9
\label{exercise:quotient-group:A4-normal-subgroup-order-4}
Prove that the (unique) subgroup of order $4$ in $A_4$ is normal and
is isomorphic to $V_4$.
\begin{proof}
  In the solution to the previous exercise,
  Exercise~\ref{exercise:quotient-group:A4-lattice}, we showed that
  $A_4$ has exactly one subgroup of order $4$, which is isomorphic to
  $V_4$. So we need only show that this subgroup is normal.

  In Exercise~\ref{exercise:quotient-group:conjugate-properties}, we
  proved that for any positive integer $n$, a subgroup of $G$
  generated by all elements of order $n$ must be normal in $G$. The
  subgroup of $A_4$ having order $4$ is generated by all elements of
  order $2$ in $A_4$, namely
  \begin{equation*}
    (1\,2)(3\,4), \quad
    (1\,3)(2\,4), \quad\text{and}\quad
    (1\,4)(2\,3).
  \end{equation*}
  Therefore $\gen{(1\,2)(3\,4), (1\,3)(2\,4)}\trianglelefteq A_4$.
\end{proof}

\Exercise{10} Find a composition series for $A_4$. Deduce that $A_4$
is solvable.
\begin{solution}
  From the previous exercise,
  Exercise~\ref{exercise:quotient-group:A4-normal-subgroup-order-4},
  the subgroup
  \begin{equation*}
    \gen{(1\,2)(3\,4), (1\,3)(2\,4)}
  \end{equation*}
  is normal in $A_4$ and is isomorphic to $V_4$. As $V_4$ is abelian,
  all of its subgroups are normal. Therefore we may take
  \begin{equation*}
    1 \trianglelefteq \gen{(1\,2)(3\,4)} \trianglelefteq
    \gen{(1\,2)(3\,4), (1\,3)(2\,4)} \trianglelefteq A_4
  \end{equation*}
  as a composition series for $A_4$. We know that the quotient groups
  are simple since they each have prime order. They are also abelian
  for the same reason, so $A_4$ is solvable.
\end{solution}

\Exercise{11} Prove that $S_4$ has no subgroup isomorphic to $Q_8$.
\begin{proof}
  If there is such a subgroup, let it be $H$. Note that $S_4$ has
  exactly $6$ elements of order $4$, namely all the $4$-cycles. And
  $Q_8$ has $6$ elements of order $4$, so $H$ must contain all
  $4$-cycles in $S_4$. Then $H$ must also contain both
  \begin{equation*}
    (1\,2\,3\,4)^2 = (1\,3)(2\,4)
  \end{equation*}
  and
  \begin{equation*}
    (1\,2\,4\,3)^2 = (1\,4)(2\,3),
  \end{equation*}
  two elements of order $2$. Along with the identity, this implies
  that $H$ has more than $8$ elements, which is impossible. Therefore
  no such subgroup exists.
\end{proof}

\Exercise{12} Prove that $A_n$ contains a subgroup isomorphic to
$S_{n-2}$ for each $n\geq3$.
\begin{proof}
  Consider the following map $\varphi\colon S_{n-2}\to A_n$:
  \begin{equation*}
    \varphi(\sigma) =
    \begin{cases}
      \sigma, & \text{if $\sigma$ is even} \\
      \sigma\circ(n-1\;\;n) & \text{if $\sigma$ is odd.}
    \end{cases}
  \end{equation*}
  Now let $\sigma$ and $\tau$ be members of $S_{n-2}$. We want to show
  first that
  \begin{equation}
    \label{eq:quotient-group:iso-A-n-sub-S-n-minus-2-hom}
    \varphi(\sigma)\varphi(\tau) = \varphi(\sigma\tau),
  \end{equation}
  so that $\varphi$ is a homomorphism. Certainly
  \eqref{eq:quotient-group:iso-A-n-sub-S-n-minus-2-hom} holds if
  $\sigma$ and $\tau$ are both even. If both are odd, then
  \begin{equation*}
    \sigma\circ(n-1\;\;n)\circ\tau\circ(n-1\;\;n)
    = \sigma\circ\tau\circ(n-1\;\;n)^2
    = \sigma\circ\tau,
  \end{equation*}
  and \eqref{eq:quotient-group:iso-A-n-sub-S-n-minus-2-hom} holds
  since $\sigma\tau$ is even. Lastly, if exactly one of $\sigma$ and
  $\tau$ is odd, then $\sigma\tau$ is odd and
  \begin{equation*}
    \varphi(\sigma)\varphi(\tau) = \sigma\circ\tau\circ(n-1\;\;n)
    = \varphi(\sigma\tau).
  \end{equation*}
  In all cases, \eqref{eq:quotient-group:iso-A-n-sub-S-n-minus-2-hom}
  holds and $\varphi$ is a homomorphism.

  Notice also that $\varphi$ is injective, for if
  $\varphi(\sigma) = \varphi(\tau)$, then either both $\sigma$ and
  $\tau$ are even, in which case $\sigma = \tau$, or both are odd in
  which case multiplication by $(n-1\;\;n)$ again gives
  $\sigma = \tau$.

  Now $\varphi(S_{n-2})$ is the image of a homomorphism into $A_n$ and
  is thus a subgroup of $A_n$. Restricting the codomain of $\varphi$
  to this subgroup then gives an isomorphism. This completes the
  proof.
\end{proof}

\Exercise{13} Prove that every element of order $2$ in $A_n$ is the
square of an element of order $4$ in $S_n$.
\begin{proof}
  Let $\sigma\in A_n$ with $\ord{\sigma} = 2$. Any element of order
  $2$ in $A_n$ is a product of an even number of disjoint
  transpositions. For each pair $(a\;b)$ and $(c\;d)$ of these
  transpositions, we can write
  \begin{equation*}
    (a\;b)(c\;d) = (a\;c\;b\;d)^2.
  \end{equation*}
  Therefore $\sigma$ can be written as a product of squares of
  disjoint $4$-cycles. Since disjoint cycles commute, we can write
  $\sigma$ as the square of an element of order $4$ in $S_n$.
\end{proof}

\Exercise{14} Prove that the subgroup of $A_4$ generated by any
element of order $2$ and any element of order $3$ is all of $A_4$.
\begin{proof}
  This follows from the lattice for $A_4$ which was proven correct in
  Exercise~\ref{exercise:quotient-group:A4-lattice}. Let $\sigma$ be
  any element of order $3$ and $\tau$ any element of order $2$ in
  $A_4$. Then $\gen{\sigma}$ is a proper subgroup of
  $\gen{\sigma,\tau}$. But we see from the lattice that $\gen{\sigma}$
  is maximal. Therefore $\gen{\sigma,\tau} = A_4$.
\end{proof}

\Exercise{15}
\label{exercise:quotient-group:3-cycles-generate-A4}
Prove that if $x$ and $y$ are distinct $3$-cycles in $S_4$ with
$x\neq y^{-1}$, then the subgroup of $S_4$ generated by $x$ and $y$ is
$A_4$.
\begin{proof}
  As with the previous exercise, this follows from the lattice for
  $A_4$. Since every $3$-cycle is even, $x$ and $y$ both belong to
  $A_4$. Since $x$ and $y$ are distinct with $x\neq y^{-1}$, they do
  not both belong to the same cyclic subgroup of $A_4$. Since
  $\gen{x}$ is maximal in $A_4$ (from the lattice), it follows that
  $\gen{x,y} = A_4$.
\end{proof}

\Exercise{16}
\label{exercise:quotient-group:3-cycles-in-S5}
Let $x$ and $y$ be distinct $3$-cycles in $S_5$ with $x\neq y^{-1}$.
\begin{enumerate}
\item Prove that if $x$ and $y$ fix a common element of
  $\{1,\dots,5\}$, then $\gen{x,y}\cong A_4$.
  \begin{proof}
    Suppose $x$ and $y$ fix the element $i$ and let $H$ be the
    subgroup of $S_5$ consisting of permutations which fix this same
    element $i$. Define the bijection
    $\varphi\colon\{1,2,3,4,5\}-\{i\}\to\{1,2,3,4\}$ by
    \begin{equation*}
      \varphi(j) =
      \begin{cases}
        j, & \text{if $j < i$,} \\
        j - 1, & \text{if $j > i$.}
      \end{cases}
    \end{equation*}
    Then $H\cong S_4$ via the isomorphism $\psi\colon H\to S_4$
    defined by
    \begin{equation*}
      \psi(z) = \varphi\circ z\circ\varphi^{-1}.
    \end{equation*}

    Now let $\sigma = \psi(x)$ and $\tau = \psi(y)$. Then $\sigma$ and
    $\tau$ are distinct $3$-cycles in $S_4$ with
    $\sigma\neq\tau^{-1}$. By
    Exercise~\ref{exercise:quotient-group:3-cycles-generate-A4}, we
    have
    \begin{equation*}
      \gen{x,y}\cong\gen{\sigma,\tau} = A_4. \qedhere
    \end{equation*}
  \end{proof}
\item Prove that if $x$ and $y$ do not fix a common element of
  $\{1,\dots,5\}$, then $\gen{x,y} = A_5$.
  \begin{proof}
    Note that $\gen{x,y}\leq A_5$. Since $x$ and $y$ do not fix a
    common element, they can be written in the form
    \begin{equation*}
      x = (a\;b\;c) \quad\text{and}\quad
      y = (a\;d\;e).
    \end{equation*}
    Then
    \begin{equation*}
      xy = (a\;d\;e\;b\;c),
    \end{equation*}
    so $\gen{x,y}$ contains $5$-cycles. Hence $\gen{x,y}$ contains a
    subgroup of order $5$. But
    \begin{equation*}
      xyx^{-1} = (a\;b\;c)(a\;d\;e)(a\;c\;b)
      = (b\;d\;e)
    \end{equation*}
    and
    \begin{equation*}
      yxy^{-1} = (a\;d\;e)(a\;b\;c)(a\;e\;d)
      = (b\;c\;d),
    \end{equation*}
    so $\gen{x,y}$ contains the two $3$-cycles $(b\;d\;e)$ and
    $(b\;c\;d)$. Neither of these is the inverse of the other, so by
    the first part of this exercise, $\gen{x,y}$ contains a subgroup
    that is isomorphic to $A_4$.

    Since $\ord{A_4} = 12$, we have shown that $\gen{x,y}$ contains
    subgroups of order $5$ and order $12$. By Lagrange's Theorem, it
    follows that $\gen{x,y}$ has an order of at least $60$. Therefore
    $\gen{x,y} = A_5$.
  \end{proof}
\end{enumerate}

\Exercise{17} If $x$ and $y$ are $3$-cycles in $S_n$, prove that
$\gen{x,y}$ is isomorphic to $Z_3$, $A_4$, $A_5$, or $Z_3\times Z_3$.
\begin{proof}
  Let $T$ be the set of elements in $\{1,2,\dots,n\}$ that are {\em
    not} fixed by $x$ or by $y$. Then $3\leq\ord{T}\leq6$. We consider
  each case in turn.

  First, if $\ord{T} = 3$ then either $x = y$ or $x = y^{-1}$. Either
  way, $\gen{x,y} = \gen{x} \cong Z_3$.

  Next, if $\ord{T} = 4$ or $5$ then, by relabeling, we can see that
  $\gen{x,y}$ is isomorphic to a subgroup of $S_5$. In both cases,
  Exercise~\ref{exercise:quotient-group:3-cycles-in-S5} shows that
  $\gen{x,y}\cong A_4$ or $\gen{x,y}\cong A_5$.

  The last remaining case is $\ord{T} = 6$. In this case $x$ and $y$
  are disjoint and therefore commute. Every element in $\gen{x,y}$ can
  then be written uniquely in the form
  \begin{equation*}
    x^ay^b, \quad \text{where $0\leq a\leq2$ and $0\leq b\leq2$.}
  \end{equation*}
  Let $z$ be a generator for $Z_3$ and define the map
  $\varphi\colon\gen{x,y}\to Z_3\times Z_3$ by
  \begin{equation*}
    \varphi(x^ay^b) = (z^a, z^b).
  \end{equation*}
  It is not difficult to check that $\varphi$ is an isomorphism, so
  that $\gen{x,y}\cong Z_3\times Z_3$ in this case.

  In each case, the subgroup $\gen{x,y}$ is isomorphic to one of
  $Z_3$, $A_4$, $A_5$, or $Z_3\times Z_3$.
\end{proof}
