\section{Transpositions and the Alternating Group}

\Exercise1 In Exercises~\ref{exercise:intro-group:perm1} and
\ref{exercise:intro-group:perm2} of Section~1.3 you were asked to find
the cycle decomposition of some permutations. Write each of these
permutations as a product of transpositions. Determine which of these
is an even permutation and which is an odd permutation.
\begin{solution}
  From Exercise~\ref{exercise:intro-group:perm1}, we have
  \begin{align*}
    \sigma &= (1\,3\,5)(2\,4) = (1\,5)(1\,3)(2\,4) \\
    \tau &= (1\,5)(2\,3) \\
    \sigma^2 &= (1\,5\,3) = (1\,3)(1\,5) \\
    \sigma\tau &= (2\,5\,3\,4) = (2\,4)(2\,3)(2\,5) \\
    \tau\sigma &= (1\,2\,4\,3) = (1\,3)(1\,4)(1\,2) \\
    \intertext{and}
    \tau^2\sigma &= (1\,3\,5)(2\,4) = (1\,5)(1\,3)(2\,4).
  \end{align*}
  Of these, we can see that $\tau$ and $\sigma^2$ are even, and the
  rest are odd.

  From Exercise~\ref{exercise:intro-group:perm2}, we get
  \begin{align*}
    \sigma &= (1\;13\;5\;10)(3\;15\;8)(4\;14\;11\;7\;12\;9) \\
           &= (1\;10)(1\;5)(1\;13)(3\;8)(3\;15)
             (4\;9)(4\;12)(4\;7)(4\;11)(4\;14) \\
    \tau &= (1\;14)(2\;9\;15\;13\;4)(3\;10)(5\;12\;7)(8\;11) \\
           &= (1\;14)(2\;4)(2\;13)(2\;15)(2\;9)(3\;10)
             (5\;7)(5\;12)(8\;11) \\
    \sigma^2 &= (1\;5)(3\;8\;15)(4\;11\;12)(7\;9\;14)(10\;13) \\
           &= (1\;5)(3\;15)(3\;8)(4\;12)(4\;11)(7\;14)(7\;9)(10\;13) \\
    \sigma\tau &= (1\;11\;3)(2\;4)(5\;9\;8\;7\;10\;15)(13\;14) \\
           &= (1\;3)(1\;11)(2\;4)(5\;15)(5\;10)(5\;7)(5\;8)(5\;9)(13\;14) \\
    \tau\sigma &= (1\;4)(2\;9)(3\;13\;12\;15\;11\;5)(8\;10\;14) \\
           &= (1\;4)(2\;9)(3\;5)(3\;11)(3\;15)(3\;12)(3\;13)(8\;14)(8\;10) \\
    \intertext{and}
    \tau^2\sigma &= (1\;2\;15\;8\;3\;4\;14\;11\;12\;13\;7\;5\;10) \\
           &= (1\;10)(1\;5)(1\;7)(1\;13)(1\;12)(1\;11)(1\;14)
             (1\;4)(1\;3)(1\;8)(1\;15)(1\;2).
  \end{align*}
  From these, we see that only $\sigma$, $\sigma^2$ and $\tau^2\sigma$
  are even. The rest are odd.
\end{solution}

\Exercise2 Prove that $\sigma^2$ is an even permutation for every
permutation $\sigma$.
\begin{proof}
  We know that $\sigma$ can be written as a product of
  transpositions. Let it be written as a product of $n$
  transpositions,
  \begin{equation*}
    \sigma = \sigma_1\sigma_2\cdots\sigma_n,
  \end{equation*}
  where each $\sigma_i$ is a transposition. Then we can write
  \begin{equation*}
    \sigma^2 = (\sigma_1\cdots\sigma_n)(\sigma_1\cdots\sigma_n).
  \end{equation*}
  Since $\sigma^2$ has been written as a product of $2n$
  transpositions (i.e., an even number of transpositions), $\sigma^2$
  is even.
\end{proof}

\Exercise3 \label{exercise:quotient-group:S-n-gen-adj}
Prove that $S_n$ is generated by
$\{(i\;\;i+1) \mid 1\leq i\leq n - 1\}$.
\begin{proof}
  We use induction on $n$. For $n = 1$, $S_n$ is trivial and is
  generated by the empty set. Now suppose the statement is true for
  $S_n$, where $n$ is some positive integer, and consider $S_{n+1}$.

  For each $n$, let
  \begin{equation*}
    T_n = \{(i\;\;i+1) \mid 1\leq i\leq n-1\}.
  \end{equation*}
  We want to show that every permutation in $S_{n+1}$ can be written
  as a product of members of $T_{n+1}$. Let $\sigma\in S_{n+1}$, and
  write $\sigma$ as a product of transpositions.

  For each transposition $(j\;k)$ with $j < k$, there are two
  cases. First, if $1\leq j < k \leq n$, then the transposition
  belongs to $S_n$ and by the induction hypothesis can be written as a
  product of members of $T_n$. In this case, we are done.

  The remaining case is where $k = n+1$. If $j = n$, then $(j\;k)$ is
  already in $T_{n+1}$ and we are done, so suppose $j < n$. Let
  \begin{equation*}
    \tau = (k - 1\;\;k)(j\;\;k-1)(k-1\;\;k).
  \end{equation*}
  Notice that $\tau(j) = k$ and $\tau(k) = j$, with all other values
  fixed. Then $(j\;\;k-1)$ can, by the inductive hypothesis, be
  written as a product of members of $T_n$. As $(k-1\;\;k)$ belongs to
  $T_{n+1}$, this completes the inductive step. By induction, the
  result holds for all positive integers $n$.
\end{proof}

\Exercise4
\label{exercise:quotient-group:S-n-gen-12-123n}
Show that $S_n = \gen{(1\,2), (1\,2\,3\,\dots\,n)}$ for all $n\geq2$.
\begin{proof}
  Note that, for $i$ with $1\leq i\leq n-2$,
  \begin{equation*}
    (1\,2\,3\,\dots\,n)(i\;\;i+1)(1\,2\,3\,\dots\,n)^{-1}
    = (i+1\;\;i+2).
  \end{equation*}
  It follows that
  \begin{equation*}
    \{(i\;\;i+1)\mid1\leq i\leq n-1\}
    \subseteq \gen{(1\,2), (1\,2\,3\,\dots\,n)},
  \end{equation*}
  so by Exercise~\ref{exercise:quotient-group:S-n-gen-adj}, we are
  done.
\end{proof}

\Exercise5 Show that if $p$ is prime, $S_p = \gen{\sigma,\tau}$ where
$\sigma$ is any transposition and $\tau$ is any $p$-cycle.
\begin{proof}
  Let $G = \gen{\sigma,\tau}$. Relabel the elements of $S_p$ so that
  \begin{equation*}
    \sigma = (0\;1\;2\;\dots\;p-1)
    \quad\text{and}\quad
    \tau = (0\;k).
  \end{equation*}
  Now observe that
  \begin{equation*}
    \sigma\tau\sigma^{-1} = (1\;\;k+1),
  \end{equation*}
  where $k+1$ is understood to be the least residue of $k+1$, modulo
  $p$. Similarly,
  \begin{equation*}
    \sigma(1\;\;k+1)\sigma^{-1} = (2\;\;k+2)
  \end{equation*}
  and, in general,
  \begin{equation*}
    \sigma(n-1\;\;k+n-1)\sigma^{-1} = (n\;\;k+n),
  \end{equation*}
  where again all terms are reduced modulo $p$. This shows that
  $(n\;\;k+n)\in G$ for all positive integers $n$.

  Taking $n = k$, we see that $(k\;2k)\in G$. With $n = 2k$, we have
  $(2k\;3k)\in G$, and in general, $((m-1)k\;\;mk)\in G$ for all
  positive integers $m$.

  But
  \begin{equation*}
    (0\;k)(k\;2k)(0\;k) = (0\;2k),
  \end{equation*}
  so $(0\;2k)\in G$. And
  \begin{equation*}
    (0\;2k)(2k\;3k)(0\;2k) = (0\;3k),
  \end{equation*}
  so $(0\;3k)\in G$. Continuing in this way, we see that
  $(0\;nk)\in G$ for all positive integers $n$.

  Since $p$ is prime, $k$ is relatively prime to $p$ and therefore has
  a multiplicative inverse mod $p$. Let $a$ be this inverse, so that
  $ak\equiv1\pmod{p}$. Then we see that $(0\;ak) = (0\;1)\in G$.

  Both $(0\,1)$ and $(0\;1\;2\;\dots\;p-1)$ belong to $G$, so by
  Exercise~\ref{exercise:quotient-group:S-n-gen-12-123n}, $G = S_p$.
\end{proof}
