\section{Composition Series and the H\"older Program}

\Exercise1 Prove that if $G$ is an abelian simple group then
$G\cong Z_p$ for some prime $p$ (do not assume $G$ is a finite group).
\begin{proof}
  Let $G$ be an abelian simple group. Then $\ord{G} > 1$ and we may
  take some nonidentity element $x$ of $G$. Now, either $\gen{x}\neq G$
  or $\gen{x} = G$.

  In the first case, $\gen{x}$ is a nontrivial proper subgroup of
  $G$. But $G$ is abelian, so $\gen{x}$ must be a normal proper
  subgroup, which contradicts the fact that $G$ is simple. Hence
  $\gen{x} = G$.

  Then $G$ is cyclic. If $G$ has infinite order, then by Theorem~7 of
  Chapter~2, $\gen{x^2}$ is a nontrivial proper subgroup, again a
  contradiction. Therefore $G$ has finite order. But then, again by
  Theorem~7 of Chapter~2, $\gen{x^n}$ is a proper subgroup for any
  proper divisor $n$ of $\ord{G}$. Therefore $\ord{G}$ is prime, and
  by Theorem~4 of Chapter~2, $G\cong Z_p$.
\end{proof}

\Exercise2 Exhibit all $3$ composition series for $Q_8$ and all $7$
composition series for $D_8$. List the composition factors in each
case.
\begin{solution}
  For $Q_8$, the composition series are
  \begin{gather*}
    1 \trianglelefteq \gen{-1}
    \trianglelefteq \gen{i}
    \trianglelefteq Q_8, \\
    1 \trianglelefteq \gen{-1}
    \trianglelefteq \gen{j}
    \trianglelefteq Q_8, \\
    \intertext{and}
    1 \trianglelefteq \gen{-1}
    \trianglelefteq \gen{k}
    \trianglelefteq Q_8.
  \end{gather*}
  Note that each subgroup has index $2$ in its containing subgroup,
  and so must be normal. Each of the composition factors is isomorphic
  to $Z_2$.

  For $D_8$, we get the following composition series:
  \begin{gather*}
    1 \trianglelefteq \gen{s}
    \trianglelefteq \gen{s,r^2}
    \trianglelefteq D_8, \\
    1 \trianglelefteq \gen{r^2s}
    \trianglelefteq \gen{s,r^2}
    \trianglelefteq D_8, \\
    1 \trianglelefteq \gen{r^2}
    \trianglelefteq \gen{s,r^2}
    \trianglelefteq D_8, \\
    1 \trianglelefteq \gen{r^2}
    \trianglelefteq \gen{r}
    \trianglelefteq D_8, \\
    1 \trianglelefteq \gen{r^2}
    \trianglelefteq \gen{rs,r^2}
    \trianglelefteq D_8, \\
    1 \trianglelefteq \gen{rs}
    \trianglelefteq \gen{rs,r^2}
    \trianglelefteq D_8, \\
    1 \trianglelefteq \gen{r^3s}
    \trianglelefteq \gen{rs,r^2}
    \trianglelefteq D_8.
  \end{gather*}
  Again, each subgroup is normal within its containing subgroup since
  they each have index $2$. Each composition factor is isomorphic to
  $Z_2$.
\end{solution}

\Exercise3 Find a composition series for the quasidihedral group of
order $16$. Deduce that $QD_{16}$ is solvable.
\begin{solution}
  Since any subgroup of index $2$ is normal, we see from the lattice
  (Exercise~\ref{exercise:lattice:QD16}) that
  \begin{equation*}
    1 \trianglelefteq \gen{\sigma^4}
    \trianglelefteq \gen{\sigma^2}
    \trianglelefteq \gen{\sigma}
    \trianglelefteq QD_{16}
  \end{equation*}
  is a composition series. Since each composition factor has order $2$
  and is thus isomorphic to the abelian group $Z_2$, we see that
  $QD_{16}$ is solvable.
\end{solution}

\Exercise4 Use Cauchy's Theorem and induction to show that a finite
abelian group has a subgroup of order $n$ for each positive divisor
$n$ of its order.
\begin{proof}
  Let $G$ be a finite abelian group. We use induction on
  $\ord{G}$. Certainly the result holds for the trivial group. And if
  $\ord{G} = p$ for some prime $p$, then the positive divisors of
  $\ord{G}$ are $1$ and $p$ and the result is again trivial.

  Now assume that the statement is true for all groups of order
  strictly smaller than $\ord{G}$, and let $n$ be a positive divisor
  of $\ord{G}$ with $n > 1$. First, if $n$ is prime then Cauchy's
  Theorem allows us to find an element $x\in G$ having order $n$. Then
  $\gen{x}$ is the desired subgroup. On the other hand, if $n$ is not
  prime, then $n$ has a prime divisor $p$, so that $n = kp$ for some
  integer $k$. Cauchy's Theorem allows us to find an element $x$
  having order $p$. Set $N = \gen{x}$. By Lagrange's Theorem,
  \begin{equation*}
    \ord{G/N} = \frac{\ord{G}}{\ord{N}} < \ord{G}.
  \end{equation*}
  Now, by the inductive hypothesis, the group $G/N$ must have a
  subgroup of order $k$. And by the Lattice Isomorphism Theorem, this
  subgroup has the form $H/N$ for some subgroup $H$ of $G$. Then
  $\ord{H} = k\ord{N} = kp = n$, so that $H$ has order $n$. This
  completes the inductive step.
\end{proof}

\Exercise5 Prove that subgroups and quotient groups of a solvable
group are solvable.
\begin{proof}
  Let $G$ be a solvable group and let $H\leq G$. Since $G$ is
  solvable, we may find a chain of subgroups
  \begin{equation*}
    1 = G_0 \trianglelefteq G_1
    \trianglelefteq G_2
    \trianglelefteq\cdots
    \trianglelefteq G_n = G
  \end{equation*}
  so that each quotient $G_{i+1}/G_i$ is abelian. For each $i$, define
  \begin{equation*}
    H_i = G_i\cap H, \quad 0\leq i\leq n.
  \end{equation*}
  Then $H_i\leq H_{i+1}$ for each $i$. Moreover, if $g\in H_{i+1}$ and
  $x\in H_i$, then in particular $g\in G_{i+1}$ and $x\in G_i$, so
  that
  \begin{equation*}
    gxg^{-1}\in G_i
  \end{equation*}
  because $G_i\trianglelefteq G_{i+1}$. But $g$ and $x$ also belong to
  $H$, so
  \begin{equation*}
    gxg^{-1}\in H_i,
  \end{equation*}
  which shows that $H_i\trianglelefteq H_{i+1}$ for each $i$.

  Next, note that
  \begin{equation*}
    H_i = G_i\cap H = (G_i\cap G_{i+1})\cap H
    = G_i\cap H_{i+1}.
  \end{equation*}
  By the Second Isomorphism Theorem, we then have
  \begin{equation*}
    H_{i+1}/H_i = H_{i+1}/(H_{i+1}\cap G_i)
    \cong H_{i+1}G_i/G_i \leq G_{i+1}/G_i.
  \end{equation*}
  Since $H_{i+1}/H_i$ is isomorphic to a subgroup of the abelian group
  $G_{i+1}/G_i$, it follows that $H_{i+1}/H_i$ is also abelian. This
  completes the proof that $H$ is solvable.

  Next, let $N\trianglelefteq G$. For each $i$, define
  \begin{equation*}
    N_i = G_iN, \quad 0\leq i\leq n.
  \end{equation*}
  Now let $g\in N_{i+1}$, where $g = g_0n_0$ with $g_0\in G_{i+1}$ and
  $n_0\in N$. Also let $x\in N_i$, where $x = g_1n_1$ with
  $g_1\in G_i$ and $n_1\in N$. Then
  \begin{equation*}
    gxg^{-1} = g_0n_0g_1n_1n_0^{-1}g_0^{-1}.
  \end{equation*}
  Now, since $N$ is normal in $G$, $Ng = gN$, so $n_0g_1 = g_1n_2$
  for some $n_2\in N$. Then
  \begin{equation*}
    gxg^{-1} = g_0g_1(n_2n_1n_0^{-1})g_0^{-1}
    = g_0g_1n_3g_0^{-1}
  \end{equation*}
  for some $n_3\in N$. Then $n_3g_0^{-1} = g_0^{-1}n_4$ for some
  $n_4\in N$. And $g_0g_1g_0^{-1}\in G_i$ since
  $G_i\trianglelefteq G_{i+1}$, so
  \begin{equation*}
    gxg^{-1} = g_0g_1g_0^{-1}n_4 \in N_i.
  \end{equation*}
  This shows that $N_i\trianglelefteq N_{i+1}$. So by the Lattice
  Isomorphism Theorem, we have $N_{i+1}/N \trianglelefteq N_i/N$. This
  shows that
  \begin{equation*}
    1 = N_0/N \trianglelefteq N_1/N \trianglelefteq N_2/N
    \trianglelefteq \cdots \trianglelefteq
    N_n/N = G/N.
  \end{equation*}
  Moreover, the Third Isomorphism Theorem says that
  \begin{equation*}
    (N_{i+1}/N)/(N_i/N) \cong N_{i+1}/N_i,
  \end{equation*}
  so the proof will be complete if we can show that $N_{i+1}/N_i$ is
  abelian.

  Let $x,y\in N_{i+1}/N_i$. Then
  \begin{equation*}
    x = (g_0n_0)N_i \quad\text{and}\quad
    y = (g_1n_1)N_i
  \end{equation*}
  for some $g_0,g_1\in G_{i+1}$ and $n_0,n_1\in N$. We have
  \begin{align*}
    xyx^{-1}y^{-1}
    &= (g_0n_0)(g_1n_1)(g_0n_0)^{-1}(g_1n_1)^{-1}N_i \\
    &= g_0n_0g_1n_1n_0^{-1}g_0^{-1}n_1^{-1}g_1^{-1}N_i.
  \end{align*}
  Since $N\trianglelefteq G$, $gN = Ng$ for any $g\in G$, so we can
  find $n_2\in N$ such that
  \begin{equation*}
    xyx^{-1}y^{-1}
    = g_0g_1g_0^{-1}g^{-1}n_2N_i.
  \end{equation*}
  Now $N_i = G_iN = NG_i$ since $N\trianglelefteq G$ (see
  Proposition~14 and its corollary). Therefore
  \begin{equation*}
    n_2N_i = n_2NG_i = NG_i = G_iN
  \end{equation*}
  and we get
  \begin{equation*}
    xyx^{-1}y^{-1}
    = g_0g_1g_0^{-1}g^{-1}G_iN
    = G_iN.
  \end{equation*}
  So $xyx^{-1}y^{-1} = 1N_i$ or $xy = yx$. This completes the proof
  that $G/N$ is solvable.
\end{proof}

\Exercise6 Prove part (1) of the Jordan--H\"older Theorem by induction
on $\ord{G}$.
\begin{proof}
  Let $G$ be a finite group with $G\neq1$. We want to show that $G$
  has a composition series. We will use induction on $\ord{G}$.

  First, if $\ord{G} = 2$, then $G\cong Z_2$ and $G$ has the
  composition series $1\trianglelefteq G$.

  Now suppose $\ord{G} > 2$, and assume that all nontrivial groups
  with order strictly less than $G$ have a composition series.

  If $G$ is simple, then it has the composition series
  $1\trianglelefteq G$ and we are done. So assume $G$ is not
  simple. Let $N$ be a normal subgroup of $G$, with $N\neq1$ and
  $N\neq G$, and choose $N$ so that no other proper normal subgroup
  has larger order.

  Since $N$ is a proper subgroup, it has by the induction hypothesis a
  composition series
  \begin{equation*}
    1 = G_0 \trianglelefteq G_1 \trianglelefteq
    G_2 \trianglelefteq\cdots\trianglelefteq
    G_r = N.
  \end{equation*}

  Consider the quotient group $G/N$. If $G/N$ is not simple, then it
  has a nontrivial proper normal subgroup $\overline{M}$. By the
  Lattice Isomorphism Theorem, there is a subgroup
  $M\trianglelefteq G$ with $N<M$, contradicting the fact that $N$ is
  maximal. So $G/N$ is simple and $G$ has the composition series
  \begin{equation*}
    1 = G_0 \trianglelefteq G_1 \trianglelefteq
    G_2 \trianglelefteq\cdots\trianglelefteq
    G_r = N \trianglelefteq G.
  \end{equation*}
  By induction, the proof is complete.
\end{proof}
