\section{Composition Series and the H\"older Program}

\Exercise1 Prove that if $G$ is an abelian simple group then
$G\cong Z_p$ for some prime $p$ (do not assume $G$ is a finite group).
\begin{proof}
  Let $G$ be an abelian simple group. Then $\ord{G} > 1$ and we may
  take some nonidentity element $x$ of $G$. Now, either $\gen{x}\neq G$
  or $\gen{x} = G$.

  In the first case, $\gen{x}$ is a nontrivial proper subgroup of
  $G$. But $G$ is abelian, so $\gen{x}$ must be a normal proper
  subgroup, which contradicts the fact that $G$ is simple. Hence
  $\gen{x} = G$.

  Then $G$ is cyclic. If $G$ has infinite order, then by Theorem~7 of
  Chapter~2, $\gen{x^2}$ is a nontrivial proper subgroup, again a
  contradiction. Therefore $G$ has finite order. But then, again by
  Theorem~7 of Chapter~2, $\gen{x^n}$ is a proper subgroup for any
  proper divisor $n$ of $\ord{G}$. Therefore $\ord{G}$ is prime, and
  by Theorem~4 of Chapter~2, $G\cong Z_p$.
\end{proof}

\Exercise2 Exhibit all $3$ composition series for $Q_8$ and all $7$
composition series for $D_8$. List the composition factors in each
case.
\begin{solution}
  For $Q_8$, the composition series are
  \begin{gather*}
    1 \trianglelefteq \gen{-1}
    \trianglelefteq \gen{i}
    \trianglelefteq Q_8, \\
    1 \trianglelefteq \gen{-1}
    \trianglelefteq \gen{j}
    \trianglelefteq Q_8, \\
    \intertext{and}
    1 \trianglelefteq \gen{-1}
    \trianglelefteq \gen{k}
    \trianglelefteq Q_8.
  \end{gather*}
  Note that each subgroup has index $2$ in its containing subgroup,
  and so must be normal. Each of the composition factors is isomorphic
  to $Z_2$.

  For $D_8$, we get the following composition series:
  \begin{gather*}
    1 \trianglelefteq \gen{s}
    \trianglelefteq \gen{s,r^2}
    \trianglelefteq D_8, \\
    1 \trianglelefteq \gen{r^2s}
    \trianglelefteq \gen{s,r^2}
    \trianglelefteq D_8, \\
    1 \trianglelefteq \gen{r^2}
    \trianglelefteq \gen{s,r^2}
    \trianglelefteq D_8, \\
    1 \trianglelefteq \gen{r^2}
    \trianglelefteq \gen{r}
    \trianglelefteq D_8, \\
    1 \trianglelefteq \gen{r^2}
    \trianglelefteq \gen{rs,r^2}
    \trianglelefteq D_8, \\
    1 \trianglelefteq \gen{rs}
    \trianglelefteq \gen{rs,r^2}
    \trianglelefteq D_8, \\
    1 \trianglelefteq \gen{r^3s}
    \trianglelefteq \gen{rs,r^2}
    \trianglelefteq D_8.
  \end{gather*}
  Again, each subgroup is normal within its containing subgroup since
  they each have index $2$. Each composition factor is isomorphic to
  $Z_2$.
\end{solution}

\Exercise3 Find a composition series for the quasidihedral group of
order $16$. Deduce that $QD_{16}$ is solvable.
\begin{solution}
  Since any subgroup of index $2$ is normal, we see from the lattice
  (Exercise~\ref{exercise:lattice:QD16}) that
  \begin{equation*}
    1 \trianglelefteq \gen{\sigma^4}
    \trianglelefteq \gen{\sigma^2}
    \trianglelefteq \gen{\sigma}
    \trianglelefteq QD_{16}
  \end{equation*}
  is a composition series. Since each composition factor has order $2$
  and is thus isomorphic to the abelian group $Z_2$, we see that
  $QD_{16}$ is solvable.
\end{solution}

\Exercise4 Use Cauchy's Theorem and induction to show that a finite
abelian group has a subgroup of order $n$ for each positive divisor
$n$ of its order.
\begin{proof}
  Let $G$ be a finite abelian group. We use induction on
  $\ord{G}$. Certainly the result holds for the trivial group. And if
  $\ord{G} = p$ for some prime $p$, then the positive divisors of
  $\ord{G}$ are $1$ and $p$ and the result is again trivial.

  Now assume that the statement is true for all groups of order
  strictly smaller than $\ord{G}$, and let $n$ be a positive divisor
  of $\ord{G}$ with $n > 1$. First, if $n$ is prime then Cauchy's
  Theorem allows us to find an element $x\in G$ having order $n$. Then
  $\gen{x}$ is the desired subgroup. On the other hand, if $n$ is not
  prime, then $n$ has a prime divisor $p$, so that $n = kp$ for some
  integer $k$. Cauchy's Theorem allows us to find an element $x$
  having order $p$. Set $N = \gen{x}$. By Lagrange's Theorem,
  \begin{equation*}
    \ord{G/N} = \frac{\ord{G}}{\ord{N}} < \ord{G}.
  \end{equation*}
  Now, by the inductive hypothesis, the group $G/N$ must have a
  subgroup of order $k$. And by the Lattice Isomorphism Theorem, this
  subgroup has the form $H/N$ for some subgroup $H$ of $G$. Then
  $\ord{H} = k\ord{N} = kp = n$, so that $H$ has order $n$. This
  completes the inductive step.
\end{proof}
