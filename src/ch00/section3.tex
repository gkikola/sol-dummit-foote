\section{\texorpdfstring{$\Z/n\Z$}{Z/nZ}: The
  Integers Modulo \texorpdfstring{$n$}{n}}

\Exercise1 Write down explicitly all the elements in the residue
classes of $\Z/18\Z$.
\begin{solution}
  The residue classes are
  \begin{align*}
    \bar{0} &= \{0, 18, -18, 36, -36, \ldots\}, \\
    \bar{1} &= \{1, 19, -17, 37, -35, \ldots\}, \\
    \bar{2} &= \{2, 20, -16, 38, -34, \ldots\}, \\
    \bar{3} &= \{3, 21, -15, 39, -33, \ldots\}, \\
    \bar{4} &= \{4, 22, -14, 40, -32, \ldots\}, \\
    \bar{5} &= \{5, 23, -13, 41, -31, \ldots\}, \\
    \bar{6} &= \{6, 24, -12, 42, -30, \ldots\}, \\
    \bar{7} &= \{7, 25, -11, 43, -29, \ldots\}, \\
    \bar{8} &= \{8, 26, -10, 44, -28, \ldots\}, \\
    \bar{9} &= \{9, 27, -9, 45, -27, \ldots\}, \\
    \overline{10} &= \{10, 28, -8, 46, -26, \ldots\}, \\
    \overline{11} &= \{11, 29, -7, 47, -25, \ldots\}, \\
    \overline{12} &= \{12, 30, -6, 48, -24, \ldots\}, \\
    \overline{13} &= \{13, 31, -5, 49, -23, \ldots\}, \\
    \overline{14} &= \{14, 32, -4, 50, -22, \ldots\}, \\
    \overline{15} &= \{15, 33, -3, 51, -21, \ldots\}, \\
    \overline{16} &= \{16, 34, -2, 52, -20, \ldots\}, \\
    \intertext{and}
    \overline{17} &= \{17, 35, -1, 53, -19, \ldots\}. \qedhere
  \end{align*}
\end{solution}

\Exercise2 Prove that the distinct equivalence classes in $\Z/n\Z$ are
precisely $\bar{0}, \bar{1}, \bar{2}, \ldots, \overline{n - 1}$ (use
the Division Algorithm).
\begin{proof}
  Consider the equivalence class $\bar k$. Using the Division
  Algorithm, we may find an integer $q$ and an integer $r$ such that
  \begin{equation*}
    k = qn + r,\quad\text{with $0\leq r<n$.}
  \end{equation*}
  Now $k\equiv r\pmod n$ and $r$ is an integer between $0$ and
  $n - 1$, so this shows that $\bar k = \bar r$. Thus the equivalence
  classes in $\Z/n\Z$ are a subset of
  $\{\bar{0}, \bar{1}, \ldots, \bar{n-1}\}$.

  Finally, we note that the equivalence classes
  $\bar0, \ldots, \overline{n-1}$ are actually distinct from each
  other. For, if not, suppose $\bar a = \bar b$ where
  $0\leq b\leq a\leq n-1$. Then $n\mid(a - b)$, and since
  $0\leq a-b\leq n-1$, we must have $a-b = 0$ so that $a =
  b$. Therefore the distinct equivalence classes are precisely
  $\bar0,\ldots,\overline{n-1}$.
\end{proof}

\Exercise3 Prove that if
$a = a_n10^n + a_{n-1}10^{n-1} + \cdots + a_110 + a_0$ is any positive
integer then $a\equiv a_n + a_{n-1} + \cdots + a_1 + a_0\pmod9$ (note
that this is the usual arithmetic rule that the remainder after
division by 9 is the same as the sum of the decimal digits mod 9 -- in
particular an integer is divisible by 9 if and only if the sum of its
digits is divisible by 9).
\begin{solution}
  Let $a$ be as stated. Since $10\equiv1\pmod9$ we may apply Theorem 3
  to write
  \begin{align*}
    a &\equiv a_n1^n + a_{n-1}1^{n-1} + \cdots + a_1 + a_0\pmod9 \\
      &\equiv a_n + a_{n-1} + \cdots + a_1 + a_0\pmod9. \qedhere
  \end{align*}
\end{solution}

\Exercise4 Compute the remainder when $37^{100}$ is divided by $29$.
\begin{solution}
  $37^2 = 1369\equiv6\pmod{29}$. Successive squaring then yields
  \begin{align*}
    37^4 &\equiv 6^2 = 36 \equiv 7\pmod{29} \\
    37^8 &\equiv 7^2 = 49 \equiv 20\pmod{29} \\
    37^{16} &\equiv 20^2 = 400 \equiv 23\pmod{29} \\
    37^{32} &\equiv 23^2 = 529 \equiv 7\pmod{29} \\
    37^{64} &\equiv 7^2 = 49 \equiv 20\pmod{29}.
  \end{align*}
  So
  \begin{equation*}
    37^{100} = 37^{64}37^{32}37^4 \equiv 20\cdot7\cdot7 \equiv 23\pmod{29}.
  \end{equation*}
  Therefore $37^{100}$ has a
  remainder of $23$ when divided by $29$.
\end{solution}

\Exercise5 Compute the last two digits of $9^{1500}$.
\begin{solution}
  $9^{1500} = 3^{3000} = 27^{1000}$. Now
  $27^2 = 729\equiv29\pmod{100}$, and successive squaring then gives
  \begin{align*}
    27^4 &\equiv 29^2 = 841 \equiv 41\pmod{100}, \\
    27^8 &\equiv 41^2 = 1681 \equiv 81\pmod{100}, \\
    27^{16} &\equiv 81^2 = 6561 \equiv 61\pmod{100}, \\
    27^{32} &\equiv 61^2 = 3721 \equiv 21\pmod{100}, \\
    27^{64} &\equiv 21^2 = 441 \equiv 41\pmod{100}.
  \end{align*}
  At this point the numbers start to repeat, so that
  $27^{128} \equiv 81\pmod{100}$, $27^{256} \equiv 61\pmod{100}$, and
  $27^{512}\equiv 21\pmod{100}$. Therefore
  \begin{align*}
    9^{1500} &= 27^{1000} = 27^{512}27^{256}27^{128}27^{64}27^{32}27^8 \\
             &\equiv 21\cdot61\cdot81\cdot41\cdot21\cdot81
               = (1281)(3321)(1701) \\
             &\equiv 81\cdot21\cdot1 \equiv 1 \pmod{100}.
  \end{align*}
  Therefore, the last two digits of $9^{1500}$ are $01$.
\end{solution}

\Exercise6 Prove that the squares of the elements in $\Z/4\Z$ are just
$\bar0$ and $\bar1$.
\begin{proof}
  \begin{align*}
    0^2 &= 0 \equiv 0 \pmod4, \\
    1^2 &= 1 \equiv 1 \pmod4, \\
    2^2 &= 4 \equiv 0 \pmod4, \\
    3^2 &= 9 \equiv 1 \pmod4. \qedhere
  \end{align*}
\end{proof}

\Exercise7 Prove for any integers $a$ and $b$ that $a^2 + b^2$ never
leaves a remainder of $3$ when divided by $4$ (use the previous
exercise).
\begin{proof}
  $a^2$ and $b^2$ are each either congruent to $0$ or to $1$, modulo
  $4$. Adding $a^2 + b^2$ then gives four cases:
  \begin{align*}
    0 + 0 \equiv 0 \pmod4, \\
    0 + 1 \equiv 1 \pmod4, \\
    1 + 0 \equiv 1 \pmod4, \\
    1 + 1 \equiv 2 \pmod4.
  \end{align*}
  In every case, $a^2+b^2$ never has a remainder of $3$ when divided
  by $4$.
\end{proof}

\Exercise8 Prove that the equation
\begin{equation}
  \label{eq:abc-squares}
  a^2 + b^2 = 3c^2
\end{equation}
has no solutions in nonzero integers $a$, $b$, and $c$.
\begin{proof}
  Consider the equation modulo $4$. From the previous exercise, the
  left-hand side cannot be congruent to $3$. However, the right-hand
  side is congruent to either $0$ or $3$, so therefore both sides must
  be congruent to $0$. That is,
  \begin{equation*}
    a^2 + b^2\equiv c^2\equiv0\pmod4.
  \end{equation*}

  This immediately implies that $c$ is even. Now, if $a$ is even, then
  $b$ must be even, since $b^2 = c^2-a^2$ is even. On the other hand,
  if $a$ is odd, then $b$ must be odd for the same reason. But if $a$
  and $b$ are both odd, then we may find integers $m$ and $n$ such
  that
  \begin{align*}
    a^2 + b^2 &= (2m+1)^2 + (2n+1)^2 \\
              &= 4m^2 + 4m + 1 + 4n^2 + 4n + 1 \\
              &\equiv 2\pmod4.
  \end{align*}
  This is impossible, so $a$, $b$, and $c$ must all be even.

  Now, if possible, suppose that $a$, $b$, and $c$ are three positive
  integers which satisfy the equation \eqref{eq:abc-squares}. Since
  all three integers must be even, their squares each contain a factor
  of $4$. Divide both sides by $4$ to get a new equation,
  \begin{equation*}
    \alpha^2 + \beta^2 = \gamma^2,
  \end{equation*}
  where $\alpha<a$, $\beta<b$, and $\gamma<c$.

  But by the same argument as before, $\alpha$, $\beta$, and $\gamma$
  must be even, so their squares are divisible by $4$ and we can again
  find an even smaller set of solutions. This process could be
  repeated indefinitely, to get smaller and smaller positive integer
  solutions. Clearly this is not possible, so there are no solutions
  in the nonzero integers.
\end{proof}

\Exercise9 Prove that the square of any odd integer always leaves a
remainder of $1$ when divided by $8$.
\begin{proof}
  If $a$ is an odd integer, then $a$ can be written as $2k + 1$ for
  some integer $k$, and
  \begin{equation*}
    a^2 = (2k + 1)^2 = 4k^2 + 4k + 1 = 4k(k + 1) + 1.
  \end{equation*}
  Now $k(k+1)$ must be even, since it is the product of consecutive
  integers. Therefore $4k(k+1)$ is divisible by $8$. Therefore
  $a^2\equiv1\pmod8$.
\end{proof}

\Exercise{10}
\label{z-mod-nz-times}
Prove that the number of elements of $(\Z/n\Z)^\times$
is $\varphi(n)$ where $\varphi$ denotes the Euler $\varphi$-function.
\begin{proof}
  We will show that the elements in $(\Z/n\Z)^\times$ are precisely
  those residue classes whose representatives are relatively prime to
  $n$.

  First suppose that $a\in(\Z/n\Z)^\times$ and let $b$ be the
  multiplicative inverse of $a$ modulo $n$, so that
  $ab\equiv1\pmod{n}$. Then $n\mid(ab - 1)$ so we may find an integer
  $m$ such that $mn = ab - 1$. Rearranging, we get $ab - mn = 1$. But
  this shows that the greatest common divisor of $a$ and $n$ is $1$
  (if not, we could factor the left-hand side to get a product of two
  integers, not both $1$, that equals $1$, which is
  impossible). Therefore any number in $(\Z/n\Z)^\times$ must be
  relatively prime to $n$.

  Now, for the other direction, suppose that $a$ is any integer
  relatively prime to $n$. Then we can use the Euclidean algorithm to
  write the common divisor $1$ as a linear combination of $a$ and $n$,
  that is,
  \begin{equation*}
    ax + ny = 1, \quad x,y\in\Z.
  \end{equation*}
  But then $ax\equiv1\pmod{n}$, so $x$ is the multiplicative inverse
  of $a$ modulo $n$, i.e., $a\in(\Z/n\Z)^\times$.

  Since there are exactly $\varphi(n)$ least residues which are
  coprime to $n$, the set $(\Z/n\Z)^\times$ has exactly $\varphi(n)$
  elements.
\end{proof}

\Exercise{11} Prove that if $\bar{a},\bar{b}\in(\Z/n\Z)^\times$, then
$\bar{a}\cdot\bar{b}\in(\Z/n\Z)^\times$.
\begin{proof}
  Let $\bar{a}$ and $\bar{b}$ be in $(\Z/n\Z)^\times$ as stated. Then
  $\bar{a}$ has a multiplicative inverse $\bar{x}$ and $\bar{b}$ has
  an inverse $\bar{y}$. Then
  \begin{equation*}
    (\bar{a}\bar{x})(\bar{b}\bar{y}) \equiv 1\cdot1 \equiv 1\pmod{n}.
  \end{equation*}
  Rearranging the left-hand side, we see that $\bar{x}\bar{y}$ is the
  multiplicative inverse of $\bar{a}\bar{b}$, so that
  $\bar{a}\bar{b}\in(\Z/n\Z)^\times$.
\end{proof}

\Exercise{12}   Let   $n\in\Z$,   $n>1$,   and   let   $a\in\Z$   with
$1\leq a\leq n$. Prove if $a$  and $n$ are not relatively prime, there
exists an integer  $b$ with $1\leq b<n$  such that $ab\equiv0\pmod{n}$
and  deduce   that  there   cannot  be  an   integer  $c$   such  that
$ac\equiv1\pmod{n}$.
\begin{proof}
  Let $d = (a, n)$ and let $b = n/d$. Then $b$ is an integer with
  $1\leq b<n$ (since $d > 1$). Similarly, $a/d$ is also an integer. So
  we have
  \begin{equation*}
    ab = a\left(\frac{n}{d}\right) = n\left(\frac{a}{d}\right) \equiv0\pmod{n}.
  \end{equation*}

  Now suppose $c$ is such that $ac\equiv1\pmod{n}$. Then
  $abc\equiv b\pmod{n}$. But this is clearly impossible, since
  $abc\equiv0\pmod{n}$ and $b\not\equiv0\pmod{n}$. Therefore such a
  $c$ cannot exist.
\end{proof}

\Exercise{13} Let $n\in\Z, n > 1$, and let $a\in\Z$ with
$1\leq a\leq n$. Prove that if $a$ and $n$ are relatively prime then
there is an integer $c$ such that $ac\equiv1\pmod{n}$.
\begin{proof}
  Since $(a, n) = 1$, we may find integers $c$ and $d$ such that
  $ac + nd = 1$. This implies that $ac\equiv1\pmod{n}$.
\end{proof}

\Exercise{14}
\label{exercise:prelim:z-mod-nz-cross}
Conclude from the previous two exercises that $(\Z/n\Z)^\times$ is the
set of elements $\bar{a}$ of $\Z/n\Z$ with $(a, n) = 1$ and hence
prove Proposition~4. Verify this directly in the case $n = 12$.
\begin{solution}
  From the previous two exercises we know that $a$ and $n$ are
  relatively prime if and only if there is an integer $c$ such that
  $ac\equiv1\pmod{n}$, i.e., if and only if $a$ has a multiplicative
  inverse modulo $n$.

  For $n = 12$, we have the following multiplication table:

  \medskip
  \begin{tabular}{r|cccccccccccc}
    & 0 & 1 & 2 & 3 & 4 & 5 & 6 & 7 & 8 & 9 & 10 & 11 \\\hline
    0 & 0 & 0 & 0 & 0 & 0 & 0 & 0 & 0 & 0 & 0 & 0 & 0 \\
    1 & 0 & \boxed{1} & 2 & 3 & 4 & 5 & 6 & 7 & 8 & 9 & 10 & 11 \\
    2 & 0 & 2 & 4 & 6 & 8 & 10 & 0 & 2 & 4 & 6 & 8 & 10 \\
    3 & 0 & 3 & 6 & 9 & 0 & 3 & 6 & 9 & 0 & 3 & 6 & 9 \\
    4 & 0 & 4 & 8 & 0 & 4 & 8 & 0 & 4 & 8 & 0 & 4 & 8 \\
    5 & 0 & 5 & 10 & 3 & 8 & \boxed{1} & 6 & 11 & 4 & 9 & 2 & 7 \\
    6 & 0 & 6 & 0 & 6 & 0 & 6 & 0 & 6 & 0 & 6 & 0 & 6 \\
    7 & 0 & 7 & 2 & 9 & 4 & 11 & 6 & \boxed{1} & 8 & 3 & 10 & 5 \\
    8 & 0 & 8 & 4 & 0 & 8 & 4 & 0 & 8 & 4 & 0 & 8 & 4 \\
    9 & 0 & 9 & 6 & 3 & 0 & 6 & 9 & 3 & 0 & 6 & 9 & 3 \\
    10 & 0 & 10 & 8 & 6 & 4 & 2 & 0 & 10 & 8 & 6 & 4 & 2 \\
    11 & 0 & 11 & 10 & 9 & 8 & 7 & 6 & 5 & 4 & 3 & 2 & \boxed{1}
  \end{tabular}
  \medskip

  The only values which have a multiplicative inverse are $1, 5, 7,$
  and $11$, which are precisely those values which are coprime to
  $12$.
\end{solution}

\Exercise{15} For each of the following pairs of integers $a$ and $n$,
show that $a$ is relatively prime to $n$ and determine the
multiplicative inverse of $\bar{a}$ in $\Z/n\Z$.
\begin{enumerate}
\item
  $a = 13, n = 20$
  \begin{solution}
    Applying the Euclidean algorithm gives
    \begin{align*}
      20 &= 1(13) + 7 \\
      13 &= 1(7) + 6 \\
      7 &= 1(6) + 1,
    \end{align*}
    so $(20, 13) = 1$. And we can write
    \begin{align*}
      1 &= 7 - 6 \\
        &= 7 - (13 - 7) \\
        &= 2(7) - 13 \\
        &= 2(20 - 13) - 13 \\
        &= 2(20) - 3(13).
    \end{align*}
    So $\overline{(-3)} = \overline{17}$ is the multiplicative inverse
    of $\overline{13}$ in $\Z/20\Z$.
  \end{solution}

\item
  $a = 69, n = 89$
  \begin{solution}
    The same procedure will show that $(69,89) = 1$ and that $\bar{a}$
    has an inverse of $\overline{40}$.
  \end{solution}

\item
  $a = 1891, n = 3797$
  \begin{solution}
    $\bar{a}$ has an inverse of $\overline{253}$.
  \end{solution}

\item
  $a = 6\,003\,722\,857, n = 77\,695\,236\,973$
  \begin{solution}
    $\bar{a}$ has an inverse of $\overline{77\,695\,236\,753}$.
  \end{solution}
\end{enumerate}
