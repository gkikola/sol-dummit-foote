\chapter{Preliminaries}

\section{Basics}

Let $\mathcal{A}$ be the set of $2\times2$ matrices over $\R$, let
\begin{equation*}
  M = \begin{pmatrix}
    1 & 1 \\
    0 & 1
  \end{pmatrix},
\end{equation*}
and let
\begin{equation*}
  \mathcal{B} = \{X\in\mathcal{A}\mid MX = XM\}.
\end{equation*}

\Exercise{1} Determine which of the following elements of
$\mathcal{A}$ lie in $\mathcal{B}$:
\begin{equation*}
  \begin{pmatrix}
    1 & 1\\
    0 & 1
  \end{pmatrix},
  \begin{pmatrix}
    1 & 1\\
    1 & 1
  \end{pmatrix},
  \begin{pmatrix}
    0 & 0\\
    0 & 0
  \end{pmatrix},
  \begin{pmatrix}
    1 & 1\\
    1 & 0
  \end{pmatrix},
  \begin{pmatrix}
    1 & 0\\
    0 & 1
  \end{pmatrix},
  \begin{pmatrix}
    0 & 1\\
    1 & 0
  \end{pmatrix}.
\end{equation*}
\begin{solution}
  It is easy to verify that
  \begin{equation*}
    \begin{pmatrix}
      1 & 1\\
      0 & 1
    \end{pmatrix},\quad
    \begin{pmatrix}
      0 & 0\\
      0 & 0
    \end{pmatrix},\quad\text{and}\quad
    \begin{pmatrix}
      1 & 0\\
      0 & 1
    \end{pmatrix}
  \end{equation*}
  all commute with $M$: the first matrix is $M$ itself, and the latter
  two are the zero matrix and the identity matrix, all of which will
  commute. So each of these matrices is in $\mathcal{B}$.

  We can check the remaining matrices individually: Let
  \begin{equation*}
    P = \begin{pmatrix}
      1 & 1\\
      1 & 1
    \end{pmatrix},\quad
    Q = \begin{pmatrix}
      1 & 1\\
      1 & 0
    \end{pmatrix},\quad\text{and}\quad
    R = \begin{pmatrix}
      0 & 1\\
      1 & 0
    \end{pmatrix}.
  \end{equation*}
  Direct computation shows that
  \begin{equation*}
    MP = \begin{pmatrix}
      2 & 2 \\
      1 & 1
    \end{pmatrix}
    \neq
    \begin{pmatrix}
      1 & 2 \\
      1 & 2
    \end{pmatrix}
    = PM,
  \end{equation*}
  \begin{equation*}
    MQ = \begin{pmatrix}
      2 & 1 \\
      1 & 0
    \end{pmatrix}
    \neq
    \begin{pmatrix}
      1 & 2 \\
      1 & 1
    \end{pmatrix}
    = QM,
  \end{equation*}
  and
  \begin{equation*}
    MR = \begin{pmatrix}
      1 & 1 \\
      1 & 0
    \end{pmatrix}
    \neq
    \begin{pmatrix}
      0 & 1 \\
      1 & 1
    \end{pmatrix}
    = RM.
  \end{equation*}
  So $P, Q, R\not\in\mathcal{B}$.
\end{solution}

\Exercise{2} Prove that if $P, Q \in\mathcal{B}$, then
$P + Q\in\mathcal{B}$.
\begin{proof}
  Let
  \begin{equation*}
    P = \begin{pmatrix}
      a & b \\ c & d
    \end{pmatrix}
    \quad\text{and}\quad
    Q = \begin{pmatrix}
      e & f \\ g & h
    \end{pmatrix}
  \end{equation*}
  be matrices in the set $\mathcal{B}$, so that $MP = PM$ and
  $MQ = QM$. Then we have
  \begin{align*}
    M(P + Q) &= \begin{pmatrix} 1 & 1 \\ 0 & 1 \end{pmatrix}
                \begin{pmatrix} a + e & b + f \\ c + g & d + h \end{pmatrix} \\
             &= \begin{pmatrix}
               a + e + c + g & b + f + d + h \\
               c + g & d + h
             \end{pmatrix} \\
             &= \begin{pmatrix}
               a + c & b + d \\
               c & d
             \end{pmatrix}
                   + \begin{pmatrix}
                     e + g & f + h \\
                     g & h
                   \end{pmatrix} \\
             &= MP + MQ \\
             &= PM + QM \\
             &= \begin{pmatrix}
               a & a + b \\
               c & c + d
             \end{pmatrix}
                   + \begin{pmatrix}
                     e & e + f \\
                     g & g + h
                   \end{pmatrix} \\
             &= \begin{pmatrix}
               a + e & a + b + e + f \\
               c + g & c + d + g + h
             \end{pmatrix} \\
             &= \begin{pmatrix}
               a + e & b + f \\
               c + g & d + h
             \end{pmatrix}
               \begin{pmatrix}
                 1 & 1 \\ 0 & 1
               \end{pmatrix} \\
             &= (P + Q)M.
  \end{align*}
  Therefore $P + Q\in\mathcal{B}$.
\end{proof}

\Exercise{3}
Prove that if $P, Q\in\mathcal{B}$, then $PQ\in\mathcal{B}$.
\begin{proof}
  A similar argument to the one in Exercise 2 above will show that
  $PQ\in\mathcal{B}$ for any $P,Q\in\mathcal{B}$.
\end{proof}

\Exercise{4} Find conditions on $p, q, r, s$ which determine precisely
when $\begin{pmatrix} p & q \\ r & s \end{pmatrix} \in \mathcal{B}$.
\begin{solution}
  Let
  \begin{equation*}
    P = \begin{pmatrix} p & q \\ r & s \end{pmatrix}.
  \end{equation*}
  Then
  \begin{equation*}
    MP = \begin{pmatrix} 1 & 1 \\ 0 & 1 \end{pmatrix}
    \begin{pmatrix} p & q \\ r & s \end{pmatrix}
    = \begin{pmatrix} p + r & q + s \\ r & s \end{pmatrix}
  \end{equation*}
  while
  \begin{equation*}
    PM = \begin{pmatrix} p & q \\ r & s \end{pmatrix}
    \begin{pmatrix} 1 & 1 \\ 0 & 1 \end{pmatrix}
    = \begin{pmatrix} p & p + q \\ r & r + s \end{pmatrix}.
  \end{equation*}
  Therefore, $MP = PM$ if and only if $r = 0$ and $p = s$. Hence
  \begin{equation*}
    \mathcal{B} = \left\{ \begin{pmatrix} p & p + q \\ 0 & p \end{pmatrix}
      \;\middle|\; p, q\in\R \right\}. \qedhere
  \end{equation*}
\end{solution}

\Exercise{5} Determine whether the following functions $f$ are well
defined:
\begin{enumerate}
\item
  $f\colon\Q\to\Z$ defined by $f(a/b) = a$.

\item
  $f\colon\Q\to\Q$ defined by $f(a/b) = a^2/b^2$.
\end{enumerate}
\begin{solution}
  \begin{enumerate}
  \item
    $f$ is not well defined since, for example,
    \begin{equation*}
      f(1/2) = 1, \quad f(2/4) = 2, \quad\text{but}\quad\frac12 = \frac24.
    \end{equation*}

  \item
    Suppose $a,b,c,d\in\Z$ with $b, d\neq 0$ are such that
    \begin{equation*}
      \frac{a}{b} = \frac{c}{d}.
    \end{equation*}
    Then
    \begin{equation*}
      f(a/b) = \frac{a^2}{b^2} = \left(\frac ab\right)^2
      = \left(\frac cd\right)^2 = \frac{c^2}{d^2} = f(c/d).
    \end{equation*}
    Therefore $f$ is well defined. \qedhere
  \end{enumerate}
\end{solution}

\Exercise{6} Determine whether the function $f\colon\R^+\to\Z$ defined
by mapping a real number $r$ to the first digit to the right of the
decimal point in a decimal expansion of $r$ is well defined.
\begin{solution}
  $f$ is not well defined since decimal expansions are not unique. For
  example, $1 = 1.0 = 0.999\ldots$ but $f(1.0) = 0$ and
  $f(0.999\ldots) = 9$.
\end{solution}

\Exercise{7} Let $f\colon A\to B$ be a surjective map of sets. Prove
that the relation
\begin{equation*}
  a\sim b \;\text{if and only if}\; f(a) = f(b)
\end{equation*}
is an equivalence relation whose equivalence classes are the fibers of
$f$.
\begin{proof}
  That $\sim$ is an equivalence relation on $A$ follows directly from
  the fact that $=$ is an equivalence relation on the set $B$.

  Now let $b\in B$ be arbitrary. Since $f$ is surjective, there is an
  $a$ in $A$ such that $f(a) = b$. Then the equivalence class of $a$
  is the set
  \begin{equation*}
    \{ x\in A \mid x\sim a \}.
  \end{equation*}
  But by definition of $\sim$, this set is equal to
  \begin{equation*}
    \{ x\in A \mid f(x) = f(a) = b \}.
  \end{equation*}
  Therefore the equivalence class of $a$ is precisely the fiber of $f$
  over $b$.
\end{proof}
