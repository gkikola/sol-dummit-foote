\section{Partially Ordered Sets and Zorn's Lemma}

\Exercise1 Let $A$ be the collection of all finite subsets of $\R$
ordered by inclusion. Discuss the existence (or nonexistence) of upper
bounds, minimal and maximal elements (where minimal elements are
defined analogously to maximal elements). Explain why this is not a
well ordering.
\begin{solution}
  Subsets of $A$ may or may not have upper bounds. For example, take
  the set of all singleton sets containing integers,
  \begin{equation*}
    X = \{\{1\}, \{2\}, \{3\}, \dots\}.
  \end{equation*}
  $X$ is a subset of $A$ but it does not have an upper bound since
  $\Z^+\not\in A$. Any finite subset of $A$, however, will have an
  upper bound, namely the union of the sets in the subset. So for
  example $\{\emptyset, \{1,2\}, \{1,3,5\}\}\subset A$ has the upper
  bound $\{1,2,3,5\}\in A$.

  $A$ does not have any maximal elements, since given any $Y$ in $A$,
  we can simply append to the set $Y$ any real number not already in
  $Y$, in order to obtain a new finite subset of $\R$ containing
  $Y$. $A$ does have a minimal element, however, namely the empty set.

  Set inclusion ($\subseteq$) is not a well ordering on $A$ since it
  is not a total ordering. That is, there exist elements $X$ and $Y$
  of $A$ such that $X\not\subseteq Y$ and $Y\not\subseteq X$ (for an
  example, take $X = \{0\}$ and $Y = \{1\}$).
\end{solution}

\Exercise2 Let $A$ be the collection of all infinite subsets of $\R$
ordered by inclusion. Discuss the existence (or nonexistence) of upper
bounds, minimal and maximal elements. Explain why this is not a well
ordering.
\begin{solution}
  In this case, every subset of $A$ has an upper bound since the union
  of the sets in the subset is a member of $A$. $A$ has one maximal
  element, $\R$ itself, but no minimal elements since, given any set
  $X\in A$, we may pick some element $x$ in $X$, so that $X - \{x\}$
  is an infinite subset of $\R$ which is contained in $X$.

  This ordering is not a well ordering for the same reason as in the
  previous exercise: set inclusion $\subseteq$ is not a total ordering
  on $A$. For example, take $X$ to be the set of even integers and
  take $Y$ to be the set of odd integers. Then $X\not\subseteq Y$ and
  $Y\not\subseteq X$.
\end{solution}

\Exercise3 Show that the following partial orderings on the given sets
are not well orderings:
\begin{enumerate}
\item $\R$ under the usual relation $\leq$.
\item $\R^+$ under the usual relation $\leq$.
\item $\R^+\cup\{0\}$ under the usual relation $\leq$.
\item $\Z$ under the usual relation $\leq$.
\end{enumerate}
\begin{proof}
  In each case, $\leq$ is a total ordering. However, it is not a well
  ordering since in each case there exist nonempty subsets which have
  no smallest element.

  For $\R$, $\R^+$, and $\R^+\cup\{0\}$ the interval $(0,1)$ is a
  subset of each but has no smallest member. For $\Z$, the entire set
  $\Z$ itself is a subset with no smallest member.
\end{proof}

\Exercise4 Show that $\Z^+$ is well ordered under the usual relation
$\leq$.
\begin{proof}
  Given any two positive integers $m$ and $n$, we must have either
  $m\leq n$, $n\leq m$, or both (if $m = n$). Therefore $\leq$ is a
  total ordering.

  Let $A$ be an arbitrary nonempty subset of $\Z^+$. Pick an integer
  $a\in A$. Then the set $\{1,2,\dots,a\}\cap A$ is nonempty and
  finite. Being finite, it must have a smallest member $b$. If
  $c\in A$ is such that $c\leq b$, then certainly $c\leq a$ (by
  transitivity), so $c\in\{1,2,\dots,a\}\cap A$. Since $b$ is the
  smallest member of this set, we must have $c = b$.

  We have shown that $\leq$ is a total ordering such that any nonempty
  subset of $\Z^+$ has a smallest member. Therefore $\Z^+$ is well
  ordered under the relation $\leq$.
\end{proof}
