\section{Partially Ordered Sets and Zorn's Lemma}

\Exercise1 Let $A$ be the collection of all finite subsets of $\R$
ordered by inclusion. Discuss the existence (or nonexistence) of upper
bounds, minimal and maximal elements (where minimal elements are
defined analogously to maximal elements). Explain why this is not a
well ordering.
\begin{solution}
  Subsets of $A$ may or may not have upper bounds. For example, take
  the set of all singleton sets containing integers,
  \begin{equation*}
    X = \{\{1\}, \{2\}, \{3\}, \dots\}.
  \end{equation*}
  $X$ is a subset of $A$ but it does not have an upper bound since
  $\Z^+\not\in A$. Any finite subset of $A$, however, will have an
  upper bound, namely the union of the sets in the subset. So for
  example $\{\emptyset, \{1,2\}, \{1,3,5\}\}\subset A$ has the upper
  bound $\{1,2,3,5\}\in A$.

  $A$ does not have any maximal elements, since given any $Y$ in $A$,
  we can simply append to the set $Y$ any real number not already in
  $Y$, in order to obtain a new finite subset of $\R$ containing
  $Y$. $A$ does have a minimal element, however, namely the empty set.

  Set inclusion ($\subseteq$) is not a well ordering on $A$ since it
  is not a total ordering. That is, there exist elements $X$ and $Y$
  of $A$ such that $X\not\subseteq Y$ and $Y\not\subseteq X$ (for an
  example, take $X = \{0\}$ and $Y = \{1\}$).
\end{solution}
