\section{Group Actions}

\Exercise1 Let $F$ be a field. Show that the multiplicative group of
nonzero elements of $F$ (denoted by $F^\times$) acts on the set $F$ by
$g\cdot a = ga$, where $g\in F^\times$, $a\in F$ and $ga$ is the usual
product in $F$ of the two field elements.
\begin{proof}
  Let $g_1,g_2\in F^\times$. Then for any $a\in F$,
  \begin{equation*}
    g_1\cdot(g_2\cdot a) = g_1\cdot g_2a = g_1(g_2a) = (g_1g_2)a
    = (g_1g_2)\cdot a,
  \end{equation*}
  where the second-to-last equality follows from the associativity of
  multiplication in $F$. Also, for any $a\in F^\times$,
  \begin{equation*}
    1\cdot a = 1a = a,
  \end{equation*}
  since $1$ is the identity of the group $F^\times$. And $1(0) = 0$
  (which follows from distributivity), so we can say that
  $1\cdot a = a$ for all $a\in F$. Therefore the mapping
  $(g, a)\mapsto ga$ of $F^\times\times F\to F$ is a group action.
\end{proof}

\Exercise2 Show that the additive group $\Z$ acts on itself by
$z\cdot a = z + a$ for all $z,a\in\Z$.
\begin{proof}
  For all $z_1,z_2,a\in\Z$, we have
  \begin{equation*}
    z_1\cdot(z_2\cdot a) = z_1 + (z_2 + a)
    = (z_1 + z_2) + a = (z_1 + z_2)\cdot a
  \end{equation*}
  and
  \begin{equation*}
    0\cdot a = 0 + a = a.
  \end{equation*}
  Therefore $\Z$ acts on itself as stated.
\end{proof}

\Exercise3 Show that the additive group $\R$ acts on the $x,y$ plane
$\R\times\R$ by $r\cdot(x,y) = (x + ry, y)$.
\begin{proof}
  For any $r_1,r_2\in\R$ and any $(x,y)\in\R^2$, we have
  \begin{align*}
    r_1\cdot(r_2\cdot(x,y))
    &= r_1\cdot(x + r_2y, y) \\
    &= (x + r_2y + r_1y, y) \\
    &= (x + (r_1 + r_2)y, y) \\
    &= (r_1 + r_2)\cdot(x,y)
  \end{align*}
  and
  \begin{equation*}
    0\cdot(x,y) = (x + 0y, y) = (x,y).
  \end{equation*}
  Therefore $\R$ acts on $\R^2$ in the manner stated above.
\end{proof}

\Exercise4 Let $G$ be a group acting on a set $A$ and fix some
$a\in A$. Show that the following sets are subgroups of $G$:
\begin{enumerate}
\item the kernel of the action
  \begin{proof}
    Suppose $g,h$ are in the kernel of the action. Then for any $b\in A$,
    \begin{equation*}
      (gh)\cdot b = g\cdot(h\cdot b) = g\cdot b = b,
    \end{equation*}
    so $gh$ is in the kernel, and the kernel is closed under the group
    operation. Moreover, if $g$ is in the kernel then
    \begin{equation*}
      b = 1\cdot b = (g^{-1}g)\cdot b
      = g^{-1}\cdot(g\cdot b) = g^{-1}\cdot b,
    \end{equation*}
    so $g^{-1}$ is in the kernel.

    Therefore the kernel of the group action is a nonempty subset of
    $G$ which is closed under the binary operation of $G$ and which is
    closed under inverses, so the kernel is a subgroup of $G$.
  \end{proof}
\item $\{g\in G\mid ga = a\}$ (called the {\em stabilizer} of $a$ in $G$)
  \begin{proof}
    The stabilizer of $a$ is nonempty since $1$ is a member. Now let
    $g,h$ be any members of the stabilizer. Then
    \begin{equation*}
      (gh)\cdot a = g\cdot(h\cdot a) = g\cdot a = a,
    \end{equation*}
    so the stabilizer is closed under the group operation. It is also
    closed under inverses, since
    \begin{equation*}
      a = 1\cdot a = (g^{-1}g)\cdot a
      = g^{-1}\cdot(g\cdot a) = g^{-1}\cdot a.
    \end{equation*}
    Therefore the stabilizer is a subgroup of $G$.
  \end{proof}
\end{enumerate}

\Exercise5 Prove that the kernel of an action of the group $G$ on the
set $A$ is the same as the kernel of the corresponding permutation
representation $G\to S_A$.
\begin{proof}
  Let $\varphi\colon G\to S_A$ be the permutation representation of
  the group action on $A$, so that for $g\in G$ and $a\in A$,
  $\varphi(g)(a) = g\cdot a$.

  If $g\in\ker\varphi$, then $\varphi(g) = 1$, where $1$ is the
  identity permutation on $A$. Then $g.a = a$ for all $a\in A$, and
  $g$ is in the kernel of the action. Conversely, if $g$ is in the
  kernel of the action, then $g.a = a$ for all $a\in A$, so that
  $\varphi(g) = 1$ and $g\in\ker\varphi$. Therefore the kernel of the
  group action and the kernel of the corresponding permutation
  representation are the same.
\end{proof}

\Exercise6 Prove that a group $G$ acts faithfully on a set $A$ if and
only if the kernel of the action is the set consisting only of the
identity.
\begin{proof}
  First, suppose that $G$ acts faithfully on $A$ and let $g$ be an
  element in the kernel of the action. Then $g\cdot a = a$ for all
  $a\in A$. However, $1\cdot a = a$ for all $a\in A$, so the elements
  $1$ and $g$ induce the same permutation on $A$. Since $G$ acts
  faithfully, this must mean that $g = 1$, so that the kernel of the
  action is the set $\{1\}$.

  For the converse, suppose that the kernel of the action is the set
  $\{1\}$. Pick two elements $g$ and $h$ in $G$ and suppose that $g$
  and $h$ induce the same permutation on $A$. Then for any $a\in A$,
  $g\cdot a = h\cdot a$. But then
  \begin{equation*}
    a = (g^{-1}g)\cdot a = g^{-1}\cdot(g\cdot a)
    = g^{-1}\cdot(h\cdot a) = (g^{-1}h)\cdot a.
  \end{equation*}
  Therefore $g^{-1}h$ is in the kernel of the action, so
  $g^{-1}h = 1$. This implies that $g = h$, so that distinct elements
  in $G$ must induce distinct permutations on $A$. This shows that $G$
  acts faithfully on $A$.
\end{proof}

\Exercise7 Prove that in Example~2 in this section the action is
faithful.
\begin{proof}
  If $V$ is a vector space over a field $F$, then the multiplicative
  group $F^\times$ acts on the set $V$ via the mapping $a\cdot v = av$
  for $a\in F^\times$ and $v\in V$. We want to show that this action
  is faithful.

  Let $a,b\in F^\times$ be such that $a\cdot v = b\cdot v$ for all
  $v\in V$. Then
  \begin{align*}
    0 &= a\cdot v + -(a\cdot v) \\
      &= a\cdot v + -(b\cdot v) \\
      &= av - bv \\
      &= (a - b)v.
  \end{align*}
  Since $(a - b)v$ is $0$ even when $v$ is nonzero, this implies that
  $a - b = 0$ or $a = b$. Therefore distinct elements in $F^\times$
  must induce distinct permutations on $V$ and the action is faithful.
\end{proof}

\Exercise8 Let $A$ be a nonempty set and let $k$ be a positive integer
with $k\leq\ord{A}$. The symmetric group $S_A$ acts on the set $B$
consisting of all subsets of $A$ of cardinality $k$ by
$\sigma\cdot\{a_1,\dots,a_k\} = \{\sigma(a_1),\dots,\sigma(a_k)\}$.
\begin{enumerate}
\item Prove that this is a group action.
  \begin{proof}
    Suppose $\sigma_1,\sigma_2\in S_A$. Then for any subset
    $\{a_1,\dots,a_k\}$ of $A$,
    \begin{align*}
      \sigma_1\cdot(\sigma_2\cdot\{a_1,\dots,a_k\})
      &= \sigma_1\cdot\{\sigma_2(a_1),\dots,\sigma_2(a_k)\} \\
      &= \{\sigma_1(\sigma_2(a_1)),\dots,\sigma_1(\sigma_2(a_k))\} \\
      &= (\sigma_1\circ\sigma_2)\cdot\{a_1,\dots,a_k\}
    \end{align*}
    and
    \begin{equation*}
      1\cdot\{a_1,\dots,a_k\}
      = \{1(a_1),\dots, 1(a_k)\}
      = \{a_1,\dots,a_k\}.
    \end{equation*}
    Therefore the specified mapping is a group action.
  \end{proof}
\item Describe explicitly how the elements $(1\,2)$ and $(1\,2\,3)$
  act on the six $2$-element subsets of $\{1,2,3,4\}$.
  \begin{solution}
    We have
    \begin{align*}
      (1\,2)\cdot\{1,2\} &= \{2,1\}, \\
      (1\,2)\cdot\{1,3\} &= \{2,3\}, \\
      (1\,2)\cdot\{1,4\} &= \{2,4\}, \\
      (1\,2)\cdot\{2,3\} &= \{1,3\}, \\
      (1\,2)\cdot\{2,4\} &= \{1,4\}, \\
      (1\,2)\cdot\{3,4\} &= \{3,4\},
    \end{align*}
    and
    \begin{align*}
      (1\,2\,3)\cdot\{1,2\} &= \{2,3\}, \\
      (1\,2\,3)\cdot\{1,3\} &= \{2,1\}, \\
      (1\,2\,3)\cdot\{1,4\} &= \{2,4\}, \\
      (1\,2\,3)\cdot\{2,3\} &= \{3,1\}, \\
      (1\,2\,3)\cdot\{2,4\} &= \{3,4\}, \\
      (1\,2\,3)\cdot\{3,4\} &= \{1,4\}. \qedhere
    \end{align*}
  \end{solution}
\end{enumerate}

\Exercise9 Do both parts of the preceding exercise with ``ordered
$k$-tuples'' in place of ``$k$-element subsets,'' where the action on
$k$-tuples is defined as above but with set braces replaced by
parentheses.
\begin{solution}
  The work is essentially the same, but with $k$-tuples replacing the
  $k$-element subsets, so we omit it. Note that in part (b) there are
  twice as many different $2$-tuples as there are $2$-element subsets,
  since the ordering of the elements is significant.
\end{solution}

\Exercise{10} With reference to the preceding two exercises determine:
\begin{enumerate}
\item for which values of $k$ the action of $S_n$ on $k$-element
  subsets is faithful
  \begin{solution}
    The action of $S_A$ on $k$-element subsets of a set $A$ is
    faithful for all integers $k$ with $1\leq k<\ord{A}$, which we
    will now show. Suppose $\sigma_1$ and $\sigma_2$ are distinct
    permutations in $S_A$. Label the elements of $A$ as
    $\{a_1,a_2,\dots,a_n\}$, where $n = \ord{A}$. Without loss of
    generality, we may suppose that $\sigma_1(a_1)\neq\sigma_2(a_1)$
    (if not, relabel the elements of $A$ so that this is true).

    Now, take any $k$-element subset $B$ of $A$ which contains $a_1$
    but which does not contain $(\sigma_1^{-1}\circ\sigma_2)(a_1)$
    (this is possible since $1\leq k<\ord{A}$). Then $\sigma_1\cdot B$
    does not contain $\sigma_2(a_1)$, however $\sigma_2\cdot B$
    does. Therefore distinct permutations in $S_A$ induce distinct
    permutations on the $k$-element subsets of $A$, so the action is
    faithful (again, assuming $1\leq k<\ord{A}$).
  \end{solution}

\item for which values of $k$ the action of $S_n$ on ordered
  $k$-tuples is faithful
  \begin{solution}
    The action of $S_A$ on ordered $k$-tuples of elements of $A$ is
    faithful for all integers $k$ with $1\leq k\leq\ord{A}$. To see
    this, suppose that $\sigma_1,\sigma_2$ are distinct permutations
    in $S_A$. Suppose for example that
    $\sigma_1(a_1)\neq\sigma_2(a_1)$ and consider the $k$-tuple
    $B = (a_1, a_2, \dots, a_k)$. Then the first coordinate in
    $\sigma_1\cdot B$ is distinct from the first coordinate of
    $\sigma_2\cdot B$. Therefore distinct permutations in $S_A$ induce
    distinct permutations on the set of $k$-tuples, so the action is
    faithful.
  \end{solution}
\end{enumerate}

\Exercise{11} Write out the cycle decomposition of the eight
permutations in $S_4$ corresponding to the elements in $D_8$ given by
the action of $D_8$ on the vertices of a square (where the vertices of
the square are labelled as in Section~2).
\begin{solution}
  Let $\varphi\colon D_8\to S_4$ be the permutation representation
  associated to the action of $D_8$ on the vertices $\{1,2,3,4\}$ of a
  square. Then
  \begin{align*}
    \varphi(1) &= 1, \\
    \varphi(r) &= (1\,2\,3\,4), \\
    \varphi(r^2) &= (1\,3)(2\,4), \\
    \varphi(r^3) &= (1\,4\,3\,2), \\
    \varphi(s) &= (2\,4), \\
    \varphi(sr) &= (1\,4)(2\,3), \\
    \varphi(sr^2) &= (1\,3), \\
    \intertext{and}
    \varphi(sr^3) &= (1\,2)(3\,4). \qedhere
  \end{align*}
\end{solution}

\Exercise{12} Assume $n$ is an even positive integer and show that
$D_{2n}$ acts on the set consisting of pairs of opposite vertices of a
regular $n$-gon. Find the kernel of this action (label vertices as
usual).
\begin{solution}
  Fix an even positive integer $n$. The set of pairs of opposite
  vertices of a regular $n$-gon, labeled in the usual way, is the set
  $P = \{P_1,P_2,\dots,P_{n/2}\}$ where
  \begin{equation*}
    P_k = \left\{k, \frac{n}2 + k\right\}.
  \end{equation*}
  $D_{2n}$ acts on $P$ by $x\cdot P_k = P_\ell$ where $P_\ell$ is the
  set of images of the vertices in $P_k$ under the symmetry $x$. For
  example, in $D_8$, $sr\cdot\{2,6\} = \{7,3\}$ because $sr$ maps
  vertex $2$ to vertex $7$ and vertex $6$ to vertex $3$.

  Let $x,y\in D_{2n}$. It is clear by definition of the action that
  $x\cdot(y\cdot P_k) = (xy)\cdot P_k$ and that $1\cdot P_k = P_k$. So
  this is a group action. The only symmetry in $D_{2n}$ which fixes
  all vertices is the identity $1$. However, since the order of
  vertices in each pair does not matter, any symmetry which only sends
  vertices to their opposites will also fix pairs of vertices. The
  only symmetry which does this is $r^{n/2}$. There is no symmetry
  which fixes only some vertices and which sends all others to their
  opposite vertices, so the kernel of the action is just the set
  $\{1, r^{n/2}\}$.
\end{solution}

\Exercise{13} Find the kernel of the left regular action.
\begin{solution}
  Let $G$ be a group. The kernel of the left regular action is the set
  \begin{equation*}
    \{g\in G\mid \text{$gh = h$ for all $h\in G$}\}.
  \end{equation*}
  By uniqueness of the identity, it is clear that this set is simply
  $\{1\}$. Therefore the left regular action is always faithful.
\end{solution}

\Exercise{14} Let $G$ be a group and let $A = G$. Show that if $G$ is
non-abelian then the maps defined by $g\cdot a = ag$ for all
$g,a\in G$ do {\em not} satisfy the axioms of a (left) group action of
$G$ on itself.
\begin{proof}
  Since $G$ is non-abelian, there exist $g_1,g_2\in G$ such that
  $g_1g_2 \neq g_2g_1$. Then $g_1\cdot(g_2\cdot a) = ag_2g_1$ but
  $(g_1g_2)\cdot a = ag_1g_2$. If $ag_2g_1 = ag_1g_2$ then the
  cancellation law gives $g_2g_1 = g_1g_2$, a contradiction. Therefore
  this map does not define a left group action.
\end{proof}

\Exercise{15} Let $G$ be any group and let $A = G$. Show that the maps
defined by $g\cdot a = ag^{-1}$ for all $g,a\in G$ {\em do} satisfy
the axioms of a (left) group action of $G$ on itself.
\begin{proof}
  Let $g_1,g_2,a\in G$ be arbitrary. Then
  \begin{equation*}
    g_1\cdot(g_2\cdot a) = g_1\cdot(ag_2^{-1})
    = ag_2^{-1}g_1^{-1} = a(g_1g_2)^{-1} = (g_1g_2)\cdot a
  \end{equation*}
  and
  \begin{equation*}
    1\cdot a = a1^{-1} = a1 = a.
  \end{equation*}
  Therefore this map does define a group action.
\end{proof}

\Exercise{16} Let $G$ be any group and let $A = G$. Show that the maps
defined by $g\cdot a = gag^{-1}$ for all $g,a\in G$ {\em do} satisfy
the axioms of a (left) group action (this action of $G$ on itself is
called {\em conjugation}).
\begin{proof}
  For any $g_1,g_2,a\in G$ we have
  \begin{equation*}
    g_1\cdot(g_2\cdot a) = g_1\cdot(g_2ag_2^{-1})
    = g_1g_2ag_2^{-1}g_1^{-1} = (g_1g_2)a(g_1g_2)^{-1}
    = (g_1g_2)\cdot a
  \end{equation*}
  and
  \begin{equation*}
    1\cdot a = 1a1 = a,
  \end{equation*}
  so this mapping does define a group action.
\end{proof}

\Exercise{17} Let $G$ be a group and let $G$ act on itself by left
conjugation, so each $g\in G$ maps $G$ to $G$ by
\begin{equation*}
  x\mapsto gxg^{-1}.
\end{equation*}
For fixed $g\in G$, prove that conjugation by $g$ is an isomorphism
from $G$ onto itself (i.e., is an automorphism of $G$). Deduce that
$x$ and $gxg^{-1}$ have the same order for all $x$ in $G$ and that for
any subset $A$ of $G$, $\ord{A} = \ord{gAg^{-1}}$ (here
$gAg^{-1} = \{gag^{-1}\mid a\in A\}$).
\label{exercise-conjugation-is-automorphism}
\begin{proof}
  Fix a $g\in G$ and let $\varphi\colon G\to G$ denote the map
  $x\mapsto gxg^{-1}$. Then for any $x_1,x_2\in G$ we have
  \begin{equation*}
    \varphi(x_1x_2) = gx_1x_2g^{-1}
    = (gx_1g^{-1})(gx_2g^{-1}) = \varphi(x_1)\varphi(x_2)
  \end{equation*}
  so $\varphi$ is a homomorphism.

  Next, suppose $\varphi(x_1) = \varphi(x_2)$. Then
  $gx_1g^{-1} = gx_2g^{-1}$ and multiplying both sides of this
  equation on the left by $g^{-1}$ and on the right by $g$ gives
  $x_1 = x_2$, so that $\varphi$ is injective.

  Now let $y\in G$ be arbitrary. Then $x = g^{-1}yg$ is such that
  $\varphi(x) = y$, so $\varphi$ is surjective. Therefore $\varphi$ is
  an automorphism.

  Since isomorphisms preserve order, we see that each element $x$ in
  $G$ has the same order as its conjugate $gxg^{-1}$. Moreover, if
  $A\subseteq G$ then the restriction of $\varphi$ to $A$,
  $\varphi|_A\colon A\to gAg^{-1}$, is still a bijection, so
  $\ord{A} = \ord{gAg^{-1}}$.
\end{proof}

\Exercise{18} Let $H$ be a group acting on a set $A$. Prove that the
relation $\sim$ on $A$ defined by
\begin{equation*}
  a\sim b
  \quad\text{if and only if}\quad
  a = hb\;\;\text{for some $h\in H$}
\end{equation*}
is an equivalence relation. (For each $x\in A$ the equivalence classes
of $x$ under $\sim$ is called the {\em orbit} of $x$ under the action
of $H$. The orbits under the action of $H$ partition the set $A$.)
\begin{proof}
  Since $a = 1a$ we have $a\sim a$. And if $a = hb$ for $h\in H$ then
  \begin{equation*}
    h^{-1}a = h^{-1}(hb) = (h^{-1}h)b = b,
  \end{equation*}
  so $a\sim b$ implies $b\sim a$.

  Lastly, suppose $a\sim b$ and $b\sim c$ and let $h_1,h_2\in H$ be
  such that $a = h_1b$ and $b = h_2c$. Then
  $a = h_1(h_2c) = (h_1h_2)c$ and $a\sim c$. Hence $\sim$ is an
  equivalence relation.
\end{proof}

\Exercise{19} Let $H$ be a subgroup of the finite group $G$ and let
$H$ act on $G$ (here $A = G$) by left multiplication. Let $x\in G$ and
let $\mathcal{O}$ be the orbit of $x$ under the action of $H$. Prove
that the map
\begin{equation*}
  H\to\mathcal{O}
  \quad\text{defined by}\quad
  h\mapsto hx
\end{equation*}
is a bijection (hence all orbits have cardinality $\ord{H}$). From
this and the preceding exercises deduce {\em Lagrange's Theorem}:
\begin{quote}
  \em if $G$ is a finite group and $H$ is a subgroup of $G$ then
  $\ord{H}$ divides $\ord{G}$.
\end{quote}
\begin{proof}
  Let $\varphi\colon H\to\mathcal{O}$ denote the map $h\mapsto
  hx$. Suppose $\varphi(h) = \varphi(k)$ for $h,k\in H$. Then
  $hx = kx$ and right cancellation implies that $h = k$, so that
  $\varphi$ is injective. And $\varphi$ is surjective by definition
  ($y\in\mathcal{O}$ means that there is $h\in H$ such that $hx =
  y$). Therefore $\varphi$ is a bijection and
  $\ord{H} = \ord{\mathcal{O}}$.

  From the previous exercise, we know that the orbits under the action
  of $H$ partition $G$. Each equivalence class $\mathcal{O}$ has
  cardinality $\ord{H}$, so $\ord{G} = n\ord{H}$ where $n$ is the
  number of orbits. Hence $\ord{H}$ divides $\ord{G}$.
\end{proof}

\Exercise{20} Show that the group of rigid motions of a tetrahedron is
isomorphic to a subgroup of $S_4$.
\begin{proof}
  Call the group of rigid motions of the tetrahedron $G$. Number each
  vertex and let $A$ denote the set $\{1,2,3,4\}$. Then each rigid
  motion $\alpha\in G$ induces a permutation $\sigma_\alpha\in S_4$ of
  $A$. $G$ acts on $A$ via the map $\alpha i = \sigma_\alpha(i)$.

  Since each distinct $\alpha\in G$ permutes the vertices in a
  different way, we get an injective homomorphism
  \begin{equation*}
    \varphi\colon G\to S_4
    \quad\text{given by}\quad
    \varphi(\alpha) = \sigma_\alpha.
  \end{equation*}
  Then $\varphi(G)$ is a subgroup of $S_4$, and by simply restricting
  the codomain of $\varphi$ we have an isomorphism from $G$ to this
  subgroup of $S_4$.
\end{proof}

\Exercise{21} Show that the group of rigid motions of a cube is
isomorphic to $S_4$.
\begin{proof}
  Again let $G$ denote the group of rigid motions and let
  $A = \{1,2,3,4\}$, where each $i\in A$ corresponds to a pair of
  opposing vertices on a cube. Each $\alpha\in G$ sends each pair of
  opposing vertices to a new pair of opposing vertices. Therefore $G$
  acts on $A$.

  Consider the homomorphism $\varphi\colon G\to S_4$ given by
  \begin{equation*}
    \varphi(\alpha)(i) = \alpha i.
  \end{equation*}
  Then $\varphi$ is injective since each distinct rigid motion
  $\alpha\in G$ gives rise to a different permutation of $A$. From
  Exercise~\ref{exercise-order-of-rigid-motions-of-a-cube} we know
  that $\ord{G} = 24 = \ord{S_4}$, so $\varphi$ is in fact an
  isomorphism.
\end{proof}

\Exercise{22} Show that the group of rigid motions of an octahedron is
isomorphic to $S_4$. Deduce that the groups of rigid motions of a cube
and an octahedron are isomorphic.
\begin{proof}
  Number each pair of opposing faces of the octahedron $1,2,3,4$. Let
  $G$ be the group of rigid motions of the octahedron and let
  $A = \{1,2,3,4\}$. Each $\alpha\in G$ sends each pair of opposing
  faces to a new pair of opposing faces, so $G$ acts on $A$.

  As in the previous exercise, we see that the homomorphism
  \begin{equation*}
    \varphi\colon G\to S_4
    \quad\text{given by}\quad
    \varphi(\alpha)(i) = \alpha i.
  \end{equation*}
  is injective. By
  Exercise~\ref{exercise-order-of-rigid-motions-of-an-octahedron} we
  have $\ord{G} = 24 = \ord{S_4}$, so $\varphi$ is an isomorphism.

  From this and the previous exercise, we see that the groups of rigid
  motions of the cube and the octahedron are isomorphic.
\end{proof}

\Exercise{23} Explain why the action of the group of rigid motions of
a cube on the set of three pairs of opposite faces is not
faithful. Find the kernel of this action.
\begin{solution}
  The group of rigid motions of a cube has order $24$ but the
  permutations on the set of pairs of opposite faces has order
  $\ord{S_3} = 6$. Therefore the action cannot be faithful.

  Construct a line through the center of each pair of opposite
  faces. Then a $180^\circ$ rotation about each of these lines will
  send each pair of opposite faces to itself. These are the only
  rotations that fix pairs of opposing faces, so the kernel of the
  action consists of these three $180^\circ$ rotations along with the
  identity transformation.
\end{solution}
