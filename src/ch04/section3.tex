\section{Groups Acting on Themselves by Conjugation}

Let $G$ be a group.

\Exercise1 Suppose $G$ has a left action on a set $A$, denoted by
$g\cdot a$ for all $g\in G$ and $a\in A$. Denote the corresponding
right action on $A$ by $a\cdot g$. Prove that the (equivalence)
relations $\sim$ and $\sim'$ defined by
\begin{equation*}
  a\sim b \quad\text{if and only if}\quad a = g\cdot b
  \quad\text{for some $g\in G$}
\end{equation*}
and
\begin{equation*}
  a\sim'b \quad\text{if and only if}\quad a = b\cdot g
  \quad\text{for some $g\in G$}
\end{equation*}
are the same relation (i.e., $a\sim b$ if and only if $a\sim'b$).
\begin{proof}
  Suppose $a\sim b$ and let $g$ be such that $a = g\cdot b$. Take
  $h = g^{-1}$. Then, since the left and right actions correspond, we
  have
  \begin{equation*}
    a = g\cdot b = b\cdot g^{-1} = b\cdot h.
  \end{equation*}
  Hence $a\sim' b$. We can also see that the implication works in the
  other direction. Therefore $\sim$ and $\sim'$ are the same relation.
\end{proof}

\Exercise2
\label{exercise:group-actions:conj-classes-D8-Q8-A4}
Find all conjugacy classes and their sizes in the following
groups:
\begin{enumerate}
\item $D_8$
  \begin{solution}
    We already know $Z(D_8) = \{1, r^2\}$. As stated in the text, for
    any $x\in D_8$ that is not in the center, $\ord{C_{D_8}(x)} =
    4$. That is, the centralizer has index $2$ and there are two
    elements in each such conjugacy class.

    First, since $srs^{-1} = r^3$, we see that $\{r, r^3\}$ is one
    conjugacy class. Next, $r(sr^2)r^{-1} = s$, so $\{s, sr^2\}$ is
    another. Lastly, $s(sr)s^{-1} = sr^3$, so $\{sr, sr^3\}$ is the
    remaining class.

    The conjugacy classes of $D_8$ are thus
    \begin{equation*}
      \{1\}, \quad
      \{r^2\}, \quad
      \{r, r^3\}, \quad
      \{s, sr^2\}, \quad\text{and}\quad
      \{sr, sr^3\}.
    \end{equation*}
    We see that the class equation is $\ord{D_8} = 2 + 2 + 2 + 2$.
  \end{solution}

\item $Q_8$
  \begin{solution}
    The center of $Q_8$ is $Z(Q_8) = \{1, -1\}$. In the text it was
    shown that $\{i, -i\}$ is one conjugacy class. In a similar way we
    can determine that
    $\ord{Q_8: C_{Q_8}(j)} = \ord{Q_8: C_{Q_8}(k)} = 2$ so the
    remaining classes have two elements. Since $iji^{-1} = -j$, we see
    that $\{j, -j\}$ is another class. And $jkj^{-1} = -k$, so the
    remaining class is $\{k, -k\}$.

    The conjugacy classes of $Q_8$ are therefore
    \begin{equation*}
      \{1\}, \quad
      \{-1\}, \quad
      \{i, -i\}, \quad
      \{j, -j\}, \quad\text{and}\quad
      \{k, -k\}.
    \end{equation*}
    The class equation is $\ord{Q_8} = 2 + 2 + 2 + 2$.
  \end{solution}

\item $A_4$
  \begin{solution}
    $A_4$ has a trivial center. Consider the even permutations
    \begin{equation*}
      1, \quad
      (1\,2\,3), \quad\text{and}\quad
      (1\,2)(3\,4).
    \end{equation*}
    These account for all the cycle types in $A_4$.

    The $3$-cycle $(1\,2\,3)$ commutes only with powers of itself and
    the identity. So $\ord{C_{A_4}((1\,2\,3))} = 3$ and
    $\ord{A_4 : C_{A_4}((1\,2\,3))} = 4$. Therefore there are four
    distinct conjugates of $(1\,2\,3)$. Since there are a total of
    eight $3$-cycles in $A_4$, they must be divided into two conjugacy
    classes. We compute,
    \begin{gather*}
      ((1\,2)(3\,4))(1\,2\,3)((1\,2)(3\,4))^{-1}
      = (1\,4\,2), \\
      ((1\,3)(2\,4))(1\,2\,3)((1\,3)(2\,4))^{-1}
      = (1\,3\,4), \\ \intertext{and}
      ((1\,4)(2\,3))(1\,2\,3)((1\,4)(2\,3))^{-1}
      = (2\,4\,3).
    \end{gather*}
    The remaining $3$-cycles would form the other class.

    Lastly, $(1\,2)(3\,4)$ commutes with $(1\,3)(2\,4)$ and
    $(1\,4)(2\,3)$, so its centralizer has four members and
    \begin{equation*}
      \ord{A_4 : C_{A_4}((1\,2)(3\,4))} = 3.
    \end{equation*}
    Therefore the three elements of order $2$ belong to a single
    conjugacy class.

    In summary, the conjugacy classes of $A_4$ are
    \begin{multline*}
      \{1\}, \quad
      \{(1\,2\,3), (1\,4\,2), (1\,3\,4), (2\,4\,3)\}, \quad
      \{(1\,3\,2), (1\,2\,4), (1\,4\,3), (2\,3\,4)\}, \\\text{and}\quad
      \{(1\,2)(3\,4), (1\,3)(2\,4), (1\,4)(2\,3)\}.
    \end{multline*}
    The class equation is $\ord{A_4} = 1 + 3 + 4 + 4$.
  \end{solution}
\end{enumerate}

\Exercise3 Find all the conjugacy classes and their sizes in the
following groups:
\begin{enumerate}
\item \label{itm:group-actions:Z2-cross-S3-conjugacy}
  $Z_2\times S_3$
  \begin{solution}
    Let $Z_2 = \gen{x}$. Since $Z_2$ is abelian, the group
    $Z_2\times S_3$ has essentially the same conjugacy classes as
    $S_3$, except that the classes come in pairs based on the first
    coordinate for each element. Again, because $Z_2$ is abelian, no
    two elements in $Z_2\times S_3$ that differ in their first
    coordinate can be conjugate to one another.

    For example, we know that the conjugacy class of $(1\,2\,3)$ in
    $S_3$ consists of the two $3$-cycles in $S_3$:
    \begin{equation*}
      \{(1\,2\,3), (1\,3\,2)\}.
    \end{equation*}
    So, the conjugacy class of $(1, (1\,2\,3))$ in $Z_2\times S_3$
    should be
    \begin{equation*}
      \{(1, (1\,2\,3)),\; (1, (1\,3\,2))\}.
    \end{equation*}
    Then we also have another class given by
    \begin{equation*}
      \{(x, (1\,2\,3)),\; (x, (1\,3\,2))\}.
    \end{equation*}

    The same reasoning applies to the $2$-cycles in $S_3$. We see
    therefore that $Z_2\times S_3$ has the following conjugacy
    classes:
    \begin{gather*}
      \{(1, 1)\}, \\
      \{(x, 1)\}, \\
      \{(1, (1\,2\,3)),\; (1, (1\,3\,2))\}, \\
      \{(x, (1\,2\,3)),\; (x, (1\,3\,2))\}, \\
      \{(1, (1\,2)),\; (1, (1\,3)),\; (1, (2\,3))\}, \\\intertext{and}
      \{(x, (1\,2)),\; (x, (1\,3)),\; (x, (2\,3))\}.
    \end{gather*}
    The class equation is $\ord{Z_2\times S_3} = 2 + 2 + 2 + 3 + 3$.
  \end{solution}

\item $S_3\times S_3$
  \begin{solution}
    Two elements $(\sigma_1, \sigma_2)$ and $(\tau_1, \tau_2)$ are
    conjugate in $S_3\times S_3$ if and only if there is
    $(\rho_1, \rho_2)$ in $S_3\times S_3$ such that
    \begin{align*}
      (\sigma_1, \sigma_2)
      &= (\rho_1, \rho_2)(\tau_1, \tau_2)(\rho_1, \rho_2)^{-1} \\
      &= (\rho_1\tau_1\rho_1^{-1}, \rho_2\tau_2\rho_2^{-1}).
    \end{align*}
    This is true if and only if
    \begin{equation*}
      \sigma_1 = \rho_1\tau_1\rho_1^{-1} \quad\text{and}\quad
      \sigma_2 = \rho_2\tau_2\rho_2^{-1}.
    \end{equation*}
    That is, if and only if $\sigma_1$ and $\tau_1$ are conjugate in
    $S_3$ and $\sigma_2$ and $\tau_2$ are also conjugate in $S_3$.

    Given this, we can determine all the conjugacy classes. To keep
    the writing compact, we will use the following labels for elements in $S_3$:
    \begin{equation*}
      a_1 = (1\,2), \quad a_2 = (1\,3), \quad a_3 = (2\,3),
    \end{equation*}
    and
    \begin{equation*}
      b_1 = (1\,2\,3), \quad b_2 = (1\,3\,2).
    \end{equation*}
    Then the conjugacy classes in $S_3\times S_3$ are
    \begin{gather*}
      \{(1, 1)\}, \\
      \{(1, b_1), (1, b_2)\}, \\
      \{(b_1, 1), (b_2, 1)\}, \\
      \{(1, a_1), (1, a_2), (1, a_3)\}, \\
      \{(a_1, 1), (a_2, 1), (a_3, 1)\}, \\
      \{(b_1, b_1), (b_1, b_2),
      (b_2, b_1), (b_2, b_2)\}, \\
      \{(a_1, b_1), (a_1, b_2),
      (a_2, b_1), (a_2, b_2),
      (a_3, b_1), (a_3, b_2)\},\\
      \{(b_1, a_1), (b_1, a_2), (b_1, a_3),
      (b_2, a_1), (b_2, a_2), (b_2, a_3)\},\\
    \end{gather*}
    and
    \begin{multline*}
      \{(a_1,a_1), (a_1, a_2), (a_1, a_3),
      (a_2,a_1), (a_2, a_2), \\ (a_2, a_3),
      (a_3,a_1), (a_3, a_2), (a_3, a_3)\}.
    \end{multline*}
    The class equation is
    $\ord{S_3\times S_3} = 1 + 2 + 2 + 3 + 3 + 4 + 6 + 6 + 9$.
  \end{solution}

\item $Z_3\times A_4$
  \begin{solution}
    Let $Z_3 = \gen{x}$. As in part
    \ref{itm:group-actions:Z2-cross-S3-conjugacy} above, elements that
    differ in their first coordinate belong to separate conjugacy
    classes. Using our result from Exercise
    \ref{exercise:group-actions:conj-classes-D8-Q8-A4}, we see that
    the conjugacy classes of $Z_3\times A_4$ are
    \begin{gather*}
      \{(1,1)\}, \quad \{(x,1)\}, \quad \{(x^2,1)\}, \\[3pt]
      \{(1, (1\,2)(3\,4)),\; (1, (1\,3)(2\,4)),\; (1, (1\,4)(2\,3))\}, \\[3pt]
      \{(x, (1\,2)(3\,4)),\; (x, (1\,3)(2\,4)),\; (x, (1\,4)(2\,3))\}, \\[3pt]
      \{(x^2, (1\,2)(3\,4)),\; (x^2, (1\,3)(2\,4)),\; (x^2, (1\,4)(2\,3))\}, \\[3pt]
      \{(1, (1\,2\,3)),\; (1, (1\,4\,2)),\; (1, (1\,3\,4)),\; (1, (2\,4\,3))\}, \\[3pt]
      \{(x, (1\,2\,3)),\; (x, (1\,4\,2)),\; (x, (1\,3\,4)),\; (x, (2\,4\,3))\}, \\[3pt]
      \{(x^2, (1\,2\,3)),\; (x^2, (1\,4\,2)),\; (x^2, (1\,3\,4)),\; (x^2, (2\,4\,3))\}, \\[3pt]
      \{(1, (1\,3\,2)),\; (1, (1\,2\,4)),\; (1, (1\,4\,3)),\; (1, (2\,3\,4))\}, \\[3pt]
      \{(x, (1\,3\,2)),\; (x, (1\,2\,4)),\; (x, (1\,4\,3)),\; (x, (2\,3\,4))\}, \\
      \intertext{and}
      \{(x^2, (1\,3\,2)),\; (x^2, (1\,2\,4)),\; (x^2, (1\,4\,3)),\; (x^2, (2\,3\,4))\}.
    \end{gather*}
    The class equation is
    \begin{equation*}
      \ord{Z_3\times A_4} = 3 + 3 + 3 + 3 + 4 + 4 + 4 + 4 + 4 + 4. \qedhere
    \end{equation*}
  \end{solution}
\end{enumerate}

\Exercise4 Prove that if $S\subseteq G$ and $g\in G$ then
\begin{equation*}
  gN_G(S)g^{-1} = N_G(gSg^{-1}) \quad\text{and}\quad
  gC_G(S)g^{-1} = C_G(gSg^{-1}).
\end{equation*}
\begin{proof}
  Let $h\in gN_G(S)g^{-1}$. Then we can write $h = gkg^{-1}$ for some
  $k\in G$ with $kSk^{-1} = S$. So
  \begin{align*}
    h(gSg^{-1})h^{-1}
    &= (gkg^{-1})gSg^{-1}(gkg^{-1})^{-1} \\
    &= g(kSk^{-1})g^{-1} \\
    &= gSg^{-1}.
  \end{align*}
  This shows that $h \in N_G(gSg^{-1})$. Conversely, if
  $h\in N_G(gSg^{-1})$ then
  \begin{equation*}
    hgSg^{-1}h^{-1} = gSg^{-1}
  \end{equation*}
  and, by multiplying on the left by $g^{-1}$ and on the right by $g$,
  we see that
  \begin{equation*}
    g^{-1}hgSg^{-1}h^{-1}g = S,
  \end{equation*}
  i.e., that $g^{-1}hg\in N_G(S)$. This implies that
  $h \in gN_G(S)g^{-1}$ as required.

  Similarly, let $h\in gC_G(S)g^{-1}$. Then we can write
  $h = gkg^{-1}$ for some $k\in G$ such that $ksk^{-1} = s$ for all
  $s\in S$. Then for any $s\in S$,
  \begin{align*}
    hgsg^{-1}h^{-1}
    &= (gkg^{-1})gsg^{-1}(gkg^{-1})^{-1} \\
    &= gksk^{-1}g^{-1} \\
    &= gsg^{-1},
  \end{align*}
  so $h\in C_G(gSg^{-1})$. Conversely, if $h\in C_G(gSg^{-1})$, then
  \begin{equation*}
    hgsg^{-1}h^{-1} = gsg^{-1} \quad \text{for all $s\in S$},
  \end{equation*}
  so
  \begin{equation*}
    g^{-1}hgsg^{-1}h^{-1}g = s \quad \text{for all $s\in S$},
  \end{equation*}
  and we see that $g^{-1}hg\in C_G(S)$ or $h \in gC_G(S)g^{-1}$,
  completing the proof.
\end{proof}

\Exercise5 If the center of $G$ is of index $n$, prove that every
conjugacy class has at most $n$ elements.
\begin{proof}
  Take any $g\in G$. We know that
  $Z(G) \leq C_G(g) \leq G$. By
  Exercise~\ref{exercise:quotient-group:subgroup-of-subgroup-index},
  we also know that
  \begin{equation*}
    n = \ord{G : Z(G)} = \ord{G : C_G(g)}\ord{C_G(g) : Z(G)}.
  \end{equation*}
  In particular, $\ord{G : C_G(g)}$ divides $n$, so
  $\ord{G : C_G(g)}\leq n$. By Proposition~6, we know that the
  conjugacy class containing $g$ has at most $n$ elements.
\end{proof}

\Exercise6 Assume $G$ is a non-abelian group of order $15$. Prove that
$Z(G) = 1$. Use the fact that $\gen{g} \leq C_G(g)$ for all $g\in G$
to show that there is at most one possible class equation for $G$.
\begin{proof}
  Suppose $Z(G)$ has a nontrivial element. Since $G$ is non-abelian
  and $Z(G)\leq G$, we must either have $\ord{Z(G)} = 3$ or
  $\ord{Z(G)} = 5$. In either case, $Z(G)$ has prime index, and
  therefore the quotient $G/Z(G)$ must be cyclic. By
  Exercise~\ref{exercise:quotient-group:center-quotient-cyclic}, $G$
  is abelian. This is a contradiction, so $Z(G)$ must be trivial.

  Next, since $G$ cannot be cyclic and has a trivial center, any
  non-identity element must have order $3$ or $5$, so its centralizer
  must have an index of $3$ or $5$. By Proposition~6, we see that the
  only possible class equation for $G$ is
  \begin{equation*}
    \ord{G} = 1 + 3 + 3 + 3 + 5,
  \end{equation*}
  since this is the only partition of $15$ with acceptable values.
\end{proof}

\Exercise7 For $n = 3$, $4$, $6$ and $7$ make lists of the partitions
of $n$ and give representatives for the corresponding conjugacy
classes of $S_n$.
\begin{solution}
  Proposition~11 and its proof shows how to form a correspondence
  between the possible cycle types of elements in $S_n$ and the
  conjugacy classes of $S_n$. For example, with $n = 3$ the possible
  partitions are $1, 1, 1$; $1, 2$; or $3$. And each of these
  corresponds to a distinct conjugacy class containing elements with
  the same cycle type:
  \begin{center}
    \begin{tabular}{r|l}
      \bf Partition of 3 & \bf Representative of Conjugacy Class \\
      \hline
      $1, 1, 1$ & $1$ \\
      $1, 2$ & $(1\,2)$ \\
      $3$ & $(1\,2\,3)$
    \end{tabular}
  \end{center}

  For $n = 4$ we get five partitions, corresponding to five conjugacy
  classes:
  \begin{center}
    \begin{tabular}{r|l}
      \bf Partition of 4 & \bf Representative of Conjugacy Class \\
      \hline
      $1, 1, 1, 1$ & $1$ \\
      $1, 1, 2$ & $(1\,2)$ \\
      $1, 3$ & $(1\,2\,3)$ \\
      $2, 2$ & $(1\,2)(3\,4)$ \\
      $4$ & $(1\,2\,3\,4)$
    \end{tabular}
  \end{center}

  For $n = 6$, we have $11$ classes:
  \begin{center}
    \begin{tabular}{r|l}
      \bf Partition of 6 & \bf Representative of Conjugacy Class \\
      \hline
      $1, 1, 1, 1, 1, 1$ & $1$ \\
      $1, 1, 1, 1, 2$ & $(1\,2)$ \\
      $1, 1, 1, 3$ & $(1\,2\,3)$ \\
      $1, 1, 2, 2$ & $(1\,2)(3\,4)$ \\
      $1, 1, 4$ & $(1\,2\,3\,4)$ \\
      $1, 2, 3$ & $(1\,2)(3\,4\,5)$ \\
      $1, 5$ & $(1\,2\,3\,4\,5)$ \\
      $2, 2, 2$ & $(1\,2)(3\,4)(5\,6)$ \\
      $2, 4$ & $(1\,2)(3\,4\,5\,6)$ \\
      $3, 3$ & $(1\,2\,3)(4\,5\,6)$ \\
      $6$ & $(1\,2\,3\,4\,5\,6)$
    \end{tabular}
  \end{center}

  Finally, for $n = 7$ we get $15$ classes:
  \begin{center}
    \begin{tabular}{r|l}
      \bf Partition of 7 & \bf Representative of Conjugacy Class \\
      \hline
      $1, 1, 1, 1, 1, 1, 1$ & $1$ \\
      $1, 1, 1, 1, 1, 2$ & $(1\,2)$ \\
      $1, 1, 1, 1, 3$ & $(1\,2\,3)$ \\
      $1, 1, 1, 2, 2$ & $(1\,2)(3\,4)$ \\
      $1, 1, 1, 4$ & $(1\,2\,3\,4)$ \\
      $1, 1, 2, 3$ & $(1\,2)(3\,4\,5)$ \\
      $1, 1, 5$ & $(1\,2\,3\,4\,5)$ \\
      $1, 2, 2, 2$ & $(1\,2)(3\,4)(5\,6)$ \\
      $1, 2, 4$ & $(1\,2)(3\,4\,5\,6)$ \\
      $1, 3, 3$ & $(1\,2\,3)(4\,5\,6)$ \\
      $1, 6$ & $(1\,2\,3\,4\,5\,6)$ \\
      $2, 2, 3$ & $(1\,2)(3\,4)(5\,6\,7)$ \\
      $2, 5$ & $(1\,2)(3\,4\,5\,6\,7)$ \\
      $3, 4$ & $(1\,2\,3)(4\,5\,6\,7)$ \\
      $7$ & $(1\,2\,3\,4\,5\,6\,7)$
    \end{tabular}
  \end{center}
\end{solution}
