\chapter{Group Actions}

\section{Group Actions and Permutation Representations}

Let $G$ be a group and let $A$ be a nonempty set.

\Exercise1
\label{exercise:group-action:stab-b-conjugate-stab-a}
Let $G$ act on the set $A$. Prove that if $a,b\in A$ and
$b = g\cdot a$ for some $g\in G$, then $G_b = gG_ag^{-1}$ ($G_a$ is
the stabilizer of $a$). Deduce that if $G$ acts transitively on $A$
then the kernel of the action is $\bigcap_{g\in G} gG_ag^{-1}$.
\begin{proof}
  Suppose $a,b\in A$ with $b = g\cdot a$ for some $g\in G$. If
  $h\in G_b$, then $h\cdot b = b$. So
  \begin{align*}
    (g^{-1}hg)\cdot a
    &= (g^{-1}h)\cdot(g\cdot a) \\
    &= g^{-1}\cdot(h\cdot b) \\
    &= g^{-1}\cdot b \\
    &= a,
  \end{align*}
  and we have $g^{-1}hg\in G_a$. Letting $k = g^{-1}hg$, we see that
  $h = gkg^{-1}$ where $k\in G_a$. This shows that
  $G_b\subseteq gG_ag^{-1}$.

  On the other hand, if $h\in gG_ag^{-1}$, then there is a $k\in G_a$
  with $h = gkg^{-1}$. Then
  \begin{align*}
    h\cdot b
    &= (gkg^{-1})\cdot b \\
    &= (gk)\cdot(g^{-1}\cdot b) \\
    &= (gk)\cdot a \\
    &= g\cdot(k\cdot a) \\
    &= g\cdot a \\
    &= b,
  \end{align*}
  so $h\in G_b$. This completes the proof that $G_b = gG_ag^{-1}$.

  We know that the kernel of the action is the set of elements of $G$
  which stabilize every point in $A$. If the action is transitive,
  then for any $a,b\in A$ there is some $g\in G$ such that
  $G_b = gG_ag^{-1}$, as has already been proved. Thus we can write
  \begin{equation*}
    \bigcap_{g\in G}gG_ag^{-1} = \bigcap_{b\in A}G_b,
  \end{equation*}
  so that the kernel of the action is $\bigcap_{g\in G}gG_ag^{-1}$.
\end{proof}

\Exercise2
\label{exercise:group-action:perm-stab-b-conjugate-stab-a}
Let $G$ be a {\em permutation group} on the set $A$ (i.e.,
$G\leq S_A$), let $\sigma\in G$ and let $a\in A$. Prove that
$\sigma G_a\sigma^{-1} = G_{\sigma(a)}$. Deduce that if $G$ acts
transitively on $A$ then
\begin{equation*}
  \bigcap_{\sigma\in G}\sigma G_a\sigma^{-1} = 1.
\end{equation*}
\begin{proof}
  Since $G$ naturally acts on $A$ via the map
  $\sigma\cdot a\mapsto\sigma(a)$, we know by the previous exercise,
  Exercise~\ref{exercise:group-action:stab-b-conjugate-stab-a}, that
  \begin{equation*}
    G_{\sigma(a)} = \sigma G_a\sigma^{-1}.
  \end{equation*}

  Since $G$ acts faithfully on $A$, the kernel of the action is
  trivial. If the action is also transitive, then (again by the
  previous exercise) we have
  \begin{equation*}
    \bigcap_{\sigma\in G}\sigma G_a\sigma^{-1} = 1. \qedhere
  \end{equation*}
\end{proof}

\Exercise3 Assume that $G$ is an abelian, transitive subgroup of
$S_A$. Show that $\sigma(a) \neq a$ for all $\sigma\in G - \{1\}$ and
all $a\in A$. Deduce that $\ord{G} = \ord{A}$.
\begin{proof}
  Let $a\in A$. Since $G$ is abelian, its subgroups are normal. In
  particular, $\sigma G_a\sigma^{-1} = G_a$. Since $G$ is also
  transitive, we know by the previous exercise,
  Exercise~\ref{exercise:group-action:perm-stab-b-conjugate-stab-a},
  that
  \begin{equation*}
    G_a = \bigcap_{\sigma\in G}\sigma G_a\sigma^{-1} = 1.
  \end{equation*}
  Thus the stabilizer of $a$ is trivial, so $\sigma(a) = a$ if and
  only if $\sigma$ is the identity.

  Finally, since $G$ has only one orbit, we have by Proposition~2 that
  \begin{equation*}
    \ord{G : G_a} = \ord{A}.
  \end{equation*}
  But by Lagrange's Theorem,
  \begin{equation*}
    \ord{G : G_a} = \frac{\ord{G}}{\ord{G_a}} = \ord{G},
  \end{equation*}
  so $\ord{G} = \ord{A}$ as required.
\end{proof}
