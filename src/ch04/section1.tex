\chapter{Group Actions}

\section{Group Actions and Permutation Representations}

Let $G$ be a group and let $A$ be a nonempty set.

\Exercise1
\label{exercise:group-action:stab-b-conjugate-stab-a}
Let $G$ act on the set $A$. Prove that if $a,b\in A$ and
$b = g\cdot a$ for some $g\in G$, then $G_b = gG_ag^{-1}$ ($G_a$ is
the stabilizer of $a$). Deduce that if $G$ acts transitively on $A$
then the kernel of the action is $\bigcap_{g\in G} gG_ag^{-1}$.
\begin{proof}
  Suppose $a,b\in A$ with $b = g\cdot a$ for some $g\in G$. If
  $h\in G_b$, then $h\cdot b = b$. So
  \begin{align*}
    (g^{-1}hg)\cdot a
    &= (g^{-1}h)\cdot(g\cdot a) \\
    &= g^{-1}\cdot(h\cdot b) \\
    &= g^{-1}\cdot b \\
    &= a,
  \end{align*}
  and we have $g^{-1}hg\in G_a$. Letting $k = g^{-1}hg$, we see that
  $h = gkg^{-1}$ where $k\in G_a$. This shows that
  $G_b\subseteq gG_ag^{-1}$.

  On the other hand, if $h\in gG_ag^{-1}$, then there is a $k\in G_a$
  with $h = gkg^{-1}$. Then
  \begin{align*}
    h\cdot b
    &= (gkg^{-1})\cdot b \\
    &= (gk)\cdot(g^{-1}\cdot b) \\
    &= (gk)\cdot a \\
    &= g\cdot(k\cdot a) \\
    &= g\cdot a \\
    &= b,
  \end{align*}
  so $h\in G_b$. This completes the proof that $G_b = gG_ag^{-1}$.

  We know that the kernel of the action is the set of elements of $G$
  which stabilize every point in $A$. If the action is transitive,
  then for any $a,b\in A$ there is some $g\in G$ such that
  $G_b = gG_ag^{-1}$, as has already been proved. Thus we can write
  \begin{equation*}
    \bigcap_{g\in G}gG_ag^{-1} = \bigcap_{b\in A}G_b,
  \end{equation*}
  so that the kernel of the action is $\bigcap_{g\in G}gG_ag^{-1}$.
\end{proof}

\Exercise2
\label{exercise:group-action:perm-stab-b-conjugate-stab-a}
Let $G$ be a {\em permutation group} on the set $A$ (i.e.,
$G\leq S_A$), let $\sigma\in G$ and let $a\in A$. Prove that
$\sigma G_a\sigma^{-1} = G_{\sigma(a)}$. Deduce that if $G$ acts
transitively on $A$ then
\begin{equation*}
  \bigcap_{\sigma\in G}\sigma G_a\sigma^{-1} = 1.
\end{equation*}
\begin{proof}
  Since $G$ naturally acts on $A$ via the map
  $\sigma\cdot a\mapsto\sigma(a)$, we know by the previous exercise,
  Exercise~\ref{exercise:group-action:stab-b-conjugate-stab-a}, that
  \begin{equation*}
    G_{\sigma(a)} = \sigma G_a\sigma^{-1}.
  \end{equation*}

  Since $G$ acts faithfully on $A$, the kernel of the action is
  trivial. If the action is also transitive, then (again by the
  previous exercise) we have
  \begin{equation*}
    \bigcap_{\sigma\in G}\sigma G_a\sigma^{-1} = 1. \qedhere
  \end{equation*}
\end{proof}

\Exercise3 Assume that $G$ is an abelian, transitive subgroup of
$S_A$. Show that $\sigma(a) \neq a$ for all $\sigma\in G - \{1\}$ and
all $a\in A$. Deduce that $\ord{G} = \ord{A}$.
\begin{proof}
  Let $a\in A$. Since $G$ is abelian, its subgroups are normal. In
  particular, $\sigma G_a\sigma^{-1} = G_a$. Since $G$ is also
  transitive, we know by the previous exercise,
  Exercise~\ref{exercise:group-action:perm-stab-b-conjugate-stab-a},
  that
  \begin{equation*}
    G_a = \bigcap_{\sigma\in G}\sigma G_a\sigma^{-1} = 1.
  \end{equation*}
  Thus the stabilizer of $a$ is trivial, so $\sigma(a) = a$ if and
  only if $\sigma$ is the identity.

  Finally, since $G$ has only one orbit, we have by Proposition~2 that
  \begin{equation*}
    \ord{G : G_a} = \ord{A}.
  \end{equation*}
  But by Lagrange's Theorem,
  \begin{equation*}
    \ord{G : G_a} = \frac{\ord{G}}{\ord{G_a}} = \ord{G},
  \end{equation*}
  so $\ord{G} = \ord{A}$ as required.
\end{proof}

\Exercise4 Let $S_3$ act on the set $\Omega$ of ordered pairs:
\begin{equation*}
  \{(i,j)\mid1\leq i,j\leq3\} \quad\text{by}\quad
  \sigma((i,j)) = (\sigma(i), \sigma(j)).
\end{equation*}
Find the orbits of $S_3$ on $\Omega$. For each $\sigma\in S_3$ find
the cycle decomposition of $\sigma$ under this action (i.e., find its
cycle decomposition when $\sigma$ is considered as an element of $S_9$
--- first fix a labelling of these nine ordered pairs). For each orbit
$\mathcal{O}$ of $S_3$ acting on these nine points pick some
$a\in\mathcal{O}$ and find the stabilizer of $a$ in $S_3$.
\begin{solution}
  We know $(1,1)$, $(2,2)$, and $(3,3)$ belong to the same orbit since
  \begin{equation*}
    (1\,2)(1,1) = (2,2) \quad\text{and}\quad
    (2\,3)(2,2) = (3,3).
  \end{equation*}
  And these form a complete orbit since if $\sigma(i,i) = (j,k)$ we
  must have $j = k$.

  We also find
  \begin{equation*}
    (1,2)
    \xrightarrow{(1\,2)} (2,1)
    \xrightarrow{(2\,1\,3)} (1,3)
    \xrightarrow{(1\,3)} (3,1)
    \xrightarrow{(1\,2)} (3,2)
    \xrightarrow{(2\,3)} (2,3),
  \end{equation*}
  so the six elements $(1,2)$, $(2,1)$, $(1,3)$, $(3,1)$, $(2,3)$, and
  $(3,2)$ belong to the other orbit.

  Now identify
  \begin{align*}
    \mathbf1 = (1,1), && \mathbf4 = (2,1), && \mathbf7 = (3,1), \\
    \mathbf2 = (1,2), && \mathbf5 = (2,2), && \mathbf8 = (3,2), \\
    \mathbf3 = (1,3), && \mathbf6 = (2,3), && \mathbf9 = (3,3).
  \end{align*}
  We get the following cycle decompositions:
  \begin{align*}
    \sigma_1
    &= (), \\
    \sigma_{(1\,2)}
    &= (\mathbf1\,\mathbf5)(\mathbf2\,\mathbf4)
      (\mathbf3\,\mathbf6)(\mathbf7\,\mathbf8), \\
    \sigma_{(1\,3)}
    &= (\mathbf1\,\mathbf9)(\mathbf2\,\mathbf8)
      (\mathbf3\,\mathbf7)(\mathbf4\,\mathbf6), \\
    \sigma_{(2\,3)}
    &= (\mathbf2\,\mathbf3)(\mathbf4\,\mathbf7)
      (\mathbf5\,\mathbf9)(\mathbf6\,\mathbf8), \\
    \sigma_{(1\,2\,3)}
    &= (\mathbf1\,\mathbf5\,\mathbf9)(\mathbf2\,\mathbf6\,\mathbf7)
      (\mathbf3\,\mathbf4\,\mathbf8), \\
    \intertext{and}
    \sigma_{(1\,3\,2)}
    &= (\mathbf1\,\mathbf9\,\mathbf5)(\mathbf2\,\mathbf7\,\mathbf6)
      (\mathbf3\,\mathbf8\,\mathbf4).
  \end{align*}

  Finally, from the first orbit we may choose $\mathbf1 = (1,1)$. The
  stabilizer of this element is
  \begin{equation*}
    \{1, (2\,3)\}.
  \end{equation*}
  For the remaining orbit, we may choose $\mathbf2 = (1,2)$. The
  stabilizer of this element is the identity.
\end{solution}

\Exercise6 As in Exercise~\ref{exercise:subgroup:poly-integer-coef} of
Section~2.2 let $R$ be the set of all polynomials with integer
coefficients in the independent variables $x_1$, $x_2$, $x_3$, $x_4$
and let $S_4$ act on $R$ by permuting the indices of the four
variables:
\begin{equation*}
  \sigma\cdot p(x_1,x_2,x_3,x_4)
  = p(x_{\sigma(1)},x_{\sigma(2)},x_{\sigma(3)},x_{\sigma(4)})
\end{equation*}
for all $\sigma\in S_4$.
\begin{enumerate}
\item Find the polynomials in the orbit of $S_4$ on $R$ containing
  $x_1 + x_2$.
  \begin{solution}
    We saw in Exercise~\ref{exercise:subgroup:poly-integer-coef} that
    the stabilizer of this polynomial has order $4$. As a subgroup of
    $S_4$, this stabilizer has index
    \begin{equation*}
      \frac{\ord{S_4}}4 = \frac{24}4 = 6.
    \end{equation*}
    By Proposition~2, the number of elements in the orbit
    $\mathcal{O}$ of $S_4$ on $R$ containing $x_1 + x_2$ is $6$. We
    find
    \begin{multline*}
      x_1 + x_2
      \xrightarrow{(1\,3)}
      x_2 + x_3
      \xrightarrow{(2\,4)}
      x_3 + x_4 \\
      \xrightarrow{(1\,3)}
      x_1 + x_4
      \xrightarrow{(3\,4)}
      x_1 + x_3
      \xrightarrow{(1\,2)(3\,4)}
      x_2 + x_4.
    \end{multline*}
    So we see that the orbit $\mathcal{O}$ consists of the above six
    polynomials.
  \end{solution}

\item Find the polynomials in the orbit of $S_4$ on $R$ containing
  $x_1x_2 + x_3x_4$.
  \begin{solution}
    Again by Exercise~\ref{exercise:subgroup:poly-integer-coef}, we
    know that the stabilizer of this polynomial has order $8$. As a
    result the orbit $\mathcal{O}$ containing this polynomial has
    order $24/8 = 3$. We find that this orbit is
    \begin{equation*}
      \mathcal{O} =
      \{x_1x_2 + x_3x_4, x_1x_3 + x_2x_4, x_2x_3 + x_1x_4\}.
      \qedhere
    \end{equation*}
  \end{solution}

\item Find the polynomials in the orbit of $S_4$ on $R$ containing
  $(x_1 + x_2)(x_3 + x_4)$.
  \begin{solution}
    We know by Exercise~\ref{exercise:subgroup:poly-integer-coef} that
    the stabilizer of this polynomial has order $8$. So the orbit
    containing it has order $3$. The orbit is
    \begin{equation*}
      \{(x_1+x_2)(x_3+x_4), (x_1+x_3)(x_2+x_4), (x_1+x_4)(x_2+x_3)\}.
      \qedhere
    \end{equation*}
  \end{solution}
\end{enumerate}
