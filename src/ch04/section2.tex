\section{Groups Acting on Themselves by Left Multiplication}

Let $G$ be a group and let $H$ be a subgroup of $G$.

\Exercise1 Let $G = \{1,a,b,c\}$ be the Klein $4$-group whose group
table is written out in Section~2.5.
\begin{enumerate}
\item \label{itm:group-actions:V4-1243}
  Label $1,a,b,c$ with the integers $1,2,4,3$, respectively, and prove
  that under the left regular representation of $G$ into $S_4$ the
  nonidentity elements are mapped as follows:
  \begin{equation*}
    a\mapsto(1\,2)(3\,4) \quad
    b\mapsto(1\,4)(2\,3) \quad
    c\mapsto(1\,3)(2\,4).
  \end{equation*}
  \begin{proof}
    We have
    \begin{align*}
      a\cdot1 &= a, & b\cdot1 &= b, & c\cdot1 &= c, \\
      a\cdot a &= 1, & b\cdot a &= c, & c\cdot a &= b, \\
      a\cdot b &= c, & b\cdot b &= 1, & c\cdot b &= a, \\
      a\cdot c &= b, & b\cdot c &= a, & c\cdot c &= 1.
    \end{align*}
    So, for example, we see that under the given labelling,
    \begin{equation*}
      \sigma_a(1) = 2, \quad
      \sigma_a(2) = 1, \quad
      \sigma_a(3) = 4, \quad\text{and}\quad
      \sigma_a(4) = 3,
    \end{equation*}
    so that $\sigma_a = (1\,2)(3\,4)$. Likewise we can see that
    $\sigma_b = (1\,4)(2\,3)$ and $\sigma_c = (1\,3)(2\,4)$.
  \end{proof}

\item Relabel $1,a,b,c$ as $1,4,2,3$, respectively, and compute the
  image of each element of $G$ under the left regular representation
  of $G$ into $S_4$. Show that the image of $G$ in $S_4$ under this
  labelling is the same {\em subgroup} as the image of $G$ in
  part~\ref{itm:group-actions:V4-1243} (even though the nonidentity
  elements individually map to different permutations under the two
  different labellings).
  \begin{solution}
    Under this labelling, we get the following mapping:
    \begin{equation*}
      a \mapsto (1\,4)(2\,3), \quad
      b \mapsto (1\,2)(3\,4), \quad\text{and}\quad
      c \mapsto (1\,3)(2\,4).
    \end{equation*}
    We see that the image of $G$ in $S_4$ under this labelling and
    also under the labelling in part~\ref{itm:group-actions:V4-1243}
    is
    \begin{equation*}
      \{1, (1\,2)(3\,4), (1\,3)(2\,4), (1\,4)(2\,3)\}. \qedhere
    \end{equation*}
  \end{solution}
\end{enumerate}

\Exercise2 List the elements of $S_3$ as $1$, $(1\,2)$, $(2\,3)$,
$(1\,3)$, $(1\,2\,3)$, $(1\,3\,2)$ and label these with the integers
$1,2,3,4,5,6$ respectively. Exhibit the image of each element of $S_3$
under the left regular representation of $S_3$ into $S_6$.
\begin{solution}
  Let $\sigma = \sigma_{(1\,2)}$ be the image of $(1\,2)$ under the
  left regular representation. Then
  \begin{align*}
    \sigma(1) &= (1\,2)\cdot1 = (1\,2) = 2, \\
    \sigma(2) &= (1\,2)\cdot(1\,2) = 1, \\
    \sigma(3) &= (1\,2)\cdot(2\,3) = (1\,2\,3) = 5, \\
    \sigma(4) &= (1\,2)\cdot(1\,3) = (1\,3\,2) = 6, \\
    \sigma(5) &= (1\,2)\cdot(1\,2\,3) = (2\,3) = 3, \\
    \intertext{and}
    \sigma(6) &= (1\,2)\cdot(1\,3\,2) = (1\,3) = 4.
  \end{align*}
  So $(1\,2)$ is mapped to $(1\,2)(3\,5)(4\,6)$ in $S_6$.

  In exactly the same manner, we can determine the images of the
  remaining elements of $S_3$. We get
  \begin{align*}
    1 &\mapsto 1,
    & (1\,3) &\mapsto (1\,4)(2\,5)(3\,6), \\
    (1\,2) &\mapsto (1\,2)(3\,5)(4\,6),
    & (1\,2\,3) &\mapsto (1\,5\,6)(2\,4\,3), \\
    (2\,3) &\mapsto (1\,3)(2\,6)(4\,5),
    & (1\,3\,2) &\mapsto (1\,6\,5)(2\,3\,4). \qedhere
  \end{align*}
\end{solution}
