\section{Groups Acting on Themselves by Left Multiplication}

Let $G$ be a group and let $H$ be a subgroup of $G$.

\Exercise1 Let $G = \{1,a,b,c\}$ be the Klein $4$-group whose group
table is written out in Section~2.5.
\begin{enumerate}
\item \label{itm:group-actions:V4-1243}
  Label $1,a,b,c$ with the integers $1,2,4,3$, respectively, and prove
  that under the left regular representation of $G$ into $S_4$ the
  nonidentity elements are mapped as follows:
  \begin{equation*}
    a\mapsto(1\,2)(3\,4) \quad
    b\mapsto(1\,4)(2\,3) \quad
    c\mapsto(1\,3)(2\,4).
  \end{equation*}
  \begin{proof}
    We have
    \begin{align*}
      a\cdot1 &= a, & b\cdot1 &= b, & c\cdot1 &= c, \\
      a\cdot a &= 1, & b\cdot a &= c, & c\cdot a &= b, \\
      a\cdot b &= c, & b\cdot b &= 1, & c\cdot b &= a, \\
      a\cdot c &= b, & b\cdot c &= a, & c\cdot c &= 1.
    \end{align*}
    So, for example, we see that under the given labelling,
    \begin{equation*}
      \sigma_a(1) = 2, \quad
      \sigma_a(2) = 1, \quad
      \sigma_a(3) = 4, \quad\text{and}\quad
      \sigma_a(4) = 3,
    \end{equation*}
    so that $\sigma_a = (1\,2)(3\,4)$. Likewise we can see that
    $\sigma_b = (1\,4)(2\,3)$ and $\sigma_c = (1\,3)(2\,4)$.
  \end{proof}

\item Relabel $1,a,b,c$ as $1,4,2,3$, respectively, and compute the
  image of each element of $G$ under the left regular representation
  of $G$ into $S_4$. Show that the image of $G$ in $S_4$ under this
  labelling is the same {\em subgroup} as the image of $G$ in
  part~\ref{itm:group-actions:V4-1243} (even though the nonidentity
  elements individually map to different permutations under the two
  different labellings).
  \begin{solution}
    Under this labelling, we get the following mapping:
    \begin{equation*}
      a \mapsto (1\,4)(2\,3), \quad
      b \mapsto (1\,2)(3\,4), \quad\text{and}\quad
      c \mapsto (1\,3)(2\,4).
    \end{equation*}
    We see that the image of $G$ in $S_4$ under this labelling and
    also under the labelling in part~\ref{itm:group-actions:V4-1243}
    is
    \begin{equation*}
      \{1, (1\,2)(3\,4), (1\,3)(2\,4), (1\,4)(2\,3)\}. \qedhere
    \end{equation*}
  \end{solution}
\end{enumerate}

\Exercise2 List the elements of $S_3$ as $1$, $(1\,2)$, $(2\,3)$,
$(1\,3)$, $(1\,2\,3)$, $(1\,3\,2)$ and label these with the integers
$1,2,3,4,5,6$ respectively. Exhibit the image of each element of $S_3$
under the left regular representation of $S_3$ into $S_6$.
\begin{solution}
  Let $\sigma = \sigma_{(1\,2)}$ be the image of $(1\,2)$ under the
  left regular representation. Then
  \begin{align*}
    \sigma(1) &= (1\,2)\cdot1 = (1\,2) = 2, \\
    \sigma(2) &= (1\,2)\cdot(1\,2) = 1, \\
    \sigma(3) &= (1\,2)\cdot(2\,3) = (1\,2\,3) = 5, \\
    \sigma(4) &= (1\,2)\cdot(1\,3) = (1\,3\,2) = 6, \\
    \sigma(5) &= (1\,2)\cdot(1\,2\,3) = (2\,3) = 3, \\
    \intertext{and}
    \sigma(6) &= (1\,2)\cdot(1\,3\,2) = (1\,3) = 4.
  \end{align*}
  So $(1\,2)$ is mapped to $(1\,2)(3\,5)(4\,6)$ in $S_6$.

  In exactly the same manner, we can determine the images of the
  remaining elements of $S_3$. We get
  \begin{align*}
    1 &\mapsto 1,
    & (1\,3) &\mapsto (1\,4)(2\,5)(3\,6), \\
    (1\,2) &\mapsto (1\,2)(3\,5)(4\,6),
    & (1\,2\,3) &\mapsto (1\,5\,6)(2\,4\,3), \\
    (2\,3) &\mapsto (1\,3)(2\,6)(4\,5),
    & (1\,3\,2) &\mapsto (1\,6\,5)(2\,3\,4). \qedhere
  \end{align*}
\end{solution}

\Exercise3 Let $r$ and $s$ be the usual generators for the dihedral
group of order $8$.
\begin{enumerate}
\item \label{itm:group-action:D8-12345678}
  List the elements of $D_8$ as $1$, $r$, $r^2$, $r^3$, $s$, $sr$,
  $sr^2$, $sr^3$ and label these with the integers $1,2,\dots,8$
  respectively. Exhibit the image of each element of $D_8$ under the
  left regular representation of $D_8$ into $S_8$.
  \begin{solution}
    We can determine the images in the same way as we did in the
    previous exercises. We find that the left regular representation
    produces the following mapping:
    \begin{align*}
      1 &\mapsto 1,
      & s &\mapsto (1\,5)(2\,6)(3\,7)(4\,8), \\
      r &\mapsto (1\,2\,3\,4)(5\,8\,7\,6),
      & sr &\mapsto (1\,6)(2\,7)(3\,8)(4\,5), \\
      r^2 &\mapsto (1\,3)(2\,4)(5\,7)(6\,8),
      & sr^2 &\mapsto (1\,7)(2\,8)(3\,5)(4\,6), \\
      r^3 &\mapsto (1\,4\,3\,2)(5\,6\,7\,8),
      & sr^3 &\mapsto (1\,8)(2\,5)(3\,6)(4\,7). \qedhere
    \end{align*}
  \end{solution}

\item \label{itm:group-action:D8-13572468} Relabel this same list of
  elements of $D_8$ with the integers $1$, $3$, $5$, $7$, $2$, $4$,
  $6$, $8$ respectively and recompute the image of each element of
  $D_8$ under the left regular representation with respect to this new
  labelling. Show that the two subgroups of $S_8$ obtained in parts
  \ref{itm:group-action:D8-12345678} and
  \ref{itm:group-action:D8-13572468} are different.
  \begin{solution}
    Under the new labelling, we find that
    \begin{align*}
      1 &\mapsto 1,
      & s &\mapsto (1\,2)(3\,4)(5\,6)(7\,8), \\
      r &\mapsto (1\,3\,5\,7)(2\,8\,6\,4),
      & sr &\mapsto (1\,4)(3\,6)(5\,8)(7\,2), \\
      r^2 &\mapsto (1\,5)(3\,7)(2\,6)(4\,8),
      & sr^2 &\mapsto (1\,6)(3\,8)(5\,2)(7\,4), \\
      r^3 &\mapsto (1\,7\,5\,3)(2\,4\,6\,8),
      & sr^3 &\mapsto (1\,8)(3\,2)(5\,4)(7\,6).
    \end{align*}
    It is clear that the image of $D_8$ in $S_8$ that was found in
    part~\ref{itm:group-action:D8-12345678} is a different subgroup of
    $S_8$ than the image found here.
  \end{solution}
\end{enumerate}

\Exercise4 Use the left regular representation of $Q_8$ to produce two
elements of $S_8$ which generate a subgroup of $S_8$ isomorphic to the
quaternion group $Q_8$.
\begin{solution}
  Label the elements $1$, $-1$, $i$, $-i$, $j$, $-j$, $k$, $-k$ of
  $Q_8$ with the integers $1,2,\dots,8$, respectively. For each
  $g\in Q_8$, let $\sigma_g$ denote the image of $g$ under the left
  regular representation of $Q_8$ into $S_8$, using this labelling.

  We have
  \begin{align*}
    i\cdot1 = i, &\quad\text{so $\sigma_i(1) = 3$}, \\
    i\cdot i = -1, &\quad\text{so $\sigma_i(3) = 2$}, \\
    i\cdot-1 = -i, &\quad\text{so $\sigma_i(2) = 4$}, \\
    i\cdot-i = 1, &\quad\text{so $\sigma_i(4) = 1$}, \\
    i\cdot j = k, &\quad\text{so $\sigma_i(5) = 7$}, \\
    i\cdot k = -j, &\quad\text{so $\sigma_i(7) = 6$}, \\
    i\cdot-j = -k, &\quad\text{so $\sigma_i(6) = 8$}, \\
    \intertext{and}
    i\cdot-k = j, &\quad\text{so $\sigma_i(8) = 5$}.
  \end{align*}
  So we see that $\sigma_i = (1\,3\,2\,4)(5\,7\,6\,8)$. In the same
  fashion, we can determine that
  $\sigma_j = (1\,5\,2\,6)(3\,8\,4\,7)$.

  Since the action of a group on itself by left multiplication is
  always faithful, we know by the First Isomorphism Theorem that $Q_8$
  is isomorphic to its image under the left regular
  representation. And since $i$ and $j$ generate $Q_8$, it follows
  that $\sigma_i$ and $\sigma_j$ generate an isomorphic subgroup of
  $S_8$. That is, the two elements
  \begin{equation*}
    (1\,3\,2\,4)(5\,7\,6\,8) \quad\text{and}\quad
    (1\,5\,2\,6)(3\,8\,4\,7)
  \end{equation*}
  generate a subgroup of $S_8$ that is isomorphic to $Q_8$.
\end{solution}

\Exercise5 Let $r$ and $s$ be the usual generators for the dihedral
group of order $8$ and let $H = \gen{s}$. List the left cosets of $H$
in $D_8$ as $1H$, $rH$, $r^2H$, and $r^3H$.
\begin{enumerate}
\item \label{itm:group-action:faithful-action-cosets-D8}
  Label these cosets with the integers $1,2,3,4$,
  respectively. Exhibit the image of each element of $D_8$ under the
  representation $\pi_H$ of $D_8$ into $S_4$ obtained from the action
  of $D_8$ by left multiplication on the set of $4$ left cosets of $H$
  in $D_8$. Deduce that this representation is faithful (i.e., the
  elements of $S_4$ obtained form a subgroup isomorphic to $D_8$).
  \begin{solution}
    In the text it was shown that
    \begin{equation*}
      \pi_H(s) = (2\,4) \quad\text{and}\quad
      \pi_H(r) = (1\,2\,3\,4).
    \end{equation*}
    Since $\pi_H$ is a homomorphism, we can find the images of the
    remaining elements by working with the images of these
    generators. For example,
    \begin{equation*}
      \pi_H(r^2) = \pi_H(r)^2 = (1\,2\,3\,4)^2 = (1\,3)(2\,4).
    \end{equation*}
    Similarly, we can find that
    \begin{align*}
      \pi_H(r^3) &= (1\,4\,3\,2), \\
      \pi_H(sr) &= (1\,4)(2\,3), \\
      \pi_H(sr^2) &= (1\,3), \\
      \pi_H(sr^3) &= (1\,2)(3\,4).
    \end{align*}

    Clearly the kernel of this action is trivial since only the
    identity is mapped to the identity permutation. We conclude that
    the representation is faithful.
  \end{solution}

\item Repeat part~\ref{itm:group-action:faithful-action-cosets-D8}
  with the list of cosets relabelled by the integers $1,3,2,4$,
  respectively. Show that the permutations obtained from this
  labelling form a subgroup of $S_4$ that is different from the
  subgroup obtained in
  part~\ref{itm:group-action:faithful-action-cosets-D8}.
  \begin{solution}
    With this labelling, we get
    \begin{align*}
      \pi_H(r) &= (1\,3\,2\,4), \\
      \pi_H(r^2) &= (1\,2)(3\,4), \\
      \pi_H(r^3) &= (1\,4\,2\,3), \\
      \pi_H(s) &= (3\,4), \\
      \pi_H(sr) &= (1\,4)(2\,3), \\
      \pi_H(sr^2) &= (1\,2), \\
      \pi_H(sr^3) &= (1\,3)(2\,4).
    \end{align*}
  \end{solution}
  The subgroup obtained in
  part~\ref{itm:group-action:faithful-action-cosets-D8} is clearly
  different from the subgroup obtained here.

\item Let $K = \gen{sr}$, list the cosets of $K$ in $D_8$ as $1K$,
  $rK$, $r^2K$, and $r^3K$, and label these with the integers
  $1,2,3,4$. Prove that, with respect to this labelling, the image of
  $D_8$ under the representation $\pi_K$ obtained from left
  multiplication on the cosets of $K$ is the same {\em subgroup} of
  $S_4$ as in part~\ref{itm:group-action:faithful-action-cosets-D8}
  (even though the subgroups $H$ and $K$ are different and some of the
  elements of $D_8$ map to different permutations under the two
  homomorphisms).
  \begin{proof}
    We get
    \begin{align*}
      \pi_K(r) &= (1\,2\,3\,4), \\
      \pi_K(r^2) &= (1\,3)(2\,4) \\
      \pi_K(r^3) &= (1\,4\,3\,2) \\
      \pi_K(s) &= (1\,2)(3\,4), \\
      \pi_K(sr) &= (2\,4), \\
      \pi_K(sr^2) &= (1\,4)(2\,3), \\
      \pi_K(sr^3) &= (1\,3).
    \end{align*}
    Comparing this to
    part~\ref{itm:group-action:faithful-action-cosets-D8}, we see that
    $\pi_H(D_8) = \pi_K(D_8)$.
  \end{proof}
\end{enumerate}

\Exercise6 Let $r$ and $s$ be the usual generators for the dihedral
group of order $8$ and let $N = \gen{r^2}$. List the left cosets of
$N$ in $D_8$ as $1N$, $rN$, $sN$ and $srN$. Label these cosets with
the integers $1,2,3,4$ respectively. Exhibit the image of each element
of $D_8$ under the representation $\pi_N$ of $D_8$ into $S_4$ obtained
from the action of $D_8$ by left multiplication on the set of $4$ left
cosets of $N$ in $D_8$. Deduce that this representation is not
faithful and prove that $\pi_N(D_8)$ is isomorphic to the Klein
4-group.
\begin{solution}
  We can see that $\pi_N(r) = (1\,2)(3\,4)$ and
  $\pi_N(s) = (1\,3)(2\,4)$, so the remaining images can be computed as
  follows:
  \begin{align*}
    \pi_N(r^2) &= \pi_N(r)^2 = 1, \\
    \pi_N(r^3) &= \pi_N(r)^3 = (1\,2)(3\,4), \\
    \pi_N(sr) &= \pi_N(s)\pi_N(r) = (1\,4)(2\,3), \\
    \pi_N(sr^2) &= \pi_N(s)\pi_N(r)^2 = (1\,3)(2\,4), \\
    \pi_N(sr^3) &= \pi_N(s)\pi_N(r)^3 = (1\,4)(2\,3).
  \end{align*}
  We see that the kernel of $\pi_N$ is $\{1, r^2\}$, so the action is
  not faithful. Lastly, we have
  \begin{equation*}
    \pi_N(D_8) = \{1, (1\,2)(3\,4), (1\,3)(2\,4), (1\,4)(2\,3)\},
  \end{equation*}
  and this subgroup is obviously not cyclic. So by
  Exercise~\ref{exercise:classify:groups-4}, we know that
  $\pi_N(D_8)\cong V_4$.
\end{solution}

\Exercise7 Let $Q_8$ be the quaternion group of order $8$.
\begin{enumerate}
\item Prove that $Q_8$ is isomorphic to a subgroup of $S_8$.
  \begin{proof}
    $Q_8$ has $8$ elements, so by Corollary~4 it is isomorphic to a
    subgroup of $S_8$.
  \end{proof}

\item Prove that $Q_8$ is not isomorphic to a subgroup of $S_n$ for
  any $n\leq7$.
  \begin{proof}
    First we will show that $Q_8$ cannot act faithfully on a set $A$
    having fewer than $8$ elements. Let $G = Q_8$, and let $G$ act on
    $A$. Choose any $a\in A$. Then the orbit of $G$ containing $a$ has
    fewer than $8$ elements. So, by Proposition~2, we have
    \begin{equation*}
      \ord{G : G_a} \leq 7.
    \end{equation*}
    By Lagrange's Theorem,
    \begin{equation*}
      \frac{\ord{G}}{\ord{G_a}} = \frac8{\ord{G_a}} \leq 7
    \end{equation*}
    and we see that $G_a$ must be a nontrivial subgroup of
    $G$. However, every nontrivial subgroup of $Q_8$ contains the
    subgroup $\gen{-1}$ (this can be seen from the lattice of
    subgroups of $Q_8$). So $-1\in G_a$, and since $a$ was arbitrary,
    we see that $-1$ stabilizes every element of $G$. Therefore $-1$
    belongs to the kernel of the group action, and $G$ does not act
    faithfully on $A$.

    Next, if $Q_8$ is isomorphic to some subgroup of $S_7$, then we
    can produce an injective homomorphism $\varphi$ from $Q_8$ into
    $S_7$. Consider the action of $Q_8$ on $\{1,2,\dots,7\}$ induced
    by $\varphi$. Since $\varphi$ is injective, its kernel must be
    trivial and this action must be faithful. But as we have shown
    above, this is impossible. Therefore $Q_8$ is not isomorphic to
    any subgroup of $S_7$ (or of $S_n$ for $1\leq n\leq7$).
  \end{proof}
\end{enumerate}

\Exercise8 Prove that if $H$ has finite index $n$ then there is a
normal subgroup $K$ of $G$ with $K\leq H$ and $\ord{G:K}\leq n!$.
\begin{proof}
  Let $G$ act by left multiplication on the set of left cosets of $H$
  in $G$, and let $\pi_H$ be the permutation representation that is
  induced by this action. Let $K = \ker\pi_H$. By Theorem~3, we know
  that $K\leq H$. And by the First Isomorphism Theorem, we know that
  $K\trianglelefteq G$ and $G/K\cong\pi_H(G)$.

  Now, $\pi_H(G)$ is a subgroup of $S_n$, so
  \begin{equation*}
    \ord{G:K} = \ord{G/K} = \ord{\pi_H(G)} \leq \ord{S_n} = n!,
  \end{equation*}
  as we wanted to show.
\end{proof}

\Exercise9 Prove that if $p$ is a prime and $G$ is a group of order
$p^\alpha$ for some $\alpha\in\Z^+$, then every subgroup of index $p$
is normal in $G$. Deduce that every group of order $p^2$ has a normal
subgroup of order $p$.
\begin{proof}
  The first statement was already proved in Corollary~5. Now suppose
  $G$ is any subgroup of order $p^2$. By Cauchy's Theorem, $G$ has an
  element $x$ of order $p$. Then
  \begin{equation*}
    \ord{G:\gen{x}} = \frac{\ord{G}}{\ord{\gen{x}}} = p,
  \end{equation*}
  so $\gen{x}$ is a normal subgroup of $G$ having order $p$.
\end{proof}

\Exercise{10} Prove that every non-abelian group of order $6$ has a
nonnormal subgroup of order $2$. Use this to classify groups of order
$6$.
\begin{proof}
  Let $G$ be a non-abelian group of order $6$. By Cauchy's Theorem,
  $G$ has at least one element of order $2$ and at least one element
  of order $3$.

  Suppose for contradiction that every subgroup of
  order $2$ is normal. Then if $a\in G$ is any element of order $2$
  and $b\in G$ is arbitrary, we see that
  \begin{equation*}
    b\gen{a} = \gen{a}b.
  \end{equation*}
  Since $a$ is the only nonidentity element in $\gen{a}$, we have
  $ba = ab$. Therefore all elements of order $2$ in $G$ belong to its
  center, $Z(G)$.

  Now take $x$ not in $Z(G)$, that is, $x\in G - Z(G)$, which is
  possible because $G$ is non-abelian. We know $Z(G)\subseteq C_G(x)$,
  but actually this containment is strict, i.e. $Z(G)\subset C_G(x)$,
  since $x\in C_G(x)$. By Lagrange's Theorem, the order of $Z(G)$ must
  divide the order of $C_G(x)$. Since $2\nmid3$, we must have
  $\ord{C_G(x)} = 6$, contradicting our choice of $x$. So $G$ must
  have a nonnormal subgroup of order $2$, as we wanted to show.

  Next we attempt to classify all groups of order $6$. Let $G$ be such
  a group. Then $G$ is either abelian or non-abelian, and we consider
  each case in turn.

  First, if $G$ is abelian, pick $x,y\in G$ with $\ord{x} = 2$ and
  $\ord{y} = 3$ (these elements exist by Cauchy's Theorem). Consider
  the element $xy$. This element must have order $2$, $3$, or $6$. But
  it cannot have order $2$, since
  \begin{equation*}
    (xy)^2 = x^2y^2 = y^2 \neq 1,
  \end{equation*}
  and it cannot have order $3$ since
  \begin{equation*}
    (xy)^3 = x^3y^3 = x \neq 1.
  \end{equation*}
  Therefore $xy$ has order $6$, and is thus a generator for $G$. Hence
  $G$ is cyclic and $G\cong Z_6$.

  Lastly, suppose $G$ is non-abelian. Let $H$ be a nonnormal subgroup
  of order $2$. Let $A$ be the set of left cosets of $H$, and let $G$
  act on $A$ by left multiplication. The permutation representation
  $\varphi$ induced by this action is a homomorphism from $G$ into
  $S_3$ (since there are exactly three cosets in $A$). By Theorem~3,
  we know that $\ker\varphi$ is the largest normal subgroup of $G$
  contained in $H$. Since $H$ is not normal, this implies that
  $\ker\varphi = 1$ and we see that $\varphi$ is injective. We now
  have an injective homomorphism from $G$ into $S_3$, and since
  $\ord{G} = \ord{S_3} = 6$, this implies that $G\cong S_3$.

  We conclude that every group of order $6$ is isomorphic to either
  $Z_6$ or $S_3$.
\end{proof}

\Exercise{11}
\label{exercise:group-action:pi-x-prod-of-m-n-cycles}
Let $G$ be a finite group and let $\pi\colon G\to S_G$ be the left
regular representation. Prove that if $x$ is an element of $G$ of
order $n$ and $\ord{G} = mn$, then $\pi(x)$ is a product of $m$
$n$-cycles. Deduce that $\pi(x)$ is an odd permutation if and only if
$\ord{x}$ is even and $\frac{\ord{G}}{\ord{x}}$ is odd.
\begin{proof}
  Let $x$ have order $n$ and let $H = \gen{x}$. We have
  $\ord{G : H} = m$, i.e. there are exactly $m$ right cosets of
  $H$. Let $Hy$ be any such right coset. Then $Hy$ is the set
  \begin{equation*}
    Hy = \{ y, xy, x^2y, \dots, x^{n-1}y \}.
  \end{equation*}
  Now we see that
  \begin{equation*}
    \pi(x)(y) = xy, \quad \pi(x)(xy) = x^2y, \quad \dots, \quad
    \pi(x)(x^{n-1}y) = y.
  \end{equation*}
  Therefore $x$ permutes the elements of $Hy$ in a cyclic fashion, so
  that
  \begin{equation*}
    (y\;xy\;x^2y\;\dots\;x^{n-1}y)
  \end{equation*}
  is an $n$-cycle in the cyclic decomposition of $\pi(x)$. Since there
  are $m$ cosets of $H$, and the elements of each are permuted in this
  way, we see that $\pi(x)$ consists of the product of $m$ disjoint
  $n$-cycles.

  By Proposition~25 of Chapter~3, we see that $\pi(x)$ is odd if and
  only if $n = \ord{x}$ is even and $m = \ord{G}/\ord{x}$ is odd.
\end{proof}

\Exercise{12}
\label{exercise:group-action:left-reg-odd-perm}
Let $G$ and $\pi$ be as in the preceding exercise. Prove that if
$\pi(G)$ contains an odd permutation then $G$ has a subgroup of index
$2$.
\begin{proof}
  We know that the alternating group $A_G$ is a normal subgroup of
  $S_G$ having prime index $2$. Now $\pi(G)$ is a subgroup of $S_G$,
  but it is not a subgroup of $A_G$ (since it contains an odd
  permutation). So by
  Exercise~\ref{exercise:quotient-group:subgrp-prime-index-consequences},
  we must have $\ord{\pi(G) : \pi(G)\cap A_G} = 2$.

  Now, since the action of a group on itself by left multiplication is
  always faithful, we know by the First Isomorphism Theorem that
  $G\cong\pi(G)$ via an isomorphism $\varphi$. Then
  $\varphi^{-1}(\pi(G)\cap A_G)$ is a subgroup of $G$ having index
  $2$.
\end{proof}

\Exercise{13} Prove that if $\ord{G} = 2k$ where $k$ is odd then $G$
has a subgroup of index $2$.
\begin{proof}
  Let $\pi\colon G\to S_G$ be the left regular representation for
  $G$. By Cauchy's Theorem, $G$ has an element $x$ of order $2$, so by
  Exercise~\ref{exercise:group-action:pi-x-prod-of-m-n-cycles},
  $\pi(G)$ contains an odd permutation, namely $\pi(x)$. Therefore, by
  Exercise~\ref{exercise:group-action:left-reg-odd-perm}, $G$ has a
  subgroup of index $2$.
\end{proof}

\Exercise{14} Let $G$ be a finite group of composite order $n$ with
the property that $G$ has a subgroup of order $k$ for each positive
integer $k$ dividing $n$. Prove that $G$ is not simple.
\begin{proof}
  Let $p$ be the smallest prime dividing $n$. We know by assumption
  that $G$ has a subgroup $H$ of order $n/p$. By Lagrange,
  $\ord{G:H} = p$ and $H$ has prime index. It now follows from
  Corollary~5 that $H$ is normal in $G$. Since $n$ is composite,
  $1 < \ord{H} < n$, and we see that $G$ has a nontrivial proper
  normal subgroup. Hence $G$ is not simple.
\end{proof}
