\section{Groups Acting on Themselves by Left Multiplication}

Let $G$ be a group and let $H$ be a subgroup of $G$.

\Exercise1 Let $G = \{1,a,b,c\}$ be the Klein $4$-group whose group
table is written out in Section~2.5.
\begin{enumerate}
\item \label{itm:group-actions:V4-1243}
  Label $1,a,b,c$ with the integers $1,2,4,3$, respectively, and prove
  that under the left regular representation of $G$ into $S_4$ the
  nonidentity elements are mapped as follows:
  \begin{equation*}
    a\mapsto(1\,2)(3\,4) \quad
    b\mapsto(1\,4)(2\,3) \quad
    c\mapsto(1\,3)(2\,4).
  \end{equation*}
  \begin{proof}
    We have
    \begin{align*}
      a\cdot1 &= a, & b\cdot1 &= b, & c\cdot1 &= c, \\
      a\cdot a &= 1, & b\cdot a &= c, & c\cdot a &= b, \\
      a\cdot b &= c, & b\cdot b &= 1, & c\cdot b &= a, \\
      a\cdot c &= b, & b\cdot c &= a, & c\cdot c &= 1.
    \end{align*}
    So, for example, we see that under the given labelling,
    \begin{equation*}
      \sigma_a(1) = 2, \quad
      \sigma_a(2) = 1, \quad
      \sigma_a(3) = 4, \quad\text{and}\quad
      \sigma_a(4) = 3,
    \end{equation*}
    so that $\sigma_a = (1\,2)(3\,4)$. Likewise we can see that
    $\sigma_b = (1\,4)(2\,3)$ and $\sigma_c = (1\,3)(2\,4)$.
  \end{proof}

\item Relabel $1,a,b,c$ as $1,4,2,3$, respectively, and compute the
  image of each element of $G$ under the left regular representation
  of $G$ into $S_4$. Show that the image of $G$ in $S_4$ under this
  labelling is the same {\em subgroup} as the image of $G$ in
  part~\ref{itm:group-actions:V4-1243} (even though the nonidentity
  elements individually map to different permutations under the two
  different labellings).
  \begin{solution}
    Under this labelling, we get the following mapping:
    \begin{equation*}
      a \mapsto (1\,4)(2\,3), \quad
      b \mapsto (1\,2)(3\,4), \quad\text{and}\quad
      c \mapsto (1\,3)(2\,4).
    \end{equation*}
    We see that the image of $G$ in $S_4$ under this labelling and
    also under the labelling in part~\ref{itm:group-actions:V4-1243}
    is
    \begin{equation*}
      \{1, (1\,2)(3\,4), (1\,3)(2\,4), (1\,4)(2\,3)\}. \qedhere
    \end{equation*}
  \end{solution}
\end{enumerate}

\Exercise2 List the elements of $S_3$ as $1$, $(1\,2)$, $(2\,3)$,
$(1\,3)$, $(1\,2\,3)$, $(1\,3\,2)$ and label these with the integers
$1,2,3,4,5,6$ respectively. Exhibit the image of each element of $S_3$
under the left regular representation of $S_3$ into $S_6$.
\begin{solution}
  Let $\sigma = \sigma_{(1\,2)}$ be the image of $(1\,2)$ under the
  left regular representation. Then
  \begin{align*}
    \sigma(1) &= (1\,2)\cdot1 = (1\,2) = 2, \\
    \sigma(2) &= (1\,2)\cdot(1\,2) = 1, \\
    \sigma(3) &= (1\,2)\cdot(2\,3) = (1\,2\,3) = 5, \\
    \sigma(4) &= (1\,2)\cdot(1\,3) = (1\,3\,2) = 6, \\
    \sigma(5) &= (1\,2)\cdot(1\,2\,3) = (2\,3) = 3, \\
    \intertext{and}
    \sigma(6) &= (1\,2)\cdot(1\,3\,2) = (1\,3) = 4.
  \end{align*}
  So $(1\,2)$ is mapped to $(1\,2)(3\,5)(4\,6)$ in $S_6$.

  In exactly the same manner, we can determine the images of the
  remaining elements of $S_3$. We get
  \begin{align*}
    1 &\mapsto 1,
    & (1\,3) &\mapsto (1\,4)(2\,5)(3\,6), \\
    (1\,2) &\mapsto (1\,2)(3\,5)(4\,6),
    & (1\,2\,3) &\mapsto (1\,5\,6)(2\,4\,3), \\
    (2\,3) &\mapsto (1\,3)(2\,6)(4\,5),
    & (1\,3\,2) &\mapsto (1\,6\,5)(2\,3\,4). \qedhere
  \end{align*}
\end{solution}

\Exercise3 Let $r$ and $s$ be the usual generators for the dihedral
group of order $8$.
\begin{enumerate}
\item \label{itm:group-action:D8-12345678}
  List the elements of $D_8$ as $1$, $r$, $r^2$, $r^3$, $s$, $sr$,
  $sr^2$, $sr^3$ and label these with the integers $1,2,\dots,8$
  respectively. Exhibit the image of each element of $D_8$ under the
  left regular representation of $D_8$ into $S_8$.
  \begin{solution}
    We can determine the images in the same way as we did in the
    previous exercises. We find that the left regular representation
    produces the following mapping:
    \begin{align*}
      1 &\mapsto 1,
      & s &\mapsto (1\,5)(2\,6)(3\,7)(4\,8), \\
      r &\mapsto (1\,2\,3\,4)(5\,8\,7\,6),
      & sr &\mapsto (1\,6)(2\,7)(3\,8)(4\,5), \\
      r^2 &\mapsto (1\,3)(2\,4)(5\,7)(6\,8),
      & sr^2 &\mapsto (1\,7)(2\,8)(3\,5)(4\,6), \\
      r^3 &\mapsto (1\,4\,3\,2)(5\,6\,7\,8),
      & sr^3 &\mapsto (1\,8)(2\,5)(3\,6)(4\,7). \qedhere
    \end{align*}
  \end{solution}

\item \label{itm:group-action:D8-13572468} Relabel this same list of
  elements of $D_8$ with the integers $1$, $3$, $5$, $7$, $2$, $4$,
  $6$, $8$ respectively and recompute the image of each element of
  $D_8$ under the left regular representation with respect to this new
  labelling. Show that the two subgroups of $S_8$ obtained in parts
  \ref{itm:group-action:D8-12345678} and
  \ref{itm:group-action:D8-13572468} are different.
  \begin{solution}
    Under the new labelling, we find that
    \begin{align*}
      1 &\mapsto 1,
      & s &\mapsto (1\,2)(3\,4)(5\,6)(7\,8), \\
      r &\mapsto (1\,3\,5\,7)(2\,8\,6\,4),
      & sr &\mapsto (1\,4)(3\,6)(5\,8)(7\,2), \\
      r^2 &\mapsto (1\,5)(3\,7)(2\,6)(4\,8),
      & sr^2 &\mapsto (1\,6)(3\,8)(5\,2)(7\,4), \\
      r^3 &\mapsto (1\,7\,5\,3)(2\,4\,6\,8),
      & sr^3 &\mapsto (1\,8)(3\,2)(5\,4)(7\,6).
    \end{align*}
    It is clear that the image of $D_8$ in $S_8$ that was found in
    part~\ref{itm:group-action:D8-12345678} is a different subgroup of
    $S_8$ than the image found here.
  \end{solution}
\end{enumerate}

\Exercise4 Use the left regular representation of $Q_8$ to produce two
elements of $S_8$ which generate a subgroup of $S_8$ isomorphic to the
quaternion group $Q_8$.
\begin{solution}
  Label the elements $1$, $-1$, $i$, $-i$, $j$, $-j$, $k$, $-k$ of
  $Q_8$ with the integers $1,2,\dots,8$, respectively. For each
  $g\in Q_8$, let $\sigma_g$ denote the image of $g$ under the left
  regular representation of $Q_8$ into $S_8$, using this labelling.

  We have
  \begin{align*}
    i\cdot1 = i, &\quad\text{so $\sigma_i(1) = 3$}, \\
    i\cdot i = -1, &\quad\text{so $\sigma_i(3) = 2$}, \\
    i\cdot-1 = -i, &\quad\text{so $\sigma_i(2) = 4$}, \\
    i\cdot-i = 1, &\quad\text{so $\sigma_i(4) = 1$}, \\
    i\cdot j = k, &\quad\text{so $\sigma_i(5) = 7$}, \\
    i\cdot k = -j, &\quad\text{so $\sigma_i(7) = 6$}, \\
    i\cdot-j = -k, &\quad\text{so $\sigma_i(6) = 8$}, \\
    \intertext{and}
    i\cdot-k = j, &\quad\text{so $\sigma_i(8) = 5$}.
  \end{align*}
  So we see that $\sigma_i = (1\,3\,2\,4)(5\,7\,6\,8)$. In the same
  fashion, we can determine that
  $\sigma_j = (1\,5\,2\,6)(3\,8\,4\,7)$.

  Since the action of a group on itself by left multiplication is
  always faithful, we know by the First Isomorphism Theorem that $Q_8$
  is isomorphic to its image under the left regular
  representation. And since $i$ and $j$ generate $Q_8$, it follows
  that $\sigma_i$ and $\sigma_j$ generate an isomorphic subgroup of
  $S_8$. That is, the two elements
  \begin{equation*}
    (1\,3\,2\,4)(5\,7\,6\,8) \quad\text{and}\quad
    (1\,5\,2\,6)(3\,8\,4\,7)
  \end{equation*}
  generate a subgroup of $S_8$ that is isomorphic to $Q_8$.
\end{solution}
