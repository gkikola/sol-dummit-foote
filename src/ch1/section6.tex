\section{Homomorphisms and Isomorphisms}

Let $G$ and $H$ be groups.

\Exercise1 Let $\varphi\colon G\to H$ be a homomorphism.
\begin{enumerate}
\item Prove that $\varphi(x^n) = \varphi(x)^n$ for all $n\in\Z^+$.
  \begin{proof}
    We use induction on $n$. The case for $n = 1$ is clear. Suppose
    $\varphi(x^n) = \varphi(x)^n$ for some particular $n\in\Z^+$. Then
    \begin{equation*}
      \varphi(x^{n+1})
      = \varphi(xx^n)
      = \varphi(x)\varphi(x^n)
      = \varphi(x)\varphi(x)^n
      = \varphi(x)^{n+1},
    \end{equation*}
    so the result holds for all $n\in\Z^+$.
  \end{proof}
\item Do part (a) for $n = -1$ and deduce that
  $\varphi(x^n) = \varphi(x)^n$ for all $n\in\Z$.
  \begin{solution}
    Let $1_G$ and $1_H$ denote the identities of $G$ and $H$,
    respectively. Then
    \begin{equation*}
      \varphi(1_G)\varphi(1_G) = \varphi(1_G) = \varphi(1_G)1_H,
    \end{equation*}
    and it follows from the cancellation law that
    $\varphi(1_G) = 1_H$. Since the identity is preserved, we will
    simply use $1$ to denote the identity of both groups from this
    point forward.

    Now, for any $x\in G$,
    \begin{equation*}
      \varphi(x)\varphi(x^{-1}) = \varphi(xx^{-1}) = \varphi(1) = 1,
    \end{equation*}
    which shows that $\varphi(x^{-1}) = \varphi(x)^{-1}$. Then for
    $n\in\Z^+$,
    \begin{equation*}
      \varphi(x^{-n}) = \varphi((x^n)^{-1}) = \varphi(x^n)^{-1}
      = \big(\varphi(x)^n\big)^{-1} = \varphi(x)^{-n}.
    \end{equation*}
    Therefore $\varphi(x^n) = \varphi(x)^n$ holds for all $n\in\Z$.
  \end{solution}
\end{enumerate}

\Exercise2 If $\varphi\colon G\to H$ is an isomorphism, prove that
$\ord{\varphi(x)} = \ord{x}$ for all $x\in G$. Deduce that any two
isomorphic groups have the same number of elements of order $n$ for
each $n\in\Z^+$. Is the result true if $\varphi$ is only assumed to be
a homomorphism?
\begin{proof}
  First suppose $\ord{x} = n<\infty$. By the previous exercise, we have
  \begin{equation*}
    \varphi(x)^n = \varphi(x^n) = \varphi(1) = 1.
  \end{equation*}
  So $\ord{\varphi(x)}\leq n$. On the other hand, if
  $\ord{\varphi(x)} = k$, then
  \begin{equation*}
    \varphi(x^k) = \varphi(x)^k = 1.
  \end{equation*}
  But $1$ is the only element in $G$ which gets sent to $1$ in $H$,
  since $\varphi$ is a bijection. This shows that $x^k = 1$, so that
  $k\geq n$. Hence $\ord{\varphi(x)} = n$.

  Now suppose $x$ has infinite order. If
  $\ord{\varphi(x)} = n < \infty$, then
  $\varphi(x^n) = \varphi(x)^n = 1$, and since $\varphi$ is a
  bijection we must have $x^n = 1$, a contradiction. Therefore
  $\varphi(x)$ must also have infinite order.

  From the above we know that $\ord{x} = \ord{\varphi(x)}$ for each
  $x$, and since $\varphi$ is a bijection this shows that $G$ and $H$
  have the same number of elements of each order.

  Finally, this result does not necessarily hold for
  homomorphisms. For example, let $H$ be the trivial group $\{1\}$ and
  take the function $\theta\colon G\to H$ defined by $\theta(x) = 1$
  for all $x\in G$. Then $\theta(x)\theta(y) = \theta(xy)$, so this is
  a homomorphism, but every element in $H$ has order $1$, which is not
  true of $G$ (unless $G$ is also trivial).
\end{proof}

\Exercise3 If $\varphi\colon G\to H$ is an isomorphism, prove that $G$
is abelian if and only if $H$ is abelian. If $\varphi\colon G\to H$ is
a homomorphism, what additional conditions on $\varphi$ (if any) are
sufficient to ensure that if $G$ is abelian, then so is $H$?
\begin{solution}
  Since $\varphi$ must be invertible (it is a bijection) and since
  $\varphi^{-1}$ must be an isomorphism from $H$ to $G$, the proof
  only needs to work in one direction. So let $x,y\in H$ be arbitrary,
  and let $a = \varphi^{-1}(x)$ and $b = \varphi^{-1}(y)$. If $G$ is
  abelian, then
  \begin{equation*}
    xy = \varphi(a)\varphi(b) = \varphi(ab) = \varphi(ba)
    = \varphi(b)\varphi(a) = yx,
  \end{equation*}
  so $H$ is also abelian, and the proof is complete.

  Note that the same result does not hold for homomorphisms. For
  instance, let $\varphi\colon\Z/2\Z\to D_6$ be given by
  $\varphi(\bar0) = 1$ and $\varphi(\bar1) = s$. Then $\varphi$ is a
  homomorphism and $\Z/2\Z$ is abelian, but $D_6$ is not abelian.

  However, if we add the constraint that $\varphi$ is surjective, then
  the result does hold: Suppose $G$ is abelian, let $x,y\in H$ be
  arbitrary, and pick $a\in\varphi^{-1}(x)$ and $b\in\varphi^{-1}(y)$
  (that is, $a$ and $b$ are chosen from the fibers of $\varphi$ over
  $x$ and $y$). Then, as before,
  \begin{equation*}
    xy = \varphi(a)\varphi(b) = \varphi(ab) = \varphi(ba)
    = \varphi(b)\varphi(a) = yx,
  \end{equation*}
  so $H$ is abelian.
\end{solution}

\Exercise4 Prove that the multiplicative groups $\R-\{0\}$ and
$\C-\{0\}$ are not isomorphic.
\begin{proof}
  Every element in $\R-\{0\}$ has infinite order, aside from $1$ and
  $-1$ which have orders $1$ and $2$, respectively. However,
  $\C-\{0\}$ has elements of order $4$, namely $i$ and $-i$. Therefore
  these groups are not isomorphic.
\end{proof}

\Exercise5 Prove that the additive groups $\R$ and $\Q$ are not
isomorphic.
\begin{proof}
  There is no bijection between $\R$ and $\Q$, since the former is
  uncountable and the latter is countable. Therefore the groups
  $(\R,+)$ and $(\Q,+)$ are not isomorphic.
\end{proof}

\Exercise6 Prove that the additive groups $\Z$ and $\Q$ are not
isomorphic.
\begin{proof}
  Suppose the contrary, and let $\varphi\colon\Q\to\Z$ be an
  isomorphism. Let
  \begin{equation*}
    a = \varphi(1).
  \end{equation*}
  Then
  \begin{equation*}
    a = \varphi\left(\frac12 + \frac12\right)
    = 2\varphi\left(\frac12\right).
  \end{equation*}
  Therefore $2$ divides $a$. For the same reason, we also have
  \begin{equation*}
    a = 3\varphi\left(\frac13\right).
  \end{equation*}
  So $3$ divides $a$. Using the same argument we see that the integer
  $a$ is actually divisible by every positive integer. The only way
  this is possible is if $a = 0$. But then, for any $n\in\Z$, we would
  have $\varphi(n) = n\varphi(1) = na = 0$. So $\varphi$ is clearly
  not an injection, and this gives the necessary
  contradiction. Therefore the additive groups $\Z$ and $\Q$ are not
  isomorphic.
\end{proof}

\Exercise7 Prove that $D_8$ and $Q_8$ are not isomorphic.
\begin{proof}
  We may simply look at the orders of the elements in each group. For
  example, $D_8$ has $4$ elements with order $2$ (namely, $s$, $sr$,
  $sr^2$, and $sr^3$), while $Q_8$ only has one element with order $2$
  (namely $-1$). Therefore $D_8\not\cong Q_8$.
\end{proof}

\Exercise8 Prove that if $n\neq m$, $S_n$ and $S_m$ are not
isomorphic.
\begin{proof}
  Since $S_n$ has order $n!$ and $S_m$ has order $m!$, there is no
  bijection from $S_n$ to $S_m$ unless $n = m$. Therefore $S_n$ and
  $S_m$ are not isomorphic when $n\neq m$.
\end{proof}

\Exercise9 Prove that $D_{24}$ and $S_4$ are not isomorphic.
\begin{proof}
  $D_{24}$ has elements of order $12$, namely $r$, $r^5$, $r^7$, and
  $r^{11}$. However, $S_4$ has no elements of order $12$, since every
  permutation in $S_4$ is either a $2$-cycle or product of $2$-cycles
  (which have order $2$), a $3$-cycle (which has order $3$), or a
  $4$-cycle (which has order $4$). Since isomorphisms must preserve
  orders of elements, $D_{24}$ and $S_4$ cannot be isomorphic.
\end{proof}

\Exercise{10} Fill in the details of the proof that the symmetric
group $S_\Delta$ and $S_\Omega$ are isomorphic if
$\ord{\Delta} = \ord{\Omega}$ as follows: let
$\theta\colon\Delta\to\Omega$ be a bijection. Define
\begin{equation*}
  \varphi\colon S_\Delta\to S_\Omega
  \quad\text{by}\quad
  \varphi(\sigma) = \theta\circ\sigma\circ\theta^{-1}
  \quad\text{for all $\sigma\in S_\Delta$}
\end{equation*}
and prove the following:
\begin{enumerate}
\item $\varphi$ is well defined, that is, if $\sigma$ is a permutation
  of $\Delta$ then $\theta\circ\sigma\circ\theta^{-1}$ is a
  permutation of $\Omega$.
  \begin{proof}
    For any permutation $\sigma$ of $\Delta$, it is clear that
    $\varphi(\sigma) = \theta\circ\sigma\circ\theta^{-1}$ is a
    function from $\Omega$ to itself. We want to show that it is a
    bijection.

    Suppose $a,b\in\Omega$ are such that
    $\varphi(\sigma)(a) = \varphi(\sigma)(b)$. Since $\theta$ is an
    injection, this implies that
    $\sigma\circ\theta^{-1}(a) = \sigma\circ\theta^{-1}(b)$. But
    $\sigma$ is also an injection, so
    $\theta^{-1}(a) = \theta^{-1}(b)$ and so $a = b$. This shows that
    $\varphi(\sigma)$ is an injection. But an injective mapping from a
    set to itself must also be surjective. Hence $\varphi(\sigma)$ is
    a bijection from $\Omega$ to itself, that is, $\varphi(\sigma)$ is
    a permutation of $\Omega$.
  \end{proof}
\item $\varphi$ is a bijection from $S_\Delta$ to $S_\Omega$.
  \begin{proof}
    Define $\psi\colon S_\Omega\to S_\Delta$ by
    \begin{equation*}
      \psi(\tau) = \theta^{-1}\circ\tau\circ\theta
      \quad\text{for any $\tau\in S_\Omega$}.
    \end{equation*}
    By the same argument as in part (a), $\psi$ is
    well-defined. Moreover, for any $\sigma\in S_\Delta$,
    \begin{equation*}
      (\psi\circ\varphi)(\sigma) = \psi(\theta\circ\sigma\circ\theta^{-1})
      = \theta^{-1}\circ\theta\circ\sigma\circ\theta^{-1}\circ\theta
      = \sigma,
    \end{equation*}
    and for any $\tau\in S_\Omega$,
    \begin{equation*}
      (\varphi\circ\psi)(\tau) = \varphi(\theta^{-1}\circ\tau\circ\theta)
      = \theta\circ\theta^{-1}\circ\tau\circ\theta\circ\theta^{-1}
      = \tau.
    \end{equation*}
    Therefore $\psi$ is a two-sided inverse of $\varphi$, so that
    $\varphi$ is a bijection.
  \end{proof}
\item $\varphi$ is a homomorphism, that is,
  $\varphi(\sigma\circ\tau) = \varphi(\sigma)\circ\varphi(\tau)$.
  \begin{proof}
    Let $\sigma,\tau\in S_\Delta$. Then
    \begin{equation*}
      \varphi(\sigma)\circ\varphi(\tau)
      = \theta\circ\sigma\circ\theta^{-1}\circ\theta\circ\tau\circ\theta^{-1}
      = \theta\circ\sigma\circ\tau\circ\theta^{-1}
      = \varphi(\sigma\circ\tau).
    \end{equation*}
    Therefore $\varphi$ is a homomorphism, and hence an isomorphism.
  \end{proof}
\end{enumerate}
