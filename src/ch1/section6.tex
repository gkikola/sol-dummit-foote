\section{Homomorphisms and Isomorphisms}

Let $G$ and $H$ be groups.

\Exercise1 Let $\varphi\colon G\to H$ be a homomorphism.
\begin{enumerate}
\item Prove that $\varphi(x^n) = \varphi(x)^n$ for all $n\in\Z^+$.
  \begin{proof}
    We use induction on $n$. The case for $n = 1$ is clear. Suppose
    $\varphi(x^n) = \varphi(x)^n$ for some particular $n\in\Z^+$. Then
    \begin{equation*}
      \varphi(x^{n+1})
      = \varphi(xx^n)
      = \varphi(x)\varphi(x^n)
      = \varphi(x)\varphi(x)^n
      = \varphi(x)^{n+1},
    \end{equation*}
    so the result holds for all $n\in\Z^+$.
  \end{proof}
\item Do part (a) for $n = -1$ and deduce that
  $\varphi(x^n) = \varphi(x)^n$ for all $n\in\Z$.
  \begin{solution}
    Let $1_G$ and $1_H$ denote the identities of $G$ and $H$,
    respectively. Then
    \begin{equation*}
      \varphi(1_G)\varphi(1_G) = \varphi(1_G) = \varphi(1_G)1_H,
    \end{equation*}
    and it follows from the cancellation law that
    $\varphi(1_G) = 1_H$. Since the identity is preserved, we will
    simply use $1$ to denote the identity of both groups from this
    point forward.

    Now, for any $x\in G$,
    \begin{equation*}
      \varphi(x)\varphi(x^{-1}) = \varphi(xx^{-1}) = \varphi(1) = 1,
    \end{equation*}
    which shows that $\varphi(x^{-1}) = \varphi(x)^{-1}$. Then for
    $n\in\Z^+$,
    \begin{equation*}
      \varphi(x^{-n}) = \varphi((x^n)^{-1}) = \varphi(x^n)^{-1}
      = \big(\varphi(x)^n\big)^{-1} = \varphi(x)^{-n}.
    \end{equation*}
    Therefore $\varphi(x^n) = \varphi(x)^n$ holds for all $n\in\Z$.
  \end{solution}
\end{enumerate}

\Exercise2 If $\varphi\colon G\to H$ is an isomorphism, prove that
$\ord{\varphi(x)} = \ord{x}$ for all $x\in G$. Deduce that any two
isomorphic groups have the same number of elements of order $n$ for
each $n\in\Z^+$. Is the result true if $\varphi$ is only assumed to be
a homomorphism?
\begin{proof}
  First suppose $\ord{x} = n<\infty$. By the previous exercise, we have
  \begin{equation*}
    \varphi(x)^n = \varphi(x^n) = \varphi(1) = 1.
  \end{equation*}
  So $\ord{\varphi(x)}\leq n$. On the other hand, if
  $\ord{\varphi(x)} = k$, then
  \begin{equation*}
    \varphi(x^k) = \varphi(x)^k = 1.
  \end{equation*}
  But $1$ is the only element in $G$ which gets sent to $1$ in $H$,
  since $\varphi$ is a bijection. This shows that $x^k = 1$, so that
  $k\geq n$. Hence $\ord{\varphi(x)} = n$.

  Now suppose $x$ has infinite order. If
  $\ord{\varphi(x)} = n < \infty$, then
  $\varphi(x^n) = \varphi(x)^n = 1$, and since $\varphi$ is a
  bijection we must have $x^n = 1$, a contradiction. Therefore
  $\varphi(x)$ must also have infinite order.

  From the above we know that $\ord{x} = \ord{\varphi(x)}$ for each
  $x$, and since $\varphi$ is a bijection this shows that $G$ and $H$
  have the same number of elements of each order.

  Finally, this result does not necessarily hold for
  homomorphisms. For example, let $H$ be the trivial group $\{1\}$ and
  take the function $\theta\colon G\to H$ defined by $\theta(x) = 1$
  for all $x\in G$. Then $\theta(x)\theta(y) = \theta(xy)$, so this is
  a homomorphism, but every element in $H$ has order $1$, which is not
  true of $G$ (unless $G$ is also trivial).
\end{proof}
