\section{Homomorphisms and Isomorphisms}

Let $G$ and $H$ be groups.

\Exercise1 Let $\varphi\colon G\to H$ be a homomorphism.
\label{exercise-homomorphism-preserves-powers}
\begin{enumerate}
\item Prove that $\varphi(x^n) = \varphi(x)^n$ for all $n\in\Z^+$.
  \begin{proof}
    We use induction on $n$. The case for $n = 1$ is clear. Suppose
    $\varphi(x^n) = \varphi(x)^n$ for some particular $n\in\Z^+$. Then
    \begin{equation*}
      \varphi(x^{n+1})
      = \varphi(xx^n)
      = \varphi(x)\varphi(x^n)
      = \varphi(x)\varphi(x)^n
      = \varphi(x)^{n+1},
    \end{equation*}
    so the result holds for all $n\in\Z^+$.
  \end{proof}
\item Do part (a) for $n = -1$ and deduce that
  $\varphi(x^n) = \varphi(x)^n$ for all $n\in\Z$.
  \begin{solution}
    Let $1_G$ and $1_H$ denote the identities of $G$ and $H$,
    respectively. Then
    \begin{equation*}
      \varphi(1_G)\varphi(1_G) = \varphi(1_G) = \varphi(1_G)1_H,
    \end{equation*}
    and it follows from the cancellation law that
    $\varphi(1_G) = 1_H$. Since the identity is preserved, we will
    simply use $1$ to denote the identity of both groups from this
    point forward.

    Now, for any $x\in G$,
    \begin{equation*}
      \varphi(x)\varphi(x^{-1}) = \varphi(xx^{-1}) = \varphi(1) = 1,
    \end{equation*}
    which shows that $\varphi(x^{-1}) = \varphi(x)^{-1}$. Then for
    $n\in\Z^+$,
    \begin{equation*}
      \varphi(x^{-n}) = \varphi((x^n)^{-1}) = \varphi(x^n)^{-1}
      = \big(\varphi(x)^n\big)^{-1} = \varphi(x)^{-n}.
    \end{equation*}
    Therefore $\varphi(x^n) = \varphi(x)^n$ holds for all $n\in\Z$.
  \end{solution}
\end{enumerate}

\Exercise2 If $\varphi\colon G\to H$ is an isomorphism, prove that
$\ord{\varphi(x)} = \ord{x}$ for all $x\in G$. Deduce that any two
isomorphic groups have the same number of elements of order $n$ for
each $n\in\Z^+$. Is the result true if $\varphi$ is only assumed to be
a homomorphism?
\begin{proof}
  First suppose $\ord{x} = n<\infty$. By the previous exercise, we have
  \begin{equation*}
    \varphi(x)^n = \varphi(x^n) = \varphi(1) = 1.
  \end{equation*}
  So $\ord{\varphi(x)}\leq n$. On the other hand, if
  $\ord{\varphi(x)} = k$, then
  \begin{equation*}
    \varphi(x^k) = \varphi(x)^k = 1.
  \end{equation*}
  But $1$ is the only element in $G$ which gets sent to $1$ in $H$,
  since $\varphi$ is a bijection. This shows that $x^k = 1$, so that
  $k\geq n$. Hence $\ord{\varphi(x)} = n$.

  Now suppose $x$ has infinite order. If
  $\ord{\varphi(x)} = n < \infty$, then
  $\varphi(x^n) = \varphi(x)^n = 1$, and since $\varphi$ is a
  bijection we must have $x^n = 1$, a contradiction. Therefore
  $\varphi(x)$ must also have infinite order.

  From the above we know that $\ord{x} = \ord{\varphi(x)}$ for each
  $x$, and since $\varphi$ is a bijection this shows that $G$ and $H$
  have the same number of elements of each order.

  Finally, this result does not necessarily hold for
  homomorphisms. For example, let $H$ be the trivial group $\{1\}$ and
  take the function $\theta\colon G\to H$ defined by $\theta(x) = 1$
  for all $x\in G$. Then $\theta(x)\theta(y) = \theta(xy)$, so this is
  a homomorphism, but every element in $H$ has order $1$, which is not
  true of $G$ (unless $G$ is also trivial).
\end{proof}

\Exercise3 If $\varphi\colon G\to H$ is an isomorphism, prove that $G$
is abelian if and only if $H$ is abelian. If $\varphi\colon G\to H$ is
a homomorphism, what additional conditions on $\varphi$ (if any) are
sufficient to ensure that if $G$ is abelian, then so is $H$?
\begin{solution}
  Since $\varphi$ must be invertible (it is a bijection) and since
  $\varphi^{-1}$ must be an isomorphism from $H$ to $G$, the proof
  only needs to work in one direction. So let $x,y\in H$ be arbitrary,
  and let $a = \varphi^{-1}(x)$ and $b = \varphi^{-1}(y)$. If $G$ is
  abelian, then
  \begin{equation*}
    xy = \varphi(a)\varphi(b) = \varphi(ab) = \varphi(ba)
    = \varphi(b)\varphi(a) = yx,
  \end{equation*}
  so $H$ is also abelian, and the proof is complete.

  Note that the same result does not hold for homomorphisms. For
  instance, let $\varphi\colon\Z/2\Z\to D_6$ be given by
  $\varphi(\bar0) = 1$ and $\varphi(\bar1) = s$. Then $\varphi$ is a
  homomorphism and $\Z/2\Z$ is abelian, but $D_6$ is not abelian.

  However, if we add the constraint that $\varphi$ is surjective, then
  the result does hold: Suppose $G$ is abelian, let $x,y\in H$ be
  arbitrary, and pick $a\in\varphi^{-1}(x)$ and $b\in\varphi^{-1}(y)$
  (that is, $a$ and $b$ are chosen from the fibers of $\varphi$ over
  $x$ and $y$). Then, as before,
  \begin{equation*}
    xy = \varphi(a)\varphi(b) = \varphi(ab) = \varphi(ba)
    = \varphi(b)\varphi(a) = yx,
  \end{equation*}
  so $H$ is abelian.
\end{solution}

\Exercise4 Prove that the multiplicative groups $\R-\{0\}$ and
$\C-\{0\}$ are not isomorphic.
\begin{proof}
  Every element in $\R-\{0\}$ has infinite order, aside from $1$ and
  $-1$ which have orders $1$ and $2$, respectively. However,
  $\C-\{0\}$ has elements of order $4$, namely $i$ and $-i$. Therefore
  these groups are not isomorphic.
\end{proof}

\Exercise5 Prove that the additive groups $\R$ and $\Q$ are not
isomorphic.
\begin{proof}
  There is no bijection between $\R$ and $\Q$, since the former is
  uncountable and the latter is countable. Therefore the groups
  $(\R,+)$ and $(\Q,+)$ are not isomorphic.
\end{proof}

\Exercise6 Prove that the additive groups $\Z$ and $\Q$ are not
isomorphic.
\begin{proof}
  Suppose the contrary, and let $\varphi\colon\Q\to\Z$ be an
  isomorphism. Let
  \begin{equation*}
    a = \varphi(1).
  \end{equation*}
  Then
  \begin{equation*}
    a = \varphi\left(\frac12 + \frac12\right)
    = 2\varphi\left(\frac12\right).
  \end{equation*}
  Therefore $2$ divides $a$. For the same reason, we also have
  \begin{equation*}
    a = 3\varphi\left(\frac13\right).
  \end{equation*}
  So $3$ divides $a$. Using the same argument we see that the integer
  $a$ is actually divisible by every positive integer. The only way
  this is possible is if $a = 0$. But then, for any $n\in\Z$, we would
  have $\varphi(n) = n\varphi(1) = na = 0$. So $\varphi$ is clearly
  not an injection, and this gives the necessary
  contradiction. Therefore the additive groups $\Z$ and $\Q$ are not
  isomorphic.
\end{proof}

\Exercise7 Prove that $D_8$ and $Q_8$ are not isomorphic.
\begin{proof}
  We may simply look at the orders of the elements in each group. For
  example, $D_8$ has $4$ elements with order $2$ (namely, $s$, $sr$,
  $sr^2$, and $sr^3$), while $Q_8$ only has one element with order $2$
  (namely $-1$). Therefore $D_8\not\cong Q_8$.
\end{proof}

\Exercise8 Prove that if $n\neq m$, $S_n$ and $S_m$ are not
isomorphic.
\begin{proof}
  Since $S_n$ has order $n!$ and $S_m$ has order $m!$, there is no
  bijection from $S_n$ to $S_m$ unless $n = m$. Therefore $S_n$ and
  $S_m$ are not isomorphic when $n\neq m$.
\end{proof}

\Exercise9 Prove that $D_{24}$ and $S_4$ are not isomorphic.
\begin{proof}
  $D_{24}$ has elements of order $12$, namely $r$, $r^5$, $r^7$, and
  $r^{11}$. However, $S_4$ has no elements of order $12$, since every
  permutation in $S_4$ is either a $2$-cycle or product of $2$-cycles
  (which have order $2$), a $3$-cycle (which has order $3$), or a
  $4$-cycle (which has order $4$). Since isomorphisms must preserve
  orders of elements, $D_{24}$ and $S_4$ cannot be isomorphic.
\end{proof}

\Exercise{10} Fill in the details of the proof that the symmetric
group $S_\Delta$ and $S_\Omega$ are isomorphic if
$\ord{\Delta} = \ord{\Omega}$ as follows: let
$\theta\colon\Delta\to\Omega$ be a bijection. Define
\begin{equation*}
  \varphi\colon S_\Delta\to S_\Omega
  \quad\text{by}\quad
  \varphi(\sigma) = \theta\circ\sigma\circ\theta^{-1}
  \quad\text{for all $\sigma\in S_\Delta$}
\end{equation*}
and prove the following:
\begin{enumerate}
\item $\varphi$ is well defined, that is, if $\sigma$ is a permutation
  of $\Delta$ then $\theta\circ\sigma\circ\theta^{-1}$ is a
  permutation of $\Omega$.
  \begin{proof}
    For any permutation $\sigma$ of $\Delta$, it is clear that
    $\varphi(\sigma) = \theta\circ\sigma\circ\theta^{-1}$ is a
    function from $\Omega$ to itself. We want to show that it is a
    bijection.

    Suppose $a,b\in\Omega$ are such that
    $\varphi(\sigma)(a) = \varphi(\sigma)(b)$. Since $\theta$ is an
    injection, this implies that
    $(\sigma\circ\theta^{-1})(a) = (\sigma\circ\theta^{-1})(b)$. But
    $\sigma$ is also an injection, so
    $\theta^{-1}(a) = \theta^{-1}(b)$ and, similarly, we have $a =
    b$. This shows that $\varphi(\sigma)$ is an injection.

    Now let $y\in\Omega$ be arbitrary. Then we may take
    \begin{equation*}
      x = \varphi(\sigma^{-1})(y)
      = (\theta\circ\sigma^{-1}\circ\theta^{-1})(y)
    \end{equation*}
    so that $\varphi(\sigma)(x) = y$. This shows that
    $\varphi(\sigma)$ is a surjection. Hence $\varphi(\sigma)$ is a
    bijection from $\Omega$ to itself, that is, $\varphi(\sigma)$ is a
    permutation of $\Omega$.
  \end{proof}
\item $\varphi$ is a bijection from $S_\Delta$ to $S_\Omega$.
  \begin{proof}
    Define $\psi\colon S_\Omega\to S_\Delta$ by
    \begin{equation*}
      \psi(\tau) = \theta^{-1}\circ\tau\circ\theta
      \quad\text{for any $\tau\in S_\Omega$}.
    \end{equation*}
    By the same argument as in part (a), $\psi$ is
    well-defined. Moreover, for any $\sigma\in S_\Delta$,
    \begin{equation*}
      (\psi\circ\varphi)(\sigma) = \psi(\theta\circ\sigma\circ\theta^{-1})
      = \theta^{-1}\circ\theta\circ\sigma\circ\theta^{-1}\circ\theta
      = \sigma,
    \end{equation*}
    and for any $\tau\in S_\Omega$,
    \begin{equation*}
      (\varphi\circ\psi)(\tau) = \varphi(\theta^{-1}\circ\tau\circ\theta)
      = \theta\circ\theta^{-1}\circ\tau\circ\theta\circ\theta^{-1}
      = \tau.
    \end{equation*}
    Therefore $\psi$ is a two-sided inverse of $\varphi$, so that
    $\varphi$ is a bijection.
  \end{proof}
\item $\varphi$ is a homomorphism, that is,
  $\varphi(\sigma\circ\tau) = \varphi(\sigma)\circ\varphi(\tau)$.
  \begin{proof}
    Let $\sigma,\tau\in S_\Delta$. Then
    \begin{equation*}
      \varphi(\sigma)\circ\varphi(\tau)
      = \theta\circ\sigma\circ\theta^{-1}\circ\theta\circ\tau\circ\theta^{-1}
      = \theta\circ\sigma\circ\tau\circ\theta^{-1}
      = \varphi(\sigma\circ\tau).
    \end{equation*}
    Therefore $\varphi$ is a homomorphism, and hence an isomorphism.
  \end{proof}
\end{enumerate}

\Exercise{11} Let $A$ and $B$ be groups. Prove that
$A\times B\cong B\times A$.
\begin{proof}
  Define the function $\varphi\colon A\times B\to B\times A$ by
  \begin{equation*}
    \varphi(a,b) = (b,a)
    \quad\text{for any $(a,b)\in A\times B$.}
  \end{equation*}
  This is a homomorphism, since
  \begin{equation*}
    \varphi((a,b)(c,d)) = \varphi(ac,bd) = (bd,ac) = (b,a)(d,c)
    = \varphi(a,b)\varphi(c,d).
  \end{equation*}
  It is also a surjection, since for any $(b,a)\in B\times A$ we can
  take $(a,b)\in A\times B$ so that $\varphi(a,b) = (b,a)$. Finally,
  if $(a,b),(c,d)\in A\times B$ are such that
  $\varphi(a,b) = \varphi(c,d)$ then $(b,a) = (d,c)$. Then $b = d$ and
  $a = c$, so $(a,b) = (c,d)$ and $\varphi$ is an injection. This
  shows that $\varphi$ is a bijection and hence an isomorphism.
\end{proof}

\Exercise{12} Let $A$, $B$, and $C$ be groups and let $G = A\times B$
and $H = B\times C$. Prove that $G\times C\cong A\times H$.
\begin{proof}
  Define $\varphi\colon G\times C\to A\times H$ as follows. For any
  $((a,b),c)\in G\times C$ by
  \begin{equation*}
    \varphi((a,b),c) = (a,(b,c)).
  \end{equation*}
  It is very straightforward to verify that $\varphi$ is a bijection
  and a homomorphism, and hence $G\times C\cong A\times H$.
\end{proof}

\Exercise{13} Let $G$ and $H$ be groups and let $\varphi\colon G\to H$
be a homomorphism. Prove that the image of $\varphi$, $\varphi(G)$, is
a subgroup of $H$. Prove that if $\varphi$ is injective then
$G\cong\varphi(G)$.
\begin{proof}
  We know that $\varphi(G)$ is nonempty, since in particular
  $\varphi(1)$ is mapped to some element in $H$ (we know from earlier
  exercises that the identity is preserved so $\varphi(1) = 1$, but we
  do not strictly need that information here). Let $a,b\in\varphi(G)$
  be arbitrary. Then there exist $\alpha,\beta\in G$ such that
  $\varphi(\alpha) = a$, and $\varphi(\beta) = b$. Then
  \begin{equation*}
    ab = \varphi(\alpha)\varphi(\beta) = \varphi(\alpha\beta),
  \end{equation*}
  so $ab\in\varphi(G)$ and $\varphi(G)$ is closed under the binary
  operation of $H$. Moreover, by
  Exercise~\ref{exercise-homomorphism-preserves-powers},
  \begin{equation*}
    a^{-1} = \varphi(\alpha)^{-1} = \varphi(\alpha^{-1}),
  \end{equation*}
  so $\varphi(G)$ is closed under inverses. Hence $\varphi(G)$ is a
  subgroup of $H$.

  Now, if we define $\varphi^*\colon G\to\varphi(G)$ by
  $\varphi^*(\gamma) = \varphi(\gamma)$ for each $\gamma\in G$, then
  $\varphi^*$ is surjective by definition. If, in addition, $\varphi$
  is injective, then $\varphi^*$ is a bijection and
  $G\cong\varphi(G)$.
\end{proof}

\Exercise{14} Let $G$ and $H$ be groups and let $\varphi\colon G\to H$
be a homomorphism. Define the {\em kernel} of $\varphi$ to be
$\{g\in G\mid \varphi(g) = 1_H\}$ (so the kernel is the set of
elements in $G$ which map to the identity of $H$, i.e., is the fiber
over the identity of $H$). Prove that the kernel of $\varphi$ is a
subgroup of $G$. Prove that $\varphi$ is injective if and only if the
kernel of $\varphi$ is the identity subgroup of $G$.
\begin{proof}
  From Exercise~\ref{exercise-homomorphism-preserves-powers} we know
  that $\varphi(1_G) = 1_H$ so the kernel of $\varphi$ is
  nonempty. Suppose $a,b\in\ker\varphi$. Then
  \begin{equation*}
    \varphi(ab) = \varphi(a)\varphi(b) = 1_H1_H = 1_H,
  \end{equation*}
  and $ab\in\ker\varphi$. Additionally, if $a\in\ker\varphi$ then
  \begin{equation*}
    \varphi(a^{-1}) = \varphi(a)^{-1} = 1_H^{-1} = 1_H
  \end{equation*}
  and $a^{-1}\in\ker\varphi$. Therefore $\ker\varphi$ is a subgroup of
  $G$.
\end{proof}

\Exercise{15} Define a map $\pi\colon\R^2\to\R$ by
$\pi((x,y))=x$. Prove that $\pi$ is a homomorphism and find the kernel
of $\pi$.
\begin{proof}
  For any $(x_1,y_1),(x_2,y_2)\in\R^2$, we have
  \begin{equation*}
    \pi((x_1,y_1) + (x_2,y_2)) = \pi(x_1 + x_2,y_1 + y_2) = x_1 + x_2
    = \pi(x_1,y_1) + \pi(x_2,y_2),
  \end{equation*}
  so $\pi$ is a homomorphism. Also,
  \begin{equation*}
    \ker\pi = \{ (0, y)\in\R^2 \mid y\in\R \}. \qedhere
  \end{equation*}
\end{proof}

\Exercise{16} Let $A$ and $B$ be groups and let $G$ be their direct
product, $A\times B$. Prove that the maps $\pi_1\colon G\to A$ and
$\pi_2\colon G\to B$ defined by $\pi_1((a,b)) = a$ and
$\pi_2((a,b)) = b$ are homomorphisms and find their kernels.
\begin{proof}
  For any $(a,b)$ and $(c,d)\in A\times B$, we have
  \begin{equation*}
    \pi_1((a,b)(c,d)) = \pi_1(ac,bd) = ac = \pi_1(a,b)\pi_1(c,d)
  \end{equation*}
  and
  \begin{equation*}
    \pi_2((a,b)(c,d)) = \pi_2(ac,bd) = bd = \pi_2(a,b)\pi_2(c,d),
  \end{equation*}
  so $\pi_1$ and $\pi_2$ are homomorphisms. Their kernels are
  \begin{equation*}
    \ker\pi_1 = \{ (1, b)\in A\times B \mid b\in B \}
  \end{equation*}
  and
  \begin{equation*}
    \ker\pi_2 = \{ (a, 1)\in A\times B \mid a\in A \}. \qedhere
  \end{equation*}
\end{proof}

\Exercise{17} Let $G$ be any group. Prove that the map from $G$ to
itself defined by $g\mapsto g^{-1}$ is a homomorphism if and only if
$G$ is abelian.
\begin{proof}
  Suppose $G$ is abelian. Then for any $a,b\in G$,
  \begin{equation*}
    (ab)^{-1} = b^{-1}a^{-1} = a^{-1}b^{-1},
  \end{equation*}
  so $g\mapsto g^{-1}$ is a homomorphism. Conversely, suppose
  $g\mapsto g^{-1}$ is a homomorphism and let $a,b\in G$ be
  arbitrary. Then $b^{-1}a^{-1} = (ba)^{-1}$ and we have
  \begin{equation*}
    ab = (a^{-1})^{-1}(b^{-1})^{-1} = (b^{-1}a^{-1})^{-1}
    = [(ba)^{-1}]^{-1} = ba,
  \end{equation*}
  so $G$ is abelian.
\end{proof}

\Exercise{18} Let $G$ be any group. Prove that the map from $G$ to
itself defined by $g\mapsto g^2$ is a homomorphism if and only if $G$
is abelian.
\begin{proof}
  Suppose $G$ is abelian. Then for any $a,b\in G$,
  \begin{equation*}
    (ab)^2 = abab = a^2b^2,
  \end{equation*}
  and $g\mapsto g^2$ is a homomorphism. Now suppose $g\mapsto g^2$ is
  a homomorphism. Then for any $a,b\in G$,
  \begin{equation*}
    a^2b^2 = (ab)^2 = abab,
  \end{equation*}
  and multiplying both sides of the equation $a^2b^2 = abab$ on the
  left by $a$ and on the right by $b$ gives $ab = ba$, so that $G$ is
  abelian.
\end{proof}

\Exercise{20} Let $G$ be a group and let $\Aut(G)$ be the set of all
isomorphisms from $G$ onto $G$. Prove that $\Aut(G)$ is a group under
function composition (called the {\em automorphism group} of $G$ and
the elements of $\Aut(G)$ are called {\em automorphisms} of $G$).
\begin{proof}
  Let $\varphi,\psi\in\Aut(G)$. Then $\varphi\circ\psi$ is a bijection
  from $G$ to itself. It is also a homomorphism, since for any
  $a,b\in G$,
  \begin{equation*}
    (\varphi\circ\psi)(ab) = \varphi(\psi(a)\psi(b))
    = (\varphi\circ\psi)(a)(\varphi\circ\psi)(b).
  \end{equation*}
  This shows that $\varphi\circ\psi\in\Aut(G)$ so $\Aut(G)$ is closed
  under composition. And function composition is always associative.

  Clearly the identity map $1\colon G\to G$ is an isomorphism, so
  $\Aut(G)$ has an identity. And for any $\varphi\in\Aut(G)$,
  $\varphi^{-1}$ must exist since $\varphi$ is a bijection. Now, for
  any $a,b\in G$ let $a^* = \varphi^{-1}(a)$ and
  $b^* = \varphi^{-1}(b)$. Since $\varphi$ is a homomorphism, we have
  \begin{equation*}
    \varphi(a^*b^*) = \varphi(a^*)\varphi(b^*) = ab,
  \end{equation*}
  which implies that $a^*b^* = \varphi^{-1}(ab)$. Then
  \begin{equation*}
    \varphi^{-1}(a)\varphi^{-1}(b) = a^*b^* = \varphi^{-1}(ab),
  \end{equation*}
  and we see that $\varphi^{-1}$ is an isomorphism and hence
  $\varphi^{-1}\in\Aut(G)$. So elements in $\Aut(G)$ have
  inverses. Therefore $\Aut(G)$ is a group under function composition.
\end{proof}

\Exercise{21} Prove that for each fixed nonzero $k\in\Q$ the map from
$\Q$ to itself defined by $q\mapsto kq$ is an automorphism of $\Q$.
\begin{proof}
  Fix a nonzero $k\in\Q$ and let $\varphi\colon\Q\to\Q$ be given by
  $\varphi(r) = kr$. Then for any $a,b\in\Q$,
  \begin{equation*}
    \varphi(a + b) = k(a + b) = ka + kb = \varphi(a) + \varphi(b),
  \end{equation*}
  so $\varphi$ is a homomorphism. To show that it is a bijection, note
  that it must be surjective since for any $a\in\Q$, we may take
  $b = a/k$ so that $\varphi(b) = a$. And $\varphi$ must be injective
  since for any $a,b\in\Q$, $\varphi(a) = \varphi(b)$ implies
  $ka = kb$ which implies $a = b$ since $k$ is nonzero. Therefore
  $\varphi$ is a bijection and hence an automorphism of $\Q$.
\end{proof}

\Exercise{22} Let $A$ be an abelian group and fix some $k\in\Z$. Prove
that the map $a\mapsto a^k$ is a homomorphism from $A$ to itself. If
$k = -1$, prove that this homomorphism is an isomorphism (i.e., is an
automorphism of $A$).
\begin{proof}
  Fix $k\in\Z$ and let $\varphi\colon A\to A$ be the mapping
  $a\mapsto a^k$. Then for any $a,b\in A$, we have
  \begin{equation*}
    \varphi(ab) = (ab)^k = a^kb^k = \varphi(a)\varphi(b),
  \end{equation*}
  where the second equality follows from the fact that $A$ is
  abelian. So $\varphi$ is a homomorphism.

  In the case where $k=-1$, $\varphi$ must be a bijection since it is
  its own inverse function. Hence $a\mapsto a^{-1}$ is an automorphism
  of $A$.
\end{proof}

\Exercise{23} Let $G$ be a finite group which possesses an
automorphism $\sigma$ such that $\sigma(g) = g$ if and only if
$g = 1$. If $\sigma^2$ is the identity map from $G\to G$, prove that
$G$ is abelian (such an automorphism $\sigma$ is called {\em fixed
  point free} of order $2$).
\begin{proof}
  Consider the map $\varphi\colon G\to G$ given by
  $\varphi(x) = x^{-1}\sigma(x)$. For any $x,y\in G$, if
  $\varphi(x) = \varphi(y)$ then
  \begin{equation*}
    x^{-1}\sigma(x) = y^{-1}\sigma(y)
  \end{equation*}
  or, rearranging,
  \begin{equation*}
    \sigma(y) = yx^{-1}\sigma(x).
  \end{equation*}
  This gives
  \begin{equation*}
    y = \sigma(\sigma(y)) = \sigma(yx^{-1}\sigma(x))
    = \sigma(yx^{-1})x
  \end{equation*}
  and multiplying on the right by $x^{-1}$ gives
  \begin{equation}
    \label{eq:y-x-inv-equals-sigma-y-x-inv}
    yx^{-1} = \sigma(yx^{-1}).
  \end{equation}
  Since $\sigma$ is fixed point free,
  \eqref{eq:y-x-inv-equals-sigma-y-x-inv} then implies that
  $yx^{-1} = 1$ or $x = y$. Therefore $\varphi$ is an injection, and
  hence a bijection since it maps the finite set $G$ to
  itself. Therefore every $x\in G$ can be written in the form
  $x = y^{-1}\sigma(y)$ for some $y\in G$.

  Now let $x\in G$ be arbitrary. Then, for some $y\in G$,
  \begin{equation*}
    \sigma(x) = \sigma(y^{-1}\sigma(y)) = \sigma(y)^{-1}y.
  \end{equation*}
  However, since $(ab)^{-1} = b^{-1}a^{-1}$ for $a,b$ in any group, we
  also have
  \begin{equation*}
    \sigma(y)^{-1}y = \sigma(y)^{-1}(y^{-1})^{-1}
    = \big(y^{-1}\sigma(y)\big)^{-1} = x^{-1}.
  \end{equation*}
  Hence $\sigma(x) = x^{-1}$ for all $x\in G$.

  Finally, let $a,b\in G$ be arbitrary. Then
  \begin{equation*}
    \sigma(ab) = (ab)^{-1} = b^{-1}a^{-1} = \sigma(b)\sigma(a) = \sigma(ba).
  \end{equation*}
  But $\sigma$ is an injection, so $ab = ba$. This shows that $G$ is
  abelian.
\end{proof}

\Exercise{24} Let $G$ be a finite group and let $x$ and $y$ be
distinct elements of order $2$ in $G$ that generate $G$. Prove that
$G\cong D_{2n}$, where $n = \ord{xy}$.
\begin{proof}
  Let $t = xy$. By
  Exercise~\ref{exercise-x-y-order-2-satisfy-dihedral-relations}, we
  have $tx = xt^{-1}$. Note also that $x$ and $t$ generate $G$, since
  $y$ can be written as $y = xt$. So by repeated application of the
  relation $tx = xt^{-1}$, we may express any member of $G$ uniquely
  in the form $x^it^j$ for some integers $i,j$ with $0\leq i\leq1$ and
  $0\leq j<n$ (the representation is unique since $t$ has order $n$,
  which implies that $t, t^2, \dots, t^{n-1}$ are all
  distinct). Therefore $\ord{G} = 2n$.

  Now let $\varphi\colon D_{2n}\to G$ be given by
  \begin{equation*}
    \varphi(s^ir^j) = x^it^j,
    \quad\text{for $i,j\in\Z$ with $0\leq i\leq1$ and $0\leq j\leq n-1$}.
  \end{equation*}
  Since every element in $D_{2n}$ can be written uniquely as $s^ir^j$
  with the above restrictions on $i$ and $j$, the function $\varphi$
  is well defined. And since $x$ and $t$ satisfy the same relations in
  $G$ that $s$ and $r$ satisfy in $D_{2n}$, $\varphi$ must be a
  homomorphism.

  We will now show that $\varphi$ is a bijection. For any $b\in G$,
  write $b = x^it^j$ for $i\in\{0,1\}$ and
  $j\in\{0,1,\dots,n-1\}$. Then if $a = s^ir^j$, we have
  $\varphi(a) = b$, which shows that $\varphi$ is surjective. Since
  $\ord{G} = \ord{D_{2n}}$, this is enough to show that $\varphi$ is a
  bijection.

  The function $\varphi$ is a bijective homomorphism, hence it is an
  isomorphism and $D_{2n}\cong G$.
\end{proof}
