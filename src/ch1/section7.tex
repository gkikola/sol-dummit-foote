\section{Group Actions}

\Exercise1 Let $F$ be a field. Show that the multiplicative group of
nonzero elements of $F$ (denoted by $F^\times$) acts on the set $F$ by
$g\cdot a = ga$, where $g\in F^\times$, $a\in F$ and $ga$ is the usual
product in $F$ of the two field elements.
\begin{proof}
  Let $g_1,g_2\in F^\times$. Then for any $a\in F$,
  \begin{equation*}
    g_1\cdot(g_2\cdot a) = g_1\cdot g_2a = g_1(g_2a) = (g_1g_2)a
    = (g_1g_2)\cdot a,
  \end{equation*}
  where the second-to-last equality follows from the associativity of
  multiplication in $F$. Also, for any $a\in F^\times$,
  \begin{equation*}
    1\cdot a = 1a = a,
  \end{equation*}
  since $1$ is the identity of the group $F^\times$. And $1(0) = 0$
  (which follows from distributivity), so we can say that
  $1\cdot a = a$ for all $a\in F$. Therefore the mapping
  $(g, a)\mapsto ga$ of $F^\times\times F\to F$ is a group action.
\end{proof}

\Exercise2 Show that the additive group $\Z$ acts on itself by
$z\cdot a = z + a$ for all $z,a\in\Z$.
\begin{proof}
  For all $z_1,z_2,a\in\Z$, we have
  \begin{equation*}
    z_1\cdot(z_2\cdot a) = z_1 + (z_2 + a)
    = (z_1 + z_2) + a = (z_1 + z_2)\cdot a
  \end{equation*}
  and
  \begin{equation*}
    0\cdot a = 0 + a = a.
  \end{equation*}
  Therefore $\Z$ acts on itself as stated.
\end{proof}

\Exercise3 Show that the additive group $\R$ acts on the $x,y$ plane
$\R\times\R$ by $r\cdot(x,y) = (x + ry, y)$.
\begin{proof}
  For any $r_1,r_2\in\R$ and any $(x,y)\in\R^2$, we have
  \begin{align*}
    r_1\cdot(r_2\cdot(x,y))
    &= r_1\cdot(x + r_2y, y) \\
    &= (x + r_2y + r_1y, y) \\
    &= (x + (r_1 + r_2)y, y) \\
    &= (r_1 + r_2)\cdot(x,y)
  \end{align*}
  and
  \begin{equation*}
    0\cdot(x,y) = (x + 0y, y) = (x,y).
  \end{equation*}
  Therefore $\R$ acts on $\R^2$ in the manner stated above.
\end{proof}

\Exercise4 Let $G$ be a group acting on a set $A$ and fix some
$a\in A$. Show that the following sets are subgroups of $G$:
\begin{enumerate}
\item the kernel of the action
  \begin{proof}
    Suppose $g,h$ are in the kernel of the action. Then for any $b\in A$,
    \begin{equation*}
      (gh)\cdot b = g\cdot(h\cdot b) = g\cdot b = b,
    \end{equation*}
    so $gh$ is in the kernel, and the kernel is closed under the group
    operation. Moreover, if $g$ is in the kernel then
    \begin{equation*}
      b = 1\cdot b = (g^{-1}g)\cdot b
      = g^{-1}\cdot(g\cdot b) = g^{-1}\cdot b,
    \end{equation*}
    so $g^{-1}$ is in the kernel.

    Therefore the kernel of the group action is a nonempty subset of
    $G$ which is closed under the binary operation of $G$ and which is
    closed under inverses, so the kernel is a subgroup of $G$.
  \end{proof}
\item $\{g\in G\mid ga = a\}$ (called the {\em stabilizer} of $a$ in $G$)
  \begin{proof}
    The stabilizer of $a$ is nonempty since $1$ is a member. Now let
    $g,h$ be any members of the stabilizer. Then
    \begin{equation*}
      (gh)\cdot a = g\cdot(h\cdot a) = g\cdot a = a,
    \end{equation*}
    so the stabilizer is closed under the group operation. It is also
    closed under inverses, since
    \begin{equation*}
      a = 1\cdot a = (g^{-1}g)\cdot a
      = g^{-1}\cdot(g\cdot a) = g^{-1}\cdot a.
    \end{equation*}
    Therefore the stabilizer is a subgroup of $G$.
  \end{proof}
\end{enumerate}

\Exercise5 Prove that the kernel of an action of the group $G$ on the
set $A$ is the same as the kernel of the corresponding permutation
representation $G\to S_A$.
\begin{proof}
  Let $\varphi\colon G\to S_A$ be the permutation representation of
  the group action on $A$, so that for $g\in G$ and $a\in A$,
  $\varphi(g)(a) = g\cdot a$.

  If $g\in\ker\varphi$, then $\varphi(g) = 1$, where $1$ is the
  identity permutation on $A$. Then $g.a = a$ for all $a\in A$, and
  $g$ is in the kernel of the action. Conversely, if $g$ is in the
  kernel of the action, then $g.a = a$ for all $a\in A$, so that
  $\varphi(g) = 1$ and $g\in\ker\varphi$. Therefore the kernel of the
  group action and the kernel of the corresponding permutation
  representation are the same.
\end{proof}

\Exercise6 Prove that a group $G$ acts faithfully on a set $A$ if and
only if the kernel of the action is the set consisting only of the
identity.
\begin{proof}
  First, suppose that $G$ acts faithfully on $A$ and let $g$ be an
  element in the kernel of the action. Then $g\cdot a = a$ for all
  $a\in A$. However, $1\cdot a = a$ for all $a\in A$, so the elements
  $1$ and $g$ induce the same permutation on $A$. Since $G$ acts
  faithfully, this must mean that $g = 1$, so that the kernel of the
  action is the set $\{1\}$.

  For the converse, suppose that the kernel of the action is the set
  $\{1\}$. Pick two elements $g$ and $h$ in $G$ and suppose that $g$
  and $h$ induce the same permutation on $A$. Then for any $a\in A$,
  $g\cdot a = h\cdot a$. But then
  \begin{equation*}
    a = (g^{-1}g)\cdot a = g^{-1}\cdot(g\cdot a)
    = g^{-1}\cdot(h\cdot a) = (g^{-1}h)\cdot a.
  \end{equation*}
  Therefore $g^{-1}h$ is in the kernel of the action, so
  $g^{-1}h = 1$. This implies that $g = h$, so that distinct elements
  in $G$ must induce distinct permutations on $A$. This shows that $G$
  acts faithfully on $A$.
\end{proof}

\Exercise7 Prove that in Example~2 in this section the action is
faithful.
\begin{proof}
  If $V$ is a vector space over a field $F$, then the multiplicative
  group $F^\times$ acts on the set $V$ via the mapping $a\cdot v = av$
  for $a\in F^\times$ and $v\in V$. We want to show that this action
  is faithful.

  Let $a,b\in F^\times$ be such that $a\cdot v = b\cdot v$ for all
  $v\in V$. Then
  \begin{align*}
    0 &= a\cdot v + -(a\cdot v) \\
      &= a\cdot v + -(b\cdot v) \\
      &= av - bv \\
      &= (a - b)v.
  \end{align*}
  Since $(a - b)v$ is $0$ even when $v$ is nonzero, this implies that
  $a - b = 0$ or $a = b$. Therefore distinct elements in $F^\times$
  must induce distinct permutations on $V$ and the action is faithful.
\end{proof}

\Exercise8 Let $A$ be a nonempty set and let $k$ be a positive integer
with $k\leq\ord{A}$. The symmetric group $S_A$ acts on the set $B$
consisting of all subsets of $A$ of cardinality $k$ by
$\sigma\cdot\{a_1,\dots,a_k\} = \{\sigma(a_1),\dots,\sigma(a_k)\}$.
\begin{enumerate}
\item Prove that this is a group action.
  \begin{proof}
    Suppose $\sigma_1,\sigma_2\in S_A$. Then for any subset
    $\{a_1,\dots,a_k\}$ of $A$,
    \begin{align*}
      \sigma_1\cdot(\sigma_2\cdot\{a_1,\dots,a_k\})
      &= \sigma_1\cdot\{\sigma_2(a_1),\dots,\sigma_2(a_k)\} \\
      &= \{\sigma_1(\sigma_2(a_1)),\dots,\sigma_1(\sigma_2(a_k))\} \\
      &= (\sigma_1\circ\sigma_2)\cdot\{a_1,\dots,a_k\}
    \end{align*}
    and
    \begin{equation*}
      1\cdot\{a_1,\dots,a_k\}
      = \{1(a_1),\dots, 1(a_k)\}
      = \{a_1,\dots,a_k\}.
    \end{equation*}
    Therefore the specified mapping is a group action.
  \end{proof}
\item Describe explicitly how the elements $(1\,2)$ and $(1\,2\,3)$
  act on the six $2$-element subsets of $\{1,2,3,4\}$.
  \begin{solution}
    We have
    \begin{align*}
      (1\,2)\cdot\{1,2\} &= \{2,1\}, \\
      (1\,2)\cdot\{1,3\} &= \{2,3\}, \\
      (1\,2)\cdot\{1,4\} &= \{2,4\}, \\
      (1\,2)\cdot\{2,3\} &= \{1,3\}, \\
      (1\,2)\cdot\{2,4\} &= \{1,4\}, \\
      (1\,2)\cdot\{3,4\} &= \{3,4\},
    \end{align*}
    and
    \begin{align*}
      (1\,2\,3)\cdot\{1,2\} &= \{2,3\}, \\
      (1\,2\,3)\cdot\{1,3\} &= \{2,1\}, \\
      (1\,2\,3)\cdot\{1,4\} &= \{2,4\}, \\
      (1\,2\,3)\cdot\{2,3\} &= \{3,1\}, \\
      (1\,2\,3)\cdot\{2,4\} &= \{3,4\}, \\
      (1\,2\,3)\cdot\{3,4\} &= \{1,4\}. \qedhere
    \end{align*}
  \end{solution}
\end{enumerate}

\Exercise9 Do both parts of the preceding exercise with ``ordered
$k$-tuples'' in place of ``$k$-element subsets,'' where the action on
$k$-tuples is defined as above but with set braces replaced by
parentheses.
\begin{solution}
  The work is essentially the same, but with $k$-tuples replacing the
  $k$-element subsets, so we omit it. Note that in part (b) there are
  twice as many different $2$-tuples as there are $2$-element subsets,
  since the ordering of the elements is significant.
\end{solution}

\Exercise{10} With reference to the preceding two exercises determine:
\begin{enumerate}
\item for which values of $k$ the action of $S_n$ on $k$-element
  subsets is faithful
  \begin{solution}
    The action of $S_A$ on $k$-element subsets of a set $A$ is
    faithful for all integers $k$ with $1\leq k<\ord{A}$, which we
    will now show. Suppose $\sigma_1$ and $\sigma_2$ are distinct
    permutations in $S_A$. Label the elements of $A$ as
    $\{a_1,a_2,\dots,a_n\}$, where $n = \ord{A}$. Without loss of
    generality, we may suppose that $\sigma_1(a_1)\neq\sigma_2(a_1)$
    (if not, relabel the elements of $A$ so that this is true).

    Now, take any $k$-element subset $B$ of $A$ which contains $a_1$
    but which does not contain $(\sigma_1^{-1}\circ\sigma_2)(a_1)$
    (this is possible since $1\leq k<\ord{A}$). Then $\sigma_1\cdot B$
    does not contain $\sigma_2(a_1)$, however $\sigma_2\cdot B$
    does. Therefore distinct permutations in $S_A$ induce distinct
    permutations on the $k$-element subsets of $A$, so the action is
    faithful (again, assuming $1\leq k<\ord{A}$).
  \end{solution}

\item for which values of $k$ the action of $S_n$ on ordered
  $k$-tuples is faithful
  \begin{solution}
    The action of $S_A$ on ordered $k$-tuples of elements of $A$ is
    faithful for all integers $k$ with $1\leq k\leq\ord{A}$. To see
    this, suppose that $\sigma_1,\sigma_2$ are distinct permutations
    in $S_A$. Suppose for example that
    $\sigma_1(a_1)\neq\sigma_2(a_1)$ and consider the $k$-tuple
    $B = (a_1, a_2, \dots, a_k)$. Then the first coordinate in
    $\sigma_1\cdot B$ is distinct from the first coordinate of
    $\sigma_2\cdot B$. Therefore distinct permutations in $S_A$ induce
    distinct permutations on the set of $k$-tuples, so the action is
    faithful.
  \end{solution}
\end{enumerate}
