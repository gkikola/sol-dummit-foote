\section{Group Actions}

\Exercise1 Let $F$ be a field. Show that the multiplicative group of
nonzero elements of $F$ (denoted by $F^\times$) acts on the set $F$ by
$g\cdot a = ga$, where $g\in F^\times$, $a\in F$ and $ga$ is the usual
product in $F$ of the two field elements.
\begin{proof}
  Let $g_1,g_2\in F^\times$. Then for any $a\in F$,
  \begin{equation*}
    g_1\cdot(g_2\cdot a) = g_1\cdot g_2a = g_1(g_2a) = (g_1g_2)a
    = (g_1g_2)\cdot a,
  \end{equation*}
  where the second-to-last equality follows from the associativity of
  multiplication in $F$. Also, for any $a\in F^\times$,
  \begin{equation*}
    1\cdot a = 1a = a,
  \end{equation*}
  since $1$ is the identity of the group $F^\times$. And $1(0) = 0$
  (which follows from distributivity), so we can say that
  $1\cdot a = a$ for all $a\in F$. Therefore the mapping
  $(g, a)\mapsto ga$ of $F^\times\times F\to F$ is a group action.
\end{proof}

\Exercise2 Show that the additive group $\Z$ acts on itself by
$z\cdot a = z + a$ for all $z,a\in\Z$.
\begin{proof}
  For all $z_1,z_2,a\in\Z$, we have
  \begin{equation*}
    z_1\cdot(z_2\cdot a) = z_1 + (z_2 + a)
    = (z_1 + z_2) + a = (z_1 + z_2)\cdot a
  \end{equation*}
  and
  \begin{equation*}
    0\cdot a = 0 + a = a.
  \end{equation*}
  Therefore $\Z$ acts on itself as stated.
\end{proof}

\Exercise3 Show that the additive group $\R$ acts on the $x,y$ plane
$\R\times\R$ by $r\cdot(x,y) = (x + ry, y)$.
\begin{proof}
  For any $r_1,r_2\in\R$ and any $(x,y)\in\R^2$, we have
  \begin{align*}
    r_1\cdot(r_2\cdot(x,y))
    &= r_1\cdot(x + r_2y, y) \\
    &= (x + r_2y + r_1y, y) \\
    &= (x + (r_1 + r_2)y, y) \\
    &= (r_1 + r_2)\cdot(x,y)
  \end{align*}
  and
  \begin{equation*}
    0\cdot(x,y) = (x + 0y, y) = (x,y).
  \end{equation*}
  Therefore $\R$ acts on $\R^2$ in the manner stated above.
\end{proof}

\Exercise4 Let $G$ be a group acting on a set $A$ and fix some
$a\in A$. Show that the following sets are subgroups of $G$:
\begin{enumerate}
\item the kernel of the action
  \begin{proof}
    Suppose $g,h$ are in the kernel of the action. Then for any $b\in A$,
    \begin{equation*}
      (gh)\cdot b = g\cdot(h\cdot b) = g\cdot b = b,
    \end{equation*}
    so $gh$ is in the kernel, and the kernel is closed under the group
    operation. Moreover, if $g$ is in the kernel then
    \begin{equation*}
      b = 1\cdot b = (g^{-1}g)\cdot b
      = g^{-1}\cdot(g\cdot b) = g^{-1}\cdot b,
    \end{equation*}
    so $g^{-1}$ is in the kernel.

    Therefore the kernel of the group action is a nonempty subset of
    $G$ which is closed under the binary operation of $G$ and which is
    closed under inverses, so the kernel is a subgroup of $G$.
  \end{proof}
\item $\{g\in G\mid ga = a\}$ (called the {\em stabilizer} of $a$ in $G$)
  \begin{proof}
    The stabilizer of $a$ is nonempty since $1$ is a member. Now let
    $g,h$ be any members of the stabilizer. Then
    \begin{equation*}
      (gh)\cdot a = g\cdot(h\cdot a) = g\cdot a = a,
    \end{equation*}
    so the stabilizer is closed under the group operation. It is also
    closed under inverses, since
    \begin{equation*}
      a = 1\cdot a = (g^{-1}g)\cdot a
      = g^{-1}\cdot(g\cdot a) = g^{-1}\cdot a.
    \end{equation*}
    Therefore the stabilizer is a subgroup of $G$.
  \end{proof}
\end{enumerate}
