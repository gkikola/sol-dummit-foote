\chapter{Introduction to Groups}

\section{Basic Axioms and Examples}

\Exercise1 Determine which of the following binary operations are
associative:
\begin{enumerate}
\item the operation $\star$ on $\Z$ defined by $a\star b = a - b$
  \begin{solution}
    $(1\star2)\star3 = -4$ while $1\star(2\star3) = 2$, so $\star$ is
    not associative.
  \end{solution}
\item the operation $\star$ on $\R$ defined by $a\star b = a + b + ab$
  \begin{solution}
    $\star$ is associative: let $a,b,c$ be real numbers. Then
    \begin{align*}
      (a\star b)\star c &= (a + b + ab)\star c \\
                        &= (a + b + ab) + c + (a + b + ab)c \\
                        &= a + b + c + ab + ac + bc + abc \\
                        &= a + (b + c + bc) + a(b + c + bc) \\
                        &= a\star(b + c + bc) \\
                        &= a\star(b\star c). \qedhere
    \end{align*}
  \end{solution}
\item the operation $\star$ on $\Q$ defined by $a\star b = (a + b)/5$
  \begin{solution}
    $(5\star20)\star15 = 4$ while $5\star(20\star15) =
    12/5$. Therefore $\star$ is not associative.
  \end{solution}
\item the operation $\star$ on $\Z\times\Z$ defined by
  $(a,b)\star(c,d) = (ad + bc, bd)$
  \begin{solution}
    $\star$ is associative: let $(a,b), (c,d), (e,f)$ be members of
    $\Z\times\Z$. Then
    \begin{align*}
      \big((a,b)\star(c,d)\big)\star(e,f)
      &= (ad + bc, bd)\star(e,f) \\
      &= \big((ad+bc)f+bde, bdf\big) \\
      &= (adf+bcf+bde, bdf) \\
      &= \big(adf+b(cf+de), bdf\big) \\
      &= (a,b)\star(cf+de,df) \\
      &= (a,b)\star\big((c,d)\star(e,f)\big). \qedhere
    \end{align*}
  \end{solution}
\item the operation $\star$ on $\Q - \{0\}$ defined by
  $a\star b = a/b$
  \begin{solution}
    $(125\star25)\star5 = 1$ while $125\star(25\star5) = 25$, so
    $\star$ is not associative.
  \end{solution}
\end{enumerate}

\Exercise2 Decide which of the binary operations in the preceding
exercises are commutative.
\begin{enumerate}
\item the operation $\star$ on $\Z$ defined by $a\star b = a - b$
  \begin{solution}
    $\star$ is not commutative since, for example, $1\star2 = -1$
    while $2\star1 = 1$.
  \end{solution}
\item the operation $\star$ on $\R$ defined by $a\star b = a + b + ab$
  \begin{solution}
    $\star$ is commutative since, for any $a,b\in\R$,
    \begin{align*}
      a\star b &= a + b + ab \\
               &= b + a + ba \\
               &= b\star a. \qedhere
    \end{align*}
  \end{solution}
\item the operation $\star$ on $\Q$ defined by $a\star b = (a + b)/5$
  \begin{solution}
    $\star$ is commutative since $+$ is commutative in $\Q$.
  \end{solution}
\item the operation $\star$ on $\Z\times\Z$ defined by
  $(a,b)\star(c,d) = (ad + bc, bd)$
  \begin{solution}
    $\star$ is commutative: Let $(a,b)$ and $(c,d)$ be elements of
    $\Z\times\Z$. Then
    \begin{align*}
      (a,b)\star(c,d) &= (ad + bc, bd) \\
                      &= (cb + da, db) \\
                      &= (c,d)\star(a,b). \qedhere
    \end{align*}
  \end{solution}
\item the operation $\star$ on $\Q - \{0\}$ defined by
  $a\star b = a/b$
  \begin{solution}
    $\star$ is not commutative since $1\star2 = 1/2$ but
    $2\star1 = 2$.
  \end{solution}
\end{enumerate}

\Exercise3 Prove that addition of residue classes in $\Z/n\Z$ is
associative (you may assume it is well defined).
\begin{proof}
  Let $\bar{a}, \bar{b}, \bar{c}$ be residue classes in $\Z/n\Z$. Then
  by Theorem~3 in Section~0.3 along with the associativity of $+$ in
  $\Z$, we may write
  \begin{align*}
    (\bar{a} + \bar{b}) + \bar{c}
    &= \overline{(a + b) + c} \\
    &= \overline{a + (b + c)} \\
    &= \bar{a} + (\bar{b} + \bar{c}).
  \end{align*}
  So addition of residue classes is associative.
\end{proof}

\Exercise4 Prove that multiplication of residue classes in $\Z/n\Z$ is
associative (you may assume it is well defined).
\begin{proof}
  As in the previous exercise, this follows from Theorem~3 in
  Section~0.3 together with the associativity of $\cdot$ in $\Z$.
\end{proof}

\Exercise5 Prove for all $n>1$ that $\Z/n\Z$ is not a group under
multiplication of residue classes.
\begin{proof}
  Let $n>1$. Then there is a residue class in $\Z/n\Z$ which does not
  contain $0$. Call this nonzero residue class $\bar{a}$. Then $\bar0$
  cannot be the identity element in $\Z/n\Z$ since
  $\bar{a}\cdot\bar0 = \bar0 \neq \bar{a}$. So suppose the identity
  element is $\bar{e}$. Then, $\bar0$ also has no inverse in $\Z/n\Z$,
  since $\bar{b}\cdot\bar0 = \bar0 \neq \bar{e}$ for any $\bar{b}$ in
  $\Z/n\Z$. Since the element $\bar0$ does not have an inverse,
  $\Z/n\Z$ is not a group under multiplication.
\end{proof}

\Exercise6 Determine which of the following sets are groups under
addition:
\begin{enumerate}
\item the set of rational numbers (including $0 = 0/1$) in lowest
  terms whose denominators are odd
  \begin{solution}
    Let the set be denoted $A$. Then $A$ is a group having identity
    $0$ and, for each $a\in A$, an inverse $-a$. To prove this, we
    need only show that $A$ is closed under addition.

    Suppose $a$ and $b$ are any elements in $A$. Then we can find
    integers $p,q,r,s$ with
    \begin{equation*}
      a = \frac{p}{q}\quad\text{and}\quad
      b = \frac{r}{s}
    \end{equation*}
    in lowest terms with $q,s$ odd. Then we have
    \begin{align*}
      a + b = \frac{ps+rq}{qs} = \frac{u}{v},
    \end{align*}
    where $u$ and $v$ are integers with $u/v$ in lowest terms. Now,
    since $u/v$ was obtained by eliminating common factors, we have
    $u\mid(ps+rq)$ and $v\mid qs$. But if $2\mid v$, then necessarily
    $2\mid qs$. But this cannot be, since $qs$ is odd, being the
    product of odd integers. Hence $A$ is closed under addition and is
    therefore a group.
  \end{solution}
\item the set of rational numbers in lowest terms whose denominators
  are even, together with $0$
  \begin{solution}
    Let $A$ denote the set. Then $A$ is not a group since $3/2\in A$
    but
    \begin{equation*}
      \frac32 + \frac32 = \frac62 = \frac31\not\in A. \qedhere
    \end{equation*}
  \end{solution}
\item the set of rational numbers of absolute value $<1$
  \begin{solution}
    Again, this set is not closed under addition since, for example,
    \begin{equation*}
      \frac34 + \frac34 > 1.
    \end{equation*}
    Therefore it is not a group.
  \end{solution}
\item the set of rational numbers of absolute value $\geq1$ together
  with $0$
  \begin{solution}
    This set is not closed under addition since, for example,
    \begin{equation*}
      \frac{12}5 - \frac85 = \frac45 \not\geq1.
    \end{equation*}
    Therefore it is not a group.
  \end{solution}
\item the set of rational numbers with denominators equal to $1$ or
  $2$
  \begin{solution}
    Denote the set by $A$. Let $a$ and $b$ be arbitrary integers. Then
    $a,b,a/2,b/2\in A$. There are several cases. First,
    \begin{equation*}
      \frac{a}1 + \frac{b}1 = \frac{a + b}1 \in A.
    \end{equation*}
    Now consider
    \begin{equation*}
      \frac{a}2 + \frac{b}2 = \frac{a + b}2.
    \end{equation*}
    If this fraction is in lowest terms, then it is in $A$. If not,
    then there must be a common factor of $2$ and the fraction can be
    written with a denominator of $1$ and thus is in $A$.

    Finally, consider
    \begin{equation*}
      \frac{a}1 + \frac{b}2 = \frac{b}2 + \frac{a}1 = \frac{2a + b}2.
    \end{equation*}
    As before, this is either in lowest terms, or can be reduced to
    lowest terms by dividing the numerator and denominator by $2$. In
    either case, this number is in $A$.

    Since $A$ is closed under addition, it is easily seen to be a
    group: the identity is $1 = 1/1$ and the inverse of $a/b\in A$ is
    $-a/b$.
  \end{solution}
\item the set of rational numbers with denominators equal to $1$, $2$,
  or $3$
  \begin{solution}
    This set is not closed under addition, since
    \begin{equation*}
      \frac12 + \frac13 = \frac56.
    \end{equation*}
    Hence this is not a group.
  \end{solution}
\end{enumerate}

\Exercise7 Let $G = \{x\in\R\mid 0\leq x<1\}$ and for $x,y\in G$ let
$x\star y$ be the fractional part of $x + y$ (i.e.,
$x\star y = x + y - [x + y]$ where $[a]$ is the greatest integer less
than or equal to $a$). Prove that $\star$ is a well defined binary
operation on $G$ and that $G$ is an abelian group under $\star$
(called the {\em real numbers mod $1$}).
\begin{proof}
  Let $x,y\in G$ be arbitrary. Then $0\leq x + y<2$. There are two
  cases: if $x + y < 1$ then $[x + y] = 0$ and $x\star y\in G$. On the
  other hand, if $1\leq x + y < 2$ then $[x + y] = 1$ and
  $x\star y = x + y - 1\in G$. Therefore $\star$ is a well defined
  binary operation on $G$.

  Let $x,y,z\in G$. If $x + y < 1$ and $y + z < 1$, then
  \begin{align*}
    (x\star y)\star z
    &= (x + y - 0)\star z \\
    &= x + y + z - [x + y + z] \\
    &= x\star(y + z - 0) \\
    &= x\star(y\star z).
  \end{align*}
  On the other hand, if $1\leq x + y < 2$ and $1\leq y + z < 2$, then
  \begin{align*}
    (x\star y)\star z
    &= (x + y - 1)\star z \\
    &= x + y + z - 1 - [x + y + z - 1] \\
    &= x\star(y + z - 1) \\
    &= x\star(y\star z).
  \end{align*}
  Finally, if $1\leq x + y < 2$ and $0\leq y + z < 1$, then
  $[x + y] = 1$, $[y + z] = 0$, and
  $[x + y + z - 1] = [x + y + z] - 1$, so
  \begin{align*}
    (x\star y)\star z
    &= (x + y - 1)\star z \\
    &= x + y + z - 1 - [x + y + z - 1] \\
    &= x + y + z - 1 - [x + y + z] + 1 \\
    &= x + y + z - [x + y + z] \\
    &= x\star(y + z - 0) \\
    &= x\star(y\star z).
  \end{align*}
  And the case where $x+y < 1$ and $y + z\geq1$ is similar. Therefore,
  $\star$ is associative.

  Since $0\in G$, $G$ has an identity ($x\star 0 = 0\star x = x$ for
  each $x$ in $G$). And every element has an inverse: the inverse of
  $0$ is $0$, and for nonzero $x\in G$, $1 - x\in G$ is an inverse
  since
  \begin{equation*}
    x\star(1-x) = x + (1 - x) - [x + (1 - x)]
    = 1 - 1 = 0.
  \end{equation*}
  Therefore $G$ is a group under $\star$.
\end{proof}

\Exercise8 Let $G = \{z\in\C \mid \text{$z^n = 1$ for some
  $n\in\Z^+$}\}$.
\begin{enumerate}
\item Prove that $G$ is a group under multiplication (called the group
  of {\em roots of unity} in $\C$).
  \begin{proof}
    Let $z,w\in G$ so that $z^n = 1$ and $w^m = 1$.

    Note that $1\in G$ since $1^1 = 1$, so $G$ has an identity. And
    every element of $G$ is nonzero, so for each $z\in G$ we may let
    $z^{-1} = 1/z$ so that every element in $G$ has an inverse (since
    $(1/z)^n = 1/z^n = 1$ so $1/z\in G$).

    By the commutativity of multiplication in $\C$, we have
    \begin{equation*}
      (zw)^{nm} = z^{nm}w^{mn} = (z^n)^m(w^m)^n = 1^m1^n = 1
    \end{equation*}
    for each $m,n\in\Z^+$. Therefore, $G$ is closed under
    multiplication. And associativity follows from associativity of
    multiplication in $\C$.

    Therefore $G$ is a group.
  \end{proof}
\item Prove that $G$ is not a group under addition.
  \begin{proof}
    $G$ is a not a group under addition since it is not closed:
    $1\in G$ but $1+1=2\not\in G$ since there is no $n\in\Z^+$ with
    $2^n = 1$.
  \end{proof}
\end{enumerate}

\Exercise9 Let $G = \{a + b\sqrt2\in\R \mid a,b\in\Q\}$.
\begin{enumerate}
\item Prove that $G$ is a group under addition.
  \begin{proof}
    Associativity of $+$ in $G$ follows from associativity of $+$ in
    $\R$. $G$ has an identity $0 = 0 + 0\sqrt2$ and for every $p$ in
    $G$ we may take $q = -p$ as its additive inverse. So we need only
    show that $G$ is closed under addition.

    Let $p = a + b\sqrt2$ and $q = c + d\sqrt2$ with
    $a,b,c,d\in\Q$. Then
    \begin{equation*}
      p+q = a + c + (b + d)\sqrt2,
      \quad\text{where $a+c\in\Q$ and $b+d\in\Q$},
    \end{equation*}
    so $p+q\in G$ and $G$ is a group.
  \end{proof}
\item Prove that the nonzero elements of $G$ are a group under
  multiplication.
  \begin{proof}
    Again, associativity follows from associativity in $\R$. This time
    the identity is $1 = 1 + 0\sqrt2$. And for any rational numbers
    $a$ and $b$ not both $0$, $a+b\sqrt2\in G-\{0\}$ and
    \begin{align*}
      \frac1{a + b\sqrt2}
      &= \frac{a - b\sqrt2}{(a + b\sqrt2)(a - b\sqrt2)} \\
      &= \frac{a - b\sqrt2}{a^2 - 2b^2} \\
      &= \frac{a}{a^2 - 2b^2} - \left(\frac{b}{a^2 - 2b^2}\right)\!\sqrt2
        \in G-\{0\},
    \end{align*}
    so every element in $G-\{0\}$ has an inverse (note that the
    denominator $a^2 - 2b^2$ is nonzero since $a,b\in\Q$ and there is
    no rational square root of $2$).

    The set is also closed under multiplication since for any
    $a,b,c,d\in\Q$ with $a,b$ not both $0$ and $c,d$ not both $0$,
    \begin{equation*}
      \left(a + b\sqrt2\right)\left(c + d\sqrt2\right)
      = ac + 2bd + (ad + bc)\sqrt2 \in G-\{0\}.
    \end{equation*}
    This shows that $G-\{0\}$ is a group under multiplication.
  \end{proof}
\end{enumerate}

\Exercise{10} Prove that a finite group is abelian if and only if its
group table is a symmetric matrix.
\begin{proof}
  List the elements of the group in a fixed order along the top row
  and first column of the group table. Then the group is abelian if
  and only if the $i,j$th entry in its group table is equal to the
  $j,i$th entry, which is true if and only if the table forms a
  symmetric matrix.
\end{proof}

\Exercise{11} Find the orders of each element of the additive group
$\Z/12\Z$.
\begin{solution}
  $\bar0$ has order $1$. $\bar1$ has order 12 since $1\cdot\bar1$,
  $2\cdot\bar1$, $\ldots$, $11\cdot\bar1$ are nonzero while
  $12\cdot\bar1 = \bar0$. Similarly, we find the following orders for
  the elements:

  \begin{center}
    \begin{tabular}{c|c|c|c|c|c|c|c|c|c|c|c|c}
      $x$ & $\bar0$ & $\bar1$ & $\bar2$ & $\bar3$ & $\bar4$ & $\bar5$ & $\bar6$
      & $\bar7$ & $\bar8$ & $\bar9$ & $\overline{10}$ & $\overline{11}$ \\\hline
      $\abs{x}$ & 1 & 12 & 6 & 4 & 3 & 12 & 2 & 12 & 3 & 4 & 6 & 12
    \end{tabular}
  \end{center}
\end{solution}

\Exercise{12} Find the orders of the following elements of the
multiplicative group $(\Z/12\Z)^\times$:
$\bar1, \overline{-1}, \bar5, \bar7, \overline{-7}, \overline{13}$.
\begin{solution}
  We get the following table:
  \begin{center}
    \begin{tabular}{c|c|c|c|c|c|c}
      $x$ & $\bar1$ & $\overline{-1}$ & $\bar5$ & $\bar7$ & $\overline{-7}$
      & $\overline{13}$ \\\hline
      $\abs{x}$ & 1 & 2 & 2 & 2 & 2 & 1
    \end{tabular}
  \end{center}
\end{solution}

\Exercise{13} Find the orders of the following elements of the
additive group $\Z/36\Z$:
$\bar1, \bar2, \bar6, \bar9, \overline{10}, \overline{12},
\overline{-1}, \overline{-10}, \overline{-18}$.
\begin{solution}
  We get the following table:

  \begin{center}
    \begin{tabular}{c|c|c|c|c|c|c|c|c|c}
      $x$ & $\bar1$ & $\bar2$ & $\bar6$ & $\bar9$ & $\overline{10}$
      & $\overline{12}$ & $\overline{-1}$ & $\overline{-10}$
      & $\overline{-18}$ \\\hline
      $\abs{x}$ & 36 & 18 & 6 & 4 & 18 & 3 & 36 & 18 & 2
    \end{tabular}
  \end{center}
\end{solution}

\Exercise{14} Find the orders of the following elements of the
multiplicative group $(\Z/36\Z)^\times$:
$\bar1, \overline{-1}, \bar5, \overline{13}, \overline{-13},
\overline{17}$.
\begin{solution}
  We have the following table:
  \begin{center}
    \begin{tabular}{c|c|c|c|c|c|c}
      $x$ & $\bar1$ & $\overline{-1}$ & $\bar5$ & $\overline{13}$
      & $\overline{-13}$ & $\overline{17}$ \\\hline
      $\abs{x}$ & 1 & 2 & 6 & 3 & 6 & 2
    \end{tabular}
  \end{center}
\end{solution}

\Exercise{15} Let $G$ be a group. Prove that
$(a_1a_2\dots a_n)^{-1} = a_n^{-1}a_{n-1}^{-1}\dots a_1^{-1}$ for all
$a_1,a_2,\dots,a_n\in G$.
\begin{proof}
  We use induction on $n$. If $n = 1$, the result is obvious. Suppose
  it holds for $n = k$, where $k\geq1$. Then for any
  $a_1,\dots,a_{k+1}$ in $G$, we have
  \begin{align*}
    (a_1\dots a_ka_{k+1})(a_{k+1}^{-1}a_k^{-1}\dots a_1^{-1})
    &= (a_1\dots a_k)(a_{k+1}a_{k+1}^{-1})(a_k^{-1}\dots a_1^{-1}) \\
    &= (a_1\dots a_k)(a_k^{-1}\dots a_1^{-1}),
  \end{align*}
  and this is equal to $1$ by the induction hypothesis. Therefore
  $(a_1\dots a_{k+1})^{-1} = a_{k+1}^{-1}\dots a_1^{-1}$ and the
  statement holds for all positive integers $n$.
\end{proof}

\Exercise{16} Let $x$ be an element of a group $G$. Prove that
$x^2 = 1$ if and only if $\abs{x}$ is either $1$ or $2$.
\begin{proof}
  First, if $\abs{x} = 1$ then $x = 1$ so $x^2 = 1^2 = 1$. If
  $\abs{x} = 2$, then $x^2 = 1$ by definition.

  For the other direction, suppose $x^2 = 1$. Then $\abs{x}\leq2$. But
  the order of an element must be at least $1$, so $\abs{x} = 1$ or
  $\abs{x} = 2$.
\end{proof}

\Exercise{17} Let $x$ be an element of a group $G$. Prove that if
$\abs{x} = n$ for some positive integer $n$ then $x^{-1} = x^{n-1}$.
\begin{proof}
  Since $\abs{x} = n$, we have $x^n = 1$. But
  $x^n = x^{n-1}x = xx^{n-1}$, so $x^{n-1}x = 1$ which shows that
  $x^{-1} = x^{n-1}$.
\end{proof}

\Exercise{18} Let $x$ and $y$ be elements of a group $G$. Prove that
$xy = yx$ if and only if $y^{-1}xy = x$ if and only if
$x^{-1}y^{-1}xy = 1$.
\begin{proof}
  If $xy = yx$, then $y^{-1}xy = y^{-1}yx = 1x = x$. Multiplying by
  $x^{-1}$ then gives $x^{-1}y^{-1}xy = 1$.

  On the other hand, if $x^{-1}y^{-1}xy = 1$, then we may multiply on
  the left by $x$ to get $y^{-1}xy = x$. Then multiplying on the left
  by $y$ gives $xy = yx$ as desired.
\end{proof}

\Exercise{19} Let $x\in G$ for $G$ a group and let $a,b\in\Z^+$.
\begin{enumerate}
\item Prove that $x^{a+b} = x^ax^b$ and $(x^a)^b = x^{ab}$.
  \label{exponent-rule}
  \begin{proof}
    $x^ax^b$ consists of a $a$ factors of $x$, multiplied by $b$
    factors of $x$, for a total of $a+b$ factors of $x$. Therefore
    $x^{a+b} = x^ax^b$ by definition. Similarly, $(x^a)^b = x^{ab}$ by
    the same reasoning.
  \end{proof}
\item Prove that $(x^a)^{-1} = x^{-a}$.
  \begin{proof}
    Since $x^{-a} = (x^{-1})^a$, we need to show that
    $(x^a)^{-1} = (x^{-1})^a$. We use induction on $a$. For $a = 1$,
    the result is trivial. Suppose it holds for $a = k$,
    $k\geq0$. Then
    \begin{equation*}
      (x^{k+1})(x^{-1})^{k+1}
      = x^k(xx^{-1})(x^{-1})^k
      = x^k(x^{-1})^k,
    \end{equation*}
    which by the induction hypothesis must be $1$. Therefore the
    result holds for all positive integers $a$.
  \end{proof}
\item Establish part \ref{exponent-rule} for arbitrary integers $a$
  and $b$ (positive, negative, or zero).
  \begin{proof}
    For any integer $a$, $x^ax^0 = x^a = x^{a+0}$ and similarly
    $x^0x^a = x^{0+a}$.

    Now suppose $a>0,b<0$. If $a + b > 0$, then
    $x^{a+b}x^{-b} = x^{(a+b)+(-b)} = x^a$ by part
    \ref{exponent-rule}. Multiplying both sides of this equation on
    the right by $x^b$ gives $x^{a+b} = x^ax^b$ as desired. On the
    other hand, if $a + b < 0$, then
    $x^{-(a+b)}x^a = x^{-(a+b)+a} = x^{-b}$. Multiplying both sides of
    this equation on the right by $x^{-a}$ gives
    $x^{-(a+b)} = x^{-b}x^{-a}$, so
    \begin{equation*}
      (x^{a+b})^{-1}
      = x^{-(a+b)} = x^{-b}x^{-a} = (x^b)^{-1}(x^a)^{-1}
      = (x^ax^b)^{-1}.
    \end{equation*}
    The last equality follows from part (4) of Proposition~1 in the
    text. Since inverses are unique (by the same proposition) we have
    $x^{a+b} = x^ax^b$.

    The case where $a < 0, b > 0$ is entirely similar to the argument
    above. Finally, if $a$ and $b$ are both negative, then
    \begin{equation*}
      x^{a+b} = (x^{-a-b})^{-1}
      = (x^{-b-a})^{-1}
      = (x^{-b}x^{-a})^{-1}
      = x^ax^b.
    \end{equation*}
    This completes the proof.
  \end{proof}
\end{enumerate}
