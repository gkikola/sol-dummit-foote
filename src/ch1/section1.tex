\chapter{Introduction to Groups}

\section{Basic Axioms and Examples}

\Exercise1 Determine which of the following binary operations are
associative:
\begin{enumerate}
\item the operation $\star$ on $\Z$ defined by $a\star b = a - b$
  \begin{solution}
    $(1\star2)\star3 = -4$ while $1\star(2\star3) = 2$, so $\star$ is
    not associative.
  \end{solution}
\item the operation $\star$ on $\R$ defined by $a\star b = a + b + ab$
  \begin{solution}
    $\star$ is associative: let $a,b,c$ be real numbers. Then
    \begin{align*}
      (a\star b)\star c &= (a + b + ab)\star c \\
                        &= (a + b + ab) + c + (a + b + ab)c \\
                        &= a + b + c + ab + ac + bc + abc \\
                        &= a + (b + c + bc) + a(b + c + bc) \\
                        &= a\star(b + c + bc) \\
                        &= a\star(b\star c). \qedhere
    \end{align*}
  \end{solution}
\item the operation $\star$ on $\Q$ defined by $a\star b = (a + b)/5$
  \begin{solution}
    $(5\star20)\star15 = 4$ while $5\star(20\star15) =
    12/5$. Therefore $\star$ is not associative.
  \end{solution}
\item the operation $\star$ on $\Z\times\Z$ defined by
  $(a,b)\star(c,d) = (ad + bc, bd)$
  \begin{solution}
    $\star$ is associative: let $(a,b), (c,d), (e,f)$ be members of
    $\Z\times\Z$. Then
    \begin{align*}
      \big((a,b)\star(c,d)\big)\star(e,f)
      &= (ad + bc, bd)\star(e,f) \\
      &= \big((ad+bc)f+bde, bdf\big) \\
      &= (adf+bcf+bde, bdf) \\
      &= \big(adf+b(cf+de), bdf\big) \\
      &= (a,b)\star(cf+de,df) \\
      &= (a,b)\star\big((c,d)\star(e,f)\big). \qedhere
    \end{align*}
  \end{solution}
\item the operation $\star$ on $\Q - \{0\}$ defined by
  $a\star b = a/b$
  \begin{solution}
    $(125\star25)\star5 = 1$ while $125\star(25\star5) = 25$, so
    $\star$ is not associative.
  \end{solution}
\end{enumerate}

\Exercise2 Decide which of the binary operations in the preceding
exercises are commutative.
\begin{enumerate}
\item the operation $\star$ on $\Z$ defined by $a\star b = a - b$
  \begin{solution}
    $\star$ is not commutative since, for example, $1\star2 = -1$
    while $2\star1 = 1$.
  \end{solution}
\item the operation $\star$ on $\R$ defined by $a\star b = a + b + ab$
  \begin{solution}
    $\star$ is commutative since, for any $a,b\in\R$,
    \begin{align*}
      a\star b &= a + b + ab \\
               &= b + a + ba \\
               &= b\star a. \qedhere
    \end{align*}
  \end{solution}
\item the operation $\star$ on $\Q$ defined by $a\star b = (a + b)/5$
  \begin{solution}
    $\star$ is commutative since $+$ is commutative in $\Q$.
  \end{solution}
\item the operation $\star$ on $\Z\times\Z$ defined by
  $(a,b)\star(c,d) = (ad + bc, bd)$
  \begin{solution}
    $\star$ is commutative: Let $(a,b)$ and $(c,d)$ be elements of
    $\Z\times\Z$. Then
    \begin{align*}
      (a,b)\star(c,d) &= (ad + bc, bd) \\
                      &= (cb + da, db) \\
                      &= (c,d)\star(a,b). \qedhere
    \end{align*}
  \end{solution}
\item the operation $\star$ on $\Q - \{0\}$ defined by
  $a\star b = a/b$
  \begin{solution}
    $\star$ is not commutative since $1\star2 = 1/2$ but
    $2\star1 = 2$.
  \end{solution}
\end{enumerate}

\Exercise3 Prove that addition of residue classes in $\Z/n\Z$ is
associative (you may assume it is well defined).
\begin{proof}
  Let $\bar{a}, \bar{b}, \bar{c}$ be residue classes in $\Z/n\Z$. Then
  by Theorem~3 in Section~0.3 along with the associativity of $+$ in
  $\Z$, we may write
  \begin{align*}
    (\bar{a} + \bar{b}) + \bar{c}
    &= \overline{(a + b) + c} \\
    &= \overline{a + (b + c)} \\
    &= \bar{a} + (\bar{b} + \bar{c}).
  \end{align*}
  So addition of residue classes is associative.
\end{proof}

\Exercise4 Prove that multiplication of residue classes in $\Z/n\Z$ is
associative (you may assume it is well defined).
\begin{proof}
  As in the previous exercise, this follows from Theorem~3 in
  Section~0.3 together with the associativity of $\cdot$ in $\Z$.
\end{proof}

\Exercise5 Prove for all $n>1$ that $\Z/n\Z$ is not a group under
multiplication of residue classes.
\begin{proof}
  Let $n>1$. Then there is a residue class in $\Z/n\Z$ which does not
  contain $0$. Call this nonzero residue class $\bar{a}$. Then $\bar0$
  cannot be the identity element in $\Z/n\Z$ since
  $\bar{a}\cdot\bar0 = \bar0 \neq \bar{a}$. So suppose the identity
  element is $\bar{e}$. Then, $\bar0$ also has no inverse in $\Z/n\Z$,
  since $\bar{b}\cdot\bar0 = \bar0 \neq \bar{e}$ for any $\bar{b}$ in
  $\Z/n\Z$. Since the element $\bar0$ does not have an inverse,
  $\Z/n\Z$ is not a group under multiplication.
\end{proof}

\Exercise6 Determine which of the following sets are groups under
addition:
\begin{enumerate}
\item the set of rational numbers (including $0 = 0/1$) in lowest
  terms whose denominators are odd
  \begin{solution}
    Let the set be denoted $A$. Then $A$ is a group having identity
    $0$ and, for each $a\in A$, an inverse $-a$. To prove this, we
    need only show that $A$ is closed under addition.

    Suppose $a$ and $b$ are any elements in $A$. Then we can find
    integers $p,q,r,s$ with
    \begin{equation*}
      a = \frac{p}{q}\quad\text{and}\quad
      b = \frac{r}{s}
    \end{equation*}
    in lowest terms with $q,s$ odd. Then we have
    \begin{align*}
      a + b = \frac{ps+rq}{qs} = \frac{u}{v},
    \end{align*}
    where $u$ and $v$ are integers with $u/v$ in lowest terms. Now,
    since $u/v$ was obtained by eliminating common factors, we have
    $u\mid(ps+rq)$ and $v\mid qs$. But if $2\mid v$, then necessarily
    $2\mid qs$. But this cannot be, since $qs$ is odd, being the
    product of odd integers. Hence $A$ is closed under addition and is
    therefore a group.
  \end{solution}
\item the set of rational numbers in lowest terms whose denominators
  are even, together with $0$
  \begin{solution}
    Let $A$ denote the set. Then $A$ is not a group since $3/2\in A$
    but
    \begin{equation*}
      \frac32 + \frac32 = \frac62 = \frac31\not\in A. \qedhere
    \end{equation*}
  \end{solution}
\item the set of rational numbers of absolute value $<1$
  \begin{solution}
    Again, this set is not closed under addition since, for example,
    \begin{equation*}
      \frac34 + \frac34 > 1.
    \end{equation*}
    Therefore it is not a group.
  \end{solution}
\item the set of rational numbers of absolute value $\geq1$ together
  with $0$
  \begin{solution}
    This set is not closed under addition since, for example,
    \begin{equation*}
      \frac{12}5 - \frac85 = \frac45 \not\geq1.
    \end{equation*}
    Therefore it is not a group.
  \end{solution}
\item the set of rational numbers with denominators equal to $1$ or
  $2$
  \begin{solution}
    Denote the set by $A$. Let $a$ and $b$ be arbitrary integers. Then
    $a,b,a/2,b/2\in A$. There are several cases. First,
    \begin{equation*}
      \frac{a}1 + \frac{b}1 = \frac{a + b}1 \in A.
    \end{equation*}
    Now consider
    \begin{equation*}
      \frac{a}2 + \frac{b}2 = \frac{a + b}2.
    \end{equation*}
    If this fraction is in lowest terms, then it is in $A$. If not,
    then there must be a common factor of $2$ and the fraction can be
    written with a denominator of $1$ and thus is in $A$.

    Finally, consider
    \begin{equation*}
      \frac{a}1 + \frac{b}2 = \frac{b}2 + \frac{a}1 = \frac{2a + b}2.
    \end{equation*}
    As before, this is either in lowest terms, or can be reduced to
    lowest terms by dividing the numerator and denominator by $2$. In
    either case, this number is in $A$.

    Since $A$ is closed under addition, it is easily seen to be a
    group: the identity is $1 = 1/1$ and the inverse of $a/b\in A$ is
    $-a/b$.
  \end{solution}
\item the set of rational numbers with denominators equal to $1$, $2$,
  or $3$
  \begin{solution}
    This set is not closed under addition, since
    \begin{equation*}
      \frac12 + \frac13 = \frac56.
    \end{equation*}
    Hence this is not a group.
  \end{solution}
\end{enumerate}

\Exercise7 Let $G = \{x\in\R\mid 0\leq x<1\}$ and for $x,y\in G$ let
$x\star y$ be the fractional part of $x + y$ (i.e.,
$x\star y = x + y - [x + y]$ where $[a]$ is the greatest integer less
than or equal to $a$). Prove that $\star$ is a well defined binary
operation on $G$ and that $G$ is an abelian group under $\star$
(called the {\em real numbers mod $1$}).
\begin{proof}
  Let $x,y\in G$ be arbitrary. Then $0\leq x + y<2$. There are two
  cases: if $x + y < 1$ then $[x + y] = 0$ and $x\star y\in G$. On the
  other hand, if $1\leq x + y < 2$ then $[x + y] = 1$ and
  $x\star y = x + y - 1\in G$. Therefore $\star$ is a well defined
  binary operation on $G$.

  Let $x,y,z\in G$. If $x + y < 1$ and $y + z < 1$, then
  \begin{align*}
    (x\star y)\star z
    &= (x + y - 0)\star z \\
    &= x + y + z - [x + y + z] \\
    &= x\star(y + z - 0) \\
    &= x\star(y\star z).
  \end{align*}
  On the other hand, if $1\leq x + y < 2$ and $1\leq y + z < 2$, then
  \begin{align*}
    (x\star y)\star z
    &= (x + y - 1)\star z \\
    &= x + y + z - 1 - [x + y + z - 1] \\
    &= x\star(y + z - 1) \\
    &= x\star(y\star z).
  \end{align*}
  Finally, if $1\leq x + y < 2$ and $0\leq y + z < 1$, then
  $[x + y] = 1$, $[y + z] = 0$, and
  $[x + y + z - 1] = [x + y + z] - 1$, so
  \begin{align*}
    (x\star y)\star z
    &= (x + y - 1)\star z \\
    &= x + y + z - 1 - [x + y + z - 1] \\
    &= x + y + z - 1 - [x + y + z] + 1 \\
    &= x + y + z - [x + y + z] \\
    &= x\star(y + z - 0) \\
    &= x\star(y\star z).
  \end{align*}
  And the case where $x+y < 1$ and $y + z\geq1$ is similar. Therefore,
  $\star$ is associative.

  Since $0\in G$, $G$ has an identity ($x\star 0 = 0\star x = x$ for
  each $x$ in $G$). And every element has an inverse: the inverse of
  $0$ is $0$, and for nonzero $x\in G$, $1 - x\in G$ is an inverse
  since
  \begin{equation*}
    x\star(1-x) = x + (1 - x) - [x + (1 - x)]
    = 1 - 1 = 0.
  \end{equation*}
  Therefore $G$ is a group under $\star$.
\end{proof}

\Exercise8 Let $G = \{z\in\C \mid \text{$z^n = 1$ for some
  $n\in\Z^+$}\}$.
\begin{enumerate}
\item Prove that $G$ is a group under multiplication (called the group
  of {\em roots of unity} in $\C$).
  \begin{proof}
    Let $z,w\in G$ so that $z^n = 1$ and $w^m = 1$.

    Note that $1\in G$ since $1^1 = 1$, so $G$ has an identity. And
    every element of $G$ is nonzero, so for each $z\in G$ we may let
    $z^{-1} = 1/z$ so that every element in $G$ has an inverse (since
    $(1/z)^n = 1/z^n = 1$ so $1/z\in G$).

    By the commutativity of multiplication in $\C$, we have
    \begin{equation*}
      (zw)^{nm} = z^{nm}w^{mn} = (z^n)^m(w^m)^n = 1^m1^n = 1
    \end{equation*}
    for each $m,n\in\Z^+$. Therefore, $G$ is closed under
    multiplication. And associativity follows from associativity of
    multiplication in $\C$.

    Therefore $G$ is a group.
  \end{proof}
\item Prove that $G$ is not a group under addition.
  \begin{proof}
    $G$ is a not a group under addition since it is not closed:
    $1\in G$ but $1+1=2\not\in G$ since there is no $n\in\Z^+$ with
    $2^n = 1$.
  \end{proof}
\end{enumerate}

\Exercise9 Let $G = \{a + b\sqrt2\in\R \mid a,b\in\Q\}$.
\begin{enumerate}
\item Prove that $G$ is a group under addition.
  \begin{proof}
    Associativity of $+$ in $G$ follows from associativity of $+$ in
    $\R$. $G$ has an identity $0 = 0 + 0\sqrt2$ and for every $p$ in
    $G$ we may take $q = -p$ as its additive inverse. So we need only
    show that $G$ is closed under addition.

    Let $p = a + b\sqrt2$ and $q = c + d\sqrt2$ with
    $a,b,c,d\in\Q$. Then
    \begin{equation*}
      p+q = a + c + (b + d)\sqrt2,
      \quad\text{where $a+c\in\Q$ and $b+d\in\Q$},
    \end{equation*}
    so $p+q\in G$ and $G$ is a group.
  \end{proof}
\item Prove that the nonzero elements of $G$ are a group under
  multiplication.
  \begin{proof}
    Again, associativity follows from associativity in $\R$. This time
    the identity is $1 = 1 + 0\sqrt2$. And for any rational numbers
    $a$ and $b$ not both $0$, $a+b\sqrt2\in G-\{0\}$ and
    \begin{align*}
      \frac1{a + b\sqrt2}
      &= \frac{a - b\sqrt2}{(a + b\sqrt2)(a - b\sqrt2)} \\
      &= \frac{a - b\sqrt2}{a^2 - 2b^2} \\
      &= \frac{a}{a^2 - 2b^2} - \left(\frac{b}{a^2 - 2b^2}\right)\!\sqrt2
        \in G-\{0\},
    \end{align*}
    so every element in $G-\{0\}$ has an inverse (note that the
    denominator $a^2 - 2b^2$ is nonzero since $a,b\in\Q$ and there is
    no rational square root of $2$).

    The set is also closed under multiplication since for any
    $a,b,c,d\in\Q$ with $a,b$ not both $0$ and $c,d$ not both $0$,
    \begin{equation*}
      \left(a + b\sqrt2\right)\left(c + d\sqrt2\right)
      = ac + 2bd + (ad + bc)\sqrt2 \in G-\{0\}.
    \end{equation*}
    This shows that $G-\{0\}$ is a group under multiplication.
  \end{proof}
\end{enumerate}

\Exercise{10} Prove that a finite group is abelian if and only if its
group table is a symmetric matrix.
\begin{proof}
  List the elements of the group in a fixed order along the top row
  and first column of the group table. Then the group is abelian if
  and only if the $i,j$th entry in its group table is equal to the
  $j,i$th entry, which is true if and only if the table forms a
  symmetric matrix.
\end{proof}

\Exercise{11} Find the orders of each element of the additive group
$\Z/12\Z$.
\begin{solution}
  $\bar0$ has order $1$. $\bar1$ has order 12 since $1\cdot\bar1$,
  $2\cdot\bar1$, $\ldots$, $11\cdot\bar1$ are nonzero while
  $12\cdot\bar1 = \bar0$. Similarly, we find the following orders for
  the elements:

  \begin{center}
    \begin{tabular}{c|c|c|c|c|c|c|c|c|c|c|c|c}
      $x$ & $\bar0$ & $\bar1$ & $\bar2$ & $\bar3$ & $\bar4$ & $\bar5$ & $\bar6$
      & $\bar7$ & $\bar8$ & $\bar9$ & $\overline{10}$ & $\overline{11}$ \\\hline
      $\ord{x}$ & 1 & 12 & 6 & 4 & 3 & 12 & 2 & 12 & 3 & 4 & 6 & 12
    \end{tabular}
  \end{center}
\end{solution}

\Exercise{12} Find the orders of the following elements of the
multiplicative group $(\Z/12\Z)^\times$: $\bar1$, $\overline{-1}$,
$\bar5$, $\bar7$, $\overline{-7}$, $\overline{13}$.
\begin{solution}
  We get the following table:
  \begin{center}
    \begin{tabular}{c|c|c|c|c|c|c}
      $x$ & $\bar1$ & $\overline{-1}$ & $\bar5$ & $\bar7$ & $\overline{-7}$
      & $\overline{13}$ \\\hline
      $\ord{x}$ & 1 & 2 & 2 & 2 & 2 & 1
    \end{tabular}
  \end{center}
\end{solution}

\Exercise{13} Find the orders of the following elements of the
additive group $\Z/36\Z$: $\bar1$, $\bar2$, $\bar6$, $\bar9$,
$\overline{10}$, $\overline{12}$, $\overline{-1}$, $\overline{-10}$,
$\overline{-18}$.
\begin{solution}
  We get the following table:
  \begin{center}
    \begin{tabular}{c|c|c|c|c|c|c|c|c|c}
      $x$ & $\bar1$ & $\bar2$ & $\bar6$ & $\bar9$ & $\overline{10}$
      & $\overline{12}$ & $\overline{-1}$ & $\overline{-10}$
      & $\overline{-18}$ \\\hline
      $\ord{x}$ & 36 & 18 & 6 & 4 & 18 & 3 & 36 & 18 & 2
    \end{tabular}
  \end{center}
\end{solution}

\Exercise{14} Find the orders of the following elements of the
multiplicative group $(\Z/36\Z)^\times$:
$\bar1, \overline{-1}, \bar5, \overline{13}, \overline{-13},
\overline{17}$.
\begin{solution}
  We have the following table:
  \begin{center}
    \begin{tabular}{c|c|c|c|c|c|c}
      $x$ & $\bar1$ & $\overline{-1}$ & $\bar5$ & $\overline{13}$
      & $\overline{-13}$ & $\overline{17}$ \\\hline
      $\ord{x}$ & 1 & 2 & 6 & 3 & 6 & 2
    \end{tabular}
  \end{center}
\end{solution}

\Exercise{15} Let $G$ be a group. Prove that
\begin{equation*}
  (a_1a_2\dots a_n)^{-1} = a_n^{-1}a_{n-1}^{-1}\dots a_1^{-1}
\end{equation*}
for all $a_1,a_2,\dots,a_n\in G$.
\begin{proof}
  We use induction on $n$. If $n = 1$, the result is obvious. Suppose
  it holds for $n = k$, where $k\geq1$. Then for any
  $a_1,\dots,a_{k+1}$ in $G$, we have
  \begin{align*}
    (a_1\dots a_ka_{k+1})(a_{k+1}^{-1}a_k^{-1}\dots a_1^{-1})
    &= (a_1\dots a_k)(a_{k+1}a_{k+1}^{-1})(a_k^{-1}\dots a_1^{-1}) \\
    &= (a_1\dots a_k)(a_k^{-1}\dots a_1^{-1}),
  \end{align*}
  and this is equal to $1$ by the induction hypothesis. Therefore
  $(a_1\dots a_{k+1})^{-1} = a_{k+1}^{-1}\dots a_1^{-1}$ and the
  statement holds for all positive integers $n$.
\end{proof}

\Exercise{16} Let $x$ be an element of a group $G$. Prove that
$x^2 = 1$ if and only if $\ord{x}$ is either $1$ or $2$.
\begin{proof}
  First, if $\ord{x} = 1$ then $x = 1$ so $x^2 = 1^2 = 1$. If
  $\ord{x} = 2$, then $x^2 = 1$ by definition.

  For the other direction, suppose $x^2 = 1$. Then $\ord{x}\leq2$. But
  the order of an element must be at least $1$, so $\ord{x} = 1$ or
  $\ord{x} = 2$.
\end{proof}

\Exercise{17} Let $x$ be an element of a group $G$. Prove that if
$\ord{x} = n$ for some positive integer $n$ then $x^{-1} = x^{n-1}$.
\begin{proof}
  Since $\ord{x} = n$, we have $x^n = 1$. But
  $x^n = x^{n-1}x = xx^{n-1}$, so $x^{n-1}x = 1$ which shows that
  $x^{-1} = x^{n-1}$.
\end{proof}

\Exercise{18} Let $x$ and $y$ be elements of a group $G$. Prove that
$xy = yx$ if and only if $y^{-1}xy = x$ if and only if
$x^{-1}y^{-1}xy = 1$.
\begin{proof}
  If $xy = yx$, then $y^{-1}xy = y^{-1}yx = 1x = x$. Multiplying by
  $x^{-1}$ then gives $x^{-1}y^{-1}xy = 1$.

  On the other hand, if $x^{-1}y^{-1}xy = 1$, then we may multiply on
  the left by $x$ to get $y^{-1}xy = x$. Then multiplying on the left
  by $y$ gives $xy = yx$ as desired.
\end{proof}

\Exercise{19} Let $x\in G$ for $G$ a group and let $a,b\in\Z^+$.
\label{exercise-exponent-rules}
\begin{enumerate}
\item Prove that $x^{a+b} = x^ax^b$ and $(x^a)^b = x^{ab}$.
  \label{exponent-rule}
  \begin{proof}
    $x^ax^b$ consists of a $a$ factors of $x$, multiplied by $b$
    factors of $x$, for a total of $a+b$ factors of $x$. Therefore
    $x^{a+b} = x^ax^b$ by definition. Similarly, $(x^a)^b = x^{ab}$ by
    the same reasoning.
  \end{proof}
\item Prove that $(x^a)^{-1} = x^{-a}$.
  \begin{proof}
    Since $x^{-a} = (x^{-1})^a$, we need to show that
    $(x^a)^{-1} = (x^{-1})^a$. We use induction on $a$. For $a = 1$,
    the result is trivial. Suppose it holds for $a = k$,
    $k\geq0$. Then
    \begin{equation*}
      (x^{k+1})(x^{-1})^{k+1}
      = x^k(xx^{-1})(x^{-1})^k
      = x^k(x^{-1})^k,
    \end{equation*}
    which by the induction hypothesis must be $1$. Therefore the
    result holds for all positive integers $a$.
  \end{proof}
\item Establish part \ref{exponent-rule} for arbitrary integers $a$
  and $b$ (positive, negative, or zero).
  \begin{proof}
    For any integer $a$, $x^ax^0 = x^a = x^{a+0}$ and similarly
    $x^0x^a = x^{0+a}$.

    Now suppose $a>0,b<0$. If $a + b > 0$, then
    $x^{a+b}x^{-b} = x^{(a+b)+(-b)} = x^a$ by part
    \ref{exponent-rule}. Multiplying both sides of this equation on
    the right by $x^b$ gives $x^{a+b} = x^ax^b$ as desired. On the
    other hand, if $a + b < 0$, then
    $x^{-(a+b)}x^a = x^{-(a+b)+a} = x^{-b}$. Multiplying both sides of
    this equation on the right by $x^{-a}$ gives
    $x^{-(a+b)} = x^{-b}x^{-a}$, so
    \begin{equation*}
      (x^{a+b})^{-1}
      = x^{-(a+b)} = x^{-b}x^{-a} = (x^b)^{-1}(x^a)^{-1}
      = (x^ax^b)^{-1}.
    \end{equation*}
    The last equality follows from part (4) of Proposition~1 in the
    text. Since inverses are unique (by the same proposition) we have
    $x^{a+b} = x^ax^b$.

    The case where $a < 0, b > 0$ is entirely similar to the argument
    above. Finally, if $a$ and $b$ are both negative, then
    \begin{equation*}
      x^{a+b} = (x^{-a-b})^{-1}
      = (x^{-b-a})^{-1}
      = (x^{-b}x^{-a})^{-1}
      = x^ax^b.
    \end{equation*}
    This completes the proof.
  \end{proof}
\end{enumerate}

\Exercise{20} For $x$ an element in $G$ a group show that $x$ and
$x^{-1}$ have the same order.
\begin{proof}
  Suppose $\ord{x} = n$ for finite $n$. Then $x^n = 1$ so
  \begin{equation*}
    (x^{-1})^n = x^{-n} = (x^n)^{-1} = 1^{-1} = 1,
  \end{equation*}
  which shows $x^{-1}$ has finite order and
  $\ord{x^{-1}} \leq \ord{x}$. On the other hand, if
  $\ord{x^{-1}} = k$ then
  \begin{equation*}
    x^k = (x^{-1})^{-k} = \left((x^{-1})^k\right)^{-1} = 1^{-1} = 1,
  \end{equation*}
  so $x$ has finite order and $\ord{x} \leq \ord{x^{-1}}$. This shows
  that $\ord{x} = \ord{x^{-1}}$ when either $x$ or $x^{-1}$ is of
  finite order. The only alternative is that $x$ and $x^{-1}$ are both
  of infinite order.
\end{proof}

\Exercise{21} Let $G$ be a finite group and let $x$ be an element of
$G$ of order $n$. Prove that if $n$ is odd, then $x = (x^2)^k$ for
some integer $k\geq1$.
\begin{proof}
  If $n$ is odd, then we may write $n = 2k-1$ for some
  $k\in\Z^+$. Then we have
  \begin{equation*}
    x^n = x^{2k-1} = 1.
  \end{equation*}
  Multiplying both sides by $x$ then gives
  \begin{equation*}
    x^{2k-1}x = x,
  \end{equation*}
  so
  \begin{equation*}
    x = x^{2k-1+1} = x^{2k} = (x^2)^k. \qedhere
  \end{equation*}
\end{proof}

\Exercise{22} If $x$ and $g$ are elements of the group $G$, prove that
$\ord{x} = \ord{g^{-1}xg}$. Deduce that $\ord{ab} = \ord{ba}$ for all
$a,b\in G$.
\begin{proof}
  A simple induction argument will show that
  $(g^{-1}xg)^k = g^{-1}x^kg$ for any $k\in\Z^+$. So if $\ord{x} = n$,
  then $x^n = 1$ and we have
  \begin{equation*}
    (g^{-1}xg)^n = g^{-1}x^ng = g^{-1}1g = 1,
  \end{equation*}
  which shows that $g^{-1}xg$ is of finite order and
  $\ord{g^{-1}xg} \leq \ord{x}$. However, if $\ord{g^{-1}xg} = k$,
  then $(g^{-1}xg)^k = 1$ so
  \begin{equation*}
    x^k = gg^{-1}x^kgg^{-1} = g(g^{-1}xg)^kg^{-1} = g1g^{-1} = 1,
  \end{equation*}
  which shows that $x$ is of finite order and
  $\ord{x} \leq \ord{g^{-1}xg}$. Therefore $\ord{x} = \ord{g^{-1}xg}$.

  This also shows that if $x$ is of infinite order, then $g^{-1}xg$ is
  of infinite order and vice versa.

  Finally, for any $a,b\in G$,
  \begin{equation*}
    \ord{ab} = \ord{b(ab)b^{-1}} = \ord{ba}. \qedhere
  \end{equation*}
\end{proof}

\Exercise{23} Suppose $x\in G$ for $G$ a group and
$\ord{x} = n < \infty$. If $n = st$ for some positive integers $s$ and
$t$, prove that $\ord{x^s} = t$.
\begin{proof}
  Let $\ord{x} = n$ where $n = st$. Then
  \begin{equation*}
    1 = x^n = x^{st} = (x^s)^t,
  \end{equation*}
  so $\ord{x^s}\leq t$. Now suppose $\ord{x^s} = r$. Then
  $(x^s)^r = x^{sr} = 1$. But $\ord{x} = st$, so we have $sr\geq st$
  or $r\geq t$, which gives $\ord{x^s}\geq t$. Therefore
  $\ord{x^s} = t$.
\end{proof}

\Exercise{24} If $a$ and $b$ are {\em commuting} elements of the group
$G$, prove that $(ab)^n = a^nb^n$ for all $n\in\Z$.
\begin{lem}
  If $a$ and $b$ are commuting elements of a group $G$, then
  $a^nb = ba^n$ for all positive integers $n$.
\end{lem}
\begin{proof}
  We use induction on $n$. The base case is trivial, so suppose
  $a^nb = ba^n$ for some positive integer $n$. Then
  \begin{equation*}
    a^{n+1}b = aa^nb = aba^n = baa^n = ba^{n+1},
  \end{equation*}
  which completes the inductive step. Hence $a^nb = ba^n$ for all
  positive $n$.
\end{proof}
\begin{proof}[Proof of main result]
  First we will use induction on $n$ to show that $(ab)^n = a^nb^n$ in
  the case where $n$ is positive. For $n = 1$, the result is
  obvious. Suppose the result is true for $n = k$, for some positive
  integer $k$. Then
  \begin{equation*}
    (ab)^{k+1} = (ab)(ab)^k = aba^kb^k
    = aa^kbb^k = a^{k+1}b^{k+1},
  \end{equation*}
  where the second-to-last equality makes use of the above lemma. This
  shows that the result holds for all positive integers $n$.

  Next, in the case where $n = 0$, we get $(ab)^0 = 1 = a^0b^0$.

  Finally, using the result from Exercise
  \ref{exercise-exponent-rules}, we have for any $n<0$,
  \begin{equation*}
    (ab)^n = (ba)^n = \left((ba)^{-n}\right)^{-1}
    = \left(b^{-n}a^{-n}\right)^{-1}
    = a^nb^n.
  \end{equation*}
  Therefore the result holds for all integers $n$.
\end{proof}

\Exercise{25} Let $G$ be a group. Prove that if $x^2 = 1$ for all
$x\in G$ then $G$ is abelian.
\begin{proof}
  For any $x\in G$, we have $x = x^{-1}$. Let $a,b\in G$ be
  arbitrary. Then we have
  \begin{equation*}
    ab = (ab)^{-1} = b^{-1}a^{-1} = ba.
  \end{equation*}
  Here we have made use of property (4) from Proposition~1. This shows
  that $G$ is abelian.
\end{proof}

\Exercise{26} Assume $H$ is a nonempty subset of the group $(G,\star)$
which is closed under the binary operation on $G$ and is closed under
inverses, i.e., for all $h$ and $k\in H$, $hk$ and $h^{-1}\in
H$. Prove that $H$ is a group under the operation $\star$ restricted
to $H$ (such a subset $H$ is called a {\em subgroup} of $G$).
\label{exercise-subgroup-conditions}
\begin{proof}
  \begin{enumerate}
  \item Associativity of $\star$ in $H$ follows from associativity of
    $\star$ in $G$.

  \item Since $H$ is nonempty, it must have an element $a$. Then by
    hypothesis $a^{-1}\in H$ and therefore $aa^{-1} = e \in H$, where
    $e$ denotes the identity of $G$. Therefore $H$ has an identity.

  \item For each $a\in H$, $a^{-1}\in H$ by hypothesis so every
    element of $H$ has an inverse in $H$.
  \end{enumerate}

  This shows that $(H,\star)$ is a group.
\end{proof}

\Exercise{27} Prove that if $x$ is an element of the group $G$ then
$\{x^n \mid n\in\Z\}$ is a subgroup of $G$ (called the {\em cyclic
  subgroup} of $G$ generated by $x$).
\begin{proof}
  Let $H$ be the subset stated above. We know $H$ is nonempty since
  $x^0 = e$ is a member of $H$. If $a = x^m$ and $b = x^n$ are any two
  elements in $H$, then $ab = x^mx^n = x^{m+n}$ by
  Exercise~\ref{exercise-exponent-rules}. So $ab\in H$ which shows
  that $H$ is closed under the binary operation of $G$. $H$ is also
  closed under inverses, since $a^{-1} = (x^m)^{-1} = x^{-m}\in
  H$. Therefore, by the previous exercise, $H$ is a subgroup of $G$.
\end{proof}

\Exercise{28} Let $(A,\star)$ and $(B,\diamond)$ be groups and let
$A\times B$ be their direct product. Verify all the group axioms for
$A\times B$:
\begin{enumerate}
\item prove that the associative law holds
\item prove that $(1,1)$ is the identity of $A\times B$, and
\item prove that the inverse of $(a,b)$ is $(a^{-1}, b^{-1})$.
\end{enumerate}
\begin{proof}
  \begin{enumerate}
  \item
    For all $(a_i, b_i)\in A\times B$ with $i = 1,2,3$ we have
    \begin{align*}
      (a_1,b_1)[(a_2,b_2)(a_3,b_3)]
      &= (a_1,b_1)(a_2a_3,b_2b_3) \\
      &= \big(a_1(a_2a_3), b_1(b_2b_3)\big) \\
      &= \big((a_1a_2)a_3, (b_1b_2)b_3\big) \\
      &= (a_1a_2,b_1b_2)(a_3,b_3) \\
      &= [(a_1,b_1)(a_2b_2)](a_3,b_3).
    \end{align*}
    This shows associativity.

  \item For any $(a,b)\in A\times B$ we have
    \begin{equation*}
      (a,b)(1,1) = (a\star1,b\diamond1) = (a,b).
    \end{equation*}
    Therefore $(1,1)$ is the identity of $A\times B$.

  \item For any $(a,b)\in A\times B$,
    \begin{equation*}
      (a,b)(a^{-1},b^{-1}) = (a\star a^{-1},b\diamond b^{-1})
      = (1,1),
    \end{equation*}
    so $(a,b)^{-1} = (a^{-1},b^{-1})$. \qedhere
  \end{enumerate}
\end{proof}

\Exercise{29} Prove that $A\times B$ is an abelian group if and only
if both $A$ and $B$ are abelian.
\begin{proof}
  First, if $A$ and $B$ are abelian and if $(a,b)$ and $(c,d)$ are any
  members of $A\times B$, then
  \begin{equation*}
    (a,b)(c,d) = (ac,bd) = (ca,db) = (c,d)(a,b),
  \end{equation*}
  so $A\times B$ is abelian.

  For the other direction, suppose $A\times B$ is abelian. Let
  $a_1,a_2\in A$ and $b_1,b_2\in B$. Then since $A\times B$ is
  abelian, we have
  \begin{equation*}
    (a_1a_2, b_1b_2) = (a_1,b_1)(a_2,b_2)
    = (a_2, b_2)(a_1,b_1) = (a_2a_1, b_2b_1).
  \end{equation*}
  Equating components shows that $a_1a_2 = a_2a_1$ and
  $b_1b_2 = b_2b_1$. Therefore $A$ and $B$ are both abelian.
\end{proof}

\Exercise{30} Prove that the elements $(a,1)$ and $(1,b)$ of
$A\times B$ commute and deduce that the order of $(a,b)$ is the least
common multiple of $\ord{a}$ and $\ord{b}$.
\begin{proof}
  If $a\in A$ and $b\in B$, then $(a,1),(1,b)\in A\times B$ and
  \begin{equation*}
    (a,1)(1,b) = (a1,1b) = (a,b) = (1a,b1) = (1,b)(a,1).
  \end{equation*}

  Now, we will show by induction on $n$ that $(a,b)^n = (a^n,b^n)$ for
  any positive integer $n$. The base case is obvious. Suppose
  $(a,b)^k = (a^k,b^k)$ for some $k>0$. Then
  \begin{equation*}
    (a,b)^{k+1} = (a,b)^k(a,b)
    = (a^k,b^k)(a,b) = (a^{k+1},b^{k+1})
  \end{equation*}
  so the statement holds for all $n\in\Z^+$. This implies that
  $\ord{(a,1)} = \ord{a}$ and $\ord{(1,b)} = \ord{b}$.

  Let the least common multiple of $\ord{a}$ and $\ord{b}$ be $\ell$
  and suppose $\ord{(a,b)} = k$. Then $m\ord{a} = n\ord{b} = \ell$ for
  some integers $m$ and $n$. Since $(a,1)$ and $(1,b)$ commute, we
  have
  \begin{align*}
    (a,b)^\ell
    &= ((a,1)(1,b))^\ell \\
    &= (a,1)^\ell(1,b)^\ell \\
    &= (a,1)^{m\ord{a}}(1,b)^{n\ord{b}} \\
    &= (1,1)(1,1) \\
    &= (1,1).
  \end{align*}
  So $k\leq\ell$. Now since $(a,b)^k = (1,1)$, we have $a^k = 1$ and
  $b^k = 1$. This implies $\ord{a}$ divides $k$ and $\ord{b}$ divides
  $k$. So $k$ is a common multiple of $\ord{a}$ and
  $\ord{b}$. Therefore $\ell\leq k$. This shows that $\ell = k$, which
  completes the proof.
\end{proof}

\Exercise{31} Prove that any finite group $G$ of even order contains
an element of order $2$.
\label{exercise-even-order-has-element-order-2}
\begin{proof}
  Define $t(G)$ to be the set $\{ g \in G \mid g \neq g^{-1} \}$. Then
  $t(G)$ must have an even number of elements because $g\in t(G)$ if
  and only if $g^{-1}\in t(G)$ and any such $g, g^{-1}$ must be
  distinct. Since $G$ also has an even number of elements, the set
  $G-t(G)$ has an even number of elements.

  Now $G-t(G)$ is nonempty since the identity $e\not\in
  t(G)$. Therefore there is a nonidentity element $a\in G - t(G)$. But
  since $a\not\in t(G)$, we have $a = a^{-1}$ so that $a^2 = e$ but
  $a\neq e$. Thus $a$ is an element of order $2$, completing the
  proof.
\end{proof}

\Exercise{32} If $x$ is an element of finite order $n$ in a group $G$,
prove that the elements $1, x, x^2, \dots, x^{n-1}$ are all
distinct. Deduce that $\ord{x}\leq\ord{G}$.
\begin{proof}
  Suppose the contrary, so that $x^s = x^t$ for $1\leq s < t<n$. Then
  $x^tx^{-s} = x^{t-s} = 1$. But $1\leq t-s < n$, so $\ord{x} < n$, a
  contradiction. This shows that each of $1, x,\dots,x^{n-1}$ are
  distinct so that $\ord{G}\geq\ord{x}$.
\end{proof}

\Exercise{33} Let $x$ be an element of finite order $n$ in the group
$G$.
\label{exercise-power-own-inverse}
\begin{enumerate}
\item Prove that if $n$ is odd then $x^i\neq x^{-i}$ for all
  $i = 1, 2, \dots, n-1$.
  \begin{proof}
    Fix a positive integer $i < n$. Then $x^ix^{n-i} = 1$ so
    $x^{-i} = x^{n-i}$. By the previous exercise, if $i\neq n-i$, then
    $x^i\neq x^{n-i}$. Since inverses are unique, we have in this case
    that $x^i\neq x^{-i}$.

    Now, if $n$ is odd, then necessarily $i\neq n-i$, so
    $x^i\neq x^{-i}$.
  \end{proof}
\item Prove that if $n = 2k$ and $1\leq i < n$ then $x^i = x^{-i}$ if
  and only if $i = k$.
  \begin{proof}
    For any $1\leq i<n$ such that $i\neq k$, we have $i\neq n-i$ so
    $x^i\neq x^{-i}$ by the argument in the first part of the
    problem. And if $i = k$, then $x^ix^i = x^{2k} = x^n = 1$, so
    $x^i = x^{-i}$ in this case (and only this case).
  \end{proof}
\end{enumerate}

\Exercise{34} If $x$ is an element of infinite order in the group $G$,
prove that the elements $x^n$, $n\in\Z$ are all distinct.
\begin{proof}
  Let $x$ have infinite order and suppose $x^m = x^n$ with $n\leq
  m$. Then $x^{m-n} = 1$. If $m-n > 0$ then $x$ has finite order,
  which is a contradiction. Therefore $m = n$. This shows that each
  $x^m$ is distinct.
\end{proof}

\Exercise{35} If $x$ is an element of finite order $n$ in a group $G$,
use the Division Algorithm to show that {\em any} integral power of
$x$ equals one of the elements in the set $\{1,x,x^2,\dots,x^{n-1}\}$.
\begin{proof}
  Let $x$ have order $n$ and suppose $k$ is any integer.

  Since $n$ must be greater than $0$, we may use the Division
  Algorithm to find integers $q$ and $r$ such that $k = qn + r$, where
  $0\leq r<n$. Then
  \begin{equation*}
    x^k = x^{qn+r} = (x^n)^qx^r = 1x^r = x^r,
    \quad \text{where $0\leq r<n$,}
  \end{equation*}
  which completes the proof.
\end{proof}

\Exercise{36} Assume $G = \{1, a, b, c\}$ is a group of order $4$ with
identity $1$. Assume also that $G$ has no elements of order $4$. Use
the cancellation laws to show that there is a unique group table for
$G$. Deduce that $G$ is abelian.
\begin{proof}
  From the previous exercises, we know that each element in $G$
  besides $1$ either has order equal to $2$ or $3$. By
  Exercise~\ref{exercise-even-order-has-element-order-2} there is an
  element in $G$ with order $2$. Without loss of generality, we may
  suppose that this element is $a$.

  Then $a^2 = 1$. Now $ab\neq1$ since that would imply
  $b = a^{-1} = a$. Next, $ab\neq a$ since otherwise the cancellation
  law would give $b = 1$. Similarly, $ab\neq b$ since otherwise
  $a = 1$. So we must have $ab = c$. Using the same reasoning, we must
  have $ba = c$ and $ac = ca = b$. Using this information, we have
  $b^2 = (ca)(ac) = c(a^2)c = c^2$.

  Now, if $b^2\neq1$ then we must have $\ord{b} = 3$ so that
  $b^3 = 1$. Then
  \begin{equation*}
    a = ab^3 = (ab)b^2 = c^3.
  \end{equation*}
  But since $c^3 = a\neq1$, we have $\ord{c} = 2$ so $1 = c^2 = b^2$
  and $\ord{b} = 2$, a contradiction. This shows that $b^2 = c^2 =
  1$. Finally,
  \begin{equation*}
    bc = (ac)c = ac^2 = a,
  \end{equation*}
  and similarly $cb = a$.

  Combining all of this information gives the following group table:
  \begin{center}
    \begin{tabular}{r|cccc}
      & $1$ & $a$ & $b$ & $c$ \\\hline
      $1$ & $1$ & $a$ & $b$ & $c$ \\
      $a$ & $a$ & $1$ & $c$ & $b$ \\
      $b$ & $b$ & $c$ & $1$ & $a$ \\
      $c$ & $c$ & $b$ & $a$ & $1$
    \end{tabular}
  \end{center}
  And we can readily see that $G$ is abelian.
\end{proof}
