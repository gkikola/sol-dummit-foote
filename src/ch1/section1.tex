\chapter{Introduction to Groups}

\section{Basic Axioms and Examples}

Let $G$ be a group.

\Exercise1 Determine which of the following binary operations are
associative:
\begin{enumerate}
\item the operation $\star$ on $\Z$ defined by $a\star b = a - b$
  \begin{solution}
    $(1\star2)\star3 = -4$ while $1\star(2\star3) = 2$, so $\star$ is
    not associative.
  \end{solution}
\item the operation $\star$ on $\R$ defined by $a\star b = a + b + ab$
  \begin{solution}
    $\star$ is associative: let $a,b,c$ be real numbers. Then
    \begin{align*}
      (a\star b)\star c &= (a + b + ab)\star c \\
                        &= (a + b + ab) + c + (a + b + ab)c \\
                        &= a + b + c + ab + ac + bc + abc \\
                        &= a + (b + c + bc) + a(b + c + bc) \\
                        &= a\star(b + c + bc) \\
                        &= a\star(b\star c). \qedhere
    \end{align*}
  \end{solution}
\item the operation $\star$ on $\Q$ defined by $a\star b = (a + b)/5$
  \begin{solution}
    $(5\star20)\star15 = 4$ while $5\star(20\star15) =
    12/5$. Therefore $\star$ is not associative.
  \end{solution}
\item the operation $\star$ on $\Z\times\Z$ defined by
  $(a,b)\star(c,d) = (ad + bc, bd)$
  \begin{solution}
    $\star$ is associative: let $(a,b), (c,d), (e,f)$ be members of
    $\Z\times\Z$. Then
    \begin{align*}
      \big((a,b)\star(c,d)\big)\star(e,f)
      &= (ad + bc, bd)\star(e,f) \\
      &= \big((ad+bc)f+bde, bdf\big) \\
      &= (adf+bcf+bde, bdf) \\
      &= \big(adf+b(cf+de), bdf\big) \\
      &= (a,b)\star(cf+de,df) \\
      &= (a,b)\star\big((c,d)\star(e,f)\big). \qedhere
    \end{align*}
  \end{solution}
\item the operation $\star$ on $\Q - \{0\}$ defined by
  $a\star b = a/b$
  \begin{solution}
    $(125\star25)\star5 = 1$ while $125\star(25\star5) = 25$, so
    $\star$ is not associative.
  \end{solution}
\end{enumerate}

\Exercise2 Decide which of the binary operations in the preceding
exercises are commutative.
\begin{enumerate}
\item the operation $\star$ on $\Z$ defined by $a\star b = a - b$
  \begin{solution}
    $\star$ is not commutative since, for example, $1\star2 = -1$
    while $2\star1 = 1$.
  \end{solution}
\item the operation $\star$ on $\R$ defined by $a\star b = a + b + ab$
  \begin{solution}
    $\star$ is commutative since, for any $a,b\in\R$,
    \begin{align*}
      a\star b &= a + b + ab \\
               &= b + a + ba \\
               &= b\star a. \qedhere
    \end{align*}
  \end{solution}
\item the operation $\star$ on $\Q$ defined by $a\star b = (a + b)/5$
  \begin{solution}
    $\star$ is commutative since $+$ is commutative in $\Q$.
  \end{solution}
\item the operation $\star$ on $\Z\times\Z$ defined by
  $(a,b)\star(c,d) = (ad + bc, bd)$
  \begin{solution}
    $\star$ is commutative: Let $(a,b)$ and $(c,d)$ be elements of
    $\Z\times\Z$. Then
    \begin{align*}
      (a,b)\star(c,d) &= (ad + bc, bd) \\
                      &= (cb + da, db) \\
                      &= (c,d)\star(a,b). \qedhere
    \end{align*}
  \end{solution}
\item the operation $\star$ on $\Q - \{0\}$ defined by
  $a\star b = a/b$
  \begin{solution}
    $\star$ is not commutative since $1\star2 = 1/2$ but
    $2\star1 = 2$.
  \end{solution}
\end{enumerate}

\Exercise3 Prove that addition of residue classes in $\Z/n\Z$ is
associative (you may assume it is well defined).
\begin{proof}
  Let $\bar{a}, \bar{b}, \bar{c}$ be residue classes in $\Z/n\Z$. Then
  by Theorem~3 in Section~0.3 along with the associativity of $+$ in
  $\Z$, we may write
  \begin{align*}
    (\bar{a} + \bar{b}) + \bar{c}
    &= \overline{(a + b) + c} \\
    &= \overline{a + (b + c)} \\
    &= \bar{a} + (\bar{b} + \bar{c}).
  \end{align*}
  So addition of residue classes is associative.
\end{proof}

\Exercise4 Prove that multiplication of residue classes in $\Z/n\Z$ is
associative (you may assume it is well defined).
\begin{proof}
  As in the previous exercise, this follows from Theorem~3 in
  Section~0.3 together with the associativity of $\cdot$ in $\Z$.
\end{proof}

\Exercise5 Prove for all $n>1$ that $\Z/n\Z$ is not a group under
multiplication of residue classes.
\begin{proof}
  Let $n>1$. Then there is a residue class in $\Z/n\Z$ which does not
  contain $0$. Call this nonzero residue class $\bar{a}$. Then $\bar0$
  cannot be the identity element in $\Z/n\Z$ since
  $\bar{a}\cdot\bar0 = \bar0 \neq \bar{a}$. So suppose the identity
  element is $\bar{e}$. Then, $\bar0$ also has no inverse in $\Z/n\Z$,
  since $\bar{b}\cdot\bar0 = \bar0 \neq \bar{e}$ for any $\bar{b}$ in
  $\Z/n\Z$. Since the element $\bar0$ does not have an inverse,
  $\Z/n\Z$ is not a group under multiplication.
\end{proof}
