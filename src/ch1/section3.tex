\section{Symmetric Groups}

\Exercise1 Let $\sigma$ be the permutation
\begin{equation*}
  1\mapsto3 \quad 2\mapsto4 \quad 3\mapsto5 \quad 4\mapsto2 \quad 5\mapsto1
\end{equation*}
and let $\tau$ be the permutation
\begin{equation*}
  1\mapsto5 \quad 2\mapsto3 \quad 3\mapsto2 \quad 4\mapsto4 \quad 5\mapsto1.
\end{equation*}
Find the cycle decompositions of each of the following permutations:
$\sigma$, $\tau$, $\sigma^2$, $\sigma\tau$, $\tau\sigma$, and
$\tau^2\sigma$.
\begin{solution}
  Applying the permutations from right to left, we get
  \begin{align*}
    \sigma &= (1\,3\,5)(2\,4) \\
    \tau &= (1\,5)(2\,3) \\
    \sigma^2 &= (1\,5\,3) \\
    \sigma\tau &= (2\,5\,3\,4) \\
    \tau\sigma &= (1\,2\,4\,3) \\
    \intertext{and}
    \tau^2\sigma &= (1\,3\,5)(2\,4). \qedhere
  \end{align*}
\end{solution}

\Exercise2 Let $\sigma$ be the permutation
\begin{align*}
  1 &\mapsto 13 &&& 2 &\mapsto 2 &&& 3 &\mapsto 15 &&& 4 &\mapsto 14
  &&& 5 &\mapsto 10 \\
  6 &\mapsto 6 &&& 7 &\mapsto 12 &&& 8 &\mapsto 3 &&& 9 &\mapsto 4
  &&& 10 &\mapsto 1 \\
  11 &\mapsto 7 &&& 12 &\mapsto 9 &&& 13 &\mapsto 5 &&& 14 &\mapsto 11
  &&& 15 &\mapsto 8
\end{align*}
and let $\tau$ be the permutation
\begin{align*}
  1 &\mapsto 14 &&& 2 &\mapsto 9 &&& 3 &\mapsto 10 &&& 4 &\mapsto 2
  &&& 5 &\mapsto 12 \\
  6 &\mapsto 6 &&& 7 &\mapsto 5 &&& 8 &\mapsto 11 &&& 9 &\mapsto 15
  &&& 10 &\mapsto 3 \\
  11 &\mapsto 8 &&& 12 &\mapsto 7 &&& 13 &\mapsto 4 &&& 14 &\mapsto 1
  &&& 15 &\mapsto 13.
\end{align*}
Find the cycle decomposition of the following permutations: $\sigma$,
$\tau$, $\sigma^2$, $\sigma\tau$, $\tau\sigma$, and $\tau^2\sigma$.
\begin{solution}
  We find
  \begin{align*}
    \sigma &= (1\;13\;5\;10)(3\;15\;8)(4\;14\;11\;7\;12\;9) \\
    \tau &= (1\;14)(2\;9\;15\;13\;4)(3\;10)(5\;12\;7)(8\;11) \\
    \sigma^2 &= (1\;5)(3\;8\;15)(4\;11\;12)(7\;9\;14)(10\;13) \\
    \sigma\tau &= (1\;11\;3)(2\;4)(5\;9\;8\;7\;10\;15)(13\;14) \\
    \tau\sigma &= (1\;4)(2\;9)(3\;13\;12\;15\;11\;5)(8\;10\;14) \\
    \intertext{and}
    \tau^2\sigma &= (1\;2\;15\;8\;3\;4\;14\;11\;12\;13\;7\;5\;10).
                   \qedhere
  \end{align*}
\end{solution}

\Exercise3 For each of the permutations whose cycle decompositions
were computed in the preceding two exercises compute its order.
\begin{solution}
  For the first exercise, $\sigma = (1\,3\,5)(2\,4)$,
  $\sigma^2 = (1\,5\,3)$, $\sigma^3 = (2\,4)$, $\sigma^4 = (1\,3\,5)$,
  $\sigma^5 = (1\,5\,3)(2\,4)$ and $\sigma^6 = 1$. Therefore
  $\abs{\sigma} = 6$. Similarly, $\tau = (1\,5)(2\,3)$ and
  $\tau^2 = 1$, so $\abs{\tau} = 2$. $\sigma^2$ is a 3-cycle and so
  has order $3$, and $\sigma\tau$ and $\tau\sigma$ are both 4-cycles
  and so have order $4$. Lastly, $\tau^2\sigma = \sigma$ so
  $\abs{\tau^2\sigma} = 6$.

  For the second exercise, we could proceed in the same way. Or we
  could observe that, since a $t$-cycle has order $t$, the order of a
  product of disjoint cycles will be the least common multiple of the
  lengths of each cycle. This gives
  \begin{align*}
    \abs{\sigma} &= [3,4,6] = 12, \\
    \abs{\tau} &= [2,3,5] = 30, \\
    \abs{\sigma^2} &= [2,3] = 6, \\
    \abs{\sigma\tau} &= [2,3,6] = 6, \\
    \abs{\tau\sigma} &= [2,3,6] = 6, \\
    \intertext{and}
    \abs{\tau^2\sigma} &= 13. \qedhere
  \end{align*}
\end{solution}
