\section{Symmetric Groups}

\Exercise1 Let $\sigma$ be the permutation
\begin{equation*}
  1\mapsto3 \quad 2\mapsto4 \quad 3\mapsto5 \quad 4\mapsto2 \quad 5\mapsto1
\end{equation*}
and let $\tau$ be the permutation
\begin{equation*}
  1\mapsto5 \quad 2\mapsto3 \quad 3\mapsto2 \quad 4\mapsto4 \quad 5\mapsto1.
\end{equation*}
Find the cycle decompositions of each of the following permutations:
$\sigma$, $\tau$, $\sigma^2$, $\sigma\tau$, $\tau\sigma$, and
$\tau^2\sigma$.
\begin{solution}
  Applying the permutations from right to left, we get
  \begin{align*}
    \sigma &= (1\,3\,5)(2\,4) \\
    \tau &= (1\,5)(2\,3) \\
    \sigma^2 &= (1\,5\,3) \\
    \sigma\tau &= (2\,5\,3\,4) \\
    \tau\sigma &= (1\,2\,4\,3) \\
    \intertext{and}
    \tau^2\sigma &= (1\,3\,5)(2\,4). \qedhere
  \end{align*}
\end{solution}

\Exercise2 Let $\sigma$ be the permutation
\begin{align*}
  1 &\mapsto 13 &&& 2 &\mapsto 2 &&& 3 &\mapsto 15 &&& 4 &\mapsto 14
  &&& 5 &\mapsto 10 \\
  6 &\mapsto 6 &&& 7 &\mapsto 12 &&& 8 &\mapsto 3 &&& 9 &\mapsto 4
  &&& 10 &\mapsto 1 \\
  11 &\mapsto 7 &&& 12 &\mapsto 9 &&& 13 &\mapsto 5 &&& 14 &\mapsto 11
  &&& 15 &\mapsto 8
\end{align*}
and let $\tau$ be the permutation
\begin{align*}
  1 &\mapsto 14 &&& 2 &\mapsto 9 &&& 3 &\mapsto 10 &&& 4 &\mapsto 2
  &&& 5 &\mapsto 12 \\
  6 &\mapsto 6 &&& 7 &\mapsto 5 &&& 8 &\mapsto 11 &&& 9 &\mapsto 15
  &&& 10 &\mapsto 3 \\
  11 &\mapsto 8 &&& 12 &\mapsto 7 &&& 13 &\mapsto 4 &&& 14 &\mapsto 1
  &&& 15 &\mapsto 13.
\end{align*}
Find the cycle decomposition of the following permutations: $\sigma$,
$\tau$, $\sigma^2$, $\sigma\tau$, $\tau\sigma$, and $\tau^2\sigma$.
\begin{solution}
  We find
  \begin{align*}
    \sigma &= (1\;13\;5\;10)(3\;15\;8)(4\;14\;11\;7\;12\;9) \\
    \tau &= (1\;14)(2\;9\;15\;13\;4)(3\;10)(5\;12\;7)(8\;11) \\
    \sigma^2 &= (1\;5)(3\;8\;15)(4\;11\;12)(7\;9\;14)(10\;13) \\
    \sigma\tau &= (1\;11\;3)(2\;4)(5\;9\;8\;7\;10\;15)(13\;14) \\
    \tau\sigma &= (1\;4)(2\;9)(3\;13\;12\;15\;11\;5)(8\;10\;14) \\
    \intertext{and}
    \tau^2\sigma &= (1\;2\;15\;8\;3\;4\;14\;11\;12\;13\;7\;5\;10).
                   \qedhere
  \end{align*}
\end{solution}

\Exercise3 For each of the permutations whose cycle decompositions
were computed in the preceding two exercises compute its order.
\begin{solution}
  For the first exercise, $\sigma = (1\,3\,5)(2\,4)$,
  $\sigma^2 = (1\,5\,3)$, $\sigma^3 = (2\,4)$, $\sigma^4 = (1\,3\,5)$,
  $\sigma^5 = (1\,5\,3)(2\,4)$ and $\sigma^6 = 1$. Therefore
  $\ord{\sigma} = 6$. Similarly, $\tau = (1\,5)(2\,3)$ and
  $\tau^2 = 1$, so $\ord{\tau} = 2$. $\sigma^2$ is a 3-cycle and so
  has order $3$, and $\sigma\tau$ and $\tau\sigma$ are both 4-cycles
  and so have order $4$. Lastly, $\tau^2\sigma = \sigma$ so
  $\ord{\tau^2\sigma} = 6$.

  For the second exercise, we could proceed in the same way. Or we
  could observe that, since a $t$-cycle has order $t$, the order of a
  product of disjoint cycles will be the least common multiple of the
  lengths of each cycle. This gives
  \begin{align*}
    \ord{\sigma} &= [3,4,6] = 12, \\
    \ord{\tau} &= [2,3,5] = 30, \\
    \ord{\sigma^2} &= [2,3] = 6, \\
    \ord{\sigma\tau} &= [2,3,6] = 6, \\
    \ord{\tau\sigma} &= [2,3,6] = 6, \\
    \intertext{and}
    \ord{\tau^2\sigma} &= 13. \qedhere
  \end{align*}
\end{solution}

\Exercise4 Compute the order of each of the elements in the following
groups:
\label{exercise-order-s3-and-s4-elements}
\begin{enumerate}
\item $S_3$
  \begin{solution}
    All elements in $S_3$ can be written as a single $t$-cycle, with
    $t$ being the order of the element:
    \begin{center}
      \begin{tabular}{r|l}
        Permutation & Order in $S_3$ \\\hline
        1 & 1 \\
        (1\,2) & 2 \\
        (1\,3) & 2 \\
        (2\,3) & 2 \\
        (1\,2\,3) & 3 \\
        (1\,3\,2) & 3
      \end{tabular}
    \end{center}
  \end{solution}
\item $S_4$
  \begin{solution}
    The order of each element in $S_4$ is simply the least common
    multiple of the lengths of each cycle in its cycle decomposition:
    \begin{center}
      \begin{tabular}{r|l}
        Permutation & Order \\\hline
        1 & 1 \\
        (1\,2) & 2 \\
        (1\,3) & 2 \\
        (1\,4) & 2 \\
        (2\,3) & 2 \\
        (2\,4) & 2 \\
        (3\,4) & 2 \\
        (1\,2\,3) & 3
      \end{tabular}
      \begin{tabular}{r|l}
        Permutation & Order \\\hline
        (1\,2\,4) & 3 \\
        (1\,3\,4) & 3 \\
        (2\,3\,4) & 3 \\
        (1\,3\,2) & 3 \\
        (1\,4\,2) & 3 \\
        (1\,4\,3) & 3 \\
        (2\,4\,3) & 3 \\
        (1\,2)(3\,4) & 2
      \end{tabular}
      \begin{tabular}{r|l}
        Permutation & Order \\\hline
        (1\,3)(2\,4) & 2 \\
        (1\,4)(2\,3) & 2 \\
        (1\,2\,3\,4) & 4 \\
        (1\,2\,4\,3) & 4 \\
        (1\,3\,2\,4) & 4 \\
        (1\,3\,4\,2) & 4 \\
        (1\,4\,2\,3) & 4 \\
        (1\,4\,3\,2) & 4
      \end{tabular}
    \end{center}
  \end{solution}
\end{enumerate}

\Exercise5 Find the order of
$(1\;12\;8\;10\;4)(2\;13)(5\;11\;7)(6\;9)$.
\begin{solution}
  Since the cycles are disjoint, the order of this element in $S_{13}$
  is the least common multiple of the cycle lengths: $[2,3,5] = 30$.
\end{solution}

\Exercise6 Write out the cycle decomposition of each element of order
$4$ in $S_4$.
\begin{solution}
  See Exercise~\ref{exercise-order-s3-and-s4-elements}.
\end{solution}

\Exercise7 Write out the cycle decomposition of each element of order
$2$ in $S_4$.
\begin{solution}
  See Exercise~\ref{exercise-order-s3-and-s4-elements}.
\end{solution}

\Exercise8 Prove that if $\Omega = \{1,2,3,\dots\}$ then $S_\Omega$ is
an infinite group.
\begin{proof}
  Let $n$ be any positive integer and consider the permutation
  $\sigma_n$ which sends $2n-1$ to $2n$ and sends $2n$ to $2n-1$,
  while fixing all other elements in $\Omega$. Clearly
  $\sigma_n\in S_\Omega$.

  Now, if $i$ and $j$ are distinct positive integers, then the numbers
  $2i-1$, $2i$, $2j-1$, $2j$ are distinct from one another, so that
  $\sigma_i$ and $\sigma_j$ have cycle decompositions that are
  disjoint. Thus $\sigma_1, \sigma_2, \dots, \sigma_n, \dots$ are
  distinct elements in $S_\Omega$, and therefore $S_\Omega$ is
  infinite.
\end{proof}

\Exercise9
\begin{enumerate}
\item Let $\sigma$ be the $12$-cycle
  $(1\;2\;3\;4\;5\;6\;7\;8\;9\;10\;11\;12)$. For which positive
  integers $i$ is $\sigma^i$ also a $12$-cycle?
  \begin{solution}
    By applying $\sigma$ twice we can determine that
    \begin{equation*}
      \sigma^2 = (1\;3\;5\;7\;9\;11)(2\;4\;6\;8\;10\;12).
    \end{equation*}
    So $\sigma^2$ is not a $12$-cycle. In this way we can also
    determine that $\sigma^3$ and $\sigma^4$ are also not
    $12$-cycles. However,
    \begin{equation*}
      \sigma^5 = (1\;6\;11\;4\;9\;2\;7\;12\;5\;10\;3\;8)
    \end{equation*}
    so $\sigma^5$ is a $12$-cycle.

    Continuing in this way, we can see that $\sigma^6$ consists of a
    product of $2$-cycles, $\sigma^7$ is a $12$-cycle, $\sigma^8$ is a
    product of $3$-cycles, $\sigma^9$ is a product of $4$-cycles,
    $\sigma^{10}$ is a product of $6$-cycles, and $\sigma^{11}$ is a
    $12$-cycle. And higher powers will simply repeat the pattern.

    Therefore, $\sigma^i$ is a $12$-cycle for $i = 1,5,7,11$ as well
    as any integers which have a remainder of $1,5,7$, or $11$ when
    divided by $12$. We can also characterize these values as being
    precisely those values of $i$ for which $(12,i) = 1$.
  \end{solution}
\item Let $\tau$ be the $8$-cycle $(1\,2\,3\,4\,5\,6\,7\,8)$. For
  which positive integers $i$ is $\tau^i$ also an $8$-cycle?
  \begin{solution}
    As in the previous part, $8$-cycles will be formed from any
    exponent $i$ which is coprime to $8$, that is, any $i$ such that
    $(8,i) = 1$. This means that $i = 1,3,5,7$ or any congruent values
    modulo $8$.
  \end{solution}
\item Let $\omega$ be the $14$-cycle
  $(1\;2\;3\;4\;5\;6\;7\;8\;9\;10\;11\;12\;13\;14)$. For which
  positive integers $i$ is $\omega^i$ also a $14$-cycle.
  \begin{solution}
    Again, it is easy to verify that values of $i$ for which
    $(14,i) = 1$ will produce $14$-cycles. So $i = 1,3,5,9,11,13$ or
    congruent values modulo $14$.
  \end{solution}
\end{enumerate}

\Exercise{10} Prove that if $\sigma$ is the $m$-cycle
$(a_1\,a_2\,\dots\,a_m)$, then for all $i\in\{1,2,\dots,m\}$,
$\sigma^i(a_k) = a_{k+i}$, where $k+i$ is replaced by its least
positive residue mod $m$. Deduce that $\ord{\sigma} = m$.
\begin{proof}
  Fix a positive integer $m$. We will use induction on $i$ to show
  that $\sigma^i(a_k) = a_{k+i}$ for each positive integer $i$. Since
  $\sigma$ cyclically permutes $a_1,\dots,a_m$, we have
  $\sigma(a_k) = a_{k+1}$ for each $k$ (taking $a_{m+1} = a_1$), so
  the base case is satisfied.

  Now suppose $\sigma^i(a_k) = a_{k+i}$ for some positive integer
  $i$. Then
  \begin{align*}
    \sigma^{i+1}(a_k) &= \sigma(\sigma^i(a_k)) \\
                      &= \sigma(a_{k+i}) \\
                      &= a_{k+i+1},
  \end{align*}
  again replacing $k+i$ and $k+i+1$ with their least positive residues
  mod $m$. This completes the inductive step, so
  $\sigma^i(a_k) = a_{k+i}$ for each integer $i > 0$.

  Finally, if $1\leq i<m$ then $\sigma^i$ sends $a_1$ to
  $a_{1+i}\neq a_1$ so that $\sigma^i$ is not the identity. But
  $\sigma^m$ sends $a_k$ to $a_{k+m} = a_k$ for each $k$. Therefore
  $\sigma^m = 1$, which shows that $\ord{\sigma} = m$.
\end{proof}

\Exercise{11} Let $\sigma$ be the $m$-cycle $(1\,2\,\dots\,m)$. Show
that $\sigma^i$ is also an $m$-cycle if and only if $i$ is relatively
prime to $m$.
\begin{proof}
  Fix a value for $i$. For the remainder of the proof, let $k^*$
  denote the least positive residue of $k$ modulo $m$. That is, let
  $k^*$ be the smallest positive integer such that
  $k^*\equiv k\pmod m$.

  Now, if $(i,m) = 1$, then the residues $i^*$, $(2i)^*$, \dots,
  $((m-1)i)^*$ must be distinct. To see this, note that $i$ has a
  multiplicative inverse (by Proposition~4 of Section~0.3), so if $s$
  and $t$ are integers with $si\equiv ti\pmod m$, it follows that
  $s\equiv t\pmod m$. Now, observe that $\sigma^i(m) = i^*$,
  $\sigma^i(i^*) = (2i)^*$, and in general,
  $\sigma^i((ki)^*) = ((k+1)i)^*$. So $\sigma^i$ is the $m$-cycle
  \begin{equation*}
    \sigma^i = (m\;i^*\;(2i)^*\;(3i)^*\;\dots\;((m-1)i)^*).
  \end{equation*}

  To prove the other direction, suppose $\sigma^i$ is an $m$-cycle and
  let $d = (i, m)$. Then there are integers $x$ and $y$ such that
  $dx = i$ and $dy = m$. Then
  \begin{equation*}
    (\sigma^i)^y = (\sigma^{dx})^y
    = (\sigma^{dy})^x = (\sigma^m)^x = 1^x = 1.
  \end{equation*}
  Therefore $\ord{\sigma^i} \leq y$. But $\sigma^i$ is an $m$-cycle,
  so its order is $m$. Therefore $y = m$ and $d = 1$. Hence $i$ is
  relatively prime to $m$.
\end{proof}

\Exercise{12}
\begin{enumerate}
\item If $\tau = (1\;2)(3\;4)(5\;6)(7\;8)(9\;10)$ determine whether
  there is an $n$-cycle $\sigma$ ($n\geq10$) with $\tau = \sigma^k$
  for some integer $k$.
  \begin{solution}
    Consider the $n$-cycle
    \begin{equation*}
      \sigma = (1\;3\;5\;7\;9\;2\;4\;6\;8\;10).
    \end{equation*}
    Then $\sigma^5 = \tau$.
  \end{solution}
\item If $\tau = (1\;2)(3\;4\;5)$ determine whether there is an
  $n$-cycle ($n\geq5$) with $\tau = \sigma^k$ for some integer $k$.
  \begin{solution}
    Suppose that it is possible, and let $\sigma$ be an $n$-cycle such
    that $\sigma^k = \tau$.

    If $n>5$ then $\sigma^k$ must fix $6, 7, \dots$. But if $\sigma$
    is an $n$-cycle, then the only way $\sigma^k$ can fix any of these
    values is if it fixes every value, that is, if
    $\sigma^k = 1 \neq \tau$. Therefore we can suppose that $n = 5$.

    Now since $\sigma^k$ is not an $n$-cycle, we know by the previous
    exercise that $k$ is not relatively prime to $m$. But $n = 5$ is
    prime, so $5\mid k$ and there is an integer $\ell$ such that
    $k = 5\ell$. Then
    $\sigma^k = (\sigma^5)^\ell = 1^\ell = 1\neq\tau$. This is a
    contradiction, so our assumption that $\sigma$ exists was invalid.
  \end{solution}
\end{enumerate}
