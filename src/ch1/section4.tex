\section{Matrix Groups}

Let $F$ be a field and let $n\in\Z^+$.

\Exercise1 Prove that $\ord{GL_2(\F_2)} = 6$.
\begin{proof}
  Consider the matrix
  \begin{equation*}
    \begin{pmatrix}
      a & b \\
      c & d
    \end{pmatrix},
    \quad a,b,c,d\in\F_2.
  \end{equation*}
  The determinant of this matrix is $ad-bc$. To be nonzero, either $a$
  and $b$ are nonzero, or $b$ and $c$ are nonzero, but not both. So
  the members of $GL_2(\F_2)$ are
  \begin{equation*}
    \begin{pmatrix}
      1 & 0 \\
      0 & 1
    \end{pmatrix},
    \begin{pmatrix}
      1 & 1 \\
      0 & 1
    \end{pmatrix},
    \begin{pmatrix}
      1 & 0 \\
      1 & 1
    \end{pmatrix},
    \begin{pmatrix}
      0 & 1 \\
      1 & 0
    \end{pmatrix},
    \begin{pmatrix}
      1 & 1 \\
      1 & 0
    \end{pmatrix},
    \;\;\text{and}\;\;
    \begin{pmatrix}
      0 & 1 \\
      1 & 1
    \end{pmatrix}. \qedhere
  \end{equation*}
\end{proof}

\Exercise2 Write out all the elements of $GL_2(\F_2)$ and compute the
order of each element.
\begin{solution}
  Direct computation produces the following orders:
  \begin{center}
    \begin{tabular}{r|c|c|c|c|c|c}
      $A$
      &
      $\begin{pmatrix}
        1 & 0 \\ 0 & 1
      \end{pmatrix}$
      &
      $\begin{pmatrix}
        1 & 1 \\ 0 & 1
      \end{pmatrix}$
      &
      $\begin{pmatrix}
        1 & 0 \\ 1 & 1
      \end{pmatrix}$
      &
      $\begin{pmatrix}
        0 & 1 \\ 1 & 0
      \end{pmatrix}$
      &
      $\begin{pmatrix}
        1 & 1 \\ 1 & 0
      \end{pmatrix}$
      &
      $\begin{pmatrix}
        0 & 1 \\ 1 & 1
      \end{pmatrix}$
      \\ \hline
      $\ord{A}$ & 1 & 2 & 2 & 2 & 3 & 3
    \end{tabular}.
  \end{center}
\end{solution}

\Exercise3 Show that $GL_2(\F_2)$ is non-abelian.
\begin{proof}
  We have
  \begin{equation*}
    \begin{pmatrix}
      1 & 1 \\ 1 & 0
    \end{pmatrix}
    \begin{pmatrix}
      1 & 1 \\ 0 & 1
    \end{pmatrix}
    =
    \begin{pmatrix}
      1 & 0 \\
      1 & 1
    \end{pmatrix}
  \end{equation*}
  but
  \begin{equation*}
    \begin{pmatrix}
      1 & 1 \\ 0 & 1
    \end{pmatrix}
    \begin{pmatrix}
      1 & 1 \\ 1 & 0
    \end{pmatrix}
    =
    \begin{pmatrix}
      0 & 1 \\
      1 & 0
    \end{pmatrix}.
  \end{equation*}
  Therefore $GL_2(\F_2)$ is non-abelian.
\end{proof}

\Exercise4 Show that if $n$ is not prime then $\Z/n\Z$ is not a field.
\begin{proof}
  Let $n$ be a composite number, so that $n = ab$ with $a,b > 1$. Then
  $(a,n) = a>1$, so by Proposition~4 of Section~0.3, $a$ does not have
  a multiplicative inverse. And $a$ is nonzero, so $\Z/n\Z$ is not a
  field.
\end{proof}

\Exercise5 Show that $GL_n(F)$ is a finite group if and only if $F$
has a finite number of elements.
\begin{proof}
  First, if $F$ is finite, then $GL_n(F)$ must be finite since there
  are only finitely many $n\times n$ matrices with entries from $F$.

  On the other hand, suppose $F$ is not finite. Then for every
  $\alpha\in F$ with $\alpha\neq0$, the matrix $\alpha I$ has nonzero
  determinant. Therefore $GL_n(F)$ is infinite.
\end{proof}

\Exercise6 If $\ord{F} = q$ is finite, prove that
$\ord{GL_n(F)} < q^{n^2}$.
\begin{proof}
  Since $F$ has $q$ elements, there are only $q^{n^2}$ possible
  $n\times n$ matrices over $F$ that can be formed. Since at least one
  of these matrices has zero determinant (take for example the zero
  matrix), it follows that $\ord{GL_n(F)} < q^{n^2}$.
\end{proof}

\Exercise7 Let $p$ be a prime. Prove that the order of $GL_2(\F_p)$ is
$p^4 - p^3 - p^2 + p$.
\begin{proof}
  Let $A$ be a $2\times2$ matrix over $\F_p$ that is {\em not} in
  $GL_2(\F_p)$. Write
  \begin{equation*}
    A =
    \begin{pmatrix}
      a & b \\
      c & d
    \end{pmatrix},
  \end{equation*}
  and note that $ad - bc = 0$. We have two cases: $a = 0$ or
  $a\neq0$.

  First, if $a = 0$ then $d$ can take any of $p$ possible values,
  while $bc = 0$. Again there are two cases: if $b = 0$ then there are
  $p$ possible values that $c$ can take. If $b \neq 0$ (this can
  happen in $p-1$ ways), then $c = b^{-1}$. So there are $p$
  possibilities for $d$, multiplied by $p + (p - 1) = 2p - 1$
  possibilities for $b$ and $c$, which gives a total of $2p^2 - p$
  choices for the case where $a = 0$.

  Next, if $a \neq 0$, this can happen in $p - 1$ ways. Then
  $d = bca^{-1}$. Now $b$ and $c$ can take any value and then $d$ is
  determined by the other variables, so there are
  $p^2(p-1) = p^3 - p^2$ possibilities for this case.

  Totaling the two cases, we find that there are
  \begin{equation*}
    (p^3 - p^2) + (2p^2 - p) = p^3 + p^2 - p
  \end{equation*}
  possible matrices that $A$ can be. Since there are $p^4$ total
  $2\times2$ matrices over $\F_p$, it follows that
  \begin{equation*}
    \ord{GL_2(\F_p)} = p^4 - p^3 - p^2 + p. \qedhere
  \end{equation*}
\end{proof}
