\section{Matrix Groups}

Let $F$ be a field and let $n\in\Z^+$.

\Exercise1 Prove that $\ord{GL_2(\F_2)} = 6$.
\begin{proof}
  Consider the matrix
  \begin{equation*}
    \begin{pmatrix}
      a & b \\
      c & d
    \end{pmatrix},
    \quad a,b,c,d\in\F_2.
  \end{equation*}
  The determinant of this matrix is $ad-bc$. To be nonzero, either $a$
  and $b$ are nonzero, or $b$ and $c$ are nonzero, but not both. So
  the members of $GL_2(\F_2)$ are
  \begin{equation*}
    \begin{pmatrix}
      1 & 0 \\
      0 & 1
    \end{pmatrix},
    \begin{pmatrix}
      1 & 1 \\
      0 & 1
    \end{pmatrix},
    \begin{pmatrix}
      1 & 0 \\
      1 & 1
    \end{pmatrix},
    \begin{pmatrix}
      0 & 1 \\
      1 & 0
    \end{pmatrix},
    \begin{pmatrix}
      1 & 1 \\
      1 & 0
    \end{pmatrix},
    \;\;\text{and}\;\;
    \begin{pmatrix}
      0 & 1 \\
      1 & 1
    \end{pmatrix}. \qedhere
  \end{equation*}
\end{proof}

\Exercise2 Write out all the elements of $GL_2(\F_2)$ and compute the
order of each element.
\begin{solution}
  Direct computation produces the following orders:
  \begin{center}
    \begin{tabular}{r|c|c|c|c|c|c}
      $A$
      &
      $\begin{pmatrix}
        1 & 0 \\ 0 & 1
      \end{pmatrix}$
      &
      $\begin{pmatrix}
        1 & 1 \\ 0 & 1
      \end{pmatrix}$
      &
      $\begin{pmatrix}
        1 & 0 \\ 1 & 1
      \end{pmatrix}$
      &
      $\begin{pmatrix}
        0 & 1 \\ 1 & 0
      \end{pmatrix}$
      &
      $\begin{pmatrix}
        1 & 1 \\ 1 & 0
      \end{pmatrix}$
      &
      $\begin{pmatrix}
        0 & 1 \\ 1 & 1
      \end{pmatrix}$
      \\ \hline
      $\ord{A}$ & 1 & 2 & 2 & 2 & 3 & 3
    \end{tabular}.
  \end{center}
\end{solution}

\Exercise3 Show that $GL_2(\F_2)$ is non-abelian.
\label{exercise-gen-lin-over-f2-non-abelian}
\begin{proof}
  We have
  \begin{equation*}
    \begin{pmatrix}
      1 & 1 \\ 1 & 0
    \end{pmatrix}
    \begin{pmatrix}
      1 & 1 \\ 0 & 1
    \end{pmatrix}
    =
    \begin{pmatrix}
      1 & 0 \\
      1 & 1
    \end{pmatrix}
  \end{equation*}
  but
  \begin{equation*}
    \begin{pmatrix}
      1 & 1 \\ 0 & 1
    \end{pmatrix}
    \begin{pmatrix}
      1 & 1 \\ 1 & 0
    \end{pmatrix}
    =
    \begin{pmatrix}
      0 & 1 \\
      1 & 0
    \end{pmatrix}.
  \end{equation*}
  Therefore $GL_2(\F_2)$ is non-abelian.
\end{proof}

\Exercise4 Show that if $n$ is not prime then $\Z/n\Z$ is not a field.
\begin{proof}
  Let $n$ be a composite number, so that $n = ab$ with $a,b > 1$. Then
  $(a,n) = a>1$, so by Proposition~4 of Section~0.3, $a$ does not have
  a multiplicative inverse. And $a$ is nonzero, so $\Z/n\Z$ is not a
  field.
\end{proof}

\Exercise5 Show that $GL_n(F)$ is a finite group if and only if $F$
has a finite number of elements.
\begin{proof}
  First, if $F$ is finite, then $GL_n(F)$ must be finite since there
  are only finitely many $n\times n$ matrices with entries from $F$.

  On the other hand, suppose $F$ is not finite. Then for every
  $\alpha\in F$ with $\alpha\neq0$, the matrix $\alpha I$ has nonzero
  determinant. Therefore $GL_n(F)$ is infinite.
\end{proof}

\Exercise6 If $\ord{F} = q$ is finite, prove that
$\ord{GL_n(F)} < q^{n^2}$.
\begin{proof}
  Since $F$ has $q$ elements, there are only $q^{n^2}$ possible
  $n\times n$ matrices over $F$ that can be formed. Since at least one
  of these matrices has zero determinant (take for example the zero
  matrix), it follows that $\ord{GL_n(F)} < q^{n^2}$.
\end{proof}

\Exercise7 Let $p$ be a prime. Prove that the order of $GL_2(\F_p)$ is
$p^4 - p^3 - p^2 + p$.
\begin{proof}
  Let $A$ be a $2\times2$ matrix over $\F_p$ that is {\em not} in
  $GL_2(\F_p)$. Write
  \begin{equation*}
    A =
    \begin{pmatrix}
      a & b \\
      c & d
    \end{pmatrix},
  \end{equation*}
  and note that $ad - bc = 0$. We have two cases: $a = 0$ or
  $a\neq0$.

  First, if $a = 0$ then $d$ can take any of $p$ possible values,
  while $bc = 0$. Again there are two cases: if $b = 0$ then there are
  $p$ possible values that $c$ can take. If $b \neq 0$ (this can
  happen in $p-1$ ways), then $c = b^{-1}$. So there are $p$
  possibilities for $d$, multiplied by $p + (p - 1) = 2p - 1$
  possibilities for $b$ and $c$, which gives a total of $2p^2 - p$
  choices for the case where $a = 0$.

  Next, if $a \neq 0$, this can happen in $p - 1$ ways. Then
  $d = bca^{-1}$. Now $b$ and $c$ can take any value and then $d$ is
  determined by the other variables, so there are
  $p^2(p-1) = p^3 - p^2$ possibilities for this case.

  Totaling the two cases, we find that there are
  \begin{equation*}
    (p^3 - p^2) + (2p^2 - p) = p^3 + p^2 - p
  \end{equation*}
  possible matrices that $A$ can be. Since there are $p^4$ total
  $2\times2$ matrices over $\F_p$, it follows that
  \begin{equation*}
    \ord{GL_2(\F_p)} = p^4 - p^3 - p^2 + p. \qedhere
  \end{equation*}
\end{proof}

\Exercise8 Show that $GL_n(F)$ is non-abelian for any $n\geq2$ and any
$F$.
\begin{proof}
  Note that every field has an additive identity $0$ and a distinct
  multiplicative identity $1$, so by restricting our proof to using
  only these two values from $F$, the result will hold for any $F$.

  We will use induction on $n$. The base case $n = 2$ was proved in
  Exercise~\ref{exercise-gen-lin-over-f2-non-abelian} (the proof works
  for any $F$ as noted above). Now assume that $GL_{n-1}(F)$ is
  non-abelian for some $n\geq3$, and let $A$ and $B$ be non-commuting
  members of $GL_{n-1}(F)$. Then, using block matrices, we get
  \begin{equation*}
    \begin{pmatrix}
      A & 0 \\
      0 & 1
    \end{pmatrix}
    \begin{pmatrix}
      B & 0 \\
      0 & 1
    \end{pmatrix}
    =
    \begin{pmatrix}
      AB & 0 \\
      0 & 1
    \end{pmatrix}
    \neq
    \begin{pmatrix}
      BA & 0 \\
      0 & 1
    \end{pmatrix}
    =
    \begin{pmatrix}
      B & 0 \\
      0 & 1
    \end{pmatrix}
    \begin{pmatrix}
      A & 0 \\
      0 & 1
    \end{pmatrix}.
  \end{equation*}
  Therefore $GL_n(F)$ is non-abelian, and this completes the proof.
\end{proof}

\Exercise9 Prove that the binary operation of matrix multiplication of
$2\times2$ matrices with real number entries is associative.
\begin{proof}
  Direct computation gives
  \begin{multline*}
    \left[
      \begin{pmatrix}
        a & b \\
        c & d
      \end{pmatrix}
      \begin{pmatrix}
        e & f \\
        g & h
      \end{pmatrix}
    \right]
    \begin{pmatrix}
      i & j \\
      k & l
    \end{pmatrix}
    =
    \begin{pmatrix}
      ae+bg & af+bh \\
      ce+dg & cf+dh
    \end{pmatrix}
    \begin{pmatrix}
      i & j \\
      k & l
    \end{pmatrix} \\
    =
    \begin{pmatrix}
      (ae+bg)i + (af+bh)k & (ae+bg)j + (af+bh)l \\
      (ce+dg)i + (cf+dh)k & (ce+dg)j + (cf+dh)l
    \end{pmatrix},
  \end{multline*}
  while
  \begin{multline*}
    \begin{pmatrix}
      a & b \\
      c & d
    \end{pmatrix}
    \left[
      \begin{pmatrix}
        e & f \\
        g & h
      \end{pmatrix}
      \begin{pmatrix}
        i & j \\
        k & l
      \end{pmatrix}
    \right]
    =
    \begin{pmatrix}
      a & b \\
      c & d
    \end{pmatrix}
    \begin{pmatrix}
      ei + fk & ej + fl \\
      gi + hk & gj + hl
    \end{pmatrix} \\
    =
    \begin{pmatrix}
      a(ei+fk) + b(gi+hk) & a(ej+fl) + b(gj+hl) \\
      c(ei+fk) + d(gi+hk) & c(ej+fl) + d(gj+hl)
    \end{pmatrix}.
  \end{multline*}
  Now, by comparing these two matrices using the associative and
  commutative properties of the real numbers, the result will follow.
\end{proof}

\Exercise{10} Let
\begin{equation*}
  G = \left\{
    \begin{pmatrix}
      a & b \\
      0 & c
    \end{pmatrix}
    \;\middle|\;
    a,b,c\in\R, a\neq0, c\neq0\right\}.
\end{equation*}
\begin{enumerate}
\item Compute the product of
  $\begin{pmatrix} a_1 & b_1 \\ 0 & c_1 \end{pmatrix}$ and
  $\begin{pmatrix} a_2 & b_2 \\ 0 & c_2 \end{pmatrix}$ to show that
  $G$ is closed under matrix multiplication.
  \begin{solution}
    We have
    \begin{equation*}
      \begin{pmatrix} a_1 & b_1 \\ 0 & c_1 \end{pmatrix}
      \begin{pmatrix} a_2 & b_2 \\ 0 & c_2 \end{pmatrix}
      =
      \begin{pmatrix}
        a_1a_2 & a_1b_2 + b_1c_2 \\
        0 & c_1c_2
      \end{pmatrix} \in G,
    \end{equation*}
    so $G$ is closed under multiplication.
  \end{solution}
\item Find the matrix inverse of
  $\begin{pmatrix} a & b \\ 0 & c \end{pmatrix}$ and deduce that $G$
  is closed under inverses.
  \begin{solution}
    Since $ac\neq0$ the matrix is invertible and we get
    \begin{equation*}
      \begin{pmatrix}
        a & b \\ 0 & c
      \end{pmatrix}^{-1}
      = \frac1{ac}
      \begin{pmatrix}
        c & -b \\ 0 & a
      \end{pmatrix}
      =
      \begin{pmatrix}
        \frac1a & -\frac{b}{ac} \\[3pt]
        0 & \frac1c
      \end{pmatrix} \in G.
    \end{equation*}
    So, $G$ is closed under inverses.
  \end{solution}
\item Deduce that $G$ is a subgroup of $GL_2(\R)$.
  \begin{solution}
    This follows from Exercise~\ref{exercise-subgroup-conditions}.
  \end{solution}
\item Prove that the set of elements of $G$ whose two diagonal entries
  are equal (i.e., $a = c$) is also a subgroup of $GL_2(\R)$.
  \begin{solution}
    Call this set $H$. We have
    \begin{equation*}
      \begin{pmatrix}
        a_1 & b_1 \\ 0 & a_1
      \end{pmatrix}
      \begin{pmatrix}
        a_2 & b_2 \\ 0 & a_2
      \end{pmatrix}
      =
      \begin{pmatrix}
        a_1a_2 & a_1b_2 + b_1a_2 \\
        0 & a_1a_2
      \end{pmatrix}
      \in H
    \end{equation*}
    and
    \begin{equation*}
      \begin{pmatrix}
        a & b \\ 0 & a
      \end{pmatrix}^{-1}
      =
      \begin{pmatrix}
        \frac1a & -\frac{b}{a^2} \\[3pt]
        0 & \frac1a
      \end{pmatrix}
      \in H.
    \end{equation*}
    $H$ is closed under matrix multiplication and inversion, so $H$ is
    a subgroup of $GL_2(\R)$.
  \end{solution}
\end{enumerate}

\Exercise{11} Let
\begin{equation*}
  H(F) = \left\{
    \begin{pmatrix}
      1 & a & b \\
      0 & 1 & c \\
      0 & 0 & 1
    \end{pmatrix}
    \;\middle|\;
    a,b,c\in F
  \right\}.
\end{equation*}
Let
\begin{equation*}
  X =
  \begin{pmatrix}
    1 & a & b \\
    0 & 1 & c \\
    0 & 0 & 1
  \end{pmatrix}
  \quad\text{and}\quad
  Y =
  \begin{pmatrix}
    1 & d & e \\
    0 & 1 & f \\
    0 & 0 & 1
  \end{pmatrix}
\end{equation*}
be elements of $H(F)$.
\begin{enumerate}
\item Compute the matrix $XY$ and deduce that $H(F)$ is closed under
  matrix multiplication. Exhibit explicit matrices such that
  $XY\neq YX$ (so that $H(F)$ is always non-abelian).
  \begin{solution}
    We have
    \begin{equation*}
      XY =
      \begin{pmatrix}
        1 & a & b \\
        0 & 1 & c \\
        0 & 0 & 1
      \end{pmatrix}
      \begin{pmatrix}
        1 & d & e \\
        0 & 1 & f \\
        0 & 0 & 1
      \end{pmatrix}
      =
      \begin{pmatrix}
        1 & a + d & af + b + e \\
        0 & 1 & c + f \\
        0 & 0 & 1
      \end{pmatrix}
      \in H(F),
    \end{equation*}
    so $H(F)$ is closed under multiplication. Moreover,
    \begin{equation*}
      \begin{pmatrix}
        1 & 1 & 0 \\
        0 & 1 & 0 \\
        0 & 0 & 1
      \end{pmatrix}
      \begin{pmatrix}
        1 & 0 & 0 \\
        0 & 1 & 1 \\
        0 & 0 & 1
      \end{pmatrix}
      =
      \begin{pmatrix}
        1 & 1 & 1 \\
        0 & 1 & 1 \\
        0 & 0 & 1
      \end{pmatrix}
    \end{equation*}
    while
    \begin{equation*}
      \begin{pmatrix}
        1 & 0 & 0 \\
        0 & 1 & 1 \\
        0 & 0 & 1
      \end{pmatrix}
      \begin{pmatrix}
        1 & 1 & 0 \\
        0 & 1 & 0 \\
        0 & 0 & 1
      \end{pmatrix}
      =
      \begin{pmatrix}
        1 & 1 & 0 \\
        0 & 1 & 1 \\
        0 & 0 & 1
      \end{pmatrix},
    \end{equation*}
    so $H(F)$ is always non-abelian.
  \end{solution}
\item Find an explicit formula for the matrix inverse $X^{-1}$ and
  deduce that $H(F)$ is closed under inverses.
  \begin{solution}
    Let
    \begin{equation*}
      Z =
      \begin{pmatrix}
        1 & -a & ca - b \\
        0 & 1 & -c \\
        0 & 0 & 1
      \end{pmatrix}.
    \end{equation*}
    By performing the multiplication, it is easily seen that
    $XZ = ZX = I$, where $I$ is the $3\times3$ identity matrix. It
    follows that $Z = X^{-1}$ and since $Z\in H(F)$, we see that
    $H(F)$ is closed under inverses.
  \end{solution}
\item Prove the associative law for $H(F)$ and deduce that $H(F)$ is a
  group of order $\ord{F}^3$.
  \begin{solution}
    We have
    \begin{multline*}
      \left[
        \begin{pmatrix}
          1 & a & b \\
          0 & 1 & c \\
          0 & 0 & 1
        \end{pmatrix}
        \begin{pmatrix}
          1 & d & e \\
          0 & 1 & f \\
          0 & 0 & 1
        \end{pmatrix}
      \right]
      \begin{pmatrix}
        1 & g & h \\
        0 & 1 & i \\
        0 & 0 & 1
      \end{pmatrix} \\
      =
      \begin{pmatrix}
        1 & a + d & af + b + e \\
        0 & 1 & c + f \\
        0 & 0 & 1
      \end{pmatrix}
      \begin{pmatrix}
        1 & g & h \\
        0 & 1 & i \\
        0 & 0 & 1
      \end{pmatrix} \\
      =
      \begin{pmatrix}
        1 & a + d + g & af + ai + b + di + e + h \\
        0 & 1 & c + f + i \\
        0 & 0 & 1
      \end{pmatrix}
    \end{multline*}
    and
    \begin{multline*}
      \begin{pmatrix}
        1 & a & b \\
        0 & 1 & c \\
        0 & 0 & 1
      \end{pmatrix}
      \left[
        \begin{pmatrix}
          1 & d & e \\
          0 & 1 & f \\
          0 & 0 & 1
        \end{pmatrix}
        \begin{pmatrix}
          1 & g & h \\
          0 & 1 & i \\
          0 & 0 & 1
        \end{pmatrix}
      \right] \\
      =
      \begin{pmatrix}
        1 & a & b \\
        0 & 1 & c \\
        0 & 0 & 1
      \end{pmatrix}
      \begin{pmatrix}
        1 & d + g & di + e + h \\
        0 & 1 & f + i \\
        0 & 0 & 1
      \end{pmatrix} \\
      =
      \begin{pmatrix}
        1 & a + d + g & af + ai + b + di + e + h \\
        0 & 1 & c + f + i \\
        0 & 0 & 1
      \end{pmatrix},
    \end{multline*}
    so multiplication in $H(F)$ is associative. This shows that $H(F)$
    is a subgroup of $GL_3(F)$.

    Now, consider the matrix $X$ above. If $\ord{F} = n < \infty$ then
    each of $a,b,c$ can take any of $n$ values each. So
    $\ord{H(F)} = n^3$.
  \end{solution}
\item Find the order of each element of the finite group $H(\Z/2\Z)$.
  \begin{solution}
    Obviously $\ord{I} = 1$. For the rest, we find
    \begin{equation*}
      \begin{pmatrix}
        1 & 0 & 0 \\
        0 & 1 & 1 \\
        0 & 0 & 1
      \end{pmatrix}^2
      =
      \begin{pmatrix}
        1 & 1 & 0 \\
        0 & 1 & 0 \\
        0 & 0 & 1
      \end{pmatrix}^2
      =
      \begin{pmatrix}
        1 & 0 & 1 \\
        0 & 1 & 0 \\
        0 & 0 & 1
      \end{pmatrix}^2
      = I,
    \end{equation*}
    so these have order $2$,
    \begin{equation*}
      \begin{pmatrix}
        1 & 1 & 1 \\
        0 & 1 & 0 \\
        0 & 0 & 1
      \end{pmatrix}^2
      =
      \begin{pmatrix}
        1 & 0 & 1 \\
        0 & 1 & 1 \\
        0 & 0 & 1
      \end{pmatrix}^2
      = I,
    \end{equation*}
    so these also have order $2$, and
    \begin{equation*}
      \begin{pmatrix}
        1 & 1 & 0 \\
        0 & 1 & 1 \\
        0 & 0 & 1
      \end{pmatrix}^4
      =
      \begin{pmatrix}
        1 & 1 & 1 \\
        0 & 1 & 1 \\
        0 & 0 & 1
      \end{pmatrix}^4
      = I,
    \end{equation*}
    so these have order $4$. $\ord{H(\Z/2\Z)} = 2^3 = 8$, so these are
    all the elements in the group.
  \end{solution}
\item Prove that every nonidentity element of the group $H(\R)$ has
  infinite order.
  \begin{solution}
    We will show by induction on $n$ that
    \begin{equation}
      \label{eq:heisenberg-power}
      \begin{pmatrix}
        1 & a & b \\
        0 & 1 & c \\
        0 & 0 & 1
      \end{pmatrix}^n
      =
      \begin{pmatrix}
        1 & na & nb + n(n-1)ac/2 \\
        0 & 1 & nc \\
        0 & 0 & 1
      \end{pmatrix}.
    \end{equation}
    Since any nonidentity element has one of $a$, $b$, or $c$ nonzero,
    this will be enough to show that the element has infinite order.

    The base case $n = 1$ is evident. Suppose
    \eqref{eq:heisenberg-power} holds for some $n\geq 1$. Then
    \begin{multline*}
      \begin{pmatrix}
        1 & a & b \\
        0 & 1 & c \\
        0 & 0 & 1
      \end{pmatrix}^{n+1}
      =
      \begin{pmatrix}
        1 & a & b \\
        0 & 1 & c \\
        0 & 0 & 1
      \end{pmatrix}
      \begin{pmatrix}
        1 & na & nb + n(n-1)ac/2 \\
        0 & 1 & nc \\
        0 & 0 & 1
      \end{pmatrix} \\
      =
      \begin{pmatrix}
        1 & (n+1)a & (n+1)b + n(n+1)ac/2 \\
        0 & 1 & (n+1)c \\
        0 & 0 & 1
      \end{pmatrix},
    \end{multline*}
    so $\eqref{eq:heisenberg-power}$ holds for all positive integers
    $n$ and the result follows.
  \end{solution}
\end{enumerate}
