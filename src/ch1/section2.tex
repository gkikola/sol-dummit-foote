\section{Dihedral Groups}

In these exercises, $D_{2n}$ has the usual presentation
\begin{equation*}
  D_{2n} = \langle r,s\mid r^n = s^2 = 1, rs = sr^{-1}\rangle.
\end{equation*}

\Exercise1 Compute the order of each of the elements in the following
groups:
\begin{enumerate}
\item $D_6$
  \begin{solution}
    The elements of $D_6$ are $1$, $r$, $r^2$, $s$, $sr$, and
    $sr^2$. We have $\ord{1} = 1$, $\ord{r} = 3$, $\ord{r^2} = 3$, and
    $\ord{s} = 2$. Since $(sr)^2 = srsr = s^2r^{-1}r = 1$ and
    $(sr^2)^2 = sr^2sr^2 = s^2r^{-2}r^2 = 1$, we have
    $\ord{sr} = \ord{sr^2} = 2$.
  \end{solution}
\item $D_8$
  \begin{solution}
    Again we have $\ord{1} = 1$, $\ord{r} = 4$, $\ord{r^2} = 2$, and
    $\ord{r^3} = 4$. For $k$ with $0\leq k\leq3$, we have
    $(sr^k)^2 = sr^ksr^k = s^2r^{-k}r^k = 1$ so $\ord{sr^k} = 2$.
  \end{solution}
\item $D_{10}$
  \begin{solution}
    Since $5$ is prime, we have for each $k$ with $0\leq k\leq4$,
    \begin{equation*}
      \ord{r^k} = 5
      \quad\text{and}\quad
      \ord{sr^k} = 2. \qedhere
    \end{equation*}
  \end{solution}
\end{enumerate}

\Exercise2 Use the generators and relations above to show that if $x$
is any element of $D_{2n}$ which is not a power of $r$, then
$rx = xr^{-1}$.
\begin{proof}
  Since any element of $D_{2n}$ that is not a power of $r$ has the
  form $sr^k$ for some integer $k$, we have
  \begin{equation*}
    rx = rsr^k = sr^{-1}r^k = sr^{-k} = sr^kr^{-1} = xr^{-1}. \qedhere
  \end{equation*}
\end{proof}

\Exercise3 Use the generators and relations above to show that every
element of $D_{2n}$ which is not a power of $r$ has order $2$. Deduce
that $D_{2n}$ is generated by the two elements $s$ and $sr$, both of
which have order $2$.
\label{exercise-dihedral-gen}
\begin{proof}
  As in the previous exercise, such elements have the form
  $sr^k$. $sr^k$ is distinct from the identity, and
  \begin{equation*}
    (sr^k)^2 = sr^ksr^k = s^2r^{-k}r^k = 1,
  \end{equation*}
  so $\ord{sr^k} = 2$.

  Now, the elements of $D_{2n}$ are $1, r, r^2, \dots, r^n$, and
  $s, sr, \dots, sr^n$. Each $r^k$ can be written as $(s(sr))^k$, and
  each $sr^k$ can be written as $s(s(sr))^k$, so $D_{2n}$ is generated
  by $\{ s, sr \}$, each element of which has order $2$.
\end{proof}

\Exercise4 If $n = 2k$ is even and $n\geq4$, show that $z = r^k$ is an
element of order $2$ which commutes with all elements of
$D_{2n}$. Show also that $z$ is the only nonidentity element of
$D_{2n}$ which commutes with all elements of $D_{2n}$.
\begin{proof}
  $r$ has order $n$, so by Exercise~\ref{exercise-power-own-inverse},
  we know that $z = z^{-1}$, that is, $r^k = r^{-k}$. Let
  $x\in D_{2n}$ be arbitrary. Then $x$ can be written either $r^\ell$
  or $sr^\ell$ for some integer $\ell$. In the first case,
  \begin{equation*}
    zx = r^kr^\ell = r^{k+\ell} = r^\ell r^k = xz,
  \end{equation*}
  and in the second case,
  \begin{equation*}
    zx = r^ksr^\ell = sr^{-k}r^\ell = sr^kr^\ell = sr^\ell r^k = xz.
  \end{equation*}
  This shows that $z$ commutes with each element of $D_{2n}$.

  Now suppose $z'$ is any nonidentity element in $D_{2n}$ which
  commutes with every element in $D_{2n}$. Then in particular $z'$
  commutes with $s$. So if $z' = r^t$ for some integer $t$, then
  $z's = sz'$, and
  \begin{equation*}
    z's = r^ts = sr^{-t}.
  \end{equation*}
  Therefore $r^t = r^{-t}$. By
  Exercise~\ref{exercise-power-own-inverse}, we must have $t = k$. On
  the other hand, if $z' = sr^t$, then
  \begin{equation*}
    z's = sr^ts = s^2r^{-t} = r^{-t}.
  \end{equation*}
  So $sr^t = r^{-t}$, but this is impossible, since a reflection
  cannot also be a rotation. Therefore $z' = z$ and $z$ is the only
  nonidentity element which commutes with all elements in the group.
\end{proof}

\Exercise5 If $n$ is odd and $n\geq3$, show that the identity is the
only element of $D_{2n}$ which commutes with all elements of $D_{2n}$.
\begin{proof}
  The proof is essentially the same as in the previous exercise,
  except the odd case from Exercise~\ref{exercise-power-own-inverse}
  is used instead of the even one.
\end{proof}

\Exercise6 Let $x$ and $y$ be elements of order $2$ in any group
$G$. Prove that if $t = xy$ then $tx = xt^{-1}$ (so that if
$n = \ord{xy} < \infty$ then $x, t$ satisfy the same relations in $G$
as $s,r$ do in $D_{2n}$).
\label{exercise-x-y-order-2-satisfy-dihedral-relations}
\begin{proof}
  Note that $x = x^{-1}$ and $y = y^{-1}$. If $t = xy$ then
  \begin{equation*}
    tx = xyx = xy^{-1}x^{-1} = x(xy)^{-1} = xt^{-1}. \qedhere
  \end{equation*}
\end{proof}

\Exercise7 Show that $\langle a,b\mid a^2 = b^2 = (ab)^n = 1\rangle$
gives a presentation in $D_{2n}$ in terms of the two generators
$a = s$ and $b = sr$ of order $2$ computed in
Exercise~\ref{exercise-dihedral-gen} above.
\begin{proof}
  Suppose $a^2 = b^2 = (ab)^n = 1$. Then $s^2 = a^2 = 1$ and
  $r^n = (s^2r)^n = (ab)^n = 1$. Since $b^2 = 1$, we have $srsr =
  1$. Multiplying each side of this equation on the right by $r^{-1}$
  and then on the left by $s$ gives $s^2(rs)1 = sr^{-1}$ or
  $rs = sr^{-1}$. This shows that the relations $s^2 = r^n = 1$ and
  $rs = sr^{-1}$ follow from the relations for $a$ and $b$.

  Now suppose $s^2 = r^n = 1$ and $rs = sr^{-1}$. Then
  $a^2 = s^2 = 1$, $b^2 = srsr = s^2r^{-1}r = 1$, and
  $(ab)^n = (s(sr))^n = (s^2r)^n = r^n = 1$. Therefore the relations
  for $a$ and $b$ follow from those for $r$ and $s$, so that the above
  is a presentation for $D_{2n}$ in terms of $a$ and $b$.
\end{proof}

\Exercise8 Find the order of the cyclic subgroup of $D_{2n}$ generated
by $r$.
\begin{solution}
  Let $G = \innerp{r}$ be the cyclic subgroup of $D_{2n}$ generated by
  $r$. Then each element of $G$ can be written $r^k$ for some integer
  $k$. If $k>0$ then $r^k$ is a clockwise rotation about the origin by
  $2k\pi/n$ radians. If $k<0$, then $r^k$ is a rotation
  counterclockwise by $-2k\pi/n$ radians. If $\ord{k}\geq n$, then the
  rotation is equivalent to a rotation $r^\ell$ where
  $0\leq\ell<n$. And $1, r, r^2, \dots, r^{n-1}$ are distinct, so $G$
  is given by
  \begin{equation*}
    G = \{ 1, r, r^2, \dots, r^{n-1} \},
  \end{equation*}
  and we have $\ord{G} = n$.
\end{solution}

\Exercise9 Let $G$ be the group of rigid motions in $\R^3$ of a
tetrahedron. Show that $\ord{G} = 12$.
\begin{proof}
  A tetrahedron has $4$ vertices. Label them from $1$ to $4$. Then a
  rigid motion in $G$ can send vertex $1$ to $4$ possible places. Once
  the new position of vertex $1$ has been chosen, there are three
  adjacent vertices at which to place vertex $2$. The positions of the
  remaining two vertices will then be completely determined by the
  positions of the first two. Therefore there are $4(3) = 12$ possible
  symmetries, so $\ord{G} = 12$.
\end{proof}

\Exercise{10} Let $G$ be the group of rigid motions in $\R^3$ of a
cube. Show that $\ord{G} = 24$.
\begin{proof}
  A cube has $8$ vertices, and each vertex has $3$ adjacent
  vertices. So there are $8$ possibilities for the position of the
  first vertex, followed by $3$ possibilities for the position of the
  second, resulting in $8(3) = 24$ symmetries. So $\ord{G} = 24$.
\end{proof}

\Exercise{11} Let $G$ be the group of rigid motions in $\R^3$ of an
octahedron. Show that $\ord{G} = 24$.
\begin{proof}
  An octahedron has $6$ vertices and each vertex has $4$ adjacent
  vertices. So, using the same reasoning as in the previous two
  exercises, we get $\ord{G} = 6(4) = 24$.
\end{proof}

\Exercise{12} Let $G$ be the group of rigid motions in $\R^3$ of a
dodecahedron. Show that $\ord{G} = 60$.
\begin{proof}
  We have $20$ vertices, and each vertex has $3$ neighboring
  vertices. So $\ord{G} = 20(3) = 60$.
\end{proof}

\Exercise{13} Let $G$ be the group of rigid motions in $\R^3$ of an
icosahedron. Show that $\ord{G} = 60$.
\begin{proof}
  We have $12$ vertices, with each vertex adjacent to $5$ vertices,
  giving $\ord{G} = 12(5) = 60$.
\end{proof}

\Exercise{14} Find a set of generators for $\Z$.
\begin{solution}
  $\Z$ is generated by $\{1\}$ since each $n\in\Z$ can be written as
  $n1 = n$.
\end{solution}

\Exercise{15} Find a set of generators and relations for $\Z/n\Z$.
\begin{proof}
  Similar to the previous exercise, $\left\{\bar1\right\}$ can
  generate $\Z/n\Z$ since every element can be expressed as a repeated
  addition of $\bar1$. $\bar1$ satisfies the relation
  $n\bar1 = \bar0$.
\end{proof}

\Exercise{16} Show that the group
$\langle x_1,y_1\mid x_1^2 = y_1^2 = (x_1y_1)^2 = 1\rangle$ is the
dihedral group $D_4$ (where $x_1$ may be replaced by the letter $r$
and $y_1$ by $s$).
\begin{proof}
  $D_4$ has the usual presentation
  $\langle r,s\mid r^2 = s^2 = 1, rs = sr^{-1}\rangle$. Since
  $r = r^{-1}$, that last relation become $rs = sr$. Let $x_1 = r$ and
  $y_1 = s$. Then $rs = sr$ implies
  $(x_1y_1)^2 = (rs)^2 = rsrs = r^2s^2 = 1$. On the other hand, if
  $(rs)^2 = 1$, then $rsrs = 1$ and, multiplying on the left by $r$
  and on the right by $s$, this becomes $sr = rs$. So the relations
  $rs = sr$ and $(x_1y_1)^2 = 1$ are equivalent. Therefore the above
  group is $D^4$.
\end{proof}

\Exercise{17} Let $X_{2n}$ be the group with presentation
\begin{equation*}
  X_{2n} = \langle x, y \mid x^n = y^2 = 1,\,xy = yx^2\rangle.
\end{equation*}
\begin{enumerate}
\item Show that if $n = 3k$ then $X_{2n}$ has order $6$, and it has
  the same generators and relations as $D_6$ when $x$ is replaced by
  $r$ and $y$ by $s$.
  \begin{proof}
    Let $n = 3k$. Suppose $x^n = y^2 = 1$ and that $xy =
    yx^2$. Note that, as shown in the text,
    \begin{equation*}
      x = xy^2 = yx^2y = yxyx^2 = y^2x^4 = x^4
    \end{equation*}
    and the cancellation law implies $x^3 = 1$. Letting $x = r$ and
    $y = s$, we then have $r^3 = s^2 = 1$ and
    \begin{equation*}
      rs = xy = yx^2 = sr^2 = sr^{-1}.
    \end{equation*}

    Now suppose that $r^3 = s^2 = 1$ and $rs = sr^{-1}$. Then
    \begin{equation*}
      x^n = x^{3k} = (r^3)^k = 1^k = 1,
    \end{equation*}
    $y^2 = s^2 = 1$, and
    \begin{equation*}
      xy = rs = sr^{-1} = sr^2 = yx^2.
    \end{equation*}

    Since the generators and relations are the same, $X_{2n}$ is $D_6$
    and has order $6$.
  \end{proof}
\item Show that if $(3,n) = 1$, then $x$ satisfies the additional
  relation: $x = 1$. In this case deduce that $X_{2n}$ has order $2$.
  \begin{proof}
    Using the same argument as in the previous part, we must have
    $x^3 = 1$. If $(3,n) = 1$ then either $n = 3k + 1$ or $n = 3k + 2$
    for some integer $k$. If $n = 3k + 1$ then
    \begin{equation*}
      x = 1^kx = (x^3)^kx = x^{3k}x = x^{3k+1} = x^n = 1,
    \end{equation*}
    and if $n = 3k + 2$ then
    \begin{equation*}
      x^{-1} = 1^{k+1}x^{-1} = (x^3)^{k+1}x^{-1} = x^{3k+3}x^{-1}
      = x^{3k+2} = x^n = 1.
    \end{equation*}
    But if $x^{-1} = 1$ then $x = 1$. In either case, the relation
    $x = 1$ holds so $X_{2n}$ is the set $\{1, y\}$ with the relation
    $y^2 = 1$, so $\ord{X_{2n}} = 2$.
  \end{proof}
\end{enumerate}

\Exercise{18} Let $Y$ be the group with presentation
\begin{equation*}
  Y = \langle u,v\mid u^4 = v^3 = 1,\, uv = v^2u^2\rangle.
\end{equation*}
\begin{enumerate}
\item Show that $v^2 = v^{-1}$.
  \begin{proof}
    $v(v^2) = v^3 = 1$ which implies $v^2 = v^{-1}$.
  \end{proof}
\item Show that $v$ commutes with $u^3$.
  \begin{proof}
    Since
    \begin{equation*}
      v^2u^3v = (v^2u^2)(uv) = (uv)(v^2u^2) = uv^3u^2 = u^3,
    \end{equation*}
    we have
    \begin{equation*}
      vu^3 = v(v^2u^3v) = v^3u^3v = u^3v,
    \end{equation*}
    so $v$ commutes with $u^3$.
  \end{proof}
\item Show that $v$ commutes with $u$.
  \begin{proof}
    Since $u^4 = 1$, we have $u^9 = u^4u^4u = u$. And since $v$
    commutes with $u^3$ we have
    \begin{equation*}
      uv = u^9v = u^6vu^3 = u^3vu^6 = vu^9 = vu.
    \end{equation*}
    Therefore $v$ commutes with $u$.
  \end{proof}
\item Show that $uv = 1$.
  \begin{proof}
    Since $u$ and $v$ commute, we get
    \begin{equation*}
      uv = (uv)(u^4v^3) = u^5v^4 = (v^2u^2)(u^3v^2)
      = (uv)(u^3v^2) = u^4v^3 = 1.\qedhere
    \end{equation*}
  \end{proof}
\item Show that $u = 1$, deduce that $v = 1$, and conclude that
  $Y = 1$.
  \begin{proof}
    Since $u^4v^3 = 1$ we have
    \begin{equation*}
      1 = u^4v^3 = u^3(uv)v^2 = u^3v^2 = u^2(uv)v = u^2v = u(uv) = u.
    \end{equation*}
    Then $v = uv = 1$ so that $1$ is the only element of $Y$. $Y$ is
    therefore the trivial group of order $1$.
  \end{proof}
\end{enumerate}
