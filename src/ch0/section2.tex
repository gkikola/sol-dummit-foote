\section{Properties of the Integers}

\Exercise{1} For each of the following pairs of integers $a$ and $b$,
determine their greatest common divisor, their least common multiple,
and write their greatest common divisor in the form $ax + by$ for some
integers $x$ and $y$.
\begin{enumerate}

\item
  $a = 20, b = 13$.
  \begin{solution}
    Applying the division algorithm repeatedly, we get
    \begin{align*}
      20 &= 1(13) + 7 \\
      13 &= 1(7) + 6 \\
      7 &= 1(6) + 1 \\
      6 &= 6(1) + 0.
    \end{align*}
    The first nonzero remainder is $1$, so $(20,13) = 1$. That is, the
    two numbers are relatively prime.

    The least common multiple, $[20, 13]$, is given by
    \begin{equation*}
      \frac{20\cdot13}{(20, 13)} = 260.
    \end{equation*}

    To write $1$ as a linear combination of $20$ and $13$, we work
    backwards and substitute:
    \begin{align*}
      1 &= 7 - 1(6) \\
        &= 7 - 1(13 - 1(7)) && \text{(Substituting $6 = 13 - 7$)} \\
        &= 2(7) - 1(13) \\
        &= 2(20 - 1(13)) - 1(13) && \text{(Substituting $7 = 20 - 13$)} \\
        &= 2(20) - 3(13). && \qedhere
    \end{align*}
  \end{solution}

\item
  $a = 69, b = 372$.
  \begin{solution}
    As above, we have
    \begin{align*}
      372 &= 5(69) + 27 \\
      69 &= 2(27) + 15 \\
      27 &= 1(15) + 12 \\
      15 &= 1(12) + 3 \\
      12 &= 4(3) + 0,
    \end{align*}
    so $(69, 372) = 3$, which gives $[69, 372] = 8556$. And again, as
    before, we write
    \begin{align*}
      3 &= 15 - 1(12) \\
        &= 15 - 1(27 - 1(15)) && \text{(Substituting $12 = 27 - 15$)} \\
        &= 2(15) - 1(27) \\
        &= 2(69 - 2(27)) - 1(27) && \text{(Substituting $15 = 69 - 2(27)$)} \\
        &= 2(69) - 5(27) \\
        &= 2(69) - 5(372 - 5(69)) && \text{(Substituting $27 = 372 - 5(69)$)} \\
        &= 27(69) - 5(372). && \qedhere
    \end{align*}
  \end{solution}

\item
  $a = 792, b = 275$.
  \begin{solution}
    \begin{align*}
      792 &= 2(275) + 242 \\
      275 &= 1(242) + 33 \\
      242 &= 7(33) + 11 \\
      33 &= 3(11) + 0.
    \end{align*}
    Hence $(792, 275) = 11$. Calculating the least common multiple
    gives $[792, 275] = 19\,800$. Then
    \begin{align*}
      11 &= 242 - 7(33) \\
         &= 242 - 7(275 - 242) \\
         &= 8(242) - 7(275) \\
         &= 8(792 - 2(275)) - 7(275) \\
         &= 8(792) - 23(275). \qedhere
    \end{align*}
  \end{solution}

\item
  $a = 11\,391, b = 5673$.
  \begin{solution}
    Using the methods above, we get
    \begin{align*}
      (11\,391, 5673) &= 3, \\
      [11\,391, 5673] &= 21\,540\,381 \\
      \intertext{and}
      -126(11\,391) + 253(5673) &= 3. \qedhere
    \end{align*}
  \end{solution}

\item
  $a = 1761, b = 1567$.
  \begin{solution}
    \begin{align*}
      (1761, 1567) &= 1, \\
      [1761, 1567] &= 2\,759\,487, \\
      \intertext{and}
      -105(1761) + 118(1567) &= 1. \qedhere
    \end{align*}
  \end{solution}

\item
  $a = 507885, b = 60808$.
  \begin{solution}
    \begin{align*}
      (507885, 60808) &= 691, \\
      [507885, 60808] &= 44\,693\,880, \\
      \intertext{and}
      -17(507885) + 142(60808) &= 691. \qedhere
    \end{align*}
  \end{solution}

\end{enumerate}

\Exercise{2} Prove that if the integer $k$ divides the integers $a$
and $b$ then $k$ divides $as + bt$ for every pair of integers $s$ and
$t$.
\begin{proof}
  Suppose $a$ and $b$ are such that $k\mid a$ and $k\mid b$. By
  definition, this means that there exists integers $m$ and $n$ such
  that $a = mk$ and $b = nk$. Therefore, for any integers $s$ and $t$,
  \begin{align*}
    as + bt &= (mk)s + (nk)t \\
            &= (ms + nt)k.
  \end{align*}
  Since $ms + nt$ must be an integer (due to closure of integer
  addition and multiplication), this shows that $k\mid(as + bt)$.
\end{proof}

\Exercise{3} Prove that if $n$ is composite then there are integers
$a$ and $b$ such that $n$ divides $ab$ but $n$ does not divide either
$a$ or $b$.
\begin{proof}
  The Fundamental Theorem of Arithmetic guarantees that $n$ is the
  product of two or more (possibly equal) prime factors. Let $a$ be
  one of the prime factors, and let $b$ be $n / a$. Note that $b$ must
  be an integer since $a\mid n$. Note also that $a,b > 1$.

  Now $n = ab$, so clearly $n \mid ab$. However, $n \nmid a$ since $a$
  is prime and $n$ is composite.

  Finally, suppose for contradiction that $n\mid b$. Then there is an
  integer $k > 1$ such that $b = kn$. Multiplying by $a$ on both sides
  gives $ab = akn$ or $n = akn$. Dividing by $n$ then gives $ak =
  1$. But this is absurd because $a$ and $k$ are both integers greater
  than $1$. This contradiction shows that $n\nmid b$, so the proof is
  complete.
\end{proof}

\Exercise{4} Let $a$, $b$, and $N$ be fixed integers with $a$ and $b$
nonzero and let $d = (a, b)$ be the greatest common divisor of $a$ and
$b$. Suppose $x_0$ and $y_0$ are particular solutions to $ax + by = N$
(i.e., $ax_0 + by_0 = N$). Prove for any integer $t$ that the integers
\begin{equation}
  \label{eq:gen-sol}
  x = x_0 + \frac{b}{d}t \quad\text{and}\quad y = y_0 - \frac{a}{d}t
\end{equation}
are also solutions to $ax + by = N$.
\begin{proof}
  Substituting for $x$ and $y$ in $ax + by$ gives
  \begin{align*}
    a\left(x_0 + \frac{b}{d}t\right)x + b\left(y_0 - \frac{a}{d}t\right)
    &= (ax_0 + by_0) + \frac{ab}{d}t - \frac{ab}dt \\
    &= ax_0 + by_0 \\
    &= N.
  \end{align*}
  This holds regardless of the value of $t$, so \eqref{eq:gen-sol} is
  always a valid solution.
\end{proof}

\Exercise{5} Determine the value $\varphi(n)$ for each integer
$n\leq30$ where $\varphi$ denotes the Euler $\varphi$-function.
\begin{solution}
  For each $n$, the value of $\varphi(n)$ can be determined by first
  finding the prime factorization of $n$,
  \begin{equation*}
    n = p_1^{\alpha_1}p_2^{\alpha_2}\cdots p_k^{\alpha_k},
    \quad\text{where each $p_i$ is prime},
  \end{equation*}
  and then by applying the formula given in the text:
  \begin{equation*}
    \varphi(n) = p_1^{\alpha_1 - 1}(p_1 - 1)p_2^{\alpha_2 - 1}(p_2 - 1)
    \cdots p_k^{\alpha_k - 1}(p_k - 1).
  \end{equation*}

  For example, to find $\varphi(18)$, we factor $18 =
  2\cdot3^2$. Applying the formula then gives
  \begin{align*}
    \varphi(18) &= 2^{1 - 1}(2 - 1)\cdot3^{2 - 1}(3 - 1) \\
                &= 1\cdot1\cdot3\cdot2 \\
                &= 6.
  \end{align*}
  Applying this process to each $n\leq30$ produces the following
  table:

  \medskip

  \begin{center}
    \begin{tabular}{r|c|c|c|c|c|c|c|c|c|c|c|c|c|c|c}
      $n$ & 1 & 2 & 3 & 4 & 5 & 6 & 7 & 8 & 9 & 10 & 11 & 12 & 13 & 14 & 15
      \\\hline
      $\varphi(n)$ & 1 & 1 & 2 & 2 & 4 & 2 & 6 & 4 & 6 & 4 & 10 & 4 & 12 & 6 & 8
    \end{tabular}

    \medskip
    \begin{tabular}{r|c|c|c|c|c|c|c|c|c|c|c|c|c}
      $n$ & 16 & 17 & 18 & 19 & 20 & 21 & 22 & 23 & 24 & 25 & 26 & 27 & 28
      \\\hline
      $\varphi(n)$
          & 8 & 16 & 6 & 18 & 8 & 12 & 10 & 22 & 8 & 20 & 12 & 18 & 12
    \end{tabular}

    \medskip
    \begin{tabular}{r|c|c}
      $n$ & 29 & 30 \\\hline
      $\varphi(n)$ & 28 & 8
    \end{tabular}
  \end{center}

  \medskip

  This process can be used to easily find $\varphi(n)$ for any $n$
  whose prime factorization is known.
\end{solution}

\Exercise{6} Prove the Well Ordering Property of $\Z$ by induction and
prove the minimal element is unique:
\begin{thm}
  If $A$ is any nonempty subset of $\Z^+$, there is some element
  $m\in A$ such that $m\leq a$, for all $a\in A$.
\end{thm}
\begin{proof}
  Suppose for contradiction that $A$ has no minimal element. We will
  prove by (strong) induction on $n$ that for each $n\in\Z^+$,
  $n\not\in A$. This will show that $A$ is the empty set, which would
  contradict the requirement that $A$ be nonempty.

  Clearly $1\not\in A$, for otherwise $1$ would be a least element
  (since $1\leq a$ for all $a\in\Z^+$). Now suppose that
  $1, 2, \ldots, k\not\in A$ for some positive integer $k$. Then
  $k + 1$ cannot be a member of $A$ since otherwise $k + 1$ would be
  the minimal element. This completes the inductive step, which shows
  that $A$ is the empty set, giving the needed contradiction to show
  that $A$ has a minimal element.

  Finally, to show that the minimal element is unique, suppose $A$ has
  two minimal elements, $a$ and $b$. Since $a$ is minimal, $a\leq
  b$. But $b$ is minimal, so $b\leq a$. So $a\leq b$ and $a\geq b$ and
  therefore $a = b$.
\end{proof}

\Exercise{7} If $p$ is a prime prove that there do not exist nonzero
integers $a$ and $b$ such that $a^2 = pb^2$ (i.e., $\sqrt p$ is not a
rational number).
\begin{proof}
  Suppose for contradiction that $a$ and $b$ are nonzero integers with
  \begin{equation*}
    a^2 = pb^2.
  \end{equation*}
  Without loss of generality we may also assume that $a$ and $b$ have
  no factors in common (if they do have factors in common, just divide
  the factors from both sides of the equation).

  Now $p \mid a^2$. And since $p$ is prime, we must also have
  $p\mid a$ (this uses the ``important property'' mentioned in item
  {\bf(8)} on page 6 of the text). Then there is an integer $m$ such
  that $a = pm$ and hence $(pm)^2 = pb^2$, or $p^2m^2 = pb^2$. This
  implies that $pm^2 = b^2$ so that $p \mid b^2$, which implies
  $p\mid b$. But $a$ and $b$ were chosen to have no factors in common,
  yet $p$ is a common factor. This gives the needed contradiction.
\end{proof}

\Exercise{8} Let $p$ be a prime, $n\in\Z^+$. Find a formula for the
largest power of $p$ which divides $n! = n(n-1)(n-2)\cdots2\cdot1$.
\begin{solution}
  The only integers less than $n$ that are divisible by $p$ are the
  multiples of $p$, of which there are
  \begin{equation*}
    \Floor{\frac{n}{p}}
  \end{equation*}
  of them, where $\floor{x}$ denotes the floor of $x$ (i.e., the
  greatest integer less than or equal to $x$).

  However, multiples of $p^2$ each contribute a second factor of
  $p$. Multiples of $p^3$ contribute a third additional factor of $p$,
  and so on. Therefore the highest power of $p$ that divides $n!$ is
  given by
  \begin{equation*}
    \Floor{\frac{n}{p}} + \Floor{\frac{n}{p^2}} + \Floor{\frac{n}{p^3}}
    + \cdots
    = \sum_{k=1}^{\floor{\log_pn}}\Floor{\frac{n}{p^k}}.\qedhere
  \end{equation*}
\end{solution}

\Exercise{10} Prove for any given positive integer $N$ there exist
only finitely many integers $n$ with $\varphi(n) = N$ where $\varphi$
denotes Euler's $\varphi$-function. Conclude in particular that
$\varphi(n)$ tends to infinity as $n$ tends to infinity.
\begin{solution}
  Fix a value of $N>0$, and let $A$ be the set of all solutions $n$ to
  the equation $\varphi(n) = N$. We must show that $A$ is a finite
  set.

  First we will show that for any $n\in A$, there cannot be a prime
  factor of $n$ larger than $N + 1$. For if there are prime factors
  larger than $N + 1$, then we may choose the smallest such prime
  $p$. Then if $q$ is any prime factor of $n$ with $q\geq p$, we may
  write $n = q^kr$, where $r$ is some positive integer relatively
  prime to $q$. Therefore we have
  \begin{align*}
    \varphi(n) &= \varphi(q^k)\varphi(r) \\
               &= q^{k-1}(q - 1)\varphi(r) \\
               &\geq q - 1 > N.
  \end{align*}
  But $\varphi(n) = N$, so this is a contradiction. This shows that
  all prime factors of $n$ must be at most $N + 1$.

  Now let $p_1, p_2, \ldots, p_m$ be all the prime factors less than
  or equal to $N + 1$ (note that this set of primes is finite). Then
  every $n\in A$ can be written in the form
  \begin{equation*}
    n = p_1^{\alpha_1}p_2^{\alpha_2}\cdots p_m^{\alpha_m},
  \end{equation*}
  where each $\alpha_i\geq 0$ and $\alpha_j > 0$ for at least one
  index $j$. Now observe that each $\alpha_i$ can be one of only
  finitely many possible values, since
  $\varphi(p_i^s) = p_i^s(p_i - 1) > N$ for sufficiently large values
  of $s$, and $N$ is the product of each $\varphi(p_i^{\alpha_i})$. So
  the distinct values of $n$ in $A$ must be finite in number, because
  there are only finitely many possible primes in their prime
  factorizations and their exponents can take only finitely many
  possible values.

  Finally, let $M$ be any positive integer. Since there are only
  finitely many values of $n$ such that $\varphi(n) \leq M$, we may
  choose the largest such $n$. Then $\varphi(m) > M$ for all $m>n$,
  which shows that $\varphi(n)$ tends to infinity as $n$ tends to
  infinity.
\end{solution}

\Exercise{11} Prove that if $d$ divides $n$ then $\varphi(d)$ divides
$\varphi(n)$ where $\varphi$ denotes Euler's $\varphi$-function.
\begin{solution}
  First consider the case where $n = p^k$ for some prime number
  $p$. Then if $d\mid n$ we must have $d = p^\ell$ for some integer
  $\ell$ with $0\leq\ell\leq k$. So
  \begin{equation*}
    \varphi(n) = \varphi(p^k) = p^{k-1}(p - 1)
    \quad\text{and}\quad
    \varphi(d) = \varphi(p^\ell) = p^{\ell-1}(p - 1).
  \end{equation*}
  Now let $a = p^{k - \ell}$. Then $a\varphi(d) = \varphi(n)$, so
  $\varphi(d)\mid\varphi(n)$.

  The more general case will follow from the fact that $\varphi$ is a
  multiplicative function: Let $n$ be a positive integer having prime
  factorization
  \begin{equation*}
    n = p_1^{\alpha_1}p_2^{\alpha_2}\cdots p_k^{\alpha_k},
  \end{equation*}
  and suppose $d$ is an integer that divides $n$. Then $d$ can be
  written as a product of these same prime factors $p_1,\ldots,p_k$,
  provided that we allow some of the exponents to be zero. That is, we
  may write
  \begin{equation*}
    d = p_1^{\beta_1}p_2^{\beta_2}\cdots p_k^{\beta_k}
    \quad\text{with $0\leq\beta_i\leq\alpha_i$ for each $i$.}
  \end{equation*}
  Then
  \begin{equation}
    \label{eq:phi_n}
    \varphi(n) = \varphi(p_1^{\alpha_1})\varphi(p_2^{\alpha_2})\cdots
    \varphi(p_k^{\alpha_k})
  \end{equation}
  and
  \begin{equation}
    \label{eq:phi_d}
    \varphi(d) = \varphi(p_1^{\beta_1})\varphi(p_2^{\beta_2})\cdots
    \varphi(p_k^{\beta_k}).
  \end{equation}
  Now each $p_i^{\beta_i}$ divides $p_i^{\alpha_i}$, so from the
  argument in the first paragraph, we know that
  $\varphi(p_i^{\beta_i}) \mid \varphi(p_i^{\alpha_i})$ for each
  $i$. Therefore we may find an integer $a_i$ such that
  $\varphi(p_i^{\alpha_i}) = a_i\varphi(p_i^{\beta_i})$. Therefore,
  equations \eqref{eq:phi_n} and \eqref{eq:phi_d} imply that
  \begin{align*}
    \varphi(n) &= a_1\varphi(p_1^{\beta_1})
                 \cdot a_2\varphi(p_2^{\beta_2})
                 \cdots a_k\varphi(p_k^{\beta_k}) \\
               &= (a_1a_2\cdots a_k)\varphi(d),
  \end{align*}
  so $\varphi(d) \mid \varphi(n)$.
\end{solution}
